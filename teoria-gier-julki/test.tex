\documentclass{article}

\usepackage{../uni-notes-eng}

\begin{document}

Wstęp do teorii gier
Lista zadań $n r 1$
Dla opisanych poniżej gier określ zbiory strategii graczy, funkcję (macierz) wypłat oraz wartości gry każdego gracza. Zakładamy, że cztery ostatnie są grami dwuosobowymi o sumie zero.

1. Uczestnicy tej wieloosobowej gry rozważają podjęcie pewnej inicjatywy - akcji. Każdy z nich ma do wyboru jedną z dwóch decyzji: "wziąć w niej udzial" lub "pozostać z boku". Jeśli więcej niż połowa uczestników przyłączy się do akcji, to każdy z nich otrzymuje wypłatę 100 . Gdy uczestnicy, którzy tak zdecydują, nie osiągną większości, to każdy z nich zapłaci "karę" wysokości 10. W każdym przypadku gracze, którzy nie przyłączą się do akcji, ani nie otrzymują premii, ani nie płaca kary.

2. Dwóch myśliwych może polować na jelenia (J) lub zające (Z). Ich decyzje zapadają równocześnie i niezależnie (tzn. bez wiedzy o decyzji drugiego). Jeleń ma wartość 4, zające po 1. Jeśli obaj zapolują na jelenia, to upoluja go, otrzymując po 2. Jeśli jeden wybierze J, a drugi Z, to pierwszy nic nie upoluje, a drugi upoluje zająca. Jeśli obaj wybiorą Z, otrzymuja po 1.

3. Dwóch pracowników wykonuje pewną pracę, przy czym każdy z nich może pracować (wtedy $s_i=1$ ) lub udawać prace (wtedy $s_i=0$ ), $i=1,2$. Jeśli gracz pracuje, ponosi koszt 3 , jeśli tylko udaje 0 . Wynik pracy wynosi $2\left(s_1+s_2\right)$ dla każdego z nich, niezależnie od tego, czy pracował, czy udawał.

4. Dwóch kierowców stoi na drodze zasypanej przez lawinę. Całkowity nakład energii potrzebny do odśnieżenia drogi wynosi $c>0$, korzyśćć każdego z nich z dojechania do domu to $b>c$. Energia (wypłata) każdego gracza, gdy obaj nic nie robią wynosi $a<b-c$.

5. Dwóch graczy ma do podziału kwotę 100. A proponuje B podział: $x$ dla $\mathrm{B}$, reszta dla $\mathrm{A}$, gdzie $x \in\{1,2, \ldots, 100\}$. Gracz B może się zgodzić i wtedy wypłaty są takie, jak zaproponował A, albo nie zgodzić i wtedy każdy otrzymuje zero.

6. Każdy gracz rzuca dwie lub trzy kostki i jednocześnie jedną monetę. Jeśli I gracz wybierze $l_1$ kostek, a drugi gracz $l_2$ kostek oraz $n$ jest liczbą orłów, które pojawiły się w rzutach obu graczy, to I gracz otrzymuje wygraną równą
$$
W_n\left(l_1, l_2\right)=\left|l_1-2 n\right|+\left|l_2-2 n\right| .
$$

7. W uproszczonej wersji włoskiej gry Morra każdy z graczy pokazuje jeden lub dwa palce i jednocześnie zgaduje, ile palców pokaże przeciwnik. Jeśli obaj odgadną albo pomylą się, wypłata wynosi zero. Jeśli tylko jeden odgadnie, otrzymuje wypłatę równą sumie palców wystawionych przez obu graczy.

\end{document}