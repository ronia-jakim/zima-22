\documentclass{article}

\usepackage{uni-notes}
%\usepackage{showframe}

%\renewcommand*\ShowFrameColor{\color{gray}}

\begin{document}

\section{NAD NIEMNEM}
\subsection{czemu to istnieje?}

Dzień {\color{def}był letni i świąteczny}. Wszystko na świecie jaśniało, kwitło, pachniało, śpiewało. Ciepło i radość lały się z błękitnego nieba i złotego słońca; radość i upojenie tryskały znad pól porosłych zielonym zbożem; radość i złota swoboda śpiewały chórem ptaków i owadów nad równiną w gorącym powietrzu, nad {\color{acc}niewielkimi wzgórzami, w okrywających} je bukietach iglastych i liściastych drzew.\bigskip

\podz{sep}\bigskip

Z jednej strony widnokręgu wznosiły się niewielkie wzgórza z ciemniejącymi na nich borkami i gajami; z drugiej wysoki brzeg Niemna piaszczystą ścianą wyrastający z zieloności ziemi, a koroną ciemnego boru oderżnięty od błękitnego nieba ogromnym półkolem obejmował równinę rozległą i gładką, z której gdzieniegdzie tylko wyrastały dzikie, pękate grusze, stare, krzywe wierzby i samotne, słupiaste topole. Dnia tego w słońcu ta piaszczysta ściana miała pozór półobręczy złotej, przepasanej jak purpurową wstęgą tkwiącą w niej warstwą czerwonego marglu. Na świetnym tym tle w zmieszanych z dala zarysach rozpoznać można było dwór obszerny i w niewielkiej od niego odległości na jednej z nim linii rozciągnięty szereg kilkudziesięciu dworków małych. Był to wraz z brzegiem rzeki zginający się nieco w półkole sznur siedlisk ludzkich, większych i mniejszych, wychylających ciemne swe profile z większych i mniejszych ogrodów. Nad niektórymi dachami, w powietrzu czystym i spokojnym wzbijały się proste i trochę tylko skłębione nici dymów; niektóre okna świeciły od słońca jak wielkie iskry; kilka strzech nowych mieszało złocistość słomy z błękitem nieba i zielonością drzew.

\pgraf
    \paxis {domain=-10:20}
        \addplot[def, ultra thick]{x * x};
    \kaxis
\kgraf

Równinę przerzynały drogi białe i trochę zieleniejące od z rzadka porastającej je trawy; ku nim, niby strumienie ku rzekom, przybiegały z pól miedze, całe błękitne od bławatków, żółte od kamioły, różowe od dzięcieliny i smółek. Z obu stron każdej drogi szerokim pasem bielały bujne rumianki i wyższe od nich kwiaty marchewnika, słały się w trawach fioletowe rohule, żółtymi gwiazdkami świeciły brodawniki i kurze ślepoty, liliowe skabiozy polne wylewały ze swych stulistnych koron miodowe wonie, chwiały się całe lasy słabej i delikatnej mietlicy, kosmate kwiaty babki stały na swych wysokich łodygach, rumianością i zawadiacką postawą stwierdzając nadaną im nazwę kozaków. Za tymi pasami roślinności dzikiej cicho w cichej pogodzie stało morze roślin uprawnych. Żyto i pszenica miały kłosy jeszcze zielone, lecz już osypane drżącymi rożkami, których obfitość wróżyła urodzaj; niższe znacznie od nich, rumianym kwiatem gęsto usiane słały się na szerokich przestrzeniach liściaste puchy koniczyny; puchem też, zda się, ale drobniejszym, delikatniejszym, z zielonością tak łagodną, że oko pieściła, młody len pokrywał gdzieniegdzie kilka zagonów, a żółta jaskrawość kwitnącego rzepaku wesołymi rzekami przepływała po łanach niskich jeszcze owsów i jęczmion.

\begin{lstlisting}[language=racket]
    (define (swietnyjezyk a b)
        (if (< a b)
            (- b a)
            (+ b a)
        )
        []
    )
\end{lstlisting}
\kdowod

Wśród tej wesołej przyrody ludzie dziś także byli weseli; mnóstwo ich ciągnęło po drogach i miedzach. Gromadami na drogach, a sznurami na miedzach szły wiejskie kobiety, których głowy ubrane w czerwone i żółte chusty tworzyły nad zbożami korowody żywych piwonii i słoneczników. Od tych gromad lały się i płynęły po łanach strumienie różnych głosów. Były to czasem rozmowy gwarne i krzykliwe, czasem śmiechy basowe lub srebrzyste, czasem płacze niemowląt u piersi w chustach niesionych, czasem też pieśni przeciągłe, głośne, których nutę porywały i przedłużały echa ze stron obu: w borkach i gajach rosnących na wzgórzach i w wielkim borze, który ciemnym pasem odcinał pozłoconą, przetkaną szkarłatem ścianę nadniemeńską od wysadzanych srebrnymi obłokami błękitów nieba. W tym ruchu ludzkim odbywającym się na urodzajnej równinie czuć było najpiękniejszy dla wiejskiej ludności moment święta: wesoły i wolny w słoneczny i wolny dzień boży powrót z kościoła.

W porze, kiedy ten ruch znacznie już stawał się mniejszy, na równinie, z dala od tu i ówdzie jeszcze ciągnących gromad, ukazały się dwie kobiety. Szły one z tej samej strony, z której wracali inni, ale zboczyły znać z prostej drogi i chwilę jakąś przebyły w jednym z rosnących na wzgórzach borków. Było to przyczyną ich spóźnienia się, tym więcej, że jedna z nich niosła ogromną więź leśnych roślin, które tylko co i dość długo zapewne zrywała. Druga kobieta zamiast kwiatów trzymała w ręku niepospolitej wielkości chustkę, która za każdym jej szerokim i zamaszystym krokiem kołysząc się wraz z długim ramieniem powiewała jak sporej wielkości chorągiew. Z dala to tylko widać było, że jedna z tych kobiet miała w ręku pęk roślin, a druga szmatę białego płótna; z bliska uderzały one niezupełnie zwykłą powierzchownością.

Kobieta z chustką była niezwykle wysoka, a wysokość tę zwiększała jeszcze chudość jej ciała, które przecież posiadało szkielet tak rozrosły i silny, że pomimo chudości ramiona jej były szerokie i wydawałyby się bardzo silne, gdyby nie małe przygarbienie pleców i karku objawiające trochę znużenia i starości, gdyby także nie ostre kości łopatek podnoszące w dwu miejscach staroświecką mantylę z długimi, co chwilę powiewającymi końcami. Oprócz tej mantyli zaopatrzonej w płócienny kołnierz miała ona na sobie czarną spódnicę, tak krótką, że spod niej aż prawie do kostek widać było dwie duże i płaskie stopy ubrane w grube pończochy i wielkie, kwieciste pantofle. Ubiór ten uzupełniony był przez stary kapelusz słomiany, którego szerokie brzegi ocieniały twarz, na pierwszy rzut oka starą, brzydką i przykrą, ale po bliższym przyjrzeniu się uwagę i ciekawość budzić mogącą. Była to mała, chuda, okrągła twarz ze skórą tak ciemną, że prawie brązową, z czołem sfałdowanym w kilka grubych zmarszczek, z wpadłymi i kościstymi policzkami, z wyrazem goryczy i złośliwości nadawanym jej przez ostrość nosa i zaciśnięcie warg, a wzmaganym przez szczególną ognistość i przenikliwość oczu. Te oczy zdawały się być jedynym bogactwem tej biednej, zestarzałej, złośliwej twarzy. Może kiedyś były one jedyną jej pięknością, a teraz, wielkie i czarne, spod czarnych, szerokich brwi oświecały ją jeszcze przejmującym blaskiem; miały one spojrzenie przenikliwe, ostre, urągliwe i płomienność nieustanną, jakby wciąż z wnętrza podsycaną, a dziwną przy tej sczerniałej, w dłoni czasu czy losu zgniecionej twarzy. Szła szerokim, zamaszystym, do pośpiechu znać przyzwyczajonym krokiem, a długie jej ramiona, u których wisiały ciemne, kościste ręce, kołysały się u jej boków w tył i naprzód, białą, rozwiniętą chustką jak chorągwią powiewając.

\end{document}