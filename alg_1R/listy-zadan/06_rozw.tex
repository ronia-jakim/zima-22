\documentclass{article}[13pt]

\usepackage{../../uni-notes-eng}
\usepackage{multicol}
\usepackage{graphicx}


\begin{document}

\subsection*{ZAD. 1.}

\emph{Let $g\in G$ be of order $n$. Show that for each $m\in \Z$ we have:}
$$g^m=e\iff n|m.$$

$\color{acc}\impliedby$

We know that there exists $k\in\Z$ such that $m=nk$. That means 
$$g^m=g^{nk}=(g^n)^k=e^k=e$$

$\color{acc}\implies$

Proof by contraposition. Let us assume that $n\nmid m$. So $m=nk+r$. Then $r\in\{1,...,n-1\}$. So we have
$$g^m=g^{nk+r}=g^{nk}g^r=eg^r\neq e$$

\subsection*{ZAD. 2.}

\emph{Let $N\leq G$. Show that the following conditions are equivalent.}

(a) $N\dzielnat G$

$\color{acc}N\dzielnat G\implies (\forall\;g\in G)\;gNg^{-1}=N$

Take any $g\in G$. Take any $x\in gNg^{-1}$. Then $x=gng^{-1}$ for some $n\in N$. We have
$$x=gng^{-1}=(gn)g^{-1}=(n'g)g^{-1}=n'\in N$$
because $gn\in gN=Ng$ so we took $n'$ such that $gn=n'g$.

Take any $n\in N$. Then 
$$n=ngg^{-1}=(ng)g^{-1}=(gn')g^{-1}=gn'g^{-1}\in gNg^{-1}.$$

$\color{acc}(b)\implies (c)$

Take any $g\in G$ and any $n\in N$. We want to show that then $gng^{-1}\in N$. 
$$gng^{-1}\in gNg^{-1}=N.$$

$\color{acc}(c)\implies (a)$

Take any $g\in G$. We want to show that $gN=Ng$. Take any $x\in gN$. It means, that for some $n\in N$ we have
$$x=gn$$
which we can multiply on both sides by $g^{-1}$ from right and get
$$xg^{-1}=gng^{-1}\in N.$$
So if we multiply it from right by $g$ we would get
$$xg^{-1}g=x\in Ng.$$
The other inclusion is analogous.

Or, if we want to be more fancy, we could write:
$$gN\subseteq Ng$$
for any $g$, so in particular for $g^{-1}$
$$g^{-1}N\subseteq Ng^{-1}$$
$$gg^{-1}Ng\subseteq gNg^{-1}g$$
$$Ng\subseteq gN$$
and the end! Only one inclusion written meticulously.

\subsection*{ZAD. 3.}

\emph{Let $g\in G$ be of order 2. Show that}
$$g\in Z(G)\iff \{e,g\}\dzielnat G$$

(we should also prove that $\{e,g\}$ is a subgroup but whooo caaareees)

$\color{acc}\implies$

Here we just want to prove that $kgk^{-1}\in \{e,g\}$ for every $k$. Since $g$ is commutative, we have
$$(kg)k^{-1}=gkk^{-1}=g\in \{e, g\}$$

$\color{acc}\impliedby$

We know that $kgk^{-1}\in\{e,g\}$ for every $k$. We want to show that $g$ is in $Z(G)$. We have two possibilities:
$$kgk^{-1}=g\implies kg=gk$$
and this is what we wanted.

$$kgk^{-1}=e\implies kg=k\implies g=e$$
which cannot be because $g$ is of order $2$ and $e$ is of order $1$.

\subsection*{ZAD. 4.}

\emph{Let $g\in G$ be a unique element of order 2 in $G$. Show that $g\in Z(G)$.}

$g\in G$ such that $ord(g)=2$ only for $g$. Let us take any other $h\in G$. We should observe that $hgh^{-1}hgh^{-1}=hggh^{-1}=hg(hg^{-1})^{-1}=(hg)(hg)^{-1}=e$. So it is either that $ghg^{-1}$ is of order 1 or order 2. 

If it is of order $1$, then
$$e=hgh^{-1}\implies h=hg\implies 2=g$$
and this cannot be!

If it is of order $2$, then we have
$$hgh^{-1}=g\implies hg=gh$$
and so $g\in Z(G)$.

{\color{def}What if $g$ is a unique element of order $k$? (show that then $g\in Z(G)$)}

\subsection*{ZAD. 5.}

\emph{Using the fundamental theorem on group homomorphism, show the following}

$$\color{acc}(\R*, \cdot)/\{1, -1\}\simeq (\R_{>0}, \cdot)$$

$f:L\to R$ such that $f(x)=x^2$. We will prove that this is indeed a homomorphism.
$$f(xy)=(xy)^2=xyxy=x^2y^2=f(x)f(y).$$

$$\color{acc}(\C, +)/\Z\simeq (\C*, \cdot)$$

$f:L\to R$ such that $f(z)=e^{2\pi iz}$ jeees

$$f(x+y)=e^{2\pi i(x+y)}=e^{\pi ix}e^{\pi iy}=f(x)f(y).$$

Shit I have no clue what is happening now. "We got the pie!!!!"

$$\color{acc}(\C*, \cdot)/\langle e^{2\pi i\frac1n}\rangle\simeq (\C*, \cdot)$$

$$f(z)=z^n$$
AAAAAAAAAAAAAAAAAAAAAAAAAAAAAAAAAAAAAAAAAAAAAAA

\section*{ZAD. 6.}

\emph{Let $p$ be a prime number and assume that $|G|=p^2$. Show that:}
$$G\simeq \Z_{p^2}\;or\; G\simeq \Z_p\times\Z_p.$$

By Lagrange theorem we know, that for any $g\in G$ $ord(g)=p$ or $ord(g)=p^2$ (because each such element generates a cyclic subgroup of $G$). So we have two possibilities:

1. There exists an element $g$ that has order $p^2$. So $G=\langle g\rangle$ and it is obvious that such a group is isomorphic with $\Z_{p^2}$.

2. There are no elements of order $p^2$, then all its elements have order $p$. So we have $g,h\in G$ such that $g\neq h$ such that $\langle g\rangle\cap \langle h\rangle=\{e\}$ and so they add up to the whole $G$. Ok so now let us consider a group
$$H=\langle g\rangle \times \langle h\rangle$$
It has order $p^2$. Because $\langle g\rangle\simeq \Z_p$ and the same goes for $\langle h\rangle$, let $\phi_g,\phi_h$ be those two isomorphisms. We want to show that
$$f(x,y)=(\phi_g(x), \phi_h(y))$$
is a homomorphism $H\simeq \Z_p\times\Z_p$. It is quite simple. Now to show that actually $G\simeq H$. Which would be
$$f(x)=\begin{cases}
    (x, e)\;x\in \langle g\rangle\\
    (e, x)\;x\in \langle h\rangle\\
    (q, p)\;if x=qp,q\in\langle g\rangle,p\in\langle h\rangle
\end{cases}$$
Ok, but what if i take $f(gh)$?

\subsection*{ZAD. 7.}

\emph{Let $\phi:\Z_2\to Aut(\Z_n)$ be the group action from Problem 2 of List 5. Show that}
$$D_n\simeq \Z_n\rtimes_\phi\Z_2$$

Ok, I really have to learn what on earth does semidirect product mean.

Let 
$$N=\{R\in D_n\;:\;R\text{ - rotation}\}$$ 
and 
$$H=\{id, S\}.$$
Then $H\cap N=\{id\}$. It is clear that $N\simeq \Z_n$. Then we can decide weather or not we will invert a rotation by symmetry, so we have that $H\simeq \Z_2$. AAAAAAAAAAAAAAAAAAAAAAAAAAAAAAAAAAAAAAAAAAAAAAA


\end{document}