\documentclass{article}[13pt]

\usepackage{../../uni-notes-eng}
\usepackage{multicol}
\usepackage{graphicx}


\begin{document}
$R$ is a commutative ring with 1 and $n\in\N_{>0}$.

\subsection*{ZAD. 1.}
\emph{Let $r_1,...,r_n\in R$. Show that $(r_1,...,r_n)=r_1R+...+r_nR$.}
\medskip

\podz{sep}
\medskip

From what I gathered, $(r_1,...,r_n)$ is the minimal ring that contains $\{r_1,...,r_n\}$. Inclusion $\subseteq$ is quite trivial but the other way around is more difficult.

Induction?

\subsection*{ZAD. 2.}
\emph{Let $I\dzielnat R$ and $\sqrt{I}:=\{a\in R\;:\;(\exists\;n\in \N)\;a^n\in I\}$. Show that $\sqrt{I}\dzielnat R$.}
\medskip

\podz{sep}
\medskip

$I$ is an ideal, which means that $a,b\in I\implies a+b\in I$ and $r\in R,a\in I\implies ra\in I$. This means that $I$ is a normal subgroup of $(R, +)$ and that notations starts to make sens to me right now.
\smallskip

First, let us tacle the multiplication. We take any $a\in I$. This means that for some $n$ $a^n\in I$ and so for any $r\in R$ we have $ra^n\in I$. So in particular for $r^n\in R$ we have $r^na^n\in I$, which means that $r^na^n=(ra)^n\in I$ and $ra\in\sqrt{I}$ for any $r\in R$.
\smallskip

Now, for the addition. This one is more difficult because we have to see that for $a^n, b^k\in I$ with assumption that $k\leq n$ we also have $(a+b)^n\in I$. But if $(a+b)^n\in I$ then in particular $(a+b)^{2n}\in I$ and the other way around. So let us start here. 
\begin{align*}
    (a+b)^{2n}&=a^{2n}+{2n\choose 1}a^{2n-1}b+...+{2n\choose 2n-1}ab^{2n-1}+b^{2n}=\\
    &=a^n(a^n+{2n\choose 1}a^{n-1}b+...+{2n\choose n}b^n)+b^k({2n\choose n-1}a^{n-1}b^{n+1-k}+...+{2n\choose 1}ab^{2n-1-k}+b^{2n-k})=\\
    &=a^n\cdot r_a+b^k\cdot r_b
\end{align*}
for $r_a,r_b\in R$ with those brutal formulas as seen above. Therefore $a^nr_a\in I$ and $b^kr_b\in I$ which means that $a^nr_a+b^kr_b\in I$ but this is equal to $(a+b)^n\in I$ so we have for $a,b\in\sqrt{I}$ $a+b\in\sqrt{I}$.

\subsection*{ZAD. 3.}
\emph{Let $f:R\to S$ be a homomorphism of commutative rings with $1$, $I\dzielnat R$, and $J\dzielnat S$. Show the following:}

\point $f^{-1}(J)\dzielnat R$

\point \emph{if $F$ is and epimorphism (onto) then }$f(I)\dzielnat S$

\point \emph{give an example of $f, I$ such that} $f(I)\not\dzielnat S$
\medskip

\podz{sep}
\medskip

$\color{def}f^{-1}(J)\dzielnat R$
\smallskip

So let us take any $a,b\in f^{-1}(J)$ we know that $f(a),f(b)\in f(J)$ so $f(a)+f(b)\in J$ as well. But $f$ is a homomorphism, so it is additive or some bullshit and we can write
$$J\ni f(a)+f(b)=f(a+b)$$
so also $a+b\in f^{-1}(J)$, which gives as the addition thingy thing.
\smallskip

Now, the dreaded multiplication. We take any $a\in f^{-1}(J)$ and any $r\in R$. We know that $f(r)\in S$ and $f(a)\in J$, which means that $f(r)f(a)\in J$. But again, $f$ is a homomorphism, so $f(r)f(a)=f(ra)\in J$ so $ra\in f^{-1}(J)$.
\medskip

$\color{def}f(I)\dzielnat S$ when $f$ is onto
\smallskip

First of all, addition. We know that $a,b\in I$ implies that $a+b\in I$. So $f(a)+f(b)=f(a+b)\in f(I)$. Now, for the multiplication. We take any $r\in R$ and any $a\in I$ and we know that $ra\in I$ so $f(ra)=f(r)f(a)\in f(I)$ and $f(r) $ can be any element in $S$ because $f$ is onto.
\medskip

{\color{def}EXAMPLE} but im too lazy to think about it right now.

\subsection*{ZAD. 4.}
\emph{Find $f\in \Q[X]$ such that $(f)=(X^2-1,x^3+1)$.}
\medskip

\podz{sep}
\medskip

So we are looking for a function for which the smallest ideal that contains it is equal to the smallest ideal that contains $X^2-1$ and $X^3-1$. Maybe first let us write down how the RHS looks like
$$(x^2-1,x^3-1)=\{ra\;:\;r\in\Q[X], a=X^2-1,X^3+1\}.$$
Ok, now how about the LHS?
$$(f)=\{rf\;:\;r\in \Q[X]\}$$
Let us take any $r\in\Q[X]$ then we have $r(x^2-1)\in (f)$ and $r(x^3-1)\in (f)$. Furthermore, we have that $(x^2-1)+(x^3-1)=x^3+x^2\in (f)$.
Maybe $f=x+1$? Yep.

We have that 
$$(x+1)(x-1)=x^2-1\in RHS$$ 
and also 
$$(x^2-x+1)(x+1)=x^3+1\in RHS$$
$$x^2(x+1)=x^3+x^2\in RHS$$

\subsection*{ZAD. 5.}
\emph{Show that the ideal $(2, X)\dzielnat\Z[X]$ is not principal.}
\medskip

\podz{sep}
\medskip

A principal ideal is an ideat $I$ generated by one element $a\in R$ through multiplication of $a$ by all elements of $R$.
\smallskip

$$(2, x)=\{ra\;:\;r\in\Z[X],a=2,x\}$$

So what if $(2,x)$ was a principal ideal? We would have an $a\in\Z[x]$ such that 
$$(\forall\;y\in(2,x))(\exists\;r\in\Z[x])\;ra=y$$
So let us start with $2$ and $x$. We assumed that $a$ as above existed, so
$$r_1\cdot a=2$$
$$r_2\cdot a=x$$
and then
$$2+x=r_1\cdot a+r_2\cdot a=(r_1+r_2)\cdot a.$$
Now, because we only have polynomials with integer coefficients, we must have $a$ of order 0, otherwise we could not obtain $2$ by multiplying $a$ by some other polynomial. We can have either $a=1$ or $a=2$. So for the second case we would need to find $r_2$ such that $r_2\cdot2=x$ and we know that $r_2$ must be of order $1$ so it must be $r_2=r_2'x$ and $r_2'x\cdot 2=x$ meaning, that $r_2'=\frac12$ which cannot be. Therefore, we are left with $a=1$ and $r_2=2$. Then, we have
$$2+x=2\cdot 1+x\cdot 1=(2+x)\cdot 2=4+2x$$
and this is a contradiction.


\end{document}