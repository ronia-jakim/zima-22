\documentclass{article}[13pt]

\usepackage{../../uni-notes-eng}
\usepackage{multicol}
\usepackage{graphicx}
\usepackage{tabularx}

\begin{document}
    \section*{Lista 1 tlumaczenie przez zjeba}

    Wyjatkowe cwiczenia 12.10.2022 (sroda) 12:15-14:00
    \bigskip

    Dla zbioru $X$, $\Po X$ jest zbiorem wszystkich podzbiorow $X$, a $S_X$ jest zbiorem wszystkich bijeckji $X\to X$
    \medskip

    1. Podaj przyklad operacji $\star$ na zbiorze $\{0,1\}$ takiej, ze
    $$0\star(0\star0)\neq(0\star0)\star0$$
    Ile jest takich operacji $\star$ na tym zbiorze?
    \medskip

    2. Zaloz, ze $\star$ jest laczna operacja na zbiorze skonczonym $A$. Pokaz, ze istnieje $a\in A$ takie, ze $a\star a=a$.
    \medskip

    3. Niech $\star$ bedzie operacja na zbiorze $X$ oraz $a,b,c\in X$. Pokaz, ze:\smallskip\\
        \indent a. Jezeli $b$ i $c$ sa neutralnymi elementami $\star$, to $b=c$.\\
        \indent b. Jezeli operacja $\star$ jest laczna, $\star$ ma element neutralny $e$, $a\star b=e$ i $c\star a=e$, to wtedy $b=c$\\
        \indent c. Jezeli $(X,\star)$ jest grupa z elementem neutralnym $e$ i $a\star b=e$, to wtedy $b\star a=e$
    \medskip

    4. Niech $f:X\to X$. Pokaz, ze\\
        \indent a. Funkcja $f$ jest na wtw jesli istnieje funkcja $g:X\to X$ taka, ze $f\circ g=id_x$\\
        \indent b. Funckja $f$ jest 1-1 wtw istnieje funckja $h:X\to X$ taka, ze $h\circ f=id_X$
    \medskip

    5. Niech $G$ bedzie grupa przeksztalcen na $X$. Pokaz, ze $id_x\in G$.
    \medskip

    6. Pokaz, ze operacja $+$ na zbiorze $\R\cup\{\infty\}$ (zdefiniowana jak na wykladzie) jest laczna i ma element neutralny, ale $(\R\cup\{\infty\},+)$ nie jest grupa.
    \medskip

    7. Pokaz, ze jezeli $|X|>1$, to $(X, L)$ nie jest grupa, gdzie dla $a,b\in X$ mamy $aLb=a$.
    \medskip

    8. Pokaz, ze jezeli $X$ jest niepusty, to\\
        \indent a. $(\Po X, \cup)$ nie jest grupa\\
        \indent b. $(\Po X, \cap )$ nie jest grupa
    \medskip

    9. Pokaz, ze grupa $S_X$ jest abelowa wtw $|X|<3$
    \medskip

    10. Sprawdz, czy ponizsza operacja $\star$ na podanym zbiorze $A$ jest laczna, przemienna i czy ma element neutralny. Sprawdz takze, czy $(A, \star)$ jest grupa\\
        \indent a. $A=\Q\setminus\{0\}$; $a\star b=\frac ab$\\
        \indent b. $A=\R$; $x\star y=x+y+2$\\
        \indent c. $A=\N_+$; $m\star n =NWD(m,n)$\\
        \indent d. $A=\N_+$; $m\star n=NWW(m,n)$\\
        \indent e. $A$ jest plaszczyzna; $P\star Q$ jest srodkowym punktem interwalu z krancami $P$ i $Q$\\
        \indent f. $A$ jest plaszczyzna; $P\star Q$ jest obrazem punktu $P$ przez odbicie wzgledem punktu $Q$

\end{document}