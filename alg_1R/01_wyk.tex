Listy zadan i cw po angielsku, wyklad po polsku + terminologia ang; kolokwia po angielsku

konsultacje: pon 11-13 (sala 502)

kolokwia w trakcie konwersatorium

\section{Wstep}
\subsection{Dzialania}

\begin{multicols}{2}
    
    {\color{def}Dzialanie} \ang{operation}\\
        \point np. dodawanie, mnozenie\\
        \point skladanie funkcji\\
    Formalnie, dzialaniem w dowolnym zbiorze $X$ to dowolna funkcja $\star:X\times X\to X$. Dla dowolnych $a,b\in X$ zamiast pisac $\star((a,b))$, uzywamy $a\star b$.
    \medskip

    {\color{acc}Przyklady:}\\
    \point na dowolnym z $\N,\Z, \R, \C, \Q$ mamy dodawanie ($+$) i mnozenie ($\cdot$)\\
    \point na $\Z, \Q,\R, \N$ mamy $\leq$ ktory daje dzialania: 
    $$a\lor b:=\min{a, b}$$
    $$a\land b:=\max{a,b}$$
    \point np na $\R$ mozemy zdefiniowac $a\star b:= a+b^2$\\
    \point niech $X$ bedzie zbiorem, a $X^X$ bedzie zbiorem wszystkich funkcji $X\to X$, wtedy skladanie funckji jest dzialaniem okreslonym w $X^X$:
    $$f\circ g\in X^X$$ 
    {\color{cyan}MOZNA DOJEBAC GRAFIK KOMUTUJACY}\\
    \point $X$ - zbior i niech $\Po X$ to zbior wszystkich podzbiorow $X$, wtedy na $\Po X$ mamy dzialanie sumy \ang{union} i przekroju \ang{intersection}\\
    \point niech $a, b\in X$, wtedy mamy rzuty na osie:
    $$aLb:=a$$
    $$aPb:=b$$
    \point na zbiorze $\R\cup\{\infty\}$ deifniujemy $(\forall\;a\in \R\cup\{\infty\})\;a+\infty=\infty=\infty+a$ oraz $(\forall\;a, b\in \R)\;a+b=a+_\R b$ (dodawanie w $\R$)
    \medskip

    Prosty opis dzialan - niech $\star$ bedzie dzialaniem okreslonym w $A=\{a_1,..., a_n\}$, to mozemy dojebac tabelke:

    \begin{center}
    \begin{tabular}{ c | c | c | c | c |}
        $\star$ & $a_1$ & $a_2$ & ... & $a_n$\\

        \hline

        $a_1$ & $a_1\star a_1$ & $a_1\star a_2$ & .. & $a_1\star a_n$\\
        
        \hline

        $a_2$ & $a_2\star a_1$ & $a_2\star a_2$ & .. & $a_2\star a_n$\\

        \hline

        ... & ... & ... & ... & ... \\
        
        \hline

        $a_n$ & $a_n\star a_1$ & $a_n\star a_2$ & .. & $a_n\star a_n$\\

        \hline
        
    \end{tabular}
    \end{center}

    Sensowne dzialania to dla nas:\\
        \point dzialania {\color{acc}laczne}. Moznaby tutaj napisac literkowa definicje, ale mi sie nie chce\\
        \point istnieje {\color{acc}element neutralny}\\
        \point czasem istnieje {\color{acc}element odwrotny} do danego $x$
    \medskip

    \podz{sep}
    \medskip

    Niech $S_X\subseteq X^X$ oznacza zbior wszystkich bijekcji $X\to X$
    \medskip

    {\color{def}Grupa przeksztalcen} \ang{transformation group} - niepusty podzbior $G\subseteq S_X$, ktory jest:\\
        \point jest zamkniety na laczenie funkcji\\
        \point $(\forall\;f\in G)\;f^{-1}\in G$\\
    \emph{Pojecie to wprowadzil Galois ok 1830, gdzie $X$ byl zbiorem pierwiastkow pewnego wielomianu.}
    \bigskip

    Pare $(G, \star)$ nazywamy {\color{def}grupa} \and{group}, gdy $\star$ jest dzialaniem okreslonym w $G$ oraz zachodzi\\
    \point $(\forall\;a,b,c\in G)\; a\star(b\star c)=(a\star b)\star c$, czyli dzialanie jest {\color{acc}laczne} \ang{associative}\\
    \point $\star$ ma element neutralny \ang{neutral element}\\
    \point $\star$ ma element odwrotny \ang{inverse element} dla kazdego elementu $G$

\end{multicols}

\subsection{Grupy}

