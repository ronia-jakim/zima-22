\section{Teoria grup}

\subsection{Grupy, pierscienie, ciala}
\begin{multicols}{2}

    {\color{def}Dzialanie} \ang{operation} na zbiorze $X$:
    $$\Phi: X\times X\to X,$$
    zwykle zapisywane jako $xy$, $x\cdot y$, $x+y$.\smallskip

    {\color{acc}Przyklady:}\\
    \point na dowolnym z $\N,\Z, \R, \C, \Q$ mamy dodawanie ($+$) i mnozenie ($\cdot$)\\
    \point na $\Z, \Q,\R, \N$ mamy $\leq$ ktory daje dzialania: 
    $$a\lor b:=\min{a, b}$$
    $$a\land b:=\max{a,b}$$
    \point np na $\R$ mozemy zdefiniowac $a\star b:= a+b^2$\\
    \point niech $X$ bedzie zbiorem, a $X^X$ bedzie zbiorem wszystkich funkcji $X\to X$, wtedy skladanie funckji jest dzialaniem okreslonym w $X^X$:
    $$f\circ g\in X^X$$ 
    {\color{cyan}MOZNA DOJEBAC GRAFIK KOMUTUJACY}\\
    \point $X$ - zbior i niech $\Po X$ to zbior wszystkich podzbiorow $X$, wtedy na $\Po X$ mamy dzialanie sumy \ang{union} i przekroju \ang{intersection}\\
    \point niech $a, b\in X$, wtedy mamy rzuty na osie:
    $$aLb:=a$$
    $$aPb:=b$$
    \point na zbiorze $\R\cup\{\infty\}$ deifniujemy $(\forall\;a\in \R\cup\{\infty\})\;a+\infty=\infty=\infty+a$ oraz $(\forall\;a, b\in \R)\;a+b=a+_\R b$ (dodawanie w $\R$)
    \medskip

    Prosty opis dzialan - niech $\star$ bedzie dzialaniem okreslonym w $A=\{a_1,..., a_n\}$, to mozemy dojebac tabelke:

    \begin{center}
    \begin{tabular}{ c | c | c | c | c |}
        $\star$ & $a_1$ & $a_2$ & ... & $a_n$\\

        \hline

        $a_1$ & $a_1\star a_1$ & $a_1\star a_2$ & .. & $a_1\star a_n$\\
        
        \hline

        $a_2$ & $a_2\star a_1$ & $a_2\star a_2$ & .. & $a_2\star a_n$\\

        \hline

        ... & ... & ... & ... & ... \\
        
        \hline

        $a_n$ & $a_n\star a_1$ & $a_n\star a_2$ & .. & $a_n\star a_n$\\

        \hline
        
    \end{tabular}
    \end{center}

    {\color{def}Element neutralny} \ang{neutral element} - takie $e$, ze dla kazdego $x\in X$ $ex=xe=x$. Dzialanie ma co najwyzej jeden element neutralny.\smallskip
    
    {\color{def}Element odwrotny} \ang{inverse element} do $x$ to takie $y$, ze $xy=yx=e$. Jesli dzialanie jest laczne \ang{associative}, to ma co najwyzej jeden element odwrotny do danego $x$.\medskip

    \podz{sep}\medskip

    {\color{def}Homomorfizm} algebry $\rodz X=(X,\{ \cdot\})$ na algebre $\rodz Y=(Y,\{\circ\})$ nazywamy przeksztalcenie $f:X\to Y$ spelniajace dla kazdego $a, b\in X$
    $$f(a\cdot b)=f(a)\circ f(b).$$
    \indent {\color{acc}$\bullet$ monomorfizm} - $f$ jest 1-1\smallskip\\
    \indent {\color{acc}$\bullet$ epimorfizm} - $f$ jest "na"\smallskip\\
    \indent {\color{acc}$\bullet$ izomorfizm} - $f$ jest 1-1 i "na"\smallskip\\
    \indent {\color{acc}$\bullet$ endomorfizm} - kiedy $\rodz Y=\rodz X$\smallskip\\
    \indent {\color{acc}$\bullet$ automorfizm} - enodmorfizm bedacy izomorfizmem\medskip

    \podz{sep}\medskip

    {\color{def}Polgrupa} to niepusty zbior z dzialaniem lacznym.\medskip
    
    {\color{def}GRUPA} \ang{group} to niepusty zbior z lacznym dzialaniem i elementem neutralnym (zwanym {\color{acc}jednoscia grupy}) oraz elementami odwrotnymi dla kazdego elementu.\smallskip\\
    \indent {\color{acc}$\hookrightarrow$ grupa abelowa} (przemienna) \ang{commutative group} - grupa z dzialaniem przemiennym\smallskip\\
    Zbior $G$ z dzialaniem $\cdot$ jest grupa, jesli:\smallskip\\
    \indent 1. $(\forall\;a,b,c\in G)\;(ab)c=a(bc)$\\
    \indent 2. $(\exists\;e\in G)(\forall\;a\in G)\;ea=ae=e$\\
    \indent 3. $(\forall\;a\in G)(\exists\;b\in G)\;ab=ba=e$\\
    \indent *4. $(\forall\;a,b\in G)\;ab=ba$ w grupie \emph{abelowej}\medskip

    {\color{def}Grupa przeksztalcen} \ang{transformation group} - niepusty podzbior $G\subseteq S_X$, ktory jest:\\
        \point jest zamkniety na laczenie funkcji\\
        \point $(\forall\;f\in G)\;f^{-1}\in G$\\
    \emph{Pojecie to wprowadzil Galois ok 1830, gdzie $X$ byl zbiorem pierwiastkow pewnego wielomianu.}
    \bigskip

    {\color{cyan}WYPADALOBY POKAZAC, ZE $(S_X, \circ)$ jest przemienna iff $|X|\leq2$}

    {\color{def}PIERSCIEN} to niepusty zbior $X$ z dwoma dzialaniami ($\cdot,\;+$, mnozenie i dodawanie) taki, ze:\smallskip\\
    \indent 1. zbior $X$ z $+$ jest grupa abelowa\smallskip\\
    \indent 2. $\cdot$ jest laczne\smallskip\\
    \indent 3. $(\forall\;x,y,z\in X)\;x\cdot(y+z)=x\cdot y+x\cdot z\;\land\;(x+y)\cdot z=x\cdot z+y\cdot z$\smallskip\\
    Kolejne dzikie nazwy $\star$:\smallskip\\
    \indent {\color{acc}$\star$ pierscien przemienny} - jesli mnozenia jest przemienne\smallskip\\
    \indent {\color{acc}$\star$ pierscien z jednoscia} - dla mnozenia istnieje element neutralny\medskip

    {\color{def}CIALO} to pierscien przemienny, ktory dla kazdego elementu $\neq 0$ ma element odwrotny

\end{multicols}\bigskip

\podz{sep2}\bigskip

\subsection{Wlasnosci}
\begin{multicols}{2}

    Niech $G$ bedzie grupa, a $e$ jej elementem neutralnym. Wowczas:\smallskip\\
    \point $a,b\in G\implies (ab)^{-1}=b^{-1}a^{-1}$\smallskip\\
    \point $a\in G$ i $n=1, ..., n$ $a^{-n} = (a^n)^{-1} =^* (a^{-1})^n$\smallskip\\
    \point dla $m,n\in\Z$ i $a\in G$ mamy $a^{mn}=^* (a^m)^n$\smallskip\\
    \point dla $G$ grupy abelowej i $n\in\Z$ $(ab)^n=^*a^nb^n$\smallskip\\
    $^*$ trzeba udowodnic, ale mi sie nie chce\medskip

    $H\subseteq G$ jest {\color{def}podgrupa} $G$, jesli jest grupa ze wzgledu na te same dzialania, czyli wystarczy, ze 
    $$(\forall\;a,b\in H)\; ab^{-1}\in H.$$

    \podz{sep}\medskip

    Jelsi $a\in G$ i istnieja $n\in\N$, $n\geq 1$, takie, ze $a^n=e$, to mowimy ze $n$ jest {\color{def}rzedem elementu} $a$ ($n=o(a)$). Jesli takie $n$ nie istnieja, to $a$ ma {\color{acc}rzad nieskonczony} ($o(a)=\infty$).\smallskip\\
    {\color{acc}\point} {\color{def}grupa torsyjna} - wszystkie elementy maja rzad skonczony\smallskip\\
    {\color{acc}\point} {\color{def}grupa beztorsyjna} - wszystkie elementy maja rzad nieskonczony\smallskip\\
    \emph{Jesli $o(a) = n$ oraz $a^N=e$ to $n|N$, fajny dowodzik, ale leniem jestem}\medskip

    {\color{def}Grupa cykliczna} to grupa zlozona z wszystkich poteg danego elementu $a$, natomiast $a$ jest nazywane {\color{acc}generatorem} tej grupy

\end{multicols}

\subsection{Permutacje :>}
\begin{multicols}{2}
    
    {\color{def}$n$-ta grupa symetryczna} [$S_n$] - grupa wszystkich permutacji zbioru $X_n = \{1, ..., n\}$. $|S_n|=n!$\medskip

    Jesli $P\in S_n$ i dla $i=1,...,n$ $P(i)=a_i$, to piszemy
    $$
        \begin{pmatrix}
            1   && 2   && ... && n  \\
            a_1 && a_2 && ... && a_n
        \end{pmatrix}
    $$

    Mnozenie permutacji:
    $$
        \begin{pmatrix}
            a_1 && a_2 && ... && a_n \\
            b_1 && b_2 && ... && b_n
        \end{pmatrix}
        \begin{pmatrix}
            c_1 && c_2 && ... && c_n \\
            a_1 && a_2 && ... && a_n
        \end{pmatrix} =
    $$
    $$=
    \begin{pmatrix}
        c_1 && c_2 && ... && c_n \\
        b_1 && b_2 && ... && b_n
    \end{pmatrix}
    $$

    {\color{def}Zbior elementow niezmienniczych} (fixpunktow) permutacji $P$ to zbior $F(P)=\{k\in X_n\;:\;P(k)=k\}$. Jego dopelnienie oznaczamy $M(P)=S_n\setminus F(P)$.\medskip

    \podz{sep}\medskip

    {\color{def}Cykl $k$-elementowy} $C$ to permutacja taka, ze $C(a_1)=a_2$, $C(a_2)=a_3$, ..., $C(a_n)=a_1$. Cykl $2$-elementowy to {\color{acc}transpozycja}. Cykle zapisujemy
    $$\begin{pmatrix}a_1, a_2,..., a_n\end{pmatrix}$$

    Kazda permutacja jest iloczynem transpozycji.\medskip

    \podz{sep}\medskip

    {\color{def}Permutacje parzyste} - iloczyn 
    $$\prod _{i<j}(a_j-a_i)$$ 
    jest dodatni (gorny row to kolejne liczby naturalne, dolny to wyrazy). Pozostale permutacje sa {\color{def}nieparzyste}.\medskip

    {\color{def}Znak permutacji} jest $+1$ gdzy permutacja jest parzysta i $-1$ wpp. Alternatywnie mozna zapisac (gorny row to $b_k$, a dolny to $c_k$)
    $$sgn\;P=\prod_{i<j}{b_j-b_i\over c_j-c_i}$$

    Dla dwoch dowolnych permutacji $P_1,P_2$ mamy
    $$sgn\;P_1P_2=sgn\;P_1\cdot sgn\;P_2$$
    $$sgn\;P_1^{-1}=sng\;P_1.$$

    {\color{def}$n$-ta grupa alternujaca} [$A_n$] - podgrupa $S_n$ zlozona ze wszystkich parzystych permutacji.
 
\end{multicols}\bigskip

\podz{sep2}\bigskip

\subsection{Grupy ilorazowe B)}
\begin{multicols}{2}
    
    {\color{def}Prawostronna warstwa} grupy $G$ wzgledem jej podgrupy $H$ wyznaczona przez $g\in G$ to zbior
    $$gH=\{gh\;:\;h\in H\},$$
    natomiast {\color{acc}lewostronna} warstwa to zbior
    $$ Hg = \{hg\;:\;h\in H\}. $$
    Dla grup abelowych sa one rowne.\medskip

    Dwa elementy $g_1,g_2\in G$ wyznaczaja te sama warstwe prawostronna wzgledem $H$, gdy $g_1^{-1}g_2\in H$, a te sama warstwe lewostronna, gdy $g_1g_2^{-1}\in H$.\medskip

    \podz{sep}\medskip

    {\color{def}Rzad grupy} skonczonej $G$ to ilosc jej elementow.\medskip

    {\color{def}Indeks} $[G : H]$ podgrupy $H$ w grupie $G$ to ilosc warstw w grupie $G$ wzgledem $H$. Dla skonczonych grup mamy:\smallskip\\
    \point $g\in G$ $o(g) | |G|$,\smallskip\\
    \point rzad i indeks kazdej podgrupy sa dzielnikami rzedu grupy,\smallskip\\
    \point jesli rzad jest liczba pierwsza, to grupa jest cykliczna\medskip

    {\color{def}Twierdzenie Lagrange'a} - dla skonczonych $G > H$:
    $$|G|=[G:H]\cdot |H|.$$

    \podz{sep}\medskip

    Podgrupa $H$ jest {\color{def}dzielnikiem normalnym} grupy $G$ [$H\triangleleft G$] jesli $(\forall\;g\in G)\;gH=Hg$. Wystarczy, ze $(\forall\;g\in G)(\forall\;h\in H)\;ghg^{-1}\in H$.\medskip

    Niech $f:G_1\to G_2$ bedzie {\color{acc}homomorfizmem}, a $e_1, e_2$ beda elementami neutralnymi grup odpowiednio $G_1,G_2$. Wtedy $f(e_1)=e_2$ oraz $f(g)^{-1}=f(g^{-1})$.\medskip

    Obraz homomorfizmu $f:G_1\to G_2$ jest {\color{acc}podgrupa grupy $G_2$} [$Im\;f < G_2$], natomiast jadro $f$ jest {\color{acc}dzielnikiem normalnym $G_1$} [$Ker\;f \dzielnat G_1$].\medskip

    \podz{sep}\medskip

    {\color{def}Grupa ilorazowa} to zbior wszyystkich warstw $H/G$, gdzie $H\dzielnat G$, z dzialaniem
    $$(g_1H)(g_2H)=(g_1g_2)H.$$
    Odwzorowanie 
    $$\phi:G\to H$$
    $$\phi(g)=gH$$
    jest {\color{acc}epimorfizmem} (czesto nazywane {\color{acc}kanonicznym homeomorfizmem $G$ na $H$}).\medskip

    {\color{cyan}[!!!]}{\color{def}Zasadnicze twierdzenie o homeomorfizmach dla grup} - jesli $f:G\to G_1$ jest epimorfizmem oraz $Ker\;f=H$, natomiast $\phi:G\to G/H$ jest dzialaniem jak wyzej, to istnieje tylko jeden izomorfizm $\psi:G/H\to G_1$ taki, ze $f=\psi\circ \phi$\medskip

    \podz{sep}\medskip

    Jezeli $\emptyset\neq A\subseteq G$ oraz $G(A)<G$ to przekroj wszystkich podgrup $G$ zawierajacych $A$, a $A\subseteq G_1<G$, to $G(A)<G_1$.\medskip

    Jezeli $K\dzielnat G$ i $H\dzielnat G$, to najmniejsza podgrupa $G$ zawierajaca $H$ i $K$ pokrywa sie ze zbiorem
    $$KH:=\{kh\;:\;k\in K,\;h\in H\}$$

    {\color{def}Pierwsze twierdzenie o izomorfizmach} - jezeli $K\dzielnat G$ i $H\dzielnat G$, to \smallskip\\
    \point $K<KH=HK<G$\smallskip\\
    \point $H\cap K\dzielnat H$ i $K\dzielnat KH$\smallskip\\
    \point $\phi:hK\to h(K\cap H)$ indukuje izomorfizm 
    $$HK/K\sim H/(H\cap K)$$

    {\color{def}Drugie twierdzenie o izomorfizmach} - jezeli $K\dzielnat G$ i $K<H<G$ i oznaczymy $\overline H = H/K$ oraz $\overline G=G/K$, to wtedy:\smallskip\\
    \point $\overline H<\overline G$\smallskip\\
    \point $\overline H\dzielnat\overline G\iff H\dzielnat G$\medskip

    \podz{sep}\medskip

    {\color{def}Automorfizm wewnetrzny }grupy $G$ wyznaczony przez $g$: $\phi_g(x)=g^{-1}xg$.\\ 
    Jesli $G$ to grupa abelowa, to dla kazdego $g$ $\phi_g(x)=x$, a wiec ma ona jedynie identycznosc.\\
    Zbior wszystkich automorfizmow wewnetrznych grupy G oznaczamy $\color{acc}I(G)$ i tworzy on grupe ze skladaniem\medskip

    {\color{def}Centrum grupy} $G$ [$Z(G)$] to zbior $x\in G$ takich, ze dla dowolnego $y\in G$ $xy=yx$. Dla kazdego $G$ $Z(G)\dzielnat G$\medskip

    Grupa $I(G)$ jest izomorficzna z $G/Z(G)$.\medskip

    Jesli $M$ to dowolny podzbior grupy $G$, to dla kazdego $g$ takiego, ze $\phi_g\in I(G)$ zbiorem {\color{def}sprzezony} do $M$ nazywamy zbior
    $$M^g=\{\phi_g(x)\;:\;x\in M\}$$
    Jesli $M=\{x\}$, to $M^g$ zawiera elementy sprzezone z x.\smallskip\\
    {\color{def}Normalizator} zbioru $M$:
    $$N_G(M)=\{g\in G\;:\;M^g=M\}$$
    {\color{def}Centralizator} zbioru $M$:
    $$C_G(M)=\{g\in G\;:\;mg=gm,\;m\in M\}$$

    Twierdzonka:\smallskip\\
    \point $(\forall\; M\subseteq G)\;C_G(M)<N_G(M)$ ($|M| = 1\implies C_G(M)=N_G(M)$)\smallskip\\
    \point $Z(G)=C_G(G)$\smallskip\\
    \point dla $M\subseteq G$ ilosc zbiorow $M^g$ jest rowna $[G\;:\;N_G(M)]$.\medskip

    Aby klasa elementow sprzezonych z $x\in G$ byla jednoelementowa wystarczy, zeby $x\in Z(G)$\medskip
    
    Jesli $G$ jest skonczona, to ilosc elementow sprzezonych z zadanym $x$ jest dzielnikiem $|G|$.\medskip

    {\color{acc}$p$-grupa} to grupa, w ktorej wszystkie elementy maja rzad $p$, gdzie $p$ jest liczba pierwsza. Jesli $|G|=p^n$ to $G$ jest $p$-grupa.\smallskip\\
    Skonczone $p$=grupy maja {\color{acc}nietrywialne centrum}.\smallskip\\
    Jesli $|G|=p^2$, to $G$ jest grupa abelowa.

\end{multicols}\bigskip

\podz{sep2}\bigskip

\subsection{Produkty grup}
\begin{multicols*}{2}
    W zbiorze $A\times B=\{(a, b)\;:\;a\in A,b\in B\}$, gdzie $A,B$ sa grupami, okreslmy
    $$(a,b)\cdot(c,d)=(ac, bd)$$
\end{multicols*}