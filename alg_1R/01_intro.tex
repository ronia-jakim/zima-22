\section{Powtorka z poprzedniego roku}

\subsection{Dzialania}
\begin{multicols*}{2}

    Dzialanie na zbiorze $X$:
    $$\Phi: X\times X\to X,$$
    zwykle zapisywane jako $xy$, $x\cdot y$, $x+y$.\smallskip

    {\color{def}Element neutralny} - takie $e$, ze dla kazdego $x\in X$ $ex=xe=x$. Dzialanie ma co najwyzej jeden element neutralny.\smallskip
    
    {\color{def}Element odwrotny} do $x$ to takie $y$, ze $xy=yx=e$. Jesli dzialanie jest laczne, to ma co najwyzej jeden element odwrotny do danego $x$.\medskip

    \podz{sep}\medskip

    {\color{def}Homomorfizm} algebry $\rodz X=(X,\{ \cdot\})$ na algebre $\rodz Y=(Y,\{\circ\})$ nazywamy przeksztalcenie $f:X\to Y$ spelniajace dla kazdego $a, b\in X$
    $$f(a\cdot b)=f(a)\circ f(b).$$
    \indent {\color{acc}$\bullet$ monomorfizm} - $f$ jest 1-1\smallskip\\
    \indent {\color{acc}$\bullet$ epimorfizm} - $f$ jest "na"\smallskip\\
    \indent {\color{acc}$\bullet$ izomorfizm} - $f$ jest 1-1 i "na"\smallskip\\
    \indent {\color{acc}$\bullet$ endomorfizm} - kiedy $\rodz Y=\rodz X$\smallskip\\
    \indent {\color{acc}$\bullet$ automorfizm} - enodmorfizm bedacy izomorfizmem

\end{multicols*}
Dzialanie na niepustym zbiorze $X$ to przeksztalcenie postaci
$$\phi:X\times X\to X,$$
co zwykle zapisujemy jako
$$x\cdot y\quad x+y\quad xy\quad x-y\quad x\star y$$
Jesli dzialanie jest laczne, to mamy
$$(x\cdot y)\cdot z = x\cdot(y\cdot z),$$
a jesli jest przemienne, to 
$$x\cdot y=y\cdot x$$

Element neutralny dzialania to $e$ takie, ze dla kazdego $x\in X$ mamy $ex=xe=x$. Nie kazde dzialanie ma element neutralny, ale jesli juz jest to jest tylko jeden.\bigskip

Element odwrotny do $x$ to element taki, ze $xy=yx=e$. Jesli dzialanie jest laczne, to moze miec co najwyzej jeden element odwrotny do danego elementu.

Niech $\rodz X=(X, \{\cdot\})$ i $\rodz Y = (Y, \{\circ\})$ beda jednodzialaniowymi algebrami, a funkcja 
$$f:\rodz X\to \rodz Y$$
niech spelnia dla kazdych $a, b\in X$
$$f(a\cdot b) = f(a)\circ f(b)$$
