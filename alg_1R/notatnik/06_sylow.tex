\section{TWIERDZENIE SYLOWA}

\subsection{I twierdzenie Sylowa}

\pdef

\textbf{\color{def}I twierdzenie Sylowa}:
\smallskip

Jeżeli $p$ jest liczbą pierwszą, a $G$ jest grupą skończoną rzędu $|G|=p^km$ dla $k\geq 1$ i $p\nmid m$, to istnieje podgrupa $H\leq G$ mająca $p^k$ elementów. Taka grupa nazywa się {\color{acc}podgrupą Sylowa}.

\kdef

{\color{def}DOWÓD:}
\medskip

Niech $G$ będzie grupą rzędu $|G|=p^km$ taką jak w twierdzeniu. Niech $X$ będzie zbiorem wszystkich $p^k$ elementowych podzbiorów grupy $G$. Możemy teraz określić działanie $\psi$ grupy $G$ na zbiór $X$. Jeśli $H=\{h_1,...,h_{p^k}\}\in X$, a $g\in G$, to
$$\psi(H)=\{gh_1,gh_2,...,gh_{p^k}\}.$$

Wiemy, że
\begin{align*}
    |H|&={p^km\choose p^k}={(p^km)!\over (p^km-p^k)!(p^k)!}=\\
    &={p^km(p^km-1)...(p^km-p^k+1)\over (p^k)!}=\prod\limits_{i=1}^{p^k}{p^km-i+1}
\end{align*}

\subsection{Twierdzenie Cauchy'ego}

{\color{def}Twierdzenie Cauchy'ego:}

Jeżeli liczba pierwsza $p$ dzieli rząd grupy $G$, to $G$ zawiera element rzędu $p$.

\subsection{p-grupy Sylowa}

\subsection{Twierdzenia Sylowa}