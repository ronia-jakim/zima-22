\section{DEFINICJA GRUPY}

\subsection{Grupy}

\pdef

{\color{def}DZIAŁANIE} w zbiorze $A$ to funkcja
$$\star:A\times A\to A$$
$$(x, y)\mapsto x\star y$$

\kdef

Algebrą nazywamy niepusty zbiór $A$ ze wszystkimi działaniami na nim określonymi, to znaczy zestawienie $(A, f_1,...,f_k)$. Mówimy, że dwie algebry $A=(A, f_1,...,f_k)$ i $B=(B, g_1,...,g_k)$ są {\color{acc}podobne}, jeśli dla każdego $i\leq k$ arność (czyli liczba argumentów) $f_i$ jest równa arności $g_i$, czyli liczbie $l_i$.

Dwie algebry są {\color{def}izomorficzne}, jeżeli istnieje $F:A\bijekcja B$ takie, że
$$(\forall\;i\leq k)(\forall\;a_1,...,a_{l_i}\in A)\;F(f_i(a_1,...,a_{l_i}))=g_i(F(a_1),...,F(a_{l_i}))$$

Działanie jest {\color{def}łączne} \ang{assosiative}, jeżeli
$$(\forall\;a,b,c\in A)\;a(bc)=(ab)c$$


\subsection{Przykłady grup}

\subsection{Podgrupy}

\subsection{Grupa cykliczna}