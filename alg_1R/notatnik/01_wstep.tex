\section{DEFINICJA GRUPY}

\subsection{Działania, struktury}

\pdef

{\color{def}DZIAŁANIE} w zbiorze $A$ to funkcja
$$\star:A\times A\to A$$
$$(x, y)\mapsto x\star y$$
\smallskip

\kdef
\medskip

Zwykle rozważamy działania binarne, ale działaniem może być funkcją z $A^n$ w $A$ (jak na przykład branie średniej arytmetycznej 3 liczb). Zdarza się też, że mamy działanie unarne, takie jak na przykład branie liczby przeciwnej do $m\in\Z$.
\medskip

Działanie jest {\color{def}łączne} \ang{assosiative}, jeżeli
$$(\forall\;a,b,c\in A)\;a(bc)=(ab)c$$
a {\color{def}przemienne} \ang{commutative}, gdy
$$(\forall\;a,b\in A)\;ab=ba$$
Tutaj warto zaznaczyć, że jeśli działanie jest łączne dla $3$ argumentów, to jest również łączne dla $k$ argumentów. Dowód przez indukcję jest trywialny.
\medskip

\podz{sep}
\bigskip

Algebrą nazywamy niepusty zbiór $A$ ze wszystkimi działaniami na nim określonymi, to znaczy zestawienie $(A, f_1,...,f_k)$. Zbiór $A$ nazywamy {\color{acc}uniwersum} lub dziedziną struktury. Mówimy, że dwie algebry $A=(A, f_1,...,f_k)$ i $B=(B, g_1,...,g_k)$ są {\color{acc}podobne}, jeśli dla każdego $i\leq k$ arność (czyli liczba argumentów) $f_i$ jest równa arności $g_i$, czyli liczbie $l_i$.

Dwie algebry są {\color{def}izomorficzne}, jeżeli istnieje $F:A\bijekcja B$ takie, że
$$(\forall\;i\leq k)(\forall\;a_1,...,a_{l_i}\in A)\;F(f_i(a_1,...,a_{l_i}))=g_i(F(a_1),...,F(a_{l_i}))$$
Struktury izomorficzne oznaczamy $\color{acc}A\izomorfizm B$. Warto zauważyć, że $\izomorfizm$ ma \emph{własności relacji równoważności}, to znaczy jest zwrotny, symetryczny i przechodni.
\medskip

\pdef
\textbf{\color{def}SŁOWNICZEK}:
\smallskip

%\begin{multicols}{2}
    \point epi-morfizm -> "na"

    \point mono-morfizm -> 1-1

    \point izo-morfizm -> bijeckja

    \point endo-morfizm -> w samego siebie

    \point auto-morfizm -> endomorfizm który jest bijekcją.
%\end{multicols}
\smallskip

\kdef
\medskip

$B=(B, g_1,...,g_k)$ jest {\color{def}podalgebrą} $A=(A, f_1,...,f_k)$, jeżeli

\point $B\subseteq A$

\point $(\forall\;i\leq k)\;g_i=f_i\obet_ B$
\medskip

Niech $B\subseteq A$, wtedy $B$ jest uniwersum podstruktury struktury $A$ z naturalnymi działaniami $\iff$ $B$ jest zamknięty na działania $f_1,...,f_k$. W takim przypadku $B$ traktujemy jako strukturę będącą podstrukturą struktury $A$.
\medskip

Jeżeli $F:A\to B$ jest homomorfizmem struktur, to $Im(F)$ jest podstrukturą $B$.
\bigskip

{\color{acc}Złożenie homomorfizmów jest homomorfizmem} a odwzorowanie odwrotne do izomorfizmu jest izomorfizmem.
\medskip

{\color{def}DOWÓD:}
\smallskip

Niech $f:(X, \cdot)\to(Y, \circ)$ i $g:(Y, \circ)\to(Z\star)$ są homomorfizmami, a $h(x)=g(f(x))$ jest ich złożeniem, to dla dowolnego $a,b\in X$ mamy
\begin{align*}
    h(a\cdot b)=g(f(a\cdot b))=g(f(a)\circ f(b))=g(f(a))\star g(f(b))=h(a)\star h(b)
\end{align*}
więc $h$ spełnia warunki homomorfizmu. Jeżeli $f, g$ były epi, mono, ... morfizmami, to zachowanie odpowiednich własności wynika z własności składania funkcji różnowartościowych, na czy bijekcji.
\smallskip

Niech $\phi:(X, \cdot)\bijekcja (Y, \circ)$ będzie izomorfizmem. Chcemy pokazać, że $\phi^{-1}$ jest homomorfizmem. Weźmy $a, b\in Y$ i $c, d\in X$ takie, że $\phi(c)=a$ oraz $\phi(d)=b$. Wtedy
$$ab=\phi(c)\phi(d)=\phi(cd),$$
czyli
$$\phi^{-1}(ab)=cd,$$
a ponieważ $\phi^{-1}(a)=c$ i $\phi^{-1}(b)=d$, to mamy
$$\phi^{-1}(ab)=cd=\phi^{-1}(a)\phi^{-1}(b).$$
Natomiast fakt, że $\phi^{-1}$ jest bijekcją wynika z tego, że $\phi$ jest bijekcją.

\subsection{Grupy}

\pdef

{\color{def}Grupa} to struktura $G=(G, \cdot)$ taka, że:

\point $\cdot$ jest działaniem łącznym

\point istnieje element neutralny $e\in G$ dla działania $\cdot$

\point dla każdego $g\in G$ istnieje element odwrotny $g^{-1}\in G$ takie, że $gg^{-1}=g^{-1}g=e$
\smallskip

\kdef

Tutaj warto zaznaczyć, że \emph{element neutralny jest jedyny}. W przeciwnym przypadku istniałyby co najmniej dwa elementy neutralne $e_1,e_2$, ale wtedy
$$e_1=e_1\cdot e_2=e_2.$$

Z łączności działania na grupie wynika, że dla każdego $g\in G$ \emph{istnieje co najwyżej jeden element odwrotny}. Gdyby $x,y$ były dwoma elementami odwrotnymi do $g$, to
$$x=xe=x(gy)=(xg)y=ey=y,$$
co prowadzi do sprzeczności.

\subsection{Podgrupy}

\subsection{Grupa cykliczna}