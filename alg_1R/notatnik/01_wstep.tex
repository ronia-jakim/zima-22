\section{DEFINICJA GRUPY}

\subsection{Działania, struktury}

\pdef

{\color{def}DZIAŁANIE} w zbiorze $A$ to funkcja
$$\star:A\times A\to A$$
$$(x, y)\mapsto x\star y$$
\smallskip

\kdef
\medskip

Zwykle rozważamy działania binarne, ale działaniem może być funkcją z $A^n$ w $A$ (jak na przykład branie średniej arytmetycznej 3 liczb). Zdarza się też, że mamy działanie unarne, takie jak na przykład branie liczby przeciwnej do $m\in\Z$.
\medskip

Działanie jest {\color{def}łączne} \ang{assosiative}, jeżeli
$$(\forall\;a,b,c\in A)\;a(bc)=(ab)c$$
a {\color{def}przemienne} \ang{commutative}, gdy
$$(\forall\;a,b\in A)\;ab=ba$$
Tutaj warto zaznaczyć, że jeśli działanie jest łączne dla $3$ argumentów, to jest również łączne dla $k$ argumentów. Dowód przez indukcję jest trywialny.
\medskip

\podz{sep}
\bigskip

Algebrą nazywamy niepusty zbiór $A$ ze wszystkimi działaniami na nim określonymi, to znaczy zestawienie $(A, f_1,...,f_k)$. Zbiór $A$ nazywamy {\color{acc}uniwersum} lub dziedziną struktury. Mówimy, że dwie algebry $A=(A, f_1,...,f_k)$ i $B=(B, g_1,...,g_k)$ są {\color{acc}podobne}, jeśli dla każdego $i\leq k$ arność (czyli liczba argumentów) $f_i$ jest równa arności $g_i$, czyli liczbie $l_i$.

Dwie algebry są {\color{def}izomorficzne}, jeżeli istnieje $F:A\bijekcja B$ takie, że
$$(\forall\;i\leq k)(\forall\;a_1,...,a_{l_i}\in A)\;F(f_i(a_1,...,a_{l_i}))=g_i(F(a_1),...,F(a_{l_i}))$$
Struktury izomorficzne oznaczamy $\color{acc}A\izomorfizm B$. Warto zauważyć, że $\izomorfizm$ ma \emph{własności relacji równoważności}, to znaczy jest zwrotny, symetryczny i przechodni.
\medskip

$B=(B, g_1,...,g_k)$ jest {\color{def}podalgebrą} $A=(A, f_1,...,f_k)$, jeżeli

\point $B\subseteq A$

\point $(\forall\;i\leq k)\;g_i=f_i\obet_ B$
\medskip

Niech $B\subseteq A$, wtedy $B$ jest uniwersum podstruktury struktury $A$ z naturalnymi działaniami $\iff$ $B$ jest zamknięty na działania $f_1,...,f_k$. W takim przypadku $B$ traktujemy jako strukturę będącą podstrukturą struktury $A$.
\medskip

\subsection{Grupy}

{\color{def}Monoid} to zbiór $X$ z działaniem łącznym oraz elementem neutralnym. Liczby naturalne z dodawaniem są przykładem monoidu.
\medskip

\pdef

{\color{def}Grupa} to struktura $G=(G, \cdot)$ taka, że:

\point $\cdot$ jest działaniem łącznym

\point istnieje element neutralny $e\in G$ dla działania $\cdot$

\point dla każdego $g\in G$ istnieje element odwrotny $g^{-1}\in G$ takie, że $gg^{-1}=g^{-1}g=e$
\smallskip

\kdef
Grupa trywialna to zbiór z działaniem zawierający jedynie jego element neutralny: $\{e\}$.
\medskip

Tutaj warto zaznaczyć, że \emph{element neutralny jest jedyny}. W przeciwnym przypadku istniałyby co najmniej dwa elementy neutralne $e_1,e_2$, ale wtedy
$$e_1=e_1\cdot e_2=e_2.$$

Z łączności działania na grupie wynika, że dla każdego $g\in G$ \emph{istnieje co najwyżej jeden element odwrotny}. Gdyby $x,y$ były dwoma elementami odwrotnymi do $g$, to
$$x=xe=x(gy)=(xg)y=ey=y,$$
co prowadzi do sprzeczności.
\medskip

Jeśli działanie grupy jest przemienne, to nazywamy ją {\color{def}grupą abelową} lub przemienną. Tak jak działanie w grupie oznaczamy zwykle przez $\cdot$, tak w grupie abelowej, aby podkreślić jego przemienność, działanie jest zwykle oznaczane przez $+$. Podobnie, potęgowanie w grupie abelowej nie oznaczamy $x^n$, a raczej $nx$.
\bigskip

\podz{sep}
\bigskip

Działanie w grupie możemy opisać za pomocą {\color{acc}tabelki}
\begin{center}
    \begin{tabular}{c | c | c | c | c }
        $\star$ & $a_1$ & $a_2$ &...& $a_k$\\

        \hline

        $a_1$ & $a_1a_1$ & $a_1a_2$ & ... & $a_1a_k$\\

        $a_2$ & $a_2a_1$ & $a_2a_2$ & ... & $a_2a_k$\\

        ... & \\

        $a_k$ & $a_ka_1$ & $a_ka_2$ & ... & $a_ka_k$
    \end{tabular}
\end{center}
\smallskip

Jeżeli działanie jest przemienne, to oczywiście taka tabelka będzie symetryczna.
\bigskip

Grupę przemienną $G=\{e,a,b,c\}$ nazywamy {\color{def}grupą czwórkową Kleina} [$K_4$]. Grupa izometrii własnych $n$-kąta foremnego [$\color{acc}D_n$] jest nazywana grupą {\color{def}dihedralną} i nie jest ona grupą abelową. Jej podgrupą jest na przykład {\color{acc}grupa obrotów własnych} $n$-kąta foremnego [$O_n$].

\podz{sep}
\bigskip

{\color{def}Pierścieniem} nazywamy zbiór $X$ z dwoma działaniami, $+$ i $\cdot$, z których $\cdot$ jest łączne, a $+$ jest przemienne. W dodatku, $\cdot$ jest rozdzielne względem $+$:
$$(\forall\;x,y,z\in X)\;x\cdot(y+z)=x\cdot y+x\cdot z$$
Jeśli dodatkowo mnożenie w pierścieniu jest działaniem przemiennym, to taki pierścień nazywamy {\color{acc}przemienny}. Jeśli zaś istnieje element neutralny dla mnożenia, to jest on {\color{acc}pierścieniem z jednością}.
\medskip

Pierścienie $K$, dla których $K\setminus\{0\}$ jest grupą przemienną względem mnożenia nazywamy {\color{def}ciałami}. Najprostszym ciało są zbiory $\Q,\R,\C$ ze zwykłym dodawaniem i mnożeniem. Zbiór $\Q(\gimel)$ wszystkich liczb zespolonych postaci $a+bi$ dla wymiernych $a,b$ jest ciałem.
\bigskip

\podz{sep}
\bigskip

Niech $H\subseteq G$ dla pewnej grupy $G$. Mówimy, że $H$ jest {\color{def}podgrupą} grupy $G$ [$\color{acc}H\leq G$], jeżeli $H$ jest grupą względem działania z $G$ (ograniczonego do $H$). Dodatkowo, jeśli $H\neq G$ to mówimy, że $H$ jest {\color{acc}podgrupą właściwą}. Na przykład
$$(\Z, +)\leq (\Q, +)\leq(\R, +).$$

Przy sprawdzaniu, czy dany zbiór $H$ jest podgrupa $G$ wystarczy sprawdzić, czy $(\forall\;x,y\in H)\;xy^{-1}\in H$.
\medskip

Jeśli $a,b\in G$, to $\color{acc}(ab)^{-1}=b^{-1}a^{-1}$.
\medskip

{\color{def}DOWÓD:}
\smallskip

Chcemy sprawdzić, że $(b^{-1}a^{-1})ab=e$
$$(b^{-1}a^{-1})ab=b^{-1}(a^{-1}a)b=b^{-1}eb=b^{-1}b=e$$
a więc dostajemy to, czego się spodziewaliśmy. 
\kdowod
\bigskip

Zdefiniujemy $a^{-n}=(a^{-1})^n$ i nie trudno pokazać, że też $a^{-n}=(a^n)^{-1}$. Dalej mamy $a^{n+m}=a^na^m$, a dla grupy przemiennej zachodzi $(ab)^n=a^nb^n$.

\subsection{Grupa cykliczna}

{\color{def}Rząd grupy} to ilość jej elementów: $org(G)=|G|$. Dla każdego $g\in G$ definiujemy {\color{def}rząd elementu} $ord(g)=N$ jako najmniejszą liczbę naturalną taką, że $g^N=e$. Znając pojęcie grup cyklicznych (niżej) możemy też podać równoważną definicję: $ord(g)=|\langle g\rangle|$.

Jeśli $ord(g)=n$ i weźmiemy $N$ takie, że $g^N=e$, to mamy pewność, że $\color{acc}n|N$. Gdyby tak nie było, to mielibyśmy $N=kn+r$, $0<r<n$ i
$$g^N=g^{kn+r}=g^{kn}g^r=(g^n)^kg^r=e^kg^r=g^r\neq e.$$
W takim razie dla $g, g^2, ..., g^n$ są elementami parami różnymi i tworzą podgrupę grupy $G$.
\medskip

{\color{def}Grupa cykliczna} to grupa utworzona przez wzięcie wszystkich potęg $g\in G$: $H=\{g, g^1, ..., g^{ord(g)}\}$, przy czym możemy mieć $ord(g)=\infty$. W takim przypadku dostajemy podgrupę nieskończoną. Dla grupy cyklicznej utworzonej przez $g$, ten element nazywamy {\color{def}generatorem}. Zauważmy, że wszystkie grupy cykliczne sa {\color{acc}abelowe}.
\smallskip

Grupa zawierająca wszystkie liczby całkowite z dodawaniem jest grupą cykliczną generowaną przez $1$ lub przez $-1$. Widzimy więc, że \emph{generator grupy nie jest wyznaczony jednoznacznie}.
\medskip

Dla $N\in \N$ definiujemy $C_N$ jako liczby naturalne $< N$ z dodawaniem modulo $N$. Zwykle oznaczamy ją $(\Z_N, +_N)$. Możemy pokazać, że każda grupa cykliczna skończona rzędu $N$ jest izomorficzne z $C_N$, natomiast grupy cykliczne nieskończone są izomorficzne z $C_\infty$.

Grupa $\Z_N*=(\Z_N*,\cdot)$ to grupa liczb naturalnych mniejszych niż $N$, które są z $N$ względnie pierwsze. Działanie na tej grupie to mnożenie modulo $N$.