\section{PIERŚCIENIE}

\subsection{Definicja}

{\color{def}Pierścień} $(R, +, \cdot)$ to zbiór z mnożeniem $\cdot$ oraz dodawaniem $+$ taki, że $(R, +)$ jest grupą przemienną, mnożenie jest łączne i posiada element neutralny. Dodatkowo, dodawanie jest rozdzielne względem mnożenia, czyli dla wszystkich $x,y,z\in A$ mamy
$$(x+y)z=xz+yz$$

Grupa $U$ zawierająca wszystkie elementy $A$ takie, że istnieją ich odwrotności dla mnożenia nazywa się {\color{def}grupą multiplikatywną} lub {\color{acc}unit group of $A$}. Czasem zapisujemy ją jako $A^*$ i nazywamy grupą elementów, które są {\color{acc}odwracalne}. Jeżeli w pierścieniu $A$ mamy $1\neq0$ i każdy jego element jest odwracalny, to $A$ jest nazywamy {\color{def}pierścieniem z dzieleniem}.

\subsection{Dzielnik zera}

\subsection{Grupa elementów odwracalnych pierścienia}

\subsection{Dziedzina}

\subsection{Ciało}