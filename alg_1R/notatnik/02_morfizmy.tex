\section{HOMOMORFIZMY}

Jeżeli $F:A\to B$ jest homomorfizmem struktur, to $Im(F)$ jest podstrukturą $B$.
\bigskip

\pdef
\textbf{\color{def}SŁOWNICZEK}:
\smallskip

%\begin{multicols}{2}
    \point epi-morfizm -> "na"

    \point mono-morfizm -> 1-1

    \point izo-morfizm -> bijeckja

    \point endo-morfizm -> w samego siebie

    \point auto-morfizm -> endomorfizm który jest bijekcją.
%\end{multicols}
\smallskip

\kdef
\medskip

{\color{acc}Złożenie homomorfizmów jest homomorfizmem} a odwzorowanie odwrotne do izomorfizmu jest izomorfizmem.
\medskip

{\color{def}DOWÓD:}
\smallskip

Niech $f:(X, \cdot)\to(Y, \circ)$ i $g:(Y, \circ)\to(Z\star)$ są homomorfizmami, a $h(x)=g(f(x))$ jest ich złożeniem, to dla dowolnego $a,b\in X$ mamy
\begin{align*}
    h(a\cdot b)=g(f(a\cdot b))=g(f(a)\circ f(b))=g(f(a))\star g(f(b))=h(a)\star h(b)
\end{align*}
więc $h$ spełnia warunki homomorfizmu. Jeżeli $f, g$ były epi, mono, ... morfizmami, to zachowanie odpowiednich własności wynika z własności składania funkcji różnowartościowych, na czy bijekcji.
\smallskip

Niech $\phi:(X, \cdot)\bijekcja (Y, \circ)$ będzie izomorfizmem. Chcemy pokazać, że $\phi^{-1}$ jest homomorfizmem. Weźmy $a, b\in Y$ i $c, d\in X$ takie, że $\phi(c)=a$ oraz $\phi(d)=b$. Wtedy
$$ab=\phi(c)\phi(d)=\phi(cd),$$
czyli
$$\phi^{-1}(ab)=cd,$$
a ponieważ $\phi^{-1}(a)=c$ i $\phi^{-1}(b)=d$, to mamy
$$\phi^{-1}(ab)=cd=\phi^{-1}(a)\phi^{-1}(b).$$
Natomiast fakt, że $\phi^{-1}$ jest bijekcją wynika z tego, że $\phi$ jest bijekcją.

\subsection{Rodzaje}

\subsection{Jądro, obraz}

Dla danego homomorfizmu $f:G\to H$ definiujemy {\color{def}jądro} $Ker\; f=\{g\in G\;:\;f(g)=e_H\}$ oraz {\color{def}obraz} $Im\;f=\{f(g)\;:\;g\in G\}$. Z tych definicji wynika, że $Ker\;f\leq G$ oraz $Im\;f\leq H$.
\medskip

Dla monomorfizmu $f:G\to H$ jądro jest trywialne $ker\;f =\{e_G\}$. Gdyby tak nie było, to dla pewnego $e_G\neq g\in G$ mielibyśmy $f(g)=e_H$, a więc dla wszystkich innych $e_G\neq h\in G$
$$f(h)=e_H\cdot f(h)=f(g)f(h)=f(gh),$$
i jest $gh\neq h$ ale $f(h)=f(gh)$.
\bigskip

Jeśli $f:X\to Y$ jest epimorfizmem, to relacja $\sim$ określona na $X$ przez
$$x\sim y\iff f(x)=f(y)$$
jest relacją równoważności, a jej klasy abstrakcji są {\color{acc}włóknami funkcji $f$}. Jeśli $K=Ker\;f$, to dla każdego $a\in X$ mamy $aK=Ka$ i warstwy $K$ w $X$ to włókna $f$.

\subsection{Zasadnicze twierdzenie o homomorfizmie}