\documentclass{article}[13pt]

\usepackage[T1]{fontenc}
\usepackage[utf8]{inputenc}
\usepackage{amssymb}
\usepackage[polish]{babel}

\usepackage{mathastext}
\usepackage{amsmath}
\usepackage{mathtools}
\usepackage{dsfont}

% fancy fonts in math mode
\usepackage{eufrak}
\usepackage{calrsfs}

\usepackage{graphicx} % for rotating \obet

\newcommand{\R}{\mathds{R}}
\newcommand{\N}{\mathds{N}}
\newcommand{\Q}{\mathds{Q}}
\newcommand{\Z}{\mathds{Z}}
\newcommand{\C}{\mathds{C}}

\usepackage{geometry}

\author{Weronika Jakimowicz}
\title{Zadania z $\star\star$ LISTA 3}
\date{32 października 2022}

\geometry {
    a4paper,
    total={180mm, 267mm}
}

\usepackage{xcolor}
\definecolor{def}{HTML}{E86354}

\begin{document}
    \maketitle

    \section*{ZAD 12.}

    Zauważmy, że jesli $f(x)\in P_2$, to $f$ możemy zapisać jako
    $$f(x)=ax^2+bx+c$$
    dla $a,b,c\in\R$ takich, że $a+b+c=1$. W takim razie, funkcja $\phi(f)$ sprowadza się do postaci:
    $$\phi(f)=\int\limits_0^1f(x)^2dx=\int\limits_0^1(ax^2+bx+c)^2dx,$$
    co z kolei jest równe:
    $$\phi(f)={a^2\over5}+{2ac+b^2\over3}+{ab\over2}+bc+c^2.$$

    Całe zadanie sprowadza się do znalezienia minimum funkcji trzech zmiennych
    $$F(a,b,c)={a^2\over5}+{2ac+b^2\over3}+{ab\over2}+bc+c^2$$
    przy warunku, że funkcja
    $$g(a,b,c)=a+b+c=1.$$

    Używając mnożników Lagrange'a dostajemy układ równań postaci
    \begin{align*}
        &\begin{cases}
            \frac25a+\frac23c+\frac b2-\lambda=0\\
            \frac23b+\frac a2+c-\lambda=0\\
            \frac23a+b+2c-\lambda=0\\
            a+b+c=1
        \end{cases}\\
        &\begin{cases}
            12a+20c+15b-30\lambda=0\\
            4b+3a+6c-6\lambda=0\\
            2a+3b+6c-3\lambda=0\\
            a+b+c=1
        \end{cases}
    \end{align*}

    {\color{def}1. $\lambda=0$}, wtedy
    \begin{align*}
        &\begin{cases}
            3a+4b+6c=0\\
            2a+3b+6c=0\\
            12a+15b+20c=0\\
            a+b+c=1
        \end{cases}\\
        &\begin{cases}
            a=-b\\
            20c=3a\\
            c=1
        \end{cases}\\
        \begin{cases}
            a=\frac{20}3\\
            b=-\frac{20}3\\
            c=1
        \end{cases}
    \end{align*}
        $$F(\frac{20}3,-\frac{20}3,1)=\frac7{27}$$

    {\color{def}2. $\lambda\neq0$}
    \medskip

    Jeśli zapiszemy je w postaci macierzy, dostajemy:
    
    $$
        \begin{pmatrix}
            12 &15& 20& -30& 0\\
            3 & 4& 6& -6& 0\\
            2 &3& 6& -3& 0\\
            1& 1& 1& 0& 1
        \end{pmatrix}
    $$

    Korzystając z metody eliminacji Gaussa, dostajemy macierz

    $$
        \begin{pmatrix}
            12 & 15	& 20&	-30&	0\\
            0&	\frac14	&  1&	\frac32&	0\\
            0	&  0	&\frac23	 &-1&	0\\
            0	&  0	&  0&	\frac92	&1\\
        \end{pmatrix}
    $$

    Która daje nam poniższe równanie:
    $$
        \begin{cases}
            \lambda=\frac29\\
            \frac23c=\lambda\\
            \frac14b+c+\frac32\lambda=0\\
            12a+15b+20c-30\lambda=0
        \end{cases}
    $$

    $$
        \begin{cases}
            \lambda=\frac29\\
            c=\frac13\\
            b=-\frac83\\
            a=\frac{10}3
        \end{cases}
    $$

    Zauważmy, że zbiór $P_2$, oraz zbior wektorów z $\R^3$ o kolejnych współrzędnych będących współczynnikami wielomianów z $P_2$, jest niezwarty i nieograniczony. Musimy więc sprawdzić, co się dzieje kiedy
    $$\|(a,b,c)\|\to\infty$$
    Wtedy $a^2+b^2+c^2\to\infty$, a wiec 
    $$F(a,b,c)\xrightarrow{}{\|(a,b,c)\|\to\infty}\infty$$

    czyli wiemy, że dla nieskończenie długich wektorów wartość funkcji jest nieskończenie wysoka.
    \medskip

    Wartość funkcji $F$ w punkcie który został otrzymany w powyższych obliczeniach wynosi
    $$F({10\over3},\frac83,\frac13)=\frac19$$
    czyli jest niższa niż dla przypadku $\lambda=0$.
    \medskip
    

    Ponieważ warunek $x+y+z=1$ każe nam szukać rozwiązań na płaszczyznie, możemy uzależnić jedną zmienną od innych, np $x$ 
    $$x=1-z-y,$$
    oraz zbadać nową funkcję, de facto funkcję dwóch zmiennych. Nazwijmy ją $G(y,z)$, ze wzorem wynikłym ze wzoru na $F$:
    $$G(y,z)=\frac1{30}(y^2+7yz+3y+16z^2+8z+6).$$
    Hesjan takiej funkcji wynosi
    $$
        \begin{bmatrix}
            2 & 7\\
            7 & 32
        \end{bmatrix}=64-49>0
    $$
    i jest niezależny od $y,z$ oraz dodatni, więc funkcja na badanej płaszczyznie jest wypukła. W takim razie znalezione przeze mnie ekstremum to minimum.

\end{document}