\documentclass{article}[13pt]

\usepackage{../uni-notes-eng}

\author{Weronika Jakimowicz}
\title{Zadania z $\star\star$ LISTA 10}
\date{25 grudnia 2023}

\begin{document}
\maketitle

\subsection*{ZAD. 17}

Zacznijmy od wskazówki, to znaczy pokazania, że
$$\lim\limits_{n\to\infty}n\Big(\int\limits_0^1f(x)dx-\frac1n\sum\limits_{i=1}^nf\Big(\frac in\Big)\Big)={f(1)-f(0)\over 2}$$

Według wzoru Taylora wiemy, że istnieje $\xi_i\in[\frac in, {i+1\over n}]$ takie, że
$$f(x)=f(\frac in)+f'(\xi_i)(x-\frac in)$$
$$f(x)-f(\frac in)=f'(\xi_i)(x-\frac in)$$

\begin{align*}
    \sum\limits_{i=1}^nf\Big(\frac in\Big)&-n\int\limits_0^1f(x)dx=\sum\limits_{i=1}^nf\Big(\frac in\Big)-n\sum\limits_{i=1}^n\int\limits_{{i-1\over n}}^\frac inf(x)dx=n\sum\limits_{i=1}^n\int\limits_{{i-1\over n}}^\frac in\big[f\Big(\frac in\Big)-f(x)\big]dx=\\
    &=n\sum\limits_{i=1}^n\int\limits_{{i-1\over n}}^\frac inf'(\xi_i)\big[x-\frac in\big]dx-{f(1)-f(0)\over2}=n\sum\limits_{i=1}^nf'(\xi_i)\Big[{x^2\over 2}-{i\over n}x\Big]_{{i-1\over n}}^\frac in=\\
    &=n\sum\limits_{i=1}^n\Big[f'(\xi_i){1\over2n^2}\Big]=\frac1{2n}\sum\limits_{i=1}^nf'(\xi_i)\xrightarrow{n\to\infty}\frac12\int\limits_0^1f'(x)dx={f(1)-f(0)\over 2}
\end{align*}

Przyjrzyjmy się teraz zależności którą mamy udowadniać.
\begin{align*}
    n\int\limits_0^1\int\limits_0^1f(x,y)dxdy&-\frac1n\sum\limits_{j=1}^n\sum\limits_{k=1}^nf\Big(\frac jn, \frac kn\Big)=n\sum\limits_{i=1}^n\int\limits_{{i-1\over n}}^\frac in\int\limits_0^1f(x,y)dxdy-\frac1n\sum\limits_{i=1}^n\sum\limits_{k=1}^nf\Big(\frac in, \frac kn\Big)=\\
    &=
\end{align*}

\end{document}