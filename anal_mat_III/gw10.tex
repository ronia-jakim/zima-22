\documentclass{article}[13pt]

\usepackage{../uni-notes-eng}

\author{Weronika Jakimowicz}
\title{Zadania z $\star\star$ LISTA 10}
\date{25 grudnia 2023}

\begin{document}
\maketitle

\subsection*{ZAD. 17}
Tak jak we wskazówce, pokażemy najpierw, że
$$\lim\limits_{n\to\infty}n\Big(\int\limits_0^1f(x)dx-\frac1n\sum\limits_{j=0}^nf(\frac jn)\Big)=\frac12(f(1)-f(0)).$$
Zauważmy najpierw, że skoro $f$ jest ciągłą funkcją całkowalną na badanym przedziale, to
$$f(1)-f(0)=\int\limits_0^1f'(x)dx$$
a z drugiej strony dla $x_i\in[\frac {i-1}n,\frac in]$
$$f(1)-f(0)=\sum\limits_{i=1}^n(f(\frac in)-f(\frac{i-1}n))=\sum\limits_{i=1}^nf'(x_i)\frac1n,$$
co dla $n\to\infty$ zmierza do $\int\limits_0^1f'(x)dx$.

Popatrzmy teraz na lewą stronę równania
\begin{align*}
    n(\int\limits_0^1f(x)dx-\frac1n\sum\limits_{i=1}^nf(\frac in))&=n\int\limits_0^1f(x)dx-\sum\limits_{i=1}^nf(\frac in)=n\sum\limits_{i=1}^n\int\limits_\frac {i-1}n^\frac{i}nf(x)dx-\sum\limits_{i=1}^nf(\frac in)=\\
    &=n\sum\limits_{i=1}^n\int\limits_{\frac{i-1}n}^\frac in[f(x)-f(\frac in)]dx=n\sum\limits_{i=1}^n\int\limits_{{i-1\over n}}^\frac in\int\limits_x^{\frac in}f'(t)dtdx
\end{align*}

Ponieważ dla każdego $\frac {i-1}n\leq x$ zachodzi
$$\int\limits_{{i-1\over n}}^\frac in |f'(t)|dt\leq \int\limits_x^\frac in |f'(t)|dt,$$
to mamy
\begin{align*}
    n\sum\limits_{i=1}^n\int\limits_{{i-1\over n}}^\frac in\int\limits_x^{\frac in}f'(t)dtdx&\leq n\sum\limits_{i=1}^n\int\limits_{{i-1\over n}}^\frac in\int\limits_x^{\frac in}|f'(t)|dtdx\leq n\sum\limits_{i=1}^n\int\limits_{{i-1\over n}}^\frac in\int\limits_{{i-1\over n}}^\frac in |f'(t)|dtdx=\\
    &=n\sum\limits_{i=1}^n\Big[(\frac in-\frac{i-1}n)\int\limits_{{i-1\over n}}^\frac in |f'(t)|dt\Big]=\sum\limits_{i=1}^n\int\limits_{{i-1\over n}}^\frac in |f'(t)|dt=\int\limits_0^1|f'(t)|dt
\end{align*}

Teraz chcemy sprawdzić co się dzieje, gdy $n\to\infty$ z wyrażeniem
$$n\Big(\int\limits_0^1f(x)dx-\frac 1n\sum\limits_{i=1}^nf(\frac in)\Big)-{f(1)-f(0)\over 2}$$

\begin{align*}
    n\Big(\int\limits_0^1f(x)dx-\frac 1n\sum\limits_{i=1}^nf(\frac in)\Big)-{f(1)-f(0)\over 2}&\leq|n\Big(\int\limits_0^1f(x)dx-\frac1n\sum\limits_{i=1}^nf(\frac in)\Big)|-\int\limits_0^1f'(x)dx\leq \int\limits_0^1|f'(x)|dx-\int\limits_0^1f'(x)dx
\end{align*}


% Teraz zauważamy, że
% $$\lim\limits_{n\to\infty}n\Big(\int\limits_0^1f(x)dx-\frac1n\sum\limits_{j=0}^nf(\frac jn)\Big)=\frac12(f(1)-f(0)),$$
% to wtedy
% $$\lim\limits_{n\to\infty}\Big[n\Big(\int\limits_0^1f(x)dx-\frac1n\sum\limits_{j=0}^nf(\frac jn)\Big)-\frac 1{2}(f(1)-f(0))\Big]=0.$$

% Korzystając z wcześniej wyliczonych wartości mamy
% \begin{align*}
%     n\Big(\int\limits_0^1f(x)dx-\frac1n\sum\limits_{j=0}^nf(\frac jn)\Big)-\frac12(f(1)-f(0))&=n\sum\limits_{i=1}^n\int\limits_{{i-1\over n}}^{i\over n}\int\limits_x^\frac in f'(t)dtdx-\sum\limits_{i=1}^nf'(x_i)\frac1n
% \end{align*}
\end{document}