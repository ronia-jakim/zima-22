\documentclass{article}[13pt]

\usepackage{../uni-notes-eng}

\author{Weronika Jakimowicz}
\title{Zadania z $\star\star$ LISTA 12}
\date{28 grudnia 2023}

\begin{document}
    \maketitle

\subsection*{ZAD. 12}
\emph{
    Niech $f$ będzie funkcją ciągłą taką, że ${\partial f\over\partial x}$ istnieje i jest ciągła. Pokazać, że
}
$${d\over dx}\int\limits_a^x\int\limits_c^df(x,y,z)dzdy=\int\limits_c^df(x,x,z)dz+\int\limits_a^x\int\limits_c^d{\partial f\over\partial x}(x,y,z)dzdy$$

Rozważmy funkcję
$$g(x,y)=\int\limits_c^df(x,y,z)dz$$
oraz
$$h(x)=\int\limits_a^xg(x,y)dy.$$
Będziemy liczyć
$${d\over dx}h(x)={d\over dx}\int\limits_a^x\int\limits_c^df(x,y,z)dzdy=\lim\limits_{t\to0}{h(x+t)-h(x)\over t}.$$
Dla ułatwienia zapisu, sprawdźmy najpierw jak będzie wyglądał licznik powyższego ułamka:
\begin{align*}
    h(x+t)-h(x)&=\int\limits_a^{x+t}g(x+t,y)dy-\int\limits_a^xg(x,y)dy=\\
    &=\int\limits_a^x[g(x+t, y)-g(x, y)]dy+\int\limits_x^{x+t}g(x+t, y)dy
\end{align*}
Wartość drugiego elementu tej sumy, to znaczy
$$\int\limits_x^{x+t}g(x+t, y)dy$$
możemy ocenić za pomocą twierdzenia o wartości średniej, czyli istnieje $\xi\in[x, x+t]$ takie, że
$$\int\limits_x^{x+t}g(x+t, y)dy=(x+t-x)f(x, \xi).$$
Teraz wracając do całości:
\begin{align*}
    {d\over dx}h(x)&=\lim\limits_{t\to0}\frac1t\Big(\int\limits_a^{x}[g(x+t, y)-g(x,y)]dy+\int\limits_x^{x+t}g(x+t, y)dy\Big)=\\
    &=\lim\limits_{t\to0}\int\limits_a^x{g(x+t, y)-g(x,y)\over t}dy+\lim\limits_{t\to0}\frac1t\int\limits_x^{x+t}g(x+t, y)dy=\\
    &=\lim\limits_{t\to0}\int\limits_a^x{g(x+t, y)-g(x,y)\over t}dy+\lim\limits_{t\to0}{x+t-x\over t}g(x+t, \xi)=\\
    &=\lim\limits_{t\to0}\int\limits_a^x{g(x+t, y)-g(x,y)\over t}dy+g(x, x),
\end{align*}
bo w drugim składniku ${x+t-x\over t}={t\over t}=1$ natomiast $\xi\in[x, x+t]$ szerokość tego przedziału zmierza do $0$, więc $\xi=x$. Teraz chcemy włożyć $\lim$ pod całkę, więc musimy pokazać, że ${g(x+t, y)-g(x, y)\over t}$ zbiega jednostajnie do ${\partial\over\partial x}g(x, y)$. Popatrzmy na ciąg
$$p_n(x, y)={g(x+\frac 1n, y)-g(x, y)\over \frac1n}.$$
Nietrudno zauważyć, że dla $n\to \infty$ mamy to samo co dla $t\to0$, natomiast z ciągłości $g$ (jako złożenia funkcji ciągłych), mamy zbieżność jednostajną tego ciągu do ${\partial\over\partial x}g(x, y)$. Czyli wracając do długiego równania, mamy
\begin{align*}
    {d\over dx}\int\limits_a^x\int\limits_c^df(x, y, z)dzdy={d\over dx}h(x)&=\int\limits_a^x{\partial\over\partial x}g(x, y)dy+g(x, x)=\\
    &=\int\limits_a^x{\partial\over\partial x}\int\limits_c^df(x, y, z)dzdy+\int\limits_c^df(x, x, z)dz=\\
    &=\int\limits_a^x\int\limits_c^d{\partial\over\partial x}f(x, y, z)dzdy+\int\limits_c^df(x, x, z)dz
\end{align*}
gdzie włożenia ${\partial\over\partial x}$ pod całkę dokonuję, bo w $\int\limits_c^df(x, y, z)dz$ funkcja $f$ zależy od $x$ a cała całka zmienia się przez zmianę wartości $f$ w $x$.

\end{document}