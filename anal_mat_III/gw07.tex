\documentclass{article}[13pt]

\usepackage[T1]{fontenc}
\usepackage[utf8]{inputenc}
\usepackage{amssymb}
\usepackage[polish]{babel}

\usepackage{mathastext}
\usepackage{amsmath}
\usepackage{mathtools}
\usepackage{dsfont}

% fancy fonts in math mode
\usepackage{eufrak}
\usepackage{calrsfs}

\usepackage{graphicx} % for rotating \obet

\newcommand{\R}{\mathds{R}}
\newcommand{\N}{\mathds{N}}
\newcommand{\Q}{\mathds{Q}}
\newcommand{\Z}{\mathds{Z}}
\newcommand{\C}{\mathds{C}}

\usepackage{geometry}

\usepackage{tikz}

\usepackage{xcolor}

\author{Weronika Jakimowicz}
\title{Zadania z $\star\star$ LISTA 7}
\date{32 października 2022}

\geometry {
    a4paper,
    total={180mm, 267mm}
}

\usepackage{xcolor}
\definecolor{def}{HTML}{E86354}

\begin{document}
\maketitle

\section*{ZAD 1.}

{\color{red}$\quad\;\;1$. RÓŻNICZKOWALNOŚĆ:}
\medskip

Aby funkcja wieloarguemntowa o wartościach wektorowych była różniczkowalna w punkcie $(x_0,y_0)$, granica
$$\lim\limits_{(x,y)\to(x_0,y_0)}{\|F(x,y)-F(x_0,y_0)-T(x-x_0,y-y_0)\|\over\|(x-x_0,y-y_0)\|}$$
musi zmieżać do 0. To znaczy, że funkcję $F$ możemy bardzo dobrze przybliżyć za pomocą przekształcenia liniowego $T$ w okolicy punktu $(x_0,y_0)$. Ponieważ operujemy na przekształceniu $\R^2\to\R^2$, to $T$ będzie miało macierz 2x2
$$\begin{pmatrix}a & b\\ c & d\end{pmatrix}$$
i po wymnożeniu z dowolnym $(x,y)$ będzie dawało wynik
$$\begin{pmatrix}
    a & b\\
    c & d
\end{pmatrix}
\begin{pmatrix}
    x\\y
\end{pmatrix}=\begin{pmatrix}
    ax+by\\
    cx+dy
\end{pmatrix}
$$
Aby taka operacja dawała wynik zbliżający się do $(x_0,y_0)$ dla punktów w jego pobliżu, to $a$ musi być małą zmianą w pierwszej współrzędem względem $x$, a $b$ małą zmianą w pierwszej współrzędnej względem $y$, analogicznie dla $c$ i $d$. Czyli $T$ jest Jakobianem. W takim razie, żeby sprawdzić czy badana przez nas funkcja jest zbieżna, wystarczy sprawdzić, czy dla dowolnego punktu poniższa granica jest równa 0:
\begin{align*}
    &{\|F(x,y)-F(x_0,y_0)-T(x-x_0,y-y_0)\|\over\|(x-x_0,y-y_0)\|}=\\
    &={\|
        \begin{pmatrix}x+f(y)\\y+f(x)\end{pmatrix}
        -\begin{pmatrix}x_0+f(y_0)\\y_0+f(x_0)\end{pmatrix}-
        \begin{pmatrix}
            x-x_0+(y-y_0)f'(y_0)\\
            (x-x_0)f'(x_0)+y-y_0
        \end{pmatrix}
    \|\over\|(x-x_0,y-y_0)\|}=\\
    &={\|
        \begin{pmatrix}
            f(y)-f(y_0)-(y-y_0)f'(y_0)\\
            f(x)-f(x_0)-(x-x_0)f'(x_0)
        \end{pmatrix}
        \|\over
        \|(x-x_0,y-y_0)\|
    }=\\
    &=\Big[{{(f(y)-f(y_0)-(y-y_0)f'(y_0))^2+(f(x)-f(x_0)-(x-x_0)f'(x_0))^2\over(x-x_0)^2+(y-y_0)^2}}\Big]^\frac12=\\
    \Big[{(f(y)-f(y_0))^2\over}\Big]
\end{align*}

\medskip


{\color{red}2. RÓŻNOWARTOŚCIOWOŚĆ:}
\medskip

Żeby funkcja była różnowartościowa, musi być odwracalna w każdym punkcie swojej dziedziny. To znaczy, że pierwsza pochodna nie może nigdy się zerowa. Przyjżyjmy się Jakobianowi funkcji $F$
\begin{align*}
    \begin{bmatrix}
        1 & f'(y)\\
        f'(x) & 1
    \end{bmatrix}
\end{align*}
Jego wyznacznik to 
$$1-f'(x)f'(y)$$
a ponieważ 
$$(\forall\;a\in\R)\;|f'(a)|<1,$$
to również
$$|f'(x)f'(y)|<1,$$
co daje nam 
$$1-f'(x)f'(y)>0,$$
a więc badana funkcja jest odwracalna w każdym punkcie dziedziny. To znaczy, że jest różnowartościowa.
\medskip

{\color{red}3. NA:}
\medskip

Jeśli funkcja jest "na", to nie może być ściśle wklęsła ani ściśle wypukła. Czyli wyznacznik jej Hesjana musi być 0.
\begin{align*}
    \begin{bmatrix}
        {d\over dx}1 & {d\over dy} f'(x)\\
        {d\over dx}f'(y) & {d\over dy}1
    \end{bmatrix}=
    \begin{bmatrix}
        0 & 0\\
        0 & 0
    \end{bmatrix}
\end{align*}
wyznacznik jest stale równy zero, więc mamy funkcje która nie jest ani wypukła ani wklęsła, więc przechodzi całą przeciwdziedzinę.


\end{document}