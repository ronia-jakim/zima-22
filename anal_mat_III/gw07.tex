\documentclass{article}[13pt]

\usepackage[T1]{fontenc}
\usepackage[utf8]{inputenc}
\usepackage{amssymb}
\usepackage[polish]{babel}

\usepackage{mathastext}
\usepackage{amsmath}
\usepackage{mathtools}
\usepackage{dsfont}

% fancy fonts in math mode
\usepackage{eufrak}
\usepackage{calrsfs}

\usepackage{graphicx} % for rotating \obet

\newcommand{\R}{\mathds{R}}
\newcommand{\N}{\mathds{N}}
\newcommand{\Q}{\mathds{Q}}
\newcommand{\Z}{\mathds{Z}}
\newcommand{\C}{\mathds{C}}

\usepackage{geometry}

\usepackage{tikz}

\usepackage{xcolor}

\author{Weronika Jakimowicz}
\title{Zadania z $\star\star$ LISTA 7}
\date{32 października 2022}

\geometry {
    a4paper,
    total={180mm, 267mm}
}

\usepackage{xcolor}
\definecolor{def}{HTML}{E86354}

\begin{document}
\maketitle

\section*{ZAD 1.}
Klasa $C^1$ - użyć złożenia dwóch funkcji ciągłych.
\medskip

{\color{red}2. RÓŻNOWARTOŚCIOWOŚĆ:}
\medskip

Zauważmy, że złożenie dwóch funkcji różnowartościowych jest różnowartościowe. Niech 
$$g(x,y)=x+y$$
$$f(x,y)=g(F(x,y))=x+f(y)+y+f(z).$$
Badamy teraz różnorodność $f$. Załóżmy, że nie jest ono różnowartościowe. Czyli w pewnym momencie funkcja zawraca na $x$ i na $y$, a więc obie pochodne cząstkowe się zerują.
$${d\over dx}F(x,y)=1$$
$${d\over dy}F(x,y)=1$$
Widzimy, że obie pochodne są funkcjami stałymi, więc nigdy się nie wyzerują, a więc $F$ jest 1-1.

{\color{red}3. NA:}
\medskip



\end{document}