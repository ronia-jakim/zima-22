\documentclass{article}[13pt]

\usepackage[T1]{fontenc}
\usepackage[utf8]{inputenc}
\usepackage{amssymb}
\usepackage[polish]{babel}

\usepackage{mathastext}
\usepackage{amsmath}
\usepackage{mathtools}
\usepackage{dsfont}

% fancy fonts in math mode
\usepackage{eufrak}
\usepackage{calrsfs}

\usepackage{graphicx} % for rotating \obet

\newcommand{\R}{\mathds{R}}
\newcommand{\N}{\mathds{N}}
\newcommand{\Q}{\mathds{Q}}
\newcommand{\Z}{\mathds{Z}}
\newcommand{\C}{\mathds{C}}


\usepackage{geometry}

\usepackage{tikz}

\usepackage{xcolor}

\author{Weronika Jakimowicz}
\title{Zadania z $\star\star$ LISTA 7}
\date{32 października 2022}

\geometry {
    a4paper,
    total={180mm, 267mm}
}

\usepackage{xcolor}
\definecolor{def}{HTML}{E86354}
\definecolor{gr}{HTML}{046137}

\begin{document}
\maketitle

\section*{ZAD 1.}

{\color{red}$\quad\;\;1$. RÓŻNICZKOWALNOŚĆ:}
\medskip

Aby funkcja wieloarguemntowa o wartościach wektorowych była różniczkowalna w punkcie $(x_0,y_0)$, granica
$$\lim\limits_{(x,y)\to(x_0,y_0)}{\|F(x,y)-F(x_0,y_0)-T(x-x_0,y-y_0)\|\over\|(x-x_0,y-y_0)\|}$$
musi zmieżać do 0. To znaczy, że funkcję $F$ możemy bardzo dobrze przybliżyć za pomocą przekształcenia liniowego $T$ w okolicy punktu $(x_0,y_0)$. Ponieważ operujemy na przekształceniu $\R^2\to\R^2$, to $T$ będzie miało macierz 2x2
$$\begin{pmatrix}a & b\\ c & d\end{pmatrix}$$
i po wymnożeniu z dowolnym $(x,y)$ będzie dawało wynik
$$\begin{pmatrix}
    a & b\\
    c & d
\end{pmatrix}
\begin{pmatrix}
    x\\y
\end{pmatrix}=\begin{pmatrix}
    ax+by\\
    cx+dy
\end{pmatrix}
$$
Aby taka operacja dawała wynik zbliżający się do $(x_0,y_0)$ dla punktów w jego pobliżu, to $a$ musi być małą zmianą w pierwszej współrzędem względem $x$, a $b$ małą zmianą w pierwszej współrzędnej względem $y$, analogicznie dla $c$ i $d$. Czyli $T$ jest Jakobianem. W takim razie, żeby sprawdzić czy badana przez nas funkcja jest zbieżna, wystarczy sprawdzić, czy dla dowolnego punktu $(x_0,y_0)$ poniższa funkcja zmieża do 0 dla $(x,y)\to(x_0,y_0)$:
\begin{align*}
    G(x,y)&={\|F(x,y)-F(x_0,y_0)-T(x-x_0,y-y_0)\|\over\|(x-x_0,y-y_0)\|}=\\
    &={\|
        \begin{pmatrix}x+f(y)\\y+f(x)\end{pmatrix}
        -\begin{pmatrix}x_0+f(y_0)\\y_0+f(x_0)\end{pmatrix}-
        \begin{pmatrix}
            x-x_0+(y-y_0)f'(y_0)\\
            (x-x_0)f'(x_0)+y-y_0
        \end{pmatrix}
    \|\over\|(x-x_0,y-y_0)\|}=\\
    &={\|
        \begin{pmatrix}
            f(y)-f(y_0)-(y-y_0)f'(y_0)\\
            f(x)-f(x_0)-(x-x_0)f'(x_0)
        \end{pmatrix}
        \|\over
        \|(x-x_0,y-y_0)\|
    }=\\
    &=\Big[{{(f(y)-f(y_0)-(y-y_0)f'(y_0))^2+(f(x)-f(x_0)-(x-x_0)f'(x_0))^2\over(x-x_0)^2+(y-y_0)^2}}\Big]^\frac12.
\end{align*}

Po pierwsze zauważmy, że podana funkcja ma wartości nieujemne, więc jeśli jest ciągła, to musi zmieżać do wartości nieujemnej. Po drugie, zauważy, że
\begin{align*}
    G(x,y)&\leq\Big[{(f(y)-f(y_0)-(y-y_0)f'(y_0))^2\over(y-y_0)^2}+{(f(x)-f(x_0)-(x-x_0)f'(x_0))^2\over(x-x_0)^2}\Big]^\frac12\leq\\
    &\leq \Big|{f(y)-f(y_0)-(y-y_0)f'(y_0)\over(y-y_0)}\Big|+\Big|{f(x)-f(x_0)-(x-x_0)f'(x_0)\over(x-x_0)}\Big|
\end{align*}
Widzimy, że funkcja o nieujemnych wartościach jest ograniczona od góry przez funkcję o wartościach nieujemnych dążącą do 0 (ponieważ $f$ jest klasy $C1$), więc i funkcja $G(x,y)\xrightarrow{(x,y)\to(x_0,y_0)}0$.

\bigskip


% {\color{red}2. RÓŻNOWARTOŚCIOWOŚĆ:}
% \medskip

% Żeby funkcja była różnowartościowa, musi być odwracalna w każdym punkcie swojej dziedziny. To znaczy, że pierwsza pochodna nie może nigdy się zerowa. Przyjżyjmy się Jakobianowi funkcji $F$
% \begin{align*}
%     \begin{bmatrix}
%         1 & f'(y)\\
%         f'(x) & 1
%     \end{bmatrix}
% \end{align*}
% Jego wyznacznik to 
% $$1-f'(x)f'(y)$$
% a ponieważ 
% $$(\forall\;a\in\R)\;|f'(a)|<1,$$
% to również
% $$|f'(x)f'(y)|<1,$$
% co daje nam 
% $$1-f'(x)f'(y)>0,$$
% a więc badana funkcja jest odwracalna w każdym punkcie dziedziny. To znaczy, że jest różnowartościowa.
% \medskip

% {\color{red}3. NA:}
% \medskip

% Jeśli funkcja jest "na", to nie może być ściśle wklęsła ani ściśle wypukła. Czyli wyznacznik jej Hesjana musi być 0.
% \begin{align*}
%     \begin{bmatrix}
%         {d\over dx}1 & {d\over dx} f'(y)\\
%         {d\over dy}f'(x) & {d\over dy}1
%     \end{bmatrix}=
%     \begin{bmatrix}
%         0 & 0\\
%         0 & 0
%     \end{bmatrix}
% \end{align*}
% wyznacznik jest stale równy zero, więc mamy funkcje która nie jest ani wypukła ani wklęsła, więc przechodzi całą przeciwdziedzinę.


{\color{red}2. RÓŻNOWARTOŚCIOWOŚĆ I NA}
\medskip

{\color{gr}Jeśli funkcja jest $1-1$ oraz "na", to jest funkcją odwracalną. Czyli wystaczy pokazać, że $F$ jest odwracalne na całej swojej dziedzinie. Z twierdzenia o funkcji odwrotnej wiemy, że $F$ jest odwracalna w pewnym otoczeniu punktu $a\in\R^2$, jeżeli
\begin{align*}\begin{bmatrix}
    {d\over dx}F_1(a) & {d\over dy}F_1(a)\\
    {d\over dx}F_2(a) & {d\over dy}F_2(a)
\end{bmatrix}&=
\begin{bmatrix}
    1 & f'(a_2)\\
    f'(a_1) & 1
\end{bmatrix}=\\
&=1-f'(a_2)f'(a_1)\geq\\
&\geq 1-|f'(a_2)||f'(a_1)|\geq 1-k^2>0.
\end{align*}
Widzimy, że jakobian tej funkcji jest niezerowy dla dowolnego $a\in\R^2$, w takim razie funkcja jest odwracalna na całej swojej dziedzinie. A więc jest bijekcją.}


\section*{ZAD 2.}

$$f:\R^n\to\R^m$$
$$m<n$$

Jakobian tej funkcji to macierz o $m$ wierszach i $n$ kolumnach:
$$
\begin{bmatrix}
    {d\over dx_1}f_1(x) & {d\over dx_2}f_1(x)&...&{d\over dx_n}f_1(x)\\
    {d\over dx_1}f_2(x) & {d\over dx_2}f_2(x)&...&{d\over dx_n}f_2(x)\\
    ... & ... & ... & ...\\
    {d\over dx_1}f_m(x) & {d\over dx_2}f_m(x)&...&{d\over dx_n}f_m(x)\\
\end{bmatrix}
$$
Wyznacznik możemy wyliczyć tylko z macierzy kwadratowych, więc w tym wypadku mamy problem. Rozważmy więc funkcję
$$F\begin{pmatrix}
    x_1\\x_2\\x_3\\x_4\\x_5\\...\\x_n
\end{pmatrix}=\begin{pmatrix}
    y_1\\y_2\\...\\y_m\\x_{m+1}\\...\\x_{n}
\end{pmatrix}$$
gdzie $y=f(x)$. Funkcja $F$ jest bijekcją wtedy i tylko wtedy gdy $f$ jest $1-1$ i na.
\smallskip

$\implies$

$\quad 1.$ Jeżeli $f$ nie jest "na", to mamy $a\in\R^m$ taki, że $(\forall\;x\in\R^n)\;f(x)\neq a$. Jeżeli teraz weźmiemy $b\in\R^n$ takie, że pierwsze $m$ współrzędnych pokrywa się z $a$, a pozostałe to $1$, to dowolny punkt $x\in\R^n$ nie pokryje pierwszych $m$ współrzędnych $b$. A więc $F$ nie może być "na".

$\quad 2.$ Jeżeli $f$ nie jest $1-1$, to dla dwóch $a \neq b$ mamy $f(a)=f(b)$. Jeżeli istnieje taka para, która różni się na pierwszych $m$ współrzędnych, a na pozostałych $n-m$ ich współrzędne się pokrywają, to nie trudno zauważyć, że $F$ nie jest $1-1$. Jeżeli z kolei różnią się na którejś z $n-m$, to wystarczy rozważyć nową funkcję $F$, która "przesuwa" y nieco niżej.
\medskip

$\impliedby$

Z różnowartościowości $f$ od razu wynika różnowartościowość $F$. Tak samo, jeżeli $f$ jest "na", to bez problemu pokrywamy pierwsze $m$ współrzędnych, natomiast fakt, że dolna część definicji $F$ jest identycznością, daje nam pokrycie pozostałych $n-m$ współrzędnych i $F$ jest na.
\medskip

Jeżeli $F$ jest bijekcją, to musi istnieć funkcja $F^{-1}$, to znaczy że $F$ musi być odwracalne dla każdego $x\in\R^n$. Korzystając z twierdzenia o funkcji odwrotnej mamy, że $F$ jest bijekcją jeśli jej jakobian jest niezerowy w każdym punkcie. Przyjrzyjmy się więc macierzy Jacobiego $F$:
\begin{align*}
    \begin{bmatrix}
        {d\over dx_1}f_1(x) & {d\over dx_2}f_1(x)&...&{d\over dx_n}f_1(x)\\
        {d\over dx_1}f_2(x) & {d\over dx_2}f_2(x)&...&{d\over dx_n}f_2(x)\\
        ... & ... & ... & ...\\
        {d\over dx_1}f_m(x) & {d\over dx_2}f_m(x)&...&{d\over dx_n}f_m(x)\\
        {d\over dx_1}id(x) & {d\over dx_{2}}id(x) &...&{d\over dx_n}id(x)\\
        ... & ... & ... & ...\\
        {d\over dx_1}id(x) & {d\over dx_{2}}id(x) &...&{d\over dx_n}id(x)\\
    \end{bmatrix}=
    \begin{bmatrix}
        {d\over dx_1}f_1(x) & {d\over dx_2}f_1(x)&...&{d\over dx_n}f_1(x)\\
        {d\over dx_1}f_2(x) & {d\over dx_2}f_2(x)&...&{d\over dx_n}f_2(x)\\
        ... & ... & ... & ...\\
        {d\over dx_1}f_m(x) & {d\over dx_2}f_m(x)&...&{d\over dx_n}f_m(x)\\
        1 & 1 &...&1\\
        ... & ... & ... & ...\\
        1 & 1&...&1\
    \end{bmatrix}
\end{align*}
Macierz ta ma wyznacznik zerowy, gdyż wszystkie jej kolumny są liniowo zależne przez obecność $1$ na ostatnich współrzędnych.

% Jakobian w tym wypadku to macierz o $m$ wierszach i $n$ kolumnach. Z kursu algebry liniowej wiemy, że rząd takiej macierzy to co najwyżej $m$. Jest tak, bo układ $n$ wektorów (kolumny) z przestrzeni $m$ wymiarowej, gdzie $m<n$ może mieć co najwyżej $m$ wektorów niezależnych liniowo.
% \medskip

% Załóżmy, nie wprost, że $f$ jest przekształceniem jednoznacznym, to znaczy bijekcją. Wtedy, jeżeli będziemy rozważać $f$ obcięte do $W\leq\R^n$ rozpiętych przez pewne $m$ wektorów bazowych, to takie obcięcie $f$ nadal będzie bijecką.
% \medskip

% Rozważmy wszystkie takie podprzestrznie $W$, które są rozpinane między innymi przez pewien wektor bazowy $e$. Dobieramy do niego $m-1$ wektorów na
% $${n\choose m-1}\geq{m+1\choose m-1}={(m+1)m\over m}$$
% sposobów, co dla $m\geq2$ jest równe co najmniej $m+1$. 
% \medskip

% Teraz, jeżeli dla punktu odpowiadającemu wektorowi $e$ policzymy Jakobian dla każdej z tych $M\geq m+1$ macierzy, to zauważmy, że ponieważ jesteśmy w przekształceniu między przestrzeniami o wymiarze $m$, to mamy tylko $m$ wektorów liniowo niezależnych. Czyli w pewnej macierzy Jakobiego wartości pochodnych w tym punkcie będą odpowiadały wektorom liniowo zależnym, czyli wyznacznik będzie zerowy. W takim razie, dla tej podprzestrzeni w której to się dzieje, obcięcie $f$ nie jest 1-1, czyli również całe przekształcenie nie może być różnowartościowe. W takim razie nie jest bijeckją i dochodzimy do sprzeczności.

% \newpage

% Załóżmy, że $f$ jest przekształceniem jednoznaczym, to znaczy bijekcją. Wtedy, jeżeli będziemy rozważać $f$ obcięte do $W\leq\R^n$ rozpiętych przez pewne $m$ wektorów bazowych, to takie obcięcie $f$ nadal będzie bijekcją. Wybrać wektory, na których rozpięta będzie popdrzestrzeń $W$ możemy na
% $${n\choose m}\geq {m+1\choose m}=m+1.$$
% Dalej zauważmy, że każde takie obcięcie jest przekształceniem między przestrzeniami o stopniu $m$, czyli macierz Jakobiego będzie kwadratowa (możemy obliczyć wyznacznik). Mamy co najmniej $m+1$ takich macierzy i dla każdego punktu co najwyżej $m$ liniowo niezależnych wektorów odpowiadających wartościom pochodnych cząstkowych w tym punkcie, które możemy wpisać w te macierze. Czyli w jednej z tych macierzy na pewno będziemy mieli wektory liniowo zależne. To znaczy, że wyznacznik się zeruje, a więc takie obcięcie $f$ do $W\leq\R^n$ nie jest 1-1. W takim razie również całe przekształcenie nie mozę być 1-1, a więc w szczególności nie jest bijekcją.

% $${(m+1)!\over(m-1)!2!}={(m+1)m\over 2}$$

\newpage




\end{document}