\documentclass{article}[13pt]

\usepackage{../uni-notes-eng}

\author{Weronika Jakimowicz}
\title{Zadania z $\star\star$ LISTA 10}
\date{25 grudnia 2023}

\allowdisplaybreaks

\color{black}
\pagecolor{white}

\begin{document}
\clearpage
\maketitle
\thispagestyle{empty}

\subsection*{ZAD. 16}
Chcemy pokazać, że
$$\lim\limits_{t\to\infty}\int\limits_a^bf(x,\sin tx)dx=\frac1{2\pi}\int\limits_a^b\int\limits_0^{2\pi}f(x,\sin y)dydx$$
% tutaj zakładam, że w poleceniu wkradł się maleńki błąd drukarski i powinno być tak jak wyżej ($dydx$) zamiast $dxdy$, bo jeśli np. $b<2\pi$, to całkując $\int_0^{2\pi}f(x,\sin y)dx$ wyszlibyśmy poza dziedzinę funkcji $f$.
\medskip

\podz{sep}
\medskip

Ustalmy z góry $x_0$. Wtedy $f(x_0, y)$ jest funkcją która zależy tylko od $y$, a $x_0$ jest traktowane jako stała. Niech więc $g(y)=f(x_0,y)$. Pokażemy najpierw, że
$$\lim\limits_{t\to\infty}\int\limits_a^bg(\sin tx)dx=\frac1{2\pi}\int\limits_a^b\int\limits_0^{2\pi}g(\sin y)dydx.$$
Przenosząc prawą stronę na lewo i wkładając ją pod granicą (bo granica sumy to suma granic, a granica z wyrażenia stałego to ono same), dostajemy
$$0=\lim\limits_{t\to\infty}\int\limits_a^bg(\sin tx)dx-\frac1{2\pi}\int\limits_a^b\int\limits_0^{2\pi}g(\sin y)dydx=\lim\limits_{t\to\infty}\int\limits_a^b\Big[g(\sin tx)-\int\limits_0^{2\pi}g(\sin y)dy\Big]dx.$$
Dalej, zauważmy, że
\begin{align*}
    \int\limits_0^{2\pi}g(\sin y)dy&=\int\limits_0^\pi g(\sin y)dy+\int\limits_\pi^{2\pi}g(\sin y)dy=\\
    &=\int\limits_0^\pi g(\sin y)dy+\int\limits_{-\pi}^0g(\sin y)dy={1\over2\pi}\int\limits_{-\pi}^\pi g(\sin y)dy,
\end{align*}
bo przesunęliśmy jedną całkę o cały okres funkcji $\sin y$, więc wartości się nie zmieniają.

Wzór Taylora na funkcję $g(\sin y)$ w pobliżu punktu $\sin tx_0$ wygląda następująco:
$$g(\sin y)=g(\sin tx_0)+[g(\sin tx_0)]'(y-tx_0)=g(\sin tx_0)+t\cos tx_0g'(\sin tx_0)(y-tx_0)$$
$$g(\sin y)-g(\sin tx_0)=t\cos tx_0g(\sin tx_0)(y-tx_0)$$

Wróćmy teraz do obliczanej granicy:
\begin{align*}
    \int\limits_a^b\Big[g(\sin tx)&-{1\over2\pi}\int\limits_{-\pi}^\pi g(\sin y)dy\Big]dx={1\over2\pi}\int\limits_a^b\int\limits_{-\pi}^\pi\Big[g(\sin tx)-g(\sin y)\Big]dydx
\end{align*}
i zauważmy, że w środku całki $\int_{-\pi}^\pi$ $x$ jest traktowane jako stała, to znaczy możemy zastosować wyżej podany wzór Taylora dla $x_0=x$, żeby otrzymać
\begin{align*}
    \int\limits_{-\pi}^\pi\Big[g(\sin x)-g(\sin y)\Big]dy&=\int\limits_{-\pi}^\pi \Big[t\cos tx g(\sin tx)(y-tx)\Big]dy=\\
    &=t\cos tx\Big[{y^2\over 2}-txy\Big]_{-\pi}^\pi=t\cos tx\Big[{\pi^2\over 2}-tx\pi-{\pi^2\over 2}+tx\pi\Big]=\\
    &=t\cos tx\cdot 0=0
\end{align*}
Czyli całe wyrażenie też jest równe 0:
\begin{align*}
    \lim\limits_{t\to\infty}\int\limits_a^b\Big[g(\sin tx)-&{1\over2\pi}\int\limits_{-\pi}^{\pi}g(\sin y)dy\Big]dx=\lim\limits_{t\to\infty}\int\limits_a^b{1\over 2\pi}\int\limits_{-\pi}^\pi\Big[g(\sin tx)-g(\sin y)\Big]dydx=\\
    &=\lim\limits_{t\to\infty}\int\limits_a^b{1\over 2\pi}\cdot 0dx=\lim\limits_{t\to\infty}0=0.
\end{align*}

\podz{sep}
\medskip

W oryginalnej wersji zadania mamy pokazać to samo dla funkcji dwóch zmiennych:
$$\lim\limits_{t\to\infty}\int\limits_a^bf(x,\sin tx)dx=\frac1{2\pi}\int\limits_a^b\int\limits_0^{2\pi}f(x,\sin y)dydx.$$
Jeżeli znowu przeniesiemy wszystko na jedną stronę i wejdziemy pod $\lim$, to dostajemy
\begin{align*}
    \lim\limits_{t\to\infty}{1\over 2\pi}\int\limits_a^b\int\limits_{-\pi}^\pi\Big[f(x,\sin tx)-f(x,\sin y)\Big]dydx,
\end{align*}
czyli wystarczy pokazać, że
$$\int\limits_{-\pi}^\pi \Big[f(x,\sin tx)-f(x, \sin y)\Big]dy=0.$$
Tutaj, tak samo jak w przypadku $g(y)$, mamy funkcję która jest różnicą funkcji całkiem niezależnej od zmiennej po której całkujemy i funkcji która od tej zmiennej zależy tylko w jednej zmiennej. Czyli dla tego danego $x$ z zewnętrznej całki możemy użyć funkcji
$$h(y)=f(x, y)$$
i wtedy mamy
$$\int\limits_{-\pi}^\pi\Big[f(x, \sin tx)-f(x, \sin y)\Big]dy=\int\limits_{-\pi}^\pi\Big[h(\sin tx)-h(\sin y)\Big]dy,$$
a my wiemy, że jest to równe zero jak wyżej. Łącząc wszystko w jedną całość mamy
\begin{align*}
    \lim\limits_{t\to\infty}\int_a^bf(x,\sin tx)dx&-{1\over2\pi}\int\limits_a^b\int\limits_0^{2\pi}f(x,\sin y)dydx=\lim\limits_{t\to\infty}\int\limits_a^b\Big[f(x,\sin x)-{1\over2\pi}\int\limits_0^{2\pi}f(x,\sin y)dy\Big]dx=\\
    &=\lim\limits_{t\to\infty}{1\over2\pi}\int\limits_a^b\int\limits_0^{2\pi}\Big[f(x,\sin tx)-f(x,\sin y)\Big]dydx=\\
    &=\lim\limits_{t\to\infty}{1\over 2\pi}\int\limits_a^b0dx=\lim\limits_{t\to\infty}0=0.
\end{align*}
$$\lim\limits_{t\to\infty}\int_a^bf(x,\sin tx)dx={1\over 2\pi}\int\limits_a^b\int\limits_0^{2\pi}f(x,\sin y)dydx$$


\subsection*{ZAD. 17}

Zacznijmy od wskazówki, to znaczy pokazania, że
$$\lim\limits_{n\to\infty}n\Big(\int\limits_0^1f(x)dx-\frac1n\sum\limits_{i=1}^nf\Big(\frac in\Big)\Big)={f(1)-f(0)\over 2}$$

Według wzoru Taylora wiemy, że istnieje $\xi_i\in[{i-1\over n}, \frac in]$ takie, że
$$f(x)=f(\frac in)+f'(\xi_i)(x-\frac in)$$
$$f(x)-f(\frac in)=f'(\xi_i)(x-\frac in)$$

\begin{align*}
    \sum\limits_{i=1}^nf\Big(\frac in\Big)&-n\int\limits_0^1f(x)dx=\sum\limits_{i=1}^nf\Big(\frac in\Big)-n\sum\limits_{i=1}^n\int\limits_{{i-1\over n}}^\frac inf(x)dx=n\sum\limits_{i=1}^n\int\limits_{{i-1\over n}}^\frac in\big[f\Big(\frac in\Big)-f(x)\big]dx=\\
    &=n\sum\limits_{i=1}^n\int\limits_{{i-1\over n}}^\frac inf'(\xi_i)\big[x-\frac in\big]dx-{f(1)-f(0)\over2}=n\sum\limits_{i=1}^nf'(\xi_i)\Big[{x^2\over 2}-{i\over n}x\Big]_{{i-1\over n}}^\frac in=\\
    &=n\sum\limits_{i=1}^n\Big[f'(\xi_i){1\over2n^2}\Big]=\frac1{2n}\sum\limits_{i=1}^nf'(\xi_i)\xrightarrow{n\to\infty}\frac12\int\limits_0^1f'(x)dx={f(1)-f(0)\over 2}
\end{align*}
\medskip

\podz{sep}
\medskip

Popatrzmy teraz na wzór Taylora dla dwóch zmiennych. Podobnie jak wyżej, istnieje $(\xi_i,\alpha_k)\in[{i-1\over n}, \frac in]\times[{k-1\over n}, \frac kn]$ takie, że
$$f(x,y)=f(\frac in, \frac kn)+(x-\frac in)f_x(\xi_i,\alpha_k)+(y-\frac kn)f_y(\xi_i,\alpha_k)$$
$$f(x,y)-f(\frac in,\frac kn)=(x-\frac in)f_x(\xi_i,\alpha_k)+(y-\frac kn)f_y(\xi_i,\alpha_k)$$
gdzie $f_x$ to pierwsza pochodna względem $x$, a $f_y$ to pierwsze pochodna względem $y$.
\begin{align*}
    n\int\limits_0^1f(x,y)dxdy&-\frac1n\sum\limits_{i=1}^n\sum\limits_{k=1}^nf(\frac in, \frac kn)=n\sum\limits_{i=1}^n\sum\limits_{k=1}^n\int\limits_{{i-1\over n}}^\frac in\int\limits_{{k-1\over n}}^\frac knf(x,y)dxdy-\frac1n\sum\limits_{i=1}^n\sum\limits_{k=1}^nf(\frac in,\frac kn)=\\
    &=n\sum\limits_{i=1}^n\sum\limits_{k=1}^n\int\limits_{{i-1\over n}}^\frac in\int\limits_{{k-1\over n}}^\frac kn \Big[f(x,y)-f(\frac in,\frac kn)\Big]dxdy\stackrel{*}{=}\\
    &=n\sum\limits_{i=1}^n\sum\limits_{k=1}^n\int\limits_{{i-1\over n}}^\frac in\int\limits_{{k-1\over n}}^\frac kn\Big[(x-\frac in)f_x(\xi_i,\alpha_k)+(y-\frac kn)f_y(\xi_i,\alpha_k)\Big]dxdy=\\
    &=n\sum\limits_{i=1}^n\sum\limits_{k=1}^n\int\limits_{{i-1\over n}}^\frac inf_x(\xi_i,\alpha_k)\int\limits_{{k-1\over n}}^\frac kn[x-\frac in]dxdy+n\sum\limits_{i=1}^n\sum\limits_{k=1}^n\int\limits_{{i-1\over n}}^\frac inf_y(\xi_i,\alpha_k)\int\limits_{{k-1\over n}}^\frac kn[y-\frac kn]dxdy\stackrel{**}{=}\\
    &=n\sum\limits_{i=1}^n\sum\limits_{k=1}^n\int\limits_{{i-1\over n}}^\frac inf_x(\xi_i,\alpha_k)\frac1{2n^2}dy+n\sum\limits_{i=1}^n\sum\limits_{k=1}^n\int\limits_{{i-1\over n}}^\frac inf_y(\xi_i,\alpha_k)\int\limits_{{k-1\over n}}^\frac kn[y-\frac kn]dydx=\\
    &=n\sum\limits_{i=1}^n\sum\limits_{k=1}^n\frac1{2n^3}f_x(\xi_i,\alpha_i)+n\sum\limits_{i=1}^n\sum\limits_{k=1}^n\int\limits_{{i-1\over n}}^\frac inf_y(\xi_i,\alpha_k)\frac1{2n^2}dx=\\
    &={1\over2n^2}\sum\limits_{i=1}^n\sum\limits_{k=1}^n\frac1{2n^3}f_x(\xi_i,\alpha_i)+{1\over2n^2}\sum\limits_{i=1}^n\sum\limits_{k=1}^n\frac1{2n^3}f_y(\xi_i,\alpha_i)
\end{align*}
Przejście w $*$ jest wykorzystaniem wzoru Taylora wyprowadzonego wyżej, natomiast przejście $**$ jest na podstawie twierdzenie Fubiniego dot. funkcji ciągłej (jaką jest $f$) na prostokącie $[{i-1\over n}, \frac in]\times[{k-1\over n}, \frac kn]$.
\begin{align*}
    \lim\limits_{n\to\infty}\Bigg({1\over2n^2}\sum\limits_{i=1}^n\sum\limits_{k=1}^n\frac1{2n^3}f_x(\xi_i,\alpha_i)&+{1\over2n^2}\sum\limits_{i=1}^n\sum\limits_{k=1}^n\frac1{2n^3}f_y(\xi_i,\alpha_i)\Bigg)=\frac12\int\limits_0^1\int\limits_0^1f_x(x, y)dxdy+\frac12\int\limits_0^1\int\limits_0^1f_y(x,y)dxdy=\\
    &=\frac12\int\limits_0^1f(1,y)-f(0,y)dy+\frac12\int\limits_0^1f(x,1)-f(x,0)dx
\end{align*}


\end{document}