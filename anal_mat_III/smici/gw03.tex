\documentclass{article}[13pt]

\usepackage[T1]{fontenc}
\usepackage[utf8]{inputenc}
\usepackage{amssymb}
\usepackage[polish]{babel}

\usepackage{mathastext}
\usepackage{amsmath}
\usepackage{mathtools}
\usepackage{dsfont}

% fancy fonts in math mode
\usepackage{eufrak}
\usepackage{calrsfs}

\usepackage{graphicx} % for rotating \obet

\newcommand{\R}{\mathds{R}}
\newcommand{\N}{\mathds{N}}
\newcommand{\Q}{\mathds{Q}}
\newcommand{\Z}{\mathds{Z}}
\newcommand{\C}{\mathds{C}}

\usepackage{geometry}

\usepackage{tikz}

\usepackage{xcolor}

\author{Weronika Jakimowicz}
\title{Zadania z $\star\star$ LISTA 3}
\date{32 października 2022}

\geometry {
    a4paper,
    total={180mm, 267mm}
}

\usepackage{xcolor}
\definecolor{def}{HTML}{E86354}

\begin{document}
    \maketitle

    \section*{ZAD 10.}

    \begin{center}
    \begin{tikzpicture}
        \draw[thick](0,0)--(0,1.73);
        \draw[thick](0,1.73)--(1.5, 0.86);
        \draw[gray, very thin](0,1.73)--(1,0);

        \draw[thick] (1.5,0.86)--(2,1.73);
        \draw[thick](-1,0)--(0,0);
        \draw[thick] (0,3.46)--(1,3.46);
        \draw[thick](-1,3.46)--(0,3.46);
        \draw[thick](-1,0)--(-2,1.73);
        \draw[thick](-2,1.73)--(-1,3.46);
        \draw[thick](2,1.73)--(1,3.46);
        \draw[thick](0, 1.73) circle (1.73);

        \draw[red, ultra  thick] (0,0)--(1,0);
        \draw[cyan, ultra thick](1,0)--(1.5,0.86);

        \node at (0.5, -0.3) {\color{red}$a_1$};
        \node at (1.6, 0.35) {\color{cyan}$a_2$};
    \end{tikzpicture}
    \end{center}

    Rozważmy dwa trójkąty prostokątne z bokami $a_1,a_2$ zaznaczonymi na obrazku wyżej. Mają wspólną przeciwprostokątna, oraz dwie przyprostokątne będące promieniem okręgu na którym opisany jest sześciokąt. W takim razie, z twierdzenia Pitagorasa, mamy
    \begin{align*}
        a_1^2+1^2&=a_2^2+1\\
        a_1^2&=a^2_2
    \end{align*}
    a ponieważ rozważamy długości boków sześciokąta, to obie wartości są dodatnie, więc
    $$a_1=a_2.$$

    Zauważamy więc, że fragmenty boków od punktu wspólnego z okręgiem do wierzchołka będą parami równe. W dodatku, ponieważ mamy okrąg jednostkowy, jeśli oznaczymy przez $\alpha_1$ kąt naprzeciwko wierzchołka sześciokąta w każdym z takich kwadratów, to jego tangens jest równy długości $a_1$.

    \begin{center}
        \begin{tikzpicture}
            \draw[thick](0,0)--(0,1.73);
            \draw[thick](0,1.73)--(1.5, 0.86);
            \draw[gray, very thin](0,1.73)--(1,0);
    
            \draw[thick] (1.5,0.86)--(2,1.73);
            \draw[thick](-1,0)--(0,0);
            \draw[thick] (0,3.46)--(1,3.46);
            \draw[thick](-1,3.46)--(0,3.46);
            \draw[thick](-1,0)--(-2,1.73);
            \draw[thick](-2,1.73)--(-1,3.46);
            \draw[thick](2,1.73)--(1,3.46);
            \draw[thick](0, 1.73) circle (1.73);
    
            \draw[red, ultra  thick] (0,0)--(1,0);
            \draw[cyan, ultra thick](1,0)--(1.5,0.86);

            \node at (0.4, 0.6) {$\alpha_1$};
            \draw[thin] (0, 0.89) arc (60:150:-0.3);
    
            \node at (0.5, -0.3) {\color{red}$a_1$};
            \node at (1.6, 0.35) {\color{cyan}$a_1$};
        \end{tikzpicture}
        \end{center}
        $$\tan\alpha_1={a_1\over 1}=a_1\quad (1)$$

        Podzielmy więc sześciokąt na 6 par przystających trójkątów prostokątnych tak jak wyżej, gdzie para o numerze $k$ będzie miała bok na zewnątrz okręgu o długości $a_k$ oraz kąt w wierzchołku będącym środkiem okregu $\alpha_k$, analogicznie jak dwa obrazka wyżej. Pole całego sześciokąta to wtedy suma pól tych 12 trójkątów, a ponieważ wysokość każdego z nich jest równa $1$, możemy napisać
        $$P=2(\sum\limits_{i=1}^6 a_n)$$
        natomiast podstawiając z obserwacji $(1)$ dostaniemy
        $$P=2(\sum\limits_{i=1}^6\tan\alpha_k).$$
        Zauważmy, że suma oznaczanych przez nas kątów jest równa $2\pi$, ponieważ musi sumuwać się do pełnego kąta, a ponieważ każdy kąt powtarza się dwa razy, to mamy warunek:
        $$\sum\limits_{i=1}^6\alpha_i=\pi.$$

        Rozważamy więc funkcję 
        $$f(\alpha_1,...,\alpha_6)=2(\sum\limits_{i=1}^6\tan\alpha_k)$$
        przy warunku
        $$g(\alpha_1,...,\alpha_6)=\sum\limits_{i=1}^6\alpha_i=\pi.$$

        Korzystając z metody mnożników Lagrange'a otrzymamy następujący układ równań:
        \begin{align*}
            \begin{cases}
                \frac2{\cos^2 \alpha_1}=\lambda\\
                ...\\
                \frac2{\cos^2\alpha_6}=\lambda\\
                \sum\limits_{i=1}^6\alpha_i=\pi
            \end{cases}
        \end{align*}
        Ponieważ po lewej stronie każdego z równań mamy wartość która nie może być równa $0$ (dzielimy $1$ przez pewna wartość), to $\lambda\neq 0$.\smallskip\\

        Niech $i\neq j$, mamy
        \begin{align*}
            &\begin{cases}
                \frac2{\cos^2 \alpha_i}=\lambda\\
                \frac2{\cos^2 \alpha_j}=\lambda
            \end{cases}\\
            &\frac2{\cos^2 \alpha_i}=\frac2{\cos^2 \alpha_j}\\
            &\cos^2 \alpha_j=\cos^2 \alpha_i
        \end{align*}
        Są dwie możliwości:

        $$1.\;\cos\alpha_j=-\cos\alpha_i$$
        co znaczy, że część kątów jest ujemna, co nie może się zdarzyć.

        $$1.\;\cos\alpha_j=\cos\alpha_i$$
        i wtedy $\alpha_j=\alpha_i$ lub $\alpha_j=2\pi-\alpha_i$. Przyjżyjmy się napierw drugiej możliwości. Każdy $\alpha_i$ możemy wrazić za pomocą $\alpha_1$, co da nam
        $$\sum\limits_{i=1}^6\alpha_i=\alpha_1+5(2\pi-\alpha_1)=10\pi-4\alpha_1$$
        żeby to było zgodne z ograniczeniem, dostaniemy
        \begin{align*}
            10\pi-4\alpha_1&=\pi\\
            \frac94\pi&=\alpha_1
        \end{align*}
        ale wtedy
        $$\alpha_k=2\pi-\frac94\pi=-\frac14\pi$$
        wszystkie kąty poza pierwszym sa ujemne, co być nie może. Pozostaje nam więc 
        $$\alpha_i=\alpha_j,$$
        co spełnia warunek dla $\alpha_i=\frac\pi6$, a więc o najmniejsze pole podejrzewamy sześciokąt foremny.
        $$P=2\sum\limits_{i=1}^6\tan{\pi\over6}={12\over\sqrt{3}}\approx 6.928$$
        
        Sprawdźmy pole innego sześciokąta, np takiego, dla którego $\alpha_1={\pi\over4}$, a dla pozostałych $i$ $\alpha_i={3\over 20}\pi$
        $$P_2=2(\tan{\pi\over4}+5\tan{3\pi\over 20})\approx7.095>P$$
        wiemy więc, że znaleziona wartość na pewno nie jest maksimum pola. Aby istniała wartość mniejsza, musielibyśmy jeden kąt zmniejszony o $\epsilon$, a pozostałe wydłużone o jakąś część tego skrócenia. Ale zauważmy, że
        $$\tan(x+y)={\tan x+\tan y\over 1-\tan x\tan y}$$
        $$\tan(x-y)={\tan x - \tan y\over 1+\tan x\tan y}$$
        \begin{align*}
            P_3&=\tan(\frac\pi6-\epsilon)+\tan(\frac\pi6+\xi_2)+...+\tan(\frac\pi6+\xi_6)=\\
            &={\tan\frac\pi6-\tan\epsilon\over 1+\tan\frac\i6\tan\epsilon}+\sum\limits_{i=2}^6{\tan\frac\pi6+\tan\xi_i\over1-\tan\frac\pi6\tan\xi_i}
        \end{align*}
        i zauważmy, że to co dodamy jest większe niż to, co odejmiemy, więc dostajemy coś większego niż oryginalne $\frac\pi6$.
        


    \section*{ZAD 12.}

    Zauważmy, że jesli $f(x)\in P_2$, to $f$ możemy zapisać jako
    $$f(x)=ax^2+bx+c$$
    dla $a,b,c\in\R$ takich, że $a+b+c=1$. W takim razie, funkcja $\phi(f)$ sprowadza się do postaci:
    $$\phi(f)=\int\limits_0^1f(x)^2dx=\int\limits_0^1(ax^2+bx+c)^2dx,$$
    co z kolei jest równe:
    $$\phi(f)={a^2\over5}+{2ac+b^2\over3}+{ab\over2}+bc+c^2.$$

    Całe zadanie sprowadza się do znalezienia minimum funkcji trzech zmiennych
    $$F(a,b,c)={a^2\over5}+{2ac+b^2\over3}+{ab\over2}+bc+c^2$$
    przy warunku, że funkcja
    $$g(a,b,c)=a+b+c=1.$$

    Używając mnożników Lagrange'a dostajemy układ równań postaci
    \begin{align*}
        &\begin{cases}
            \frac25a+\frac23c+\frac b2-\lambda=0\\
            \frac23b+\frac a2+c-\lambda=0\\
            \frac23a+b+2c-\lambda=0\\
            a+b+c=1
        \end{cases}\\
        &\begin{cases}
            12a+20c+15b-30\lambda=0\\
            4b+3a+6c-6\lambda=0\\
            2a+3b+6c-3\lambda=0\\
            a+b+c=1
        \end{cases}
    \end{align*}

    {\color{def}1. $\lambda=0$}, wtedy
    \begin{align*}
        &\begin{cases}
            3a+4b+6c=0\\
            2a+3b+6c=0\\
            12a+15b+20c=0\\
            a+b+c=1
        \end{cases}\\
        &\begin{cases}
            a=-b\\
            20c=3a\\
            c=1
        \end{cases}\\
        &\begin{cases}
            a=\frac{20}3\\
            b=-\frac{20}3\\
            c=1
        \end{cases}
    \end{align*}
        $$F(\frac{20}3,-\frac{20}3,1)=\frac7{27}$$

    {\color{def}2. $\lambda\neq0$}
    \medskip

    Jeśli zapiszemy je w postaci macierzy, dostajemy:
    
    $$
        \begin{pmatrix}
            12 &15& 20& -30& 0\\
            3 & 4& 6& -6& 0\\
            2 &3& 6& -3& 0\\
            1& 1& 1& 0& 1
        \end{pmatrix}
    $$

    Korzystając z metody eliminacji Gaussa, dostajemy macierz

    $$
        \begin{pmatrix}
            12 & 15	& 20&	-30&	0\\
            0&	\frac14	&  1&	\frac32&	0\\
            0	&  0	&\frac23	 &-1&	0\\
            0	&  0	&  0&	\frac92	&1\\
        \end{pmatrix}
    $$

    Która daje nam poniższe równanie:
    $$
        \begin{cases}
            \lambda=\frac29\\
            \frac23c=\lambda\\
            \frac14b+c+\frac32\lambda=0\\
            12a+15b+20c-30\lambda=0
        \end{cases}
    $$

    $$
        \begin{cases}
            \lambda=\frac29\\
            c=\frac13\\
            b=-\frac83\\
            a=\frac{10}3
        \end{cases}
    $$

    Zauważmy, że zbiór $P_2$, oraz zbior wektorów z $\R^3$ o kolejnych współrzędnych będących współczynnikami wielomianów z $P_2$, jest niezwarty i nieograniczony. Musimy więc sprawdzić, co się dzieje kiedy
    $$\|(a,b,c)\|\to\infty$$
    $$b=1-a-c$$
    \begin{align*}
        F(a,b,c)&=\frac1{30}(6a^2+20ac+10b^2+15ab+30bc+30c^2)=\\
                &=\frac1{30}(\frac14(2a-5c-5)^2+\frac52c^2+\frac54(c-1)^2+\frac52)\to\infty
    \end{align*}
    Czyli dla wektorów o długości dążącej do nieskończoności mamy wartość $F$ dodatnią i dążącą do nieskończoności, więc nie istnieje wartość maksymalna.

    Wartość funkcji $F$ w punkcie który został otrzymany w powyższych obliczeniach wynosi
    $$F({10\over3},-\frac83,\frac13)=\frac19$$
    czyli jest niższa niż dla przypadku $\lambda=0$.
    \medskip
    

    Ponieważ warunek $x+y+z=1$ każe nam szukać rozwiązań na płaszczyznie, możemy uzależnić jedną zmienną od innych, np $x$ 
    $$x=1-z-y,$$
    oraz zbadać nową funkcję, de facto funkcję dwóch zmiennych. Nazwijmy ją $G(y,z)$, ze wzorem wynikłym ze wzoru na $F$:
    $$G(y,z)=\frac1{30}(y^2+7yz+3y+16z^2+8z+6).$$
    Hesjan takiej funkcji wynosi
    $$
        \begin{bmatrix}
            2 & 7\\
            7 & 32
        \end{bmatrix}=64-49>0
    $$
    i jest niezależny od $y,z$ oraz dodatni, więc funkcja na badanej płaszczyznie jest wypukła. W takim razie znalezione przeze mnie ekstremum to minimum.

\end{document}