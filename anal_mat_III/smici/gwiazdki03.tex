\documentclass{article}[13pt]

\usepackage{../uni-notes-eng}
\usepackage{multicol}
\usepackage{graphicx}

\begin{document}

    \section*{ZAD 12.}

    Zauwazmy, ze jesli $f(x)\in P_2$, to $f$ mozemy zapisac jako
    $$f(x)=ax^2+bx+c$$
    dla $a,b,c\in\R$ takich, ze $a+b+c=1$. W takim razie funkcja $\phi(f)$ sprowadza sie do postaci:
    $$\phi(f)=\int\limits_0^1f(x)^2dx=\int\limits_0^1(ax^2+bx+c)^2dx,$$
    co z kolei jest rowne:
    $$\phi(f)={a^2\over5}+{2ac+b^2\over3}+{ab\over2}+bc+c^2.$$

    Cale zadanie sprowadza sie do znalezienia minimum funkcji trzech zmiennych
    $$F(a,b,c)={a^2\over5}+{2ac+b^2\over3}+{ab\over2}+bc+c^2$$
    przy warunku, ze funkcja
    $$g(a,b,c)=a+b+c=1.$$

    Uzywajac mnoznikow Lagrange'a dostajemy uklad rownan postaci
    \begin{align*}
        &\begin{cases}
            \frac25a+\frac23c+\frac b2-\lambda=0\\
            \frac23b+\frac a2+c-\lambda=0\\
            \frac23a+b+2c-\lambda=0\\
            a+b+c=1
        \end{cases}\\
        &\begin{cases}
            12a+20c+15b-30\lambda=0\\
            4b+3a+6c-6\lambda=0\\
            2a+3b+6c-3\lambda=0\\
            a+b+c=1
        \end{cases}
    \end{align*}

    {\color{def}1. $\lambda=0$}, wtedy
    \begin{align*}
        &\begin{cases}
            3a+4b+6c=0\\
            2a+3b+6c=0\\
            12a+15b+20c=0\\
            a+b+c=1
        \end{cases}\\
        &\begin{cases}
            a=-b\\
            20c=3a\\
            c=1
        \end{cases}\\
        \begin{cases}
            a=\frac{20}3\\
            b=-\frac{20}3\\
            c=1
        \end{cases}
    \end{align*}
        $$F(\frac{20}3,-\frac{20}3,1)=\frac7{27}$$

    {\color{def}2. $\lambda\neq0$}
    \medskip

    Jesli zapiszemy je w postaci macierzy, dostajemy:
    
    $$
        \begin{pmatrix}
            12 &15& 20& -30& 0\\
            3 & 4& 6& -6& 0\\
            2 &3& 6& -3& 0\\
            1& 1& 1& 0& 1
        \end{pmatrix}
    $$

    Korzystajac z metody eliminacji Gaussa, dostajemy macierz

    $$
        \begin{pmatrix}
            12 & 15	& 20&	-30&	0\\
            0&	\frac14	&  1&	\frac32&	0\\
            0	&  0	&\frac23	 &-1&	0\\
            0	&  0	&  0&	\frac92	&1\\
        \end{pmatrix}
    $$

    Ktora daje nam ponizsze rownanie:
    $$
        \begin{cases}
            \lambda=\frac29\\
            \frac23c=\lambda\\
            \frac14b+c+\frac32\lambda=0\\
            12a+15b+20c-30\lambda=0
        \end{cases}
    $$

    $$
        \begin{cases}
            \lambda=\frac29\\
            c=\frac13\\
            b=-\frac83\\
            a=\frac{10}3
        \end{cases}
    $$

    Zauwazmy, ze zbior $P_2$, oraz zbior wektorow z $\R^3$ o kolejnych wspolrzednych bedacych wspolczynnikami wielomianow z $P_2$, jest niezwarty i nieograniczony. Musimy wiec sprawdzic, co sie dzieje kiedy
    $$\|(a,b,c)\|\to\infty$$
    Wtedy $a^2+b^2+c^2\to\infty$, a wiec 
    $$F(a,b,c)\rarrow{}{\|(a,b,c)\|\to\infty}\infty$$

    czyli wiemy, ze dla nieskonczenie dlugich wektorow wartosc funkcji jest nieskonczenie wysoka.
    \medskip

    Wartosc funkcji $F$ w punkcie ktory zostal otrzymany w powyzszych obliczeniach wynosi
    $$F({10\over3},\frac83,\frac13)=\frac19$$
    czyli jest nizsza niz dla przypadku $\lambda=0$.
    \medskip
    

    Poniewaz warunek $x+y+z=1$ kaze nam szukac rozwiazan na plaszczyznie, mozemy uzaleznic jedna zmienna od innych, np $x$ 
    $$x=1-z-y,$$
    oraz zbadac nowa funkcje, de facto funkcje dwoch zmiennych. Nazwijmy ja $G(y,z)$, ze wzorem wyniklym ze wzoru na $F$:
    $$G(y,z)=\frac1{30}(y^2+7yz+3y+16z^2+8z+6).$$
    Hesjan takiej funkcji wynosi
    $$
        \begin{bmatrix}
            2 & 7\\
            7 & 32
        \end{bmatrix}=64-49>0
    $$
    i jest niezalezny od $y,z$ oraz dodatni, wiec funkcja na badanej plaszczyznie jest wypukla. W takim razie znalezione przeze mnie ekstremum to minimum.

\end{document}