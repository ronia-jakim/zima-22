\documentclass{article}[16pt]

\usepackage{../../../uni-notes-eng}
\usepackage{multicol}
\usepackage{graphicx}

\title{Ujebanko przez kolanko}
\date{69}
\author{maruda}

\begin{document}
\maketitle

\subsection*{ZAD.2.}

Jezu nie rozumiem o co chodzi w tych zadaniach.
\medskip

Wiemy, że $S_m(t^2)=P_{2m}(t)$ oraz
$$\int\limits_{-a}^ap(x)P_{2m}(x)P_{2k}(x)dx=0$$
dla $m\neq k$. Czyli jest
\begin{align*}
    0&=\int\limits_{-a}^ap(x)P_{2m}(x)P_{2k}(x)dx=\int\limits_{-a}^ap(x)S_m(x^2)S_k(x^2)dx=2\int\limits_0^ap(x)S_m(x^2)S_k(x^2)dx=\\
    &=\begin{bmatrix} \sqrt u=x\\ {1\over 2\sqrt u}du=dx\end{bmatrix}=2\int\limits_0^{a^2}{p(\sqrt u)\over 2\sqrt u}S_m(u)S_k(u)du=\int\limits_0^{a^2}{p(\sqrt u)\over \sqrt u}S_m(u)S_k(u)du
\end{align*}
czyli jeśli $p'(x)={p(\sqrt x)\over \sqrt x}$, to wielomiany $S_m$ są ortogonalne na przedziale $[0,a^2]$.
\bigskip

Wiemy, że $P_{2m+1}=xR_m(x^2)$ 
\begin{align*}
    0&=\int\limits_{-a}^ap(x)P_{2m+1}(x)P_{2k+1}(x)dx=\int\limits_{-a}^ap(x)xR_m(x^2)xR_k(x^2)dx=\int\limits_{-a}^ap(x)x^2R_m(x^2)R_k(x^2)dx=\\
    &=2\int\limits_0^ap(x)x^2R_m(x^2)R_k(x^2)dx=\begin{bmatrix}\sqrt u=x\\{1\over 2\sqrt u}du=dx\end{bmatrix}=2\int\limits_0^{a^2}{p(\sqrt u)\over 2\sqrt u}uR_m(u)R_k(u)du=\\
    &=\int\limits_0^{a^2}{p(\sqrt u)}\sqrt{u}R_m(u)R_k(u)du
\end{align*}
czyli jeśli $p'(x)=\sqrt xp(x)$, to wielomiany $R_m$ sa ortogonalne na przedziale $[0, a^2]$.

\subsection*{ZAD. 3.}

% W tym zadaniu zakładam, że waga to $p(x)=(1-x^2)^{-\frac12}$?

% Kwadrat normy dowolnego $\overline T_n$ to
% $$\|\overline T_n\|^2=\int\limits_{-1}^1(1-x^2)^{-\frac12}2^{2-2n}T_n(x)T_n(x)dx=2^{2-2n}\int\limits_{-1}^1(1-x^2)^{-\frac12}T_n(x)T_n(x)dx$$
% wiemy, że $\int\limits_{-1}^1(1-x^2)^{-\frac12}T_n(x)T_n(x)dx=\frac\pi2$ (lub $\pi$ gdy $n=0$).
% \medskip

Załóżmy teraz, że istnieje wielomian
$$s(x)=x^n+a_{n-1}x^{n-1}+...+a_1x+a_0$$
taki, że $\|s(x)\|^2<\|\overline T_n\|^2$. Ponieważ, tak jak na liście 7, wielomiany $\overline T_n$ są wielomianami ortogonalnymi, to
$$s(x)=\sum\limits_{i=0}^nb_iT_i=b_nT_n+\sum\limits_{i=0}^{n-1}b_iT_i,$$
ale ponieważ $s$ jak i $T_n$ mają przy $x^n$ jedynkę, to $b_n=1$, czyli
$$s(x)=T_n+\sum\limits_{i=0}^{n-1}b_iT_i$$
Mamy więc
\begin{align*}
    \|s\|^2&=\langle s,s\rangle=\langle\sum\limits_{i=0}^nb_iT_i,\sum\limits_{j=0}^nb_jT_j\rangle=\sum\limits_{i=0}^n\sum\limits_{j=0}^nb_ib_j\langle T_i,T_j\rangle=\sum\limits_{i=0}^nb_i^2\langle T_i,T_i\rangle=\\
    &=\langle T_n,T_n\rangle+\sum\limits_{i=0}^{n-1}b_i^2\langle T_i,T_i\rangle=\|T_n\|^2+\sum\limits_{i=0}^{n-1}b_i^2\|T_i\|^2,
\end{align*}
ale ponieważ $s(x)\neq T_n(x)$, to co najmniej jedno $b_i$ musi być różne od zera, a więc mamy, że $\|s\|=\|T_n\|+A$, gdzie $A>0$, czyli każdy wielomian z wiodącym $x^n$ ma normę większą niż $T_n$.

\end{document}