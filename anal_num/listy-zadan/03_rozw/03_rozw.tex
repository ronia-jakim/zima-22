\documentclass{article}[13pt]

\usepackage{../../../uni-notes-eng}
\usepackage{multicol}
\usepackage{graphicx}

\begin{document}
    \section*{ZAD 1.}
    
    Niech $f(x)=\frac 1x-c$. Zauwazajac, ze $f(\frac1c)=0$, mozemy przyblizyc $\frac1c$ uzywajac metody Newtona dla funkcji $f$.
    \medskip

    Rozwazmy wiec punkt $x_1$, ktory przyblizamy za pomoca:
    $$x_1=x_0-{f(x_0)\over f'(x_0)}=x_0+{\frac1{x_0}-c\over \frac1{x_0^2}}=x_0+{x^2_0-cx_0^3\over x_0}=x_0+x_0-cx_0^2=2x_0-cx_0^2$$
    Czyli w punkcie $x_0$ rysujemy styczna do funkcji $f$ i szukamy jej punktu przeciecia z osia OX.
    \bigskip

    \begin{align*}
        x_2&=x_1(2-cx_1)=\\
        &=(x_0(2-cx_0))(2-c(x_0(2-cx_0)))=\\
        &=(2x_0-cx_0^2)(2-2cx_0+c^2x_0^2)=\\
        &=4x_0-4cx_0^2+2c^2x_0^3-2cx_0^2+2c^2x_0^3-c^3x_0^4=\\
        &=4x_0-6cx_0^2+4c^2x_0^3-c^3x_0^4
    \end{align*}

    Czyli dla coraz to wiekszego $n$ mamy coraz to wieksze potegi $x_0$, czyli generalnie to nie bedziemy chcieli brac zbyd duzego $x_0$, najlepiej cos ponizej $1$. W dodatku przy wyzszych potegach $x_0$ mamy wyzsza potege $c$, przy czym potega przy $x_0$ jest zwykle o jeden wieksza. Mozemy przewidziec, ze tak bedzie dalej. Najlepiej wiec, zeby $x_0^2\cdot c<1$, a im mniejsze tym lepiej.


    \section*{ZAD 2.}

    Ciag przyblizen za pomoca metody newtona jest zbiezny liniowo do pierwiastka funkcji $f$.
    
    $$x_{x+1}=x_n-{f(x_n)\over f'(x_0)}$$
    Chce, zeby bylo liniowe

\end{document}