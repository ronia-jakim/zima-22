\documentclass{article}[13pt]

\usepackage{../../../uni-notes-eng}
\usepackage{multicol}
\usepackage{graphicx}

\begin{document}
    \section*{ZAD 1.}
    
    Niech $f(x)=\frac 1x-c$. Zauwazajac, ze $f(\frac1c)=0$, mozemy przyblizyc $\frac1c$ uzywajac metody Newtona dla funkcji $f$.
    \medskip

    Rozwazmy wiec punkt $x_1$, ktory przyblizamy za pomoca:
    $$x_1=x_0-{f(x_0)\over f'(x_0)}=x_0+{\frac1{x_0}-c\over \frac1{x_0^2}}=x_0+{x^2_0-cx_0^3\over x_0}=x_0+x_0-cx_0^2=2x_0-cx_0^2$$
    Czyli w punkcie $x_0$ rysujemy styczna do funkcji $f$ i szukamy jej punktu przeciecia z osia OX.
    \bigskip

    \begin{align*}
        x_2&=x_1(2-cx_1)=\\
        &=(x_0(2-cx_0))(2-c(x_0(2-cx_0)))=\\
        &=(2x_0-cx_0^2)(2-2cx_0+c^2x_0^2)=\\
        &=4x_0-4cx_0^2+2c^2x_0^3-2cx_0^2+2c^2x_0^3-c^3x_0^4=\\
        &=4x_0-6cx_0^2+4c^2x_0^3-c^3x_0^4
    \end{align*}

    Czyli dla coraz to wiekszego $n$ mamy coraz to wieksze potegi $x_0$, czyli generalnie to nie bedziemy chcieli brac zbyd duzego $x_0$, najlepiej cos ponizej $1$. W dodatku przy wyzszych potegach $x_0$ mamy wyzsza potege $c$, przy czym potega przy $x_0$ jest zwykle o jeden wieksza. Mozemy przewidziec, ze tak bedzie dalej. Najlepiej wiec, zeby $x_0^2\cdot c<1$, a im mniejsze tym lepiej.


    \section*{ZAD 2.}

    Ciag przyblizen za pomoca metody newtona jest zbiezny liniowo do pierwiastka funkcji $f$.
    
    $$x_{x+1}=x_n-{f(x_n)\over f'(x_0)}$$
    Chce, zeby bylo liniowe

    \section*{ZAD 4.}

    Aby ciag byl zbiezny kwadratowo, musi zachodzic dla $\alpha=0$
    $${x_{n+1}-\alpha\over (x_n-\alpha)^2}\to K\in(0,\infty).$$

    \podz{sep}

    $${1\over n^2}$$
    $${\frac1{n^2}\over (\frac1{n^2})^2}={n^4\over n^2}\to\infty$$
    Czyli ten ciag nie jest zbiezny kwadratowo.\medskip

    \podz{sep}\medskip

    $${1\over2^{2^n}}$$
    $${2^{2^n}\over2^{2^{n+1}}}=2^{2^n-2^{n+1}}=2^{2^n(1-2)}=2^{-2^n}\to 0$$

    Czyli ciag jest zbiezny kwadratowo\medskip

    \podz{sep}\medskip

    $$\frac1{\sqrt{n}}$$
    $${\frac1{\sqrt{n}}\over \frac1n}={n\over\sqrt{n}}\to\infty$$

    Czyli ciag nie jest zbiezny kwadratowo\medskip

    \podz{sep}\medskip

    $${1\over e^n}$$
    $${{1\over e^n}\over {1\over 3^{2n}}}={e^{2n}\over e^n}=e^n\to\infty$$

    Czyli znowu nie zbiezny kwadratowo\medskip

    \podz{sep}\medskip
    
    $$\frac1{n^n}$$
    $${\frac1{n^n}\over \frac1{n^{2n}}}={n^{2n}\over n^n}=n^n\to\infty$$

    \section*{ZAD 6.}

    $$x_{n+1}=x_n-rf(x_n)$$

    Ciag jest zbiezny liniowo, gdy 
    $${x_{n+1}-\alpha\over x_n-\alpha}\to K\quad 0<K<1$$

    Szukamy $\alpha$ takiego, ze $f(\alpha)=0$. Rozpatrzmy funkcje $\phi$ taka, ze
    $$\phi(\alpha)=\alpha$$
    czyli miejsce zerowe $f$ to jej punkt staly.
    Dalej, niech
    $$\phi(x_n)=x_{n+1},$$
    $$\phi(x_n)=x_n+rf(x_n).$$

    Zauwazmy, ze istnieje $\xi\in[\alpha,x_n]$ takie, ze
    $$x_{n+1}=\phi(\alpha)+\phi'(\xi)(x_n-\alpha)+...+{\phi^{(p)}(\xi)\over p!}(x_n-\alpha)^p$$
    natomiast dla zbieznosci liniowej potrzebne nam jest $p=1$, wiec
    $$\phi(x_n)=\alpha+\phi'(\xi)(x_n-\alpha)$$
    $$\phi(x_n)-\alpha=\phi'(\xi)(x_n-\alpha)$$

    Sprawdzmy kiedy funkcja $\phi(x_n)$ zbiega do $\alpha$, czyli kiedy
    $$|\phi'(x)|=|1-rf'(x)|<1$$
    bo w przeciwnym wypadku mamy ciag ktory nie jest zbiezny liniowo. 
    \begin{align*}
        -1<&1-rf'(x)<1\\
        0<&rf'(x)<2
    \end{align*}
    

\end{document}