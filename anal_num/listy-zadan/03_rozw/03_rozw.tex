\documentclass{article}[13pt]

\usepackage{../../../uni-notes-eng}
\usepackage{multicol}
\usepackage{graphicx}


\begin{document}
    ZAZNACZYLAM WSZYSTKO POZA 1 I 7!!!

    \section*{ZAD 1.}
    
    Niech $f(x)=\frac 1x-c$. Zauwazajac, ze $f(\frac1c)=0$, mozemy przyblizyc $\frac1c$ uzywajac metody Newtona dla funkcji $f$.
    \medskip

    Rozpatrzmy ciag okreslony:
    $$x_{n+1}=x_n-{f(x_n)\over f'(x_n)}=x_n-{\frac 1{x_n}-c\over -\frac 1{x_n^2}}=x_n+{(1-cx_n)x_n^2\over x_n}=x_n+x_n-cx_n^2=x_n(2-cx_n)$$

    Niech $\phi$ bedzie funkcja taka, ze $\phi(\frac 1c)=\frac 1c$ oraz
    $$\phi(x_n)=x_{n+1}$$
    $$\phi(x)=x(2-cx)$$

    Aby rozpatrywany przez nas ciag byl zbiezny, funkcje $\phi$ musi byc funkcja zwiezajaca [kontrakcja], czyli szukamy zbioru $D$ takiego, ze dla $x_{n},\frac1c\in D$ oraz liczby $K$ dla ktorej zachodzi
    $$|x_{n+1}-\phi(\frac1c)|=|\phi(x_n)-\frac1c|<K|x_n-\frac1c|$$
    i dalej
    $$|x_{n}-\frac1c|<K^2|x_{n-1}-\frac1c|<K^3|x_{n-1}-\frac1c|<...<K^n|x_1-\frac1c|$$

    \begin{align*}
        |\phi(x_{n+1})-\frac1c|&<K|x_n-\frac1c|\\
        \Big |{\phi(x_{n+1})-\frac1c\over x_n-\frac1c}\Big |&<K\\
        \Big |{x_n(2-cx_n)-\frac1c\over x_n-\frac1c}\Big |&<K
    \end{align*}


    Uzyjmy szeregu Taylora do przyblizenia funkcji $\phi$ w okolicy $\alpha=\frac1c$:
    \begin{align*}
        x_{n+1}=\phi(x_n)=\phi(\alpha)+\phi'(\alpha)(x_n-\alpha)+{\phi''(\alpha)\over 2!}(x_n-\alpha)^2+...+{\phi^{(p1)(\alpha)}\over (p-1)!}(x_n-\alpha)^{p-1}+{\phi^{(p)}(\xi_n)\over p!}(x_n\alpha)^p
    \end{align*}


    \section*{ZAD 2.}

    Ciag przyblizen za pomoca metody newtona jest zbiezny liniowo do pierwiastka funkcji $f$.
    
    $$x_{x+1}=x_n-{f(x_n)\over f'(x_0)}$$
    Chce, zeby bylo liniowe.
    \medskip

    Rozpatrzmy funkcje dla $\alpha$ takiego, ze $f(\alpha)=0$:
    \begin{align*}
        \phi(\alpha)&=\alpha\\
        \phi(x_n)&=x_{n+1}\\
        \phi(x)&=x-{f(x)\over f'(x)}
    \end{align*}
    czyli w szczegolnosci $f(\alpha)=\alpha-{f(\alpha)\over f'(\alpha)}$.
    $$\phi'(x)=1-{f'(x)^2-f(x)f''(x)\over f'(x)^2}={f(x)f''(x)\over f'(x)^2}$$
    $$\phi'(\alpha)={f(\alpha)f''(\alpha)\over f'(\alpha)^2}.$$
    
    Ciag jest zbiezny liniowo, gdy
    $$\Big |{x_{n+1}-\alpha\over x_n-\alpha}\Big |\to K < 1\quad K>0$$

    $$\lim\limits_{n\to\infty}{E_{n+1}\over E_n}=\lim\limits_{n\to\infty}{x_{n+1}-\alpha\over x_n-\alpha}=\phi'(\alpha)\in (0,1)$$

    Czyli zauwazmy, ze aby to bylo prawdziwe, to $f'(\alpha)\neq 0$, czyli $\alpha$ jest pojedynczym pierwiastkiem funkcji ktora ma co najmniej 2 niezerowe pochodne, np $x^2$ dla $\alpha=0$.
    \medskip

    Wtedy
    $$\phi(x)=x-{x^2\over 2x}=x-\frac x2=\frac x2$$
    $$x_{n+1}=\frac{x_n}2-\alpha=\frac{x_n}2$$
    czyli $E_{n+1}=\frac12E_n$, a wiec
    $${E_{n+1}\over E_n}\to \frac12$$
    co spelnia wymagania.

    \section*{ZAD 3.}

    Zalozmy, ze mamy funckje rosnaca, wtedy regula falsi daje nam ciag
    $$x_1={af(b)-bf(a)\over f(b)-f(a)}$$
    $$x_{n+1}={x_n f(b)-bf(x_n)\over f(b)-f(x_n)}$$
    Dla wygody oznaczmy $B=f(b)$
    $$x_{n+1}=x_n{B\over B-f(x_n)}-b{f(x_n)\over B-f(x_n)}$$
    rozpatrzmy funkcje $\phi$ taka, ze
    \begin{align*}
        f(\alpha)&=0\\
        \phi(\alpha)&=\alpha\\
        \phi(x)&=x{B\over B-f(x)}-b{f(x)\over B-f(x)}\\
        \phi(x_n)&=x_{n+1}
    \end{align*}

    Aby metoda byla zbiezna liniowo, to 
    $$\lim\limits_{n\to\infty}{E_{n+1}\over E_n}=\lim\limits_{n\to\infty}\Big |{x_{n+1}-\alpha\over x_n-\alpha}\Big |=\phi'(\alpha)= K\in (0,1)$$

    $$\phi'(x)={Bf'(x)\over B-f(x)}$$
    $$\phi'(\alpha)={Bf'(\alpha)\over B}=f'(\alpha)$$
    co jest prawda na przyklad dla funkcji
    $$f(x)=x^3-1$$
    

    \section*{ZAD 4.}

    Aby ciag byl zbiezny kwadratowo, musi zachodzic dla $\alpha=0$
    $${x_{n+1}-\alpha\over (x_n-\alpha)^2}\to K\in(0,\infty).$$

    \podz{sep}

    $${1\over n^2}$$
    $${\frac1{n^2}\over (\frac1{n^2})^2}={n^4\over n^2}\to\infty$$
    Czyli ten ciag nie jest zbiezny kwadratowo.\medskip

    \podz{sep}\medskip

    $${1\over2^{2^n}}$$
    $${2^{2^n}\over2^{2^{n+1}}}=2^{2^n-2^{n+1}}=2^{2^n(1-2)}=2^{-2^n}\to 0$$

    Czyli ciag jest zbiezny kwadratowo\medskip

    \podz{sep}\medskip

    $$\frac1{\sqrt{n}}$$
    $${\frac1{\sqrt{n}}\over \frac1n}={n\over\sqrt{n}}\to\infty$$

    Czyli ciag nie jest zbiezny kwadratowo\medskip

    \podz{sep}\medskip

    $${1\over e^n}$$
    $${{1\over e^n}\over {1\over 3^{2n}}}={e^{2n}\over e^n}=e^n\to\infty$$

    Czyli znowu nie zbiezny kwadratowo\medskip

    \podz{sep}\medskip
    
    $$\frac1{n^n}$$
    $${\frac1{n^n}\over \frac1{n^{2n}}}={n^{2n}\over n^n}=n^n\to\infty$$


    \section*{ZAD 5.}

    Jest to jedyny pierwiastek, gdyz w przeciwnym wypadku w pewnym punkcie funkcja $f$ musialaby sie przegiac, wiec miec zerowa pierwsza pochodna.
    \medskip

    Poniewaz $f''(x)>0$, to funkcja $f$ jest wypukla. W dodatku, $f'(x)>0$, wiec funkcja ta jest rosnaca. Oznaczmy blad metody Newtona przez
    $$e_n=x_n-\alpha.$$
    W zalozeniu mamy, ze $f''$ jest ciagla i pokazalismy, ze $\alpha$ jest jedynym pierwiastkiem tej funkcji. Z definicji wyrazu w metodzie Newtona mamy
    \begin{align*}
        e_{n+1}=x_{n+1}-\alpha=x_n-{f(x_n)\over f'(x_n)}-\alpha=e_n-{f(x_n)\over f'(x_n)}
    \end{align*}
    a na mocy wzoru Taylora:
    \begin{align*}
        0=f(\alpha)=f(x_n-e_n)=f(x_n)-e_nf'(x_n)+\frac12e_n^2f''(\xi_n)\quad(1)
    \end{align*}
    dla pewnego $\xi_n\in[x_n,r]$. To daje
    $$e_nf'(x_n)-f(x_n)=\frac12f''(\xi_n)e^2n$$
    $$e_{n+1}=\frac12{f''(\xi_n)\over f'(x_n)}e_n^2\approx\frac12{f''(\alpha)\over f'(\alpha)}e_n^2=Ce^2_n\quad(2)$$
    czyli ta metoda jest zbiezna kwadratowo.
    \medskip

    Wobec wzoru $(2)$ mamy
    $$e_{n+1}>0,$$ czyli $x_n>r$ dla $n\geq 1$. Z tej wlasnosci wynika tez, ze $f(x_n)>f(\alpha)=0$ Dlatego, na mocy $(1)$ $e_{n+1}<e_n$. W takim razie ciag bledow jest malejacy i ograniczny z dolu. Dzieki temu, granice
    $$e=\lim\limits_{n\to\infty}e_n$$
    $$x=\lim\limits_{n\to\infty}x_n$$
    istnieja. Z $(1)$ wynika tez, ze $e=e-{f(x)\over f'(x)}$, czyli $f(x)=0$ i $x=\alpha$.

    \section*{ZAD 6.}

    $$x_{n+1}=x_n-rf(x_n)$$

    Ciag jest zbiezny liniowo, gdy 
    $${x_{n+1}-\alpha\over x_n-\alpha}\to K\quad 0<K<1$$

    Szukamy $\alpha$ takiego, ze $f(\alpha)=0$. Rozpatrzmy funkcje $\phi$ taka, ze
    $$\phi(\alpha)=\alpha$$
    czyli miejsce zerowe $f$ to jej punkt staly.
    Dalej, niech
    $$\phi(x_n)=x_{n+1},$$
    $$\phi(x_n)=x_n+rf(x_n).$$

    Zauwazmy, ze istnieje $\xi\in[\alpha,x_n]$ takie, ze
    $$x_{n+1}=\phi(\alpha)+\phi'(\xi)(x_n-\alpha)+...+{\phi^{(p)}(\xi)\over p!}(x_n-\alpha)^p$$
    natomiast dla zbieznosci liniowej potrzebne nam jest $p=1$, wiec
    $$\phi(x_n)=\alpha+\phi'(\xi)(x_n-\alpha)$$
    $$\phi(x_n)-\alpha=\phi'(\xi)(x_n-\alpha)$$

    Sprawdzmy kiedy funkcja $\phi(x_n)$ zbiega do $\alpha$, czyli kiedy
    $$|\phi'(x)|=|1-rf'(x)|<1$$
    bo w przeciwnym wypadku mamy ciag ktory nie jest zbiezny liniowo. 
    \begin{align*}
        -1<&1-rf'(x)<1\\
        0<&rf'(x)<2
    \end{align*}
    

    \section*{ZAD 8.}

    Rozwazmy funkcje
    $$f(x)=x^2-R$$
    jej miejscem zerowym jest $\sqrt R$. Dalej, niech
    $$g(x)={f(x)\over f'(x)}={x^2-R\over\sqrt{2x}}$$
    Miejscem zerowym funkcji $g$ jest nadal liczba $\sqrt R$. Jesli rozwazymy metode Newtona dla $g$, to otrzymamy
    \begin{align*}
        x_{n+1}=x_n-{(x^2-R)2x_n\over4x^4_n-(x^2-R)}=x_n-{2x_n^3-2Rx_n\over3x_n^2+R}={4x_n^3+Rx_n-2x_n^3+2Rx_n\over3x_n^2+R}=x_n{x_n^2+3R\over3x_n^2+R}
    \end{align*}
    Czyli chcemy sprawdzic, czy metoda Newtona dla funkcji $g$ jest zbiezna szesciennie. Tak jak poprzednio, potrzebujemy funkcji 
    $$\phi(x)=x-{2x^3-2Rx\over 3x^2+R}$$
    i zeby byla to zbieznosc szescienna, to
    $$\frac16\phi^{(3)}(\sqrt{R})<\infty$$
    \begin{align*}
        \phi^{(3)}(\sqrt{R})&=-(12\sqrt{R}+48\sqrt{R}^3-24\sqrt{R}R)(3R+R)^{-2}+2(3R+R)^{-3}(6R-2R+12R^2-12R^2)6\sqrt{R}=\\
        &=2{3\sqrt{R}\cdot R\over 16R^3}-{\sqrt{R}(12+24R)\over16R^2}={6R^{\frac32}-12R^{\frac32}-24R^{\frac52}\over16R^3}={-6R^{\frac32}-24R^{\frac52}\over16R^3}={-6R^{-1}-24\over16R^{\frac12}}<\infty
    \end{align*}

    \section*{ZAD 9.}

    Metoda Newtona dla funkcji
    $$g={f\over \sqrt{f'}}$$
    $$g'={2(f')^2-f''f\over 2(f')^{\frac32}}$$
    wyglada nastepujaco:
    $$x_{n+1}=x_n-{g\over g'}=x_n-{f\over \sqrt{f'}}{2(f')^{\frac32}\over2(f')^2-f''f}=x_n-{f'f\over(f')^2-{f''f\over2}}$$

\end{document}