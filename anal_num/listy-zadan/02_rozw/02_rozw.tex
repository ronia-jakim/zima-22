\documentclass{article}[13pt]

\usepackage{../../../uni-notes-eng}
\usepackage{multicol}
\usepackage{graphicx}

\begin{document}
    \section*{ZAD 1.}
    Po pierwsze zauwazmy, ze dla $u<1$, co $u=2^{-t-1}$ z pewnoscia spelnia, zachodzi
    $$u\leq {u\over 1-u}$$
    $$u(1-u)\leq u$$
    $$u-u^2\leq u$$
    $$0\leq u^2.$$
    Dla $n=1$ dziala, zalozmy wiec, ze dla wszystkich $n$ mamy
    $$1-{nu\over 1-nu}\leq\prod\limits_{j=1}^n(1+\alpha_j)^{p_j}\leq 1+{nu\over 1-nu}$$
    Wtedy mamy
    \begin{align*}
        \prod\limits_{j=1}^{n+1}(1+\alpha_j)^{p_j}&=1+\theta_{n+1}=\\
        &=(1+\alpha_{n+1})^{p_j}(1+\theta_n)
    \end{align*}
    czyli mamy, ze 
    \begin{align*}
        \theta_{n+1}&\leq\alpha_{n+1}+\theta_n+\alpha_{n+1}\theta_n\leq\\
        &\leq {u\over 1-u}+{nu\over 1-nu}+{nu\over1-nu}{u\over 1-u}=\\
        &={u(1-nu)+nu(1-u)+nu^2\over (1-nu)(1-u)}=\\
        &={u-nu^2+nu-nu^2+nu^2\over (1-nu)(1-u)}=\\
        &={u(1-nu+n)\over (1-nu)(1-u)}
    \end{align*}
    ale ja potrzebuje
    $$\theta_{n+1}\leq {(n+1)u\over 1-(n+1)u}$$

    To sprawdzmy, czy
    \begin{align*}
        {u(1-nu+n)\over (1-nu)(1-u)}&\leq{(n+1)u\over 1-(n+1)u}\\
        {(1-nu+n)\over (1-nu)(1-u)}&\leq{(n+1)\over 1-(n+1)u}\\
    \end{align*}
    $$1-nu+n\leq n+1$$
    bo $nu<1$, czyli
    $$1-nu+n\leq n\leq n+1$$
    oraz
    $$(1-nu)(1-u)=1-u-nu+nu^2\geq 1-nu-u$$
    co jest prawda, bo $nu^2\geq 0$.
    \medskip

    To mamy ogranicznie od gory, teraz musze zrobic od dolu
    \begin{align*}
        \prod\limits_{j=1}^{n+1}(1+\alpha_j)^{p_j}&=1+\theta_{n+1}=\\
        &=(1+\theta_n)(1+\alpha_{n+1})^{p_{n+1}}\\
        &\geq(1+\theta_n)(1+\alpha_{n})^{-1}\geq\\
        &\geq(1-{nu\over 1-nu})(1+{u\over 1-u})^{-1}=\\
        &={1-2nu\over 1-nu}({1\over 1-u})^{-1}=\\
        &={(1-2nu)(1-u)\over 1-nu}
    \end{align*}
    teraz potrzebuje pokazac, zeby
    $${(1-2nu)(1-u)\over 1-nu}\geq 1-{(n+1)u\over{1-(n+1)u}}={1-2(n+1)u\over 1-(n+1)u}$$
    $$(1-2nu)(1-u)(1-nu-u)\geq (1-nu)(1-2(n+1)u)$$
    $$(1-nu-u)(1-u-2nu+2nu^2)\geq (1-nu)(1-2nu-2u)$$
    $$(1-nu)(1-u-2nu+2nu^2)-u(1-u-2nu+2nu^2)\geq(1-nu)(1-2nu-2u)$$
    

    \section{ZAD 2.}

    Znowu indukcja, czyli zakladamy, ze zachodzi dla wszystkich $n$, wtedy mamy
    \begin{align*}
        \prod\limits_{j=1}^{n+1}(1+\alpha_J)^{p_j}=1+\theta_{n+1}=(1+\theta_n)(1+\alpha_{n+1})=1+\theta_n+\alpha_{n+1}+\theta_n\alpha_{n+1}
    \end{align*}
    czyli
    $$\theta_{n+1}=\theta_n+\alpha_{n+1}+\theta_n\alpha_{n+1}\leq1.01nu+u+1.01nu^2$$
    Wystarczy mi, zeby



    \section*{ZAD 3. x, y - liczby maszynowe takie, ze $|y|\leq \frac12 u|x|$, pokazac, ze $fl(x+y)=x$}

    Zakladam sobie, ze $x>0$, bo tak mi latwiej bedzie w zyciu.

    $$fl(x+y)=(x+y)(1+\epsilon)$$

    \begin{align*}
        (x+y)(1+\epsilon)\leq (x+y)(1+u)\leq &(x+\frac12ux)(1+u)=\\
        &=x(1+\frac12u)(1+u)=\\
        &=x(1+u+\frac12u+\frac12 u^2)=\\
        &=x+\underbrace{xu(1+\frac12u+\frac12u^2)}
    \end{align*}
    zaznaczony fragment jest w okolicach bedu bezwzglednego pomiaru, wiec mozemy go pominac

    \begin{align*}
        (x+y)(1+\epsilon)\geq (x+y)(1-u)\geq &(x-\frac12ux)(1-u)=\\
        &x(1-\frac12u)(1-u)=\\
        &=x(1-u-\frac12u+\frac12u^2)=\\
        &=x-xu(1-\frac12u+\frac12u^2)
    \end{align*}
    i takiie samo wytlumaczenie jak poprzednio. Czyli mamy wyrazenie ograniczone od gory i od dolu przez x, czyli jest rowne x 
    
    :v

    \section*{ZAD 6.}

    Chce pokazac, ze to, co wypluje komputer rozni sie niewiele od tego, co powinno sie pojawic na ekranie :v
    \medskip

    Po pierwsze, przy podawaniu danych mamy blad bo nie kazda liczba zmiennoprzecikowa jest liczba maszynowa, czyli podajemy
    $$a_n'=a_n+{|a_n'-a_n|\over |a_n|}\leq a_n+{u\over 1-u}$$

    Algorytm Hornera to tak naprawde
    $$fl(a0+x(a_1+x(a_2+...+x(a_{n-1}+x\cdot a_n))))=(a_0+fl(x\cdot(a_1+...)))(1+\epsilon_0)$$

    \section*{ZAD 8.}
    a) $f(x)={1\over x^2+c}$

    Jezeli $c>=0$ to smiga, ale jesli $c<0$, to w okolicy $x=\sqrt{-c}$ caly przyklad sie jebie.

    $${d\over dx}{1\over x^2+c}=-{2x\over (c+x^2)^2}$$
    $$C_f(x)={2x\over (c+x^2)^2}\cdot{x^2+c\over 1}={2x\over c+x^2}={2\over \frac{c}{x}+x}$$
    \medskip

    b) $f(x)={1-\cos x\over x^2}$
    $${d\over dx}{1-\cos x\over x^2}={x\sin x+2\cos x-2\over x^3}$$
    $$C_f(x)={x\sin x+2\cos x-2\over x^3}\cdot {x^2\over 1-\cos x}={x\sin x+2\cos x-2\over x(1-cos x)}$$
    Jebie sie dla $x = cos^{-1}1$, czyli dla $x=(2k+1){\pi\over 2}$ dla $k=0,1, ..$
    $$\lim\limits_{x\to 0}{x\sin x+2\cos x-2\over x(1-cos x)}=\lim\limits_{x\to 0}{x\cos x-\sin x\over x\sin x-\cos x+1}=\lim\limits_{x\to 0}{-x\sin x\over 2\sin x+x\cos x}=\lim\limits_{x\to 0}{-\sin x-x\cos x \over 3\cos x-x\sin x}=\frac03=0$$
\end{document}