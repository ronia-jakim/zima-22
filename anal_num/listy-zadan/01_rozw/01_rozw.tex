\documentclass{article}[13pt]

\usepackage{../../../uni-notes-eng}
\usepackage{multicol}
\usepackage{graphicx}

\begin{document}
    \subsection*{1. Niech $B$ bedzie liczba naturalna wieksza od 1. Wykazac, ze kazda niezerowa liczba rzeczywista $x$ ma jednoznaczne przedstawienie w postaci znormalizowanej $x=smB^c$, gdzie $s$ jest znakiem liczby $x$, $c$ - liczba calkowita (cecha), a $m$ - liczba z przedzialu $[1, B)$, zwana mantysa.}

    1. istnienie:
    \medskip

    Niech $0\neq x\in\R$. Wtedy istnieje $c$ takie, ze $B^c\leq |x| < B^{c+1}$, czyli $c=\lfloor\log_B |x|\rfloor$. Niech $m={|x|\over B^c}$, wtedy $|x|=B^2\cdot m$. W koncu, niech $s={|x|\over x}$. Mamy $x=sB^cm$.
    \bigskip

    2. jedynosc:
    \medskip

    \indent A. jedynosc $s$ jest oczywista
    \medskip

    \indent B. jedynosc $c$: zalozmy, nie wprost, ze istnieja $c_1, c_2$ takie, ze
    $$x=sB^{c_1}m=sB^{c_2}m'$$
    Wtedy
    $$B^{c_1}m=B^{c_2}m'$$
    Jesli $m=m'$ oczywiste. W przeciwnym wypadku
    $$c_1\log_B m=c_2\log_B m'$$
    mozemy zalozyc, ze $c_1<c_2$ oraz $(\exists\;k\in\N)\;c_1+k=c_2$, czyli
    $$c_1\log_B m=(c_1+k)\log_B m'$$
    $$0 = k\log_B m'$$
    W takim razie albo $m'=1$, wtedy $x=2^c_1$, albo $k=0$, czyli $c_1=c_2$.
    \medskip

    \indent C. jedynosc $m$: zalozmy, nie wprost, ze istnieja $m_1, m_2$ takie, ze (...), wtedy
    $$x=sB^{c}m_1=sB^{c}m_2.$$
    $c$ jest jedyne, gdyz $c=\lfloor \log_B|x|\rfloor$ i to dzialanie ma jednoznaczny wynik. Czyli
    $$sB^cm_1=sB^cm_2$$
    $$m_1=m_2$$

    \kdowod

    \subsection*{2. Ile jest liczb zmiennopozycyjnych w arytmetyce double w standardzie IEE754?}

    Przy 64 bitach mamy $2^{64}$ mozliwosci ich zapalenia. 

    \subsection*{3. Czesc rozw w pliku .jl}

    \begin{lstlisting}[language=juleczka]
function frst_exp(x, s, t, r)
    ret = zero(x)
    ret = (x^3) - (s*(x^2)) + t*x - r
    print("  ", typeof(x), " wynik: ", ret, "\n")
end

function snd_exp(x, s, t, r)
    ret = zero(x)
    ret = ((x - s) * x + t) * x - r
    print("  ", typeof(x), " wynik: ", ret, "\n")
end

frst_exp(Float16(4.71), Float16(6), Float16(3), Float16(0.149)) # -14.58
frst_exp(Float32(4.71), Float32(6), Float32(3), Float32(0.149)) # -14.6365
frst_exp(Float64(4.71), Float64(6), Float64(3), Float64(0.149)) # -14.636489000000006

print("alternatywne wyrazenie:\n")

snd_exp(Float16(4.71), Float16(6), Float16(3), Float16(0.149)) # -14.63
snd_exp(Float32(4.71), Float32(6), Float32(3), Float32(0.149)) # -14.63649
snd_exp(Float64(4.71), Float64(6), Float64(3), Float64(0.149)) # -14.636489
    \end{lstlisting}

    -14.636489 - wartosc prawidlowa

    {\renewcommand{\arraystretch}{2}
    \begin{tabular}{| c | c | c |}
        \hline

        Float16 & ${|-14.636489+14.58|\over 14.636489}=$ ${0.056489\over14.636489}=$ $0.003859463837$ & ${|-14.636489 + 14.63|\over14.636489}=$ ${0.006489\over14.636489}=$ $4.43344029\cdot 10^{-4}$ \\

        \hline

        Float32 & ${|-14.636489 + 14.6365|\over14.636489}=$ ${0.000011\over14.636489}=7.51546358$ & ${|-14.636489 + 14.63649|\over14.636489}=$ ${0.000001\over14.636489}$ \\

        \hline

        Float64 & ${0.000000000000006\over14.636489}=$ $7.51546358\cdot10^{-7}$ & ${|-14.636489 + 14.636489|\over14.636489}=0$ \\

        \hline
    \end{tabular}}

\end{document}