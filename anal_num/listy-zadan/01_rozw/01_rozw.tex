\documentclass{article}[13pt]

\usepackage{../../../uni-notes-eng}
\usepackage{multicol}
\usepackage{graphicx}

\begin{document}
    \subsection*{1. Niech $B$ bedzie liczba naturalna wieksza od 1. Wykazac, ze kazda niezerowa liczba rzeczywista $x$ ma jednoznaczne przedstawienie w postaci znormalizowanej $x=smB^c$, gdzie $s$ jest znakiem liczby $x$, $c$ - liczba calkowita (cecha), a $m$ - liczba z przedzialu $[1, B)$, zwana mantysa.}

    1. istnienie:
    \medskip

    Niech $0\neq x\in\R$. Wtedy istnieje $c$ takie, ze $B^c\leq |x| < B^{c+1}$, czyli $c=\lfloor\log_B |x|\rfloor$. Niech $m={|x|\over B^c}$, wtedy $|x|=B^2\cdot m$. W koncu, niech $s={|x|\over x}$. Mamy $x=sB^cm$.
    \bigskip

    2. jedynosc:
    \medskip

    \indent A. jedynosc $s$ jest oczywista
    \medskip

    \indent B. jedynosc $c$: zalozmy, nie wprost, ze istnieja $c_1, c_2$ takie, ze
    $$x=sB^{c_1}m=sB^{c_2}m'$$
    Wtedy
    $$B^{c_1}m=B^{c_2}m'$$
    Jesli $m=m'$ oczywiste. W przeciwnym wypadku
    $$c_1\log_B m=c_2\log_B m'$$
    mozemy zalozyc, ze $c_1<c_2$ oraz $(\exists\;k\in\N)\;c_1+k=c_2$, czyli
    $$c_1\log_B m=(c_1+k)\log_B m'$$
    $$0 = k\log_B m'$$
    W takim razie albo $m'=1$, wtedy $x=2^c_1$, albo $k=0$, czyli $c_1=c_2$.
    \medskip

    \indent C. jedynosc $m$: zalozmy, nie wprost, ze istnieja $m_1, m_2$ takie, ze (...), wtedy
    $$x=sB^{c}m_1=sB^{c}m_2.$$
    $c$ jest jedyne, gdyz $c=\lfloor \log_B|x|\rfloor$ i to dzialanie ma jednoznaczny wynik. Czyli
    $$sB^cm_1=sB^cm_2$$
    $$m_1=m_2$$

    \kdowod

    \subsection*{2. Ile jest liczb zmiennopozycyjnych w arytmetyce double w standardzie IEE754?}

    Przy 64 bitach mamy $2^{64}$ mozliwosci ich zapalenia. Liczby NaN to liczby majace wszystkie bity w mantysie zapalone, a takich jest $2^{53}$. To daje nam $2^{64}-2^{53}$ liczb, ale 0 jest reprezentowane na dwa sposoby, wiec wystarczy $2^{64}-2^{53}-1$. 
    
    Liczby subnormalne maja na pierwszym miejscu mantysy 0, podczas gdy cala reszta zaczyna mantyse od 1 i one juz sie wliczaja.

    \subsection*{3. Czesc rozw w pliku .jl}

\begin{lstlisting}[language=juleczka]
function frst_exp(x, s, t, r)
    ret = zero(x)
    ret = (x^3) - (s*(x^2)) + t*x - r
    print("  ", typeof(x), " wynik: ", ret, "\n")
end

function snd_exp(x, s, t, r)
    ret = zero(x)
    ret = ((x - s) * x + t) * x - r
    print("  ", typeof(x), " wynik: ", ret, "\n")
end

function rel_error(val, exp, mess)
    bez = zero(val)
    if val > exp
        bez = val - exp
    else
        bez = exp - val
    end

    println("Blad wzgledny ", mess, " wynosi: ", bez / exp)
end
\end{lstlisting}

    -14.636489 - wartosc prawidlowa

    {\renewcommand{\arraystretch}{2}
    \begin{tabular}{| c | c | c |}
        \hline

        Float16 & -0.003987568330082385 & -0.0002511872895199931 \\

        \hline

        Float32 & -7.760455081662309e-7 & -5.931504907397523e-8 \\

        \hline

        Float64 & -4.85459822885188e-16 & 0 \\

        \hline
    \end{tabular}}

    \subsection*{Zad 4. Za dluga tresc}

    $$\sum\limits_{k=t+2}^\infty \frac1{2^k} = 2\frac1{2^{t+2}} = 2^{-t-1}$$
    \begin{align*}
        |rd(x)-x|&=|s(1+\sum\limits_{k=1}^\infty e_{-k}2^{-k})2^c-s(1+\sum\limits_{k=1}^t e_{-k}2^{-k} + e_{-t-1}2^{-t})2^c|=\\
        &=2^c|\sum\limits_{k=1}^\infty e_{-k}2^{-k} - \sum\limits_{k=1}^t e_{-k}2^{-k} - e_{-t-1}2^{-t}| =\\
        &= 2^c|\sum\limits_{k=t+1}^\infty e_{-k}2^{-k} - e_{-t-1}2^{-t}| \leq\\
        &\leq 2^c|\sum\limits_{k=t+1}^\infty2^{-k} - e_{-t-1}2^{-t}|=\\
        &= 2^c|\sum\limits_{k=t+2}^\infty2^{-k}+e_{-t-1}2^{-t-1} - e_{-t-1}2^{-t}|=\\
        &= 2^c|\sum\limits_{k=t+2}^\infty2^{-k}+e_{-t-1}({1\over 2^{t+1}} - {1\over 2^t})| =\\
        &= 2^c|\frac1{2^{t+1}}-{e_{-t-1}\over 2^{t+1}}| \leq\\
        &\leq 2^c\cdot 2^{-t-1}=2^c\cdot u
    \end{align*}

    \begin{align*}
        {|rd(x)-x|\over |x|} = {2^c|m-\overline m|\over 2^c |m|}= {|\sum\limits_{k=t+1}^\infty e_{-k}2^{-k}-...|\over |1+\sum\limits_{k=1}^\infty e_{-k}2^{-k}|}\leq \frac u 1 = u
    \end{align*}

    \subsection*{5. nah}

    $${|\overline m - m|\over |m|}\leq \frac u{1+u}$$
    $$(1+u)|\overline m - m|\leq u|m|$$

    Z poprzedniego zadania wiemy, ze $|\overline m-m|\leq u$. Wystarczy wiec, ze rozwazymy dwa przypadki:

    1. $(1+u)\leq m$ - oczywiste

    2. $(1+u) > m$

    \begin{align*}
        u &> m - 1\\
        \overline mu &> m\overline m - \overline m  >  m  -  \overline m \\
        \overline m u +  m u&> m  -  \overline m  +  m u\\
        m u &>  m - \overline m  +  m u -  \overline m u\\
        m u &>  m (1+u) -  \overline m (1+u)\\
        m u&>(1+u)( m - \overline m )\\
        {u\over 1+u} &> { m - \overline m \over  m }
    \end{align*}

    \subsection*{6. w pliiikuuuuuu}

\begin{lstlisting}
using Base

function to_number(str::AbstractString)
    if length(str) <= 64
        bias = Float64(2^10 - 1)

        ret = Float64(0)
        cech = Float64(0) # 11
        man = Float64(1) # 53

        c = Float64(1)

        for i = 1:11
            cech += c * (Float64(str[13 - i]) - Float64(48))
            c *= Float64(2)
        end

        c = Float64(0.5)

        for i = 1:52
            man += c * (Float64(str[i+12]) - Float64(48))

            c /= Float64(2)
        end

        cech -= bias

        ret = man * (2 ^ cech)

        if str[1] == '1'
            ret *= Float64(-1)
        end

        return ret
    else
        print("this is not a representation of a Float64")
        return NaN
    end
end
\end{lstlisting}

    \subsection*{7.}

    Liczba 1.000000057228997

\begin{lstlisting}
function fl()
    x = one(Float64)

    while x < 2.0
        if x * (1/ x) != 1
            println(x)
            println(x * (1 / x))
            break
        end
        x += eps(x)
    end
end
\end{lstlisting}

\end{document}