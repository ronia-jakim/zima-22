\documentclass{article}[16pt]

\usepackage{../../../uni-notes-eng}
\usepackage{multicol}
\usepackage{graphicx}


\begin{document}

\section*{ZAD 1.}

moze potem

\section*{ZAD 2.}

\begin{align*}
    f[x_0,x_1,...,x_k]={f[x_1,x_2,...,x_k]-f[x_0,x_1,...,x_{k-1}]\over x_k-x_0}
\end{align*}

DIVIDED DIFFERENCE

\section*{ZAD 4.}

\begin{center}\begin{tabular}{ | c | c | c | c | c | c | c |}
    \hline

    x & -2 & -1 & 0 & 1 & 2 & 3\\

    \hline

    p(x) & 31 & 5 & 1 & 1 & 11 & 61\\

    \hline
\end{tabular}\end{center}

Bedziemy uzywac wzoru interpolacyjnego Newtona, czyli potrzebujemy roznicy dzielonej $y$:
\begin{align*}
    \begin{matrix}
        31\\
        & & -26\\
        5 & & & -{11\over 15}\\
        & & -4\\
        1\\
        & & 0\\
        1\\
        & & 10\\
        11\\
        & & 50\\
        61
    \end{matrix}
\end{align*}

\end{document}