\documentclass{article}[16pt]

\usepackage{../../../uni-notes-eng}
\usepackage{multicol}
\usepackage{graphicx}

\title{Lista 5\\\large Analiza Numeryczna M}
\author{Weronika Jakimowicz}
\date{16.11.2022}

\begin{document}
\maketitle

\section*{ZAD 1.}

a.
$$\sum\limits_{k=0}^n\lambda_k(x)\equiv 1$$

Zauważmy, że
$$\sum\limits_{k=0}^n\lambda_k(x)=\sum\limits_{k=0}^n1\cdot\lambda_k(x)$$
Czyli $\sum\limits_{k=0}^n$ interpoluje funkcję w węzłach
$$(x_0, 1),(x_1,1),(x_2,1),...,(x_k,1).$$
Bez trudu można zauważyć, że jedynym wielomianem który to robi jest wielomian $q(x)=1$, więc
$$\sum\limits_{k=0}^n\lambda_k(x)=q(x)=1.$$
\medskip

b.
$$\sum\limits_{k=0}^n\lambda_k(0)x_k^j=\begin{cases}1\quad j=0\\0\quad wpp\end{cases}$$

Jeżeli $j=0$, to mamy
$$\sum\limits_{k=0}^n\lambda_k(0)x_k^j=\sum\limits_{x=0}^n\lambda_k(0)$$
co z poprzedniego podpunktu jest zawsze równe 1.
\smallskip

Jeżeli $j\neq 0$, to interpolujemy funkcję $f$ w węzłach
$$(x_0,x_0^j),(x_1,x_1^j),...,(x_k,x_k^j).$$
W tym celu możemy użyc $q(x)=x^j$, czyli
$$\sum\limits_{k=0}^n\lambda_k(x)x_k^j=q(x)=x^j$$
co po wstawieniu $x=0$ daje
$$\sum\limits_{k=0}^n\lambda_k(0)x_k^j=q(0)=0^j=0.$$

\section*{ZAD 2.}

\begin{align*}
    f[x_0,x_1,...,x_k]={f[x_1,x_2,...,x_k]-f[x_0,x_1,...,x_{k-1}]\over x_k-x_0}
\end{align*}

Twierdzenie 6.2.1. z "Analiza numeryczna" ~ Kincaid.

Po pierwsze, jeżeli interpolujemy funkcję $f$ przez wielomain $p_k$ stopnia $k+1$-tego w węzłach $(x_0,y_0), (x_1,y_1),...,(x_k, y_k)$, to jest ten wielomian wyrażony jednoznacznie (Twierdzenie 6.1.1. - Kicnaid). Niech więc $p_k,p_{k-1}$ interpolują funkcję $f$ w odpowiednio węzłach $(x_0,y_0),...,(x_k,y_k)$ i $(x_0,y_0),...,(x_{k-1},y_{k-1})$. Dalej, niech $q$ będzie wielomianek interpolującym $f$ w węzłach $(x_1,y_1),...,(x_k,y_k)$.
\medskip

Dla ułatwienia dowodu, wprowadźmy lemat że 
$$q(x)+{x-x_k\over x_k-x_0}[q(x)-p_{k-1}(x)]$$
interpoluje funkcję w węzłach $(x_0,y_0),(x_1,y_1),...,(x_k,y_k)$.

Wielomian $q(x)-p_{k-1}(x)$ ma miejsca zerowe dla 
$$x_1,...,x_{k-1},$$ 
więc nie wpływa na węzły $(x_1,y_1),...,(x_{k-1},y_{k-1})$. Co więcej, przechodzi on przez $(x_k,y_k)$ i $(x_0, -y_0)$. W takim razie, wielomian 
$$(x_k-x)(q(x)-p_{k-1}(x))$$ 
ma miejsce zerowe dodatkowo dla $x_k$ i przechodzi przez $(x_0,-y_0)$. Ponieważ $x_k>x_0$, to $x_0-x_k<0$, więc
$${x_k-x\over x_0-x_k}[q(x)-p_{k-1}(x)]={x-x_k\over x_k-x_0}[q(x)-p_{k-1}(x)]$$
przechodzi tylko przez $(x_0,y_0)$, a w pozostałych węzłach ma miejsca zerowe. Czyli
$$q(x)+{x-x_k\over x_k-x_0}[q(x)-p_{k-1}(x)]$$
jest przechodzące przez wszystkie $k+1$ węzłów.
\medskip

Skoro istnieje tylko jeden wielomian $k$-tego stopnia przechodzący przez ustalone węzły $(x_0,y_0),...,(x_k,y_k)$, to zachodzi
\begin{align*}
    p_k(x)&=q(x)+{x-x_k\over x_k-x_0}[q(x)-p_{k-1}(x)].
\end{align*}
Współczynnik przy $x^k$ po lewej stronie to
$$f[x_0,...,x_k],$$
natomiast współczynnik przy $x^k$ po prawej stronie to
$${f[x_1,...,x_k]-f[x_0,...,x_{k-1}]\over x_k-x_0}$$
co daje nam dowodzoną zależność:
$$f[x_0,...,x_k]={f[x_1,...,x_k]-f[x_0,...,x_{k-1}]\over x_k-x_0}.$$

\section*{ZAD 3.}

    Dla $k=1$
    $$p[x, x_1]={p[x_1]-p[x]\over x_1-x}={p(x_1)-p(x)\over x_1-x}$$
    co jest wielomianem stopnia $n-1$, bo dzielimy wielomian stopnia $n$ przez wielomian stopnia $1$.
    \smallskip

    Załóżmy, że jest teza jest prawdziwa dla wszystkich $\leq k$, wtedy
    $$p[x, x_1,...,x_{k+1}]={p[x_1, ..., x_{k+1}]-p[x, x_1,...,x_k]\over x_{k+1}-x}.$$
    Z założenia indukcyjnego wiemy, że $p[x, x_1, ..., x_k]$ jest wielomianem stopnia $n-k$. Pierwsza część róznicy w liczniku jest wartością niezależną od $x$. W takim razie do stopnia szukanego wielomianu przyczynia się tylko wielomian stopnia $n-k$ dzielony przez wielomian stopnia $1$. Daje nam to wielomian stopnia $n-k-1$, czyli $n-(k+1)$ co kończy dowód.

\section*{ZAD 4.}

\begin{center}\begin{tabular}{ | c | c | c | c | c | c | c |}
    \hline

    x & -2 & -1 & 0 & 1 & 2 & 3\\

    \hline

    p(x) & 31 & 5 & 1 & 1 & 11 & 61\\

    \hline
\end{tabular}\end{center}

Bedziemy uzywac wzoru interpolacyjnego Newtona, czyli potrzebujemy roznicy dzielonej $y$:
\begin{center}
\begin{tabular} {c c c c c c c c c c c c c c c c c c}
    $x_0$ & & $x_1$ & & $x_2$ & & $x_3$ & & $x_4$ & & $x_5$\\
    & \\
    31 & & 5 & & 1 & & 1 & & 11 & & 61\\
    & -26 & & -4 && 0 & & 10 & & 50\\
    & & 11 & & 2 & & 5 & & 20\\
    & & & -3 & & 1 & & 5\\
    & & & & 1 & & 1\\
    & & & & & 0
\end{tabular}
\end{center}

Wzór interpolacyjny Newtona:
$$p(x)=\sum\limits_{k=0}^5[y_0,...,y_k]\prod\limits_{j=0}^{i-1}(x-x_j)$$

\begin{align*}
    p(x)&=\sum\limits_{k=0}^5[y_0,...,y_k]\prod\limits_{j=0}^{i-1}(x-x_j)=\\
    &=31 -26(x+2)+11(x+2)(x+1)-3(x+2)(x+1)x+(x+2)(x+1)x(x-1)
\end{align*}

Dla drugiego wielomianu zmienia się jedynie wartość na szczycie, czyli $[x_0,x_1,...,x_5]$. Wynosi ono wtedy $-{31\over120}$ zamiast $0$ i mamy
\begin{align*}
    q(x)&=p(x)-{31\over120}(x+2)(x+1)x+(x+2)(x+1)x(x-1)(x-2)
\end{align*}


\section*{ZAD 5.}

Dla dowolnego $x\in[a,b]$ istnieje $k$ takie, że $x\in[x_k,x_{k+1}]$. Wtedy iloczyn $(x-x_k)(x-x_{k+1})$ jest największy gdy $x={x_k+x_{k+1}\over2}$, czyli jest równo pomiędzy tymi dwoma punktami. Wtedy
\begin{align*}
    (x-x_k)(x-x_{k+1})&={x_k+x_{k+1}-2x_k\over2}{x_k+x_{k+1}-2x_{k+1}\over2}=\\
    &=-{x_{k+1}-x_k\over 2}{x_{k+1}-x_k\over 2}=-\frac14h^2
\end{align*}

Największą odległość jaką może mieć $x$ względem $x_{k-1}$ jest $2h$:
$$(x_{k+1}-x_{k-1})=a+(k+1)h-a-(k-1)h=2h.$$
Analogicznie, największa odległość od $x_{k-2}$ to $3h$ etc. Dla $x_{k+2}$ również największa odległość to $2h$:
$$(x_k-x_{k+2})=a+kh-a-(k+2)h=-2h$$
i tak samo dla $x_{k+3}$ to $3h$. Odległość $x_0$ to $kh$, a $x_n$ to $(n-k)h$. Daje to nam poniższą nierówność:
\begin{align*}
    \prod\limits_{i=0}^n|x-x_i|&\leq kh\cdot (k-1)h\cdot...\cdot 2h\cdot \frac14h^2\cdot 2h\cdot...\cdot (n-k-1)h\cdot (n-k)\cdot h=\\
    &=\frac14 k!\cdot {n!\over k!} h^{k-1}h^2h^{n-k-1}=\frac14n!h^n
\end{align*}


\section*{ZAD 7.}

Twierdzenie ze slajdów:

Jeżeli funkcja $f$ ma w przedziale $[a,b]$ ciągłą $(n+1)$-sza pochodną, a wielomian $L_n\in\Pi_n$ interpoluję tę funkcją w parami różnych punktach $x_0,x_1,...,x_n\in[a,b]$ to dla każdego $x\in[a,b]$ zachodzi równość
$$f(x)-L_n(x)={1\over (n+1)!}f^{(n+1)}(\xi_x)p_{n+1}(x)$$
gdzie $p_{n+1}(x)=\prod\limits_{k=0}^n(x-x_k)$.

Czyli
$$f(x)-L_1(x)=\frac1{2!}f^{(n+1)}(\xi_x)(x-x_0)(x-x_1)$$
i dla wartości bezwzględnej:
\begin{align*}
    |f(x)-L_1(x)|&= \frac12|f^{(n+1)}(\xi_x)|x-x_0||x-x_1|\leq\\
    &\leq \frac12\max(|f''(x)|)|{x_0+x_1\over2}-x_0||{x_0+x_1\over2}-x_1|=\\
    &=\frac12M_2|{x_1-x_0\over2}||{x_0-x_1\over2}|=\\
    &=\frac12M_2(x_1-x_0)^2\frac14=\frac18M_2(x_1-x_0)^2
\end{align*}

\section*{ZAD 8.}

Twierdzenia 6.1.7. z Kincaid-Cheney:

Jeżeli węzły $x_i$ są zerami wilomianu Czebyszewa $T_{n+1}$, to dla $|x|\leq 1$ jest
$$|f(x)-L_n(x)|\leq{1\over 2^n(n+1)!}\|f^{(n+1)}\|_{[-1,1]}$$

Rozważana przez nas norma $\|f\|_{[a,b]}$ to norma maksimum, więc
$$\|f^{(n+1)}\|_{[-1,1]}=\|e^x\|_{[-1,1]}=e^1=e.$$

Chcemy więc znaleźć najmniejsze takie $n$, że
$$10^{-5}\leq {e\over 2^n(n+1)!}$$
dla $n=6$
$$2^67!=322560\approx 3\cdot 10^5$$
$${e\over 2^67!}\approx {3\over 3\cdot 10^5}=10^{-5}$$


\end{document}