\documentclass{article}[13pt]

\usepackage{../../uni-notes-eng}
\usepackage{multicol}
\usepackage{graphicx}

\begin{document}
    \subsection*{2. Niech $B$ bedzie liczba naturalna wieksza od 1. Wykazac, ze kazda niezerowa liczba rzeczywista $x$ ma jednoznaczne przedstawienie w postaci znormalizowanej $x=smB^c$, gdzie $s$ jest znakiem liczby $x$, $c$ - liczba calkowita (cecha), a $m$ - liczba z przedzialu $[1, B)$, zwana mantysa.}

    Zalozmy nie wprost, ze istnieje taka liczba $x\in\R$, dla ktorej nie istniaja takie $c\in\Z$ oraz $m\in[1, B)$, ze $x=smB^c$.
    \smallskip

    Dla ulatwienia dowodu skupimy sie na liczbach $x>0$ - wowczas $s=1$, a dla liczb $x<0$ wystarczy obrocic znaki nierownosci oraz przyjac $s=-1$.
    Niech $c_0, c_1\in\Z$ beda takimi liczbami, ze 
    $$B^{c_0}\leq x< B^{c_1}$$
    oraz $|c_1-c_0|=1$. Wowczas liczba 
    $$y=\frac{x}{B^{c_0}}$$
    spelnia $y\in [1, B)$, gdyz $x\in [B^{c_0}, B^{c_0+1})$. W takim razie mozemy powiedziec, ze
    $$x=y\cdot B^{c_0}.$$
    Ale wtedy $x$ mozna zapisac wedlug zasad opisanych w tresci - sprzecznosc.

    \kdowod

    \subsection{3. Czesc rozw w pliku .jl}

    \begin{lstlisting}[language=juleczka]
function frst_exp(x, s, t, r)
    ret = zero(x)
    ret = (x^3) - (s*(x^2)) + t*x - r
    print("  ", typeof(x), " wynik: ", ret, "\n")
end

function snd_exp(x, s, t, r)
    ret = zero(x)
    ret = ((x - s) * x + t) * x - r
    print("  ", typeof(x), " wynik: ", ret, "\n")
end

frst_exp(Float16(4.71), Float16(6), Float16(3), Float16(0.149)) # -14.58
frst_exp(Float32(4.71), Float32(6), Float32(3), Float32(0.149)) # -14.6365
frst_exp(Float64(4.71), Float64(6), Float64(3), Float64(0.149)) # -14.636489000000006

print("alternatywne wyrazenie:\n")

snd_exp(Float16(4.71), Float16(6), Float16(3), Float16(0.149)) # -14.63
snd_exp(Float32(4.71), Float32(6), Float32(3), Float32(0.149)) # -14.63649
snd_exp(Float64(4.71), Float64(6), Float64(3), Float64(0.149)) # -14.636489
    \end{lstlisting}

    -14.636489 - wartosc prawidlowa

    {\renewcommand{\arraystretch}{2}
    \begin{tabular}{| c | c | c |}
        \hline

        Float16 & ${|-14.636489+14.58|\over 14.636489}=$ ${0.056489\over14.636489}=$ $0.003859463837$ & ${|-14.636489 + 14.63|\over14.636489}=$ ${0.006489\over14.636489}=$ $4.43344029\cdot 10^{-4}$ \\

        \hline

        Float32 & ${|-14.636489 + 14.6365|\over14.636489}=$ ${0.000011\over14.636489}=7.51546358$ & ${|-14.636489 + 14.63649|\over14.636489}=$ ${0.000001\over14.636489}$ \\

        \hline

        Float64 & ${0.000000000000006\over14.636489}=$ $7.51546358\cdot10^{-7}$ & ${|-14.636489 + 14.636489|\over14.636489}=0$ \\

        \hline
    \end{tabular}}

\end{document}