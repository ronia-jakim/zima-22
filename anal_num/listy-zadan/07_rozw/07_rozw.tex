\documentclass{article}[16pt]

\usepackage{../../../uni-notes-eng}
\usepackage{multicol}
\usepackage{graphicx}

\begin{document}

Co mam: 1, 4, 6, 8 (4)

MAX: 10

\section*{ZAD. 1}

Chcemy udowodnić, że
$$\int\limits_{-1}^1(1-x^2)^{-\frac12}T_k(x)T_l(x)dx=\begin{cases}
    0\quad k\neq l\\
    \pi\quad k=l=0\\
    \frac\pi2\quad k=l\neq 0
\end{cases}$$

Po pierwsze zauważmy, że
$$\int\limits_{-\pi}^\pi \cos(k\theta)\cos(l\theta)d\theta=\begin{cases}
    0\quad k\neq l\\
    2\pi\quad k=l=0\\
    \pi
\end{cases}$$

Korzystając z zależności
\begin{align*}
    \frac12(\cos(x-y)+\cos(x+y))&=\frac12(\cos(x)\cos(y)-\sin(x)\sin(y)+\cos(x)\cos(y)+\sin(x)\sin(y))=\\
    &=\frac12(2\cos(x)\cos(y))=\cos(x)\cos(y)
\end{align*}

Jeśli $k=l=0$, to mamy

\begin{align*}
    \int\limits_{-\pi}^\pi\cos^2(0\cdot\theta)d\theta=\int\limits_{-\pi}^\pi1d\theta=\pi-(-\pi)=2\pi
\end{align*}

natomiast jeśli $k=l\neq0$, to jest

\begin{align*}
    \int\limits_{-\pi}^\pi\cos^2(k\theta)d\theta&={2k(\pi+\pi)+\sin(2k\pi)-\sin(-2k\pi)\over 4k}={4k\pi\over 4k}=\pi
\end{align*}

Jeśli $k\neq l$:
\begin{align*}
    \int\limits_{-\pi}^\pi\cos(k\theta)\cos(l\theta)d\theta&=\int\limits_{-\pi}^\pi\frac12(\cos((k-l)\theta)+\cos((k+l)\theta))d\theta=\\
    &=\frac12\int\limits_{-\pi}^\pi\cos((k-l)\theta)d\theta+\frac12\int\limits_{-\pi}^\pi\cos((k+l)\theta)d\theta=\\
    &=\frac12{\sin((k-l)\theta)\over k-l}+\frac12{\sin((k+l)\theta)\over k-l}=\\
    &=\frac12\Big({\sin((k-l)\pi)-\sin((l-k)\pi)\over k-l}+{\sin((k+l)\pi)-\sin((-k-l)\pi)\over k+l}\Big)=\\
    &={\sin(\pi{k-l+l-k\over2})\cos(\pi{k-l+l-k\over2})\over k-l}+{\sin(\pi{k+l-k-l\over2})\cos(\pi{k+l-k-l\over2})\over k+l}=\\
    &=0+0=0
\end{align*}

Wracając do Czebyszewa, wiemy, że $T_n(\cos(\theta))=\cos(n\theta)$. W takim razie
\begin{align*}\int\limits_{-1}^1(1-x^2)^{-\frac12}T_k(x)T_l(x)dx&=\begin{bmatrix}
    x=\cos(\theta)\\
    dx=-\sin(\theta)d\theta=-\sqrt{1-\cos^2(\theta)}d\theta
\end{bmatrix}=-\int\limits_{\pi}^0\cos(k\theta)\cos(l\theta)d\theta=\\
&=\int\limits_0^\pi\cos(k\theta)\cos(l\theta)d\theta
\end{align*}
A ponieważ $\cos(x)$ jest funkcją parzystą, to
$$\int\limits_{-\pi}^\pi\cos(k\theta)\cos(l\theta)d\theta=2\int\limits_{0}^\pi\cos(k\theta)\cos(l\theta)d\theta$$

\subsection*{ZAD. 2.}

{\color{def}(a)} !!! zmieniam oznaczenie funkcji wagowej na $w(x)$ bo tak.

(to jest z Kincaida)
\smallskip

Tak jak na slajdzie zauważono, jeżeli $P_n(x)=a_nx^n+...$, to $P_n=a_np_n(x)$ gdzie $p_n$ to standardowy wielomian ortogonalny.

Standardowe wielomiany ortogonalne są definiowane w następujący sposób:
$$p_0(x)=1$$
$$p_1(x)=(x-a_1)p_0(x)$$
$$p_k(x)=(x-a_k)p_{k-1}(x)-b_kp_{k-2}(x),$$
gdzie
\begin{align*}
    a_n&={\langle xp_{n-1},p_{n-1}\rangle\over\langle p_{n-1},p_{n-1}\rangle}\\
    b_n&={\langle xp_{n-1},p_{n-2}\rangle\over\langle p_{n-2},p_{n-2}\rangle}
\end{align*}

Udowodnimy przez indukcję względem $n$, że $\langle p_n,p_i\rangle=0$ dla $i<n$.

Jeżeli $n=1$, to mamy
\begin{align*}
    0=\langle p_1, p_0\rangle &= \langle (x-a_1)p_0, p_0\rangle=\langle xp_0,p_0\rangle-a_1\langle p_0,p_0\rangle
\end{align*}
$$a_1={\langle xp_0,p_0\rangle\over \langle p_0,p_0\rangle}={\int w(x)xdx\over \int w(x)dx}$$
co jest wyznaczone jednoznacznie.

Jeżeli $n\geq 2$ i $\langle p_{n-1},p_i\rangle=0$ dla $i<n-1$, to chcemy
\begin{align*}
    0=\langle p_n,p_{n-1}\rangle&=\langle (x-a_n)p_{n-1}-b_np_{n-2},p_{n-1}\rangle=\\
    &=\langle (x-a_n)p_{n-1},p_{n-1}\rangle-b_n\langle p_{n-2},p_{n-1}\rangle=\\
    &=\langle xp_{n-1},p_{n-1}\rangle-a_n\langle p_{n-1},p_{n-1}\rangle -b_n\cdot 0=\\
    &=\langle xp_{n-1},p_{n-1}\rangle-a_n\langle p_{n-1},p_{n-1}\rangle
\end{align*}
$$a_n={\langle xp_{n-1},p_{n-1}\rangle\over \langle p_{n-1},p_{n-1}\rangle},$$
co jest zdefiniowane jednoznacznie.

\begin{align*}
    0=\langle p_n,p_{n-2}\rangle&=\langle (x-a_n)p_{n-1}-b_np_{n-2}, p_{n-2}\rangle=\\
    &=\langle (x-a_n)p_{n-1},p_{n-2}\rangle-b_n\langle p_{n-2},p_{n-2}\rangle=\\
    &=\langle xp_{n-1},p_{n-2}\rangle-b_n\langle p_{n-2},p_{n-2}\rangle
\end{align*}
$$b_n={\langle xp_{n-1},p_{n-2}\rangle\over \langle p_{n-2},p_{n-2}\rangle}$$
co również jest jednoznaczne z poprzednich definicji.
\medskip

Dalej, dla dowolnego $i<n-1$ mamy
\begin{align*}
    \langle p_n,p_i\rangle&=\langle (x-a_n)p_{n-1}-b_np_{n-2},p_i\rangle=\\
    &=\langle xp_{n-1},p_i\rangle-a_n\langle p_{n-1},p_i\rangle -b_n\langle p_{n-2},p_i\rangle=\\
    &=\langle p_{n-1},xp_i\rangle=\langle p_{n-1},p_{i+1}+a_{i+1}p_i+b_{i+1}p_{i-1}\rangle =0
\end{align*}

{\color{def}(b)}

To, że lnz to zadanie 6, więc sobie pominę. Jest ich $n+1$ sztuk, a przestrzeń $\Pi_n$ jest wymiaru $n+1$, bo jej bazą standardową jest
$$1,x,x^2,...,x^n$$
czyli mamy maksymalny układ liniowo niezależny, ergo baza.

{\color{def}(c)}

W poprzednim podpunkcie pokazaliśmy, że wielomiany ortogonalne $P_0,...,P_{k-1}$ tworzą bazę $\Pi_{k-1}$. W takim razie
$$Q=\sum\limits_{i=0}^{k-1}c_iP_i$$
i mamy
\begin{align*}
    \langle Q,P_k\rangle &=\langle \sum\limits_{i=0}^{k-1}c_iP_i,P_k\rangle=\sum\limits_{i=0}^{k-1}c_i\langle P_i,P_k\rangle=\sum\limits_{i=0}^{k-1}c_i\cdot 0=0.
\end{align*}


\subsection*{ZAD. 3.}

Wielomiany ortogonalne są definiowane w następujący sposób:
$$P_0(x)=a_0$$
$$P_1(x)=(a_1x-b_1)P_0(x)$$
$$P_k(x)=(a_kx-b_k)P_{k-1}(x)-c_kP_{k-2}(x).$$
Chcemy pokazać, że zera wielomianu $P_n$ są rzeczywiste, pojedyncze i leżą w przedziale $(a, b)$, możemy więc ograniczyć się do przestrzeni $\Pi_n$ rozpiętej przez wielomiany $P_0,...,P_n$, bo tylko te mają wpływ na wartość $P_n$.

Załóżmy, nie wprost, że $P_n$ ma $m$ miejsc zerowych $x_1,...,x_m$ dla $0\leq m<n$. Wiemy, że
$$\langle P_0,P_n\rangle=\int\limits_{a}^ba_0p(x)P_n(x)dx=a_0\int\limits_a^bp(x)P_n(x)dx=0$$
$$\int\limits_a^bp(x)P_n(x)dx=0$$

Ponieważ $p(x)$ nie zmienia znaku na przedziale $[a,b]$, to $P_n(x)$ nie może takie być. W takim razie dla pewnego $i\in\{1,...,m\}$ $P_n(x)$ zmienia znak w $x_i$. Oznaczmy wielomian $Z(x)=(x-x_1)...(x-x_m)$. Wtedy $P_n(x)Z(x)p(x)$ jest wielomianem niezmieniającym znak na przedziale $[a, b]$, czyli
$$c=\int\limits_a^bZ(x)P_n(x)p(x)dx\neq 0$$

Ale z drugiej strony wiemy, że $Z$ jest wielomianem stopnia $m<n$, więc może zostać zapisane w postaci
$$Z=\sum\limits_{k=0}^mx_kP_k,$$
co daje nam
\begin{align*}
    c&=\int\limits_a^bp(x)Z(x)P_n(x)dx=\int\limits_a^bp(x)P_n(x)\sum\limits_{k=0}^mx_kP_k(x)dx=\\
    &=\sum\limits_{k=0}^mx_k\int\limits_a^bp(x)P_k(x)P_n(x)dx=\sum\limits_{k=0}^mx_k\cdot 0=0
\end{align*}
i jest sprzeczność.

\subsection*{ZAD. 4.}

$$\Big\|\sum\limits_{j=0}^nc_jf_j\Big\|_2^2=\sum\limits_{j=0}^n|c_j|^2\|f_j\|_2^2$$

Norma:
$$\|f\|_2=\Big(\int_{-1}^1f^2(x)(1-x^2)^{-\frac12}dx\Big)^{\frac12}$$

\begin{align*}
    \|\sum\limits_{j=0}^{n}c_jf_j\|_2^2&=\langle \sum\limits_{j=0}^nc_jf_j, \sum\limits_{i=0}^nc_if_i\rangle=\sum\limits_{j=0}^n|c_j|\langle f_j,\sum\limits_{i=0}^nc_if_j\rangle=\\
    &=\sum\limits_{j=0}^n|c_j|\sum\limits_{i=0}^n|\overline{c_i}|\langle f_j,f_i\rangle=\sum\limits_{j=0}^n\sum\limits_{i=0}^n|c_j||\overline{c_i}|\langle f_j,f_i\rangle=\\
    &=\sum\limits_{j=0}^n|c_j|^2\langle f_j,f_j\rangle=\sum\limits_{j=0}^n|c_j|^2\|f_j\|^2_2
\end{align*}

\subsection*{ZAD. 6.}

Wiem, że $\langle f_i,f_j\rangle=0$ dla każdego $i\neq j$. Chce pokazać, że
$$\sum\limits_{i=0}^ma_if_i=0\iff a_i=0$$

Załóżmy nie wprost, że co najmniej jedno $a_k\neq 0$. Weźmy to $k$ i wtedy:

\begin{align*}
    0=\langle0,f_k\rangle=\langle \sum\limits_{i=0}^ma_if_i,f_k\rangle&=\sum\limits_{i=0}^n|a_i|\langle f_i,f_k\rangle=|a_k|\langle f_k,f_k\rangle=|a_k|\|f_k\|^2\neq 0
\end{align*}
co daje sprzeczność.

\subsection*{ZAD. 8.}

\begin{center}
\begin{tabular}{c || c | c | c | c | c | c | c | c | c |}
    T & 0 & 10 & 20 & 30 & 40 & 80 & 90 & 95\\

    \hline

    S & 68 & 67.1 & 66.4 & 65.6 & 64.6 & 61.8 & 61.0 & 60
\end{tabular}
\end{center}

Średnia wartość $x$ to $\overline x=45.625$, wartość średnia $y$ - $\overline{y}=64.3125$, średnie nachylenie:
$$\overline{a}={\sum(x_i-\overline{x})(y_i-\overline{y})\over \sum(x_i-\overline{x})^2}=-0.07993035770813546$$

I chcemy, żeby to przechodziło przez średni punkt:
$$64.3125=-0.07993\cdot 45.625+b\implies b = 67.9593$$

a więc
$$S=-0.07993T+67.9593$$

\begin{center}
\begin{tikzpicture}
    \begin{axis}
        \addplot[domain=-5:100, color=acc]{-0.07993*x+67.9593};
        \addplot[only marks] coordinates {
            (0, 68)(10,67.1)(20,66.4)(30,65.6)(40,64.6)(80,61.8)(90,61)(95,60)
        };
    \end{axis}
\end{tikzpicture}    
\end{center}

\end{document}