\documentclass{article}[16pt]

\usepackage{../../../uni-notes-eng}
\usepackage{multicol}
\usepackage{graphicx}


\begin{document}
\large
    \section*{ZAD 1.}
    
    Mamy zadany wielomian
    $$p(z)=a_0+z(a_1+z(a_1+...+z(a_{n-1}+za_n)))$$
    który możemy zapisać jako
    $$p(z)=(z-z_0)q(z)+p(z_o),\quad (\kawa)$$
    gdzie $p(z_0)$ to reszta z dzielenia przez $(z-z_0)$, a $q(z)$ to iloraz:
    $$q(z)=b_0+zb_1+z^2b_2+...+z^{n-1}b_{n-1}.$$
    Mamy wiec rownosc
    \begin{align*}
        a_0+za_1+z^2a_3+...+z^na_n&=(z-z_0)(b_0+zb_1+...+z^{n-1}b_{n-1})+p(z_0)
    \end{align*}
    i wspolczynniki przy zmiennych musza byc sobie rowne, wiec
    \begin{align*}
        a_n&=b_{n-1}\\
        a_{n-1}&=b_{n-2}-z_0b_{n-1}\\
        a_{k+1}&=b_k-z_0b_{k+1}\quad\text{dla }k=0,...,n-2\\
        a_0&=-z_0b_0+p(z_0)
    \end{align*}
    W takim razie
    $$p(z_0)=a_0+z_0b_0.$$

    Schemat Hornera do znalezienia $p(z_0)$ wyglada wtedy nastepujaco:
    \begin{align*}
        b_{n-1}&:=a_{n}\\
        b_{k}&:=a_{k+1}+z_0b_{k+1}\quad\text{dla }k=n-2,...,-1
    \end{align*}
    i szukana wartosc to $b_{-1}$.\podz{sep}
    \bigskip

    Analogicznie, dla pochodnej $p(z)$ mamy
    \begin{align*}
        a_1+2a_2z+...+na_nz^{n-1}=&(z-z_0)q'(z)+q(z)\\
        a_1+2a_2z+...+na_nz^{n-1}=&(z-z_0)(b_1+2zb_2+3z^2b_3+...+(n-1)z^{n-2}b_{n-1})+\\
        &+(b_0+zb_1+...+z^{n-1}b_{n-1})
    \end{align*}
    i aby wyrazy po obu stronach sie zgadzaly:
    \begin{align*}
        na_n&=b_{n-1}+(n-1)b_{n-1}\\
        (n-1)a_{n-1}&=(n-2)b_{n-2}+b_{n-2}-z_0b_{n-1}\\
        a_{k+1}&=b_k-z_0b_{k+1}\\
        a_1&=b_0-z_0b_1
    \end{align*}
    W trakcie rozniczkowania $p(z_0)$ sie uproscilo, ale mozemy zapisac, ze
    $$p'(z_0)=b_0=a_1+z_0b_1.$$
    Schemat Hornera dla znalezienia $p'(z_0)$ to w takim razie
    \begin{align*}
        b_{n-1}&:=a_n\\
        b_k&:=a_{k+1}+z_0b_{k+1}\quad\text{dla}k=n-2,...,0
    \end{align*}

    i wyraz $b_0$ to szukane $p'(z_0)$.\podz{sep}
    \medskip

    To teraz dla $p''(z_0)$
    \begin{align*}
        2a_2+6za_3+...+n(n-1)a_nz^{n-2}=&(z-z_0)q''(z)+2q'(z)\\
        2a_2+6za_3+...+n(n-1)a_nz^{n-2}=&(z-z_0)(2b_2+6zb_3+...+(n-1)(n-2)z^{n-3}b_{n-1})+\\
        &+2(b_1+2zb_2+3z^2b_3+...+(n-1)z^{n-2}b_{n-1})
    \end{align*}

    \begin{align*}
        n(n-1)a_n&=(n-1)(n-2)b_{n-1}+2(n-1)b_{n-1}\\
        a_{n-1}&=b_{n-2}-z_0b_{n-1}\\
        a_{k+1}&=b_{k}-z_0b_{k+1}\\
        2a_2&=2b_1-2z_0b_2
    \end{align*}

    Schemat Hornera dla $p''(z_0)$:
    \begin{align*}
        b_{n-1}&:=a_n\\
        b_k&:={k+1\over2}(a_{k+1}+z_0b_{k+1})\quad\text{dla }k=n-2,...,1
    \end{align*}


    \section*{ZAD 2.}

    Mamy dana funkcje
    $$f(z)=f(x+iy)=u(x,y)+iv(x,y).$$

    \begin{align*}
        z_1&=x_0+iy_0+{u(x_0,y_0)+iv(x_0,y_0)\over u'(x_0,y_0)+iv'(x_0,y_0)}\\
        &=x_0+iy_0+{(u(x_0,y_0)+iv(x_0,y_0))(u'(x_0,y_0)-iv'(x_0,y_0))\over u'(x_0,y_k)^2-v'(x_0,y_0)^2}
    \end{align*}
    
    Niech $u_k=u(x_k,y_k)$, $v_k=v(x_k,y_k)$.

    \begin{align*}
        x_{1}&=x_0+{u_0(u')_0-v_0(v')_0\over (u')_0^2-(v')_0^2}\\
        y_{1}&=y_0+{v_0(u')_0-u_0(v')_0\over (u')_0^2-(v')_0^2}\\
        x_{k+1}&=x_k+{u_k(u')_k-v_k(v')_k\over (u')_k^2-(v')_k^2}\\
        y_{k+1}&=y_k+{v_k(u')_k-u_k(v')_k\over (u')_k^2-(v')_k^2}
    \end{align*}


    \section*{ZAD 4.}

    Metoda $\phi$ oblicza pierwiastek jakiejs funkcji $f$, ktory wypada w $\alpha$. Blad dla $n$-tego kroku wynosi
    $$E_n=x_n-\alpha$$
    a jezeli metoda jest rzedu $p+1$, to
    \begin{align*}
        \Big |{E_{n+1}\over (E_n)^{p+1}}\Big |< \infty
    \end{align*}

    \begin{align*}
        x_{k+1}&=\phi(x_k)=\\
        &=\phi(\alpha)+(x_k-\alpha)\phi'(\alpha)+(x_k-\alpha)^2{\phi''(\alpha)\over 2!}+...+(x_k-\alpha)^p{\phi^{(p)}(\alpha)\over p!}+(x_k-\alpha)^{p+1}{\phi^{(p+1)}(\xi_k)\over (p+1)!}
    \end{align*}
    dla pewnego $\xi_k\in[x_k,\alpha]$.
    \medskip

    Ale wszystko poza ostatnim wyrazem sie zeruje, wiec mamy
    $$x_{k+1}=\phi(x_k)=\alpha+(x_k-\alpha)^{p+1}{\phi^{(p+1)}(\xi_k)\over(p+1)!}$$
    i to do $\alpha$ bedzie zbiegac w potedze $p+1$.

    \begin{align*}
        {x_{k+1}-\alpha\over (x_k-\alpha)^{p+1}}&={\phi(x_k)-\alpha\over (x_k-\alpha)^{p+1}}=\\
        &={(x_k-\alpha)^{p+1}{\phi^{(p+1)}(\xi_k)\over(p+1)!}\over (x_k-\alpha)^{p+1}}=\\
        &={\phi^{(p+1)}(\xi_k)\over(p+1)!}\rarrow{}{k\to\infty}{\phi^{(p+1)}(\alpha)\over(p+1)!}
    \end{align*}

    
    \section*{ZAD 5.}

    


    \section*{ZAD 6.}

    Jejku chce umrzyc.


\end{document}