\documentclass{article}[13pt]

\usepackage{../../../uni-notes-eng}
\usepackage{multicol}
\usepackage{graphicx}


\begin{document}
    \section*{ZAD 1.}
    \begin{align*}
        p(x)&=a_0+a_1x+a_2x^2+...+a_nx^n=\\
            &=\sum\limits{i=1}^na_ix^i\\
            &=a_0+x(a_1+x(a_2+...+x(a_{n-1}+xa_n)))\\
        p'(x)&=a_1+2a_2x+...+na_nx^{n-1}=\\
            &=\sum\limits_{i=0}^nia_ix^{i-1}=\\
            &=a_1+x(2a_2+x(3a_3+...+x((n-1)a_{n-1}+xna_n))\\
        p''(x)&=2a_2+3\cdot2a_3x+...+n(n-1)a_n^{n-2}\\
            &=\sum\limits_{i=1}^ni(i-1)a_ix^{i-2}=\\
            &=2a_2+x(3\cdot2a_3+...+x((n-1)(n-2)a_{n-1}+xn(n-1)a_n))
    \end{align*}

    Zeby obliczyc $p(z_0)$ postepujemy tak samo jak w zwyklym schemacie Hornera, czyli
\begin{lstlisting}
i = n
x = a_i

dopoki i >= 0
    x = a_i + x * z_0
    i = i - 1

zwroc x
\end{lstlisting}

    Dla $p'(z_0)$ trzeba tylko przemnozyc kazdy wyraz przez odpowiedni indeks:

\begin{lstlisting}
i = n
x = i * a_i

dopoki i >= 0
    x = i * a_i + x * z_0
    i = i - 1

zwroc x
\end{lstlisting}

    Analogicznie dla $p''$ i $p'''$.

    \section*{ZAD 2.}

    


\end{document}