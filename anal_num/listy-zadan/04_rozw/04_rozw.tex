\documentclass{article}[16pt]

\usepackage{../../../uni-notes-eng}
\usepackage{multicol}
\usepackage{graphicx}


\begin{document}
\large

    \begin{center}\begin{tabular}{| c | c | c | c | c | c | c | c |}
        \hline
        1 & 2 & 3 & 4 & 5 & 6 & 7 & 8\\

        \hline
        
        X & X & - & X & X & X & - & -\\

        \hline
    \end{tabular}\end{center}

    \section*{ZAD 1.}
    
    Mamy zadany wielomian
    $$p(z)=a_0+z(a_1+z(a_1+...+z(a_{n-1}+za_n)))$$
    który możemy zapisać jako
    $$p(z)=(z-z_0)q(z)+p(z_o),\quad (\kawa)$$
    gdzie $p(z_0)$ to reszta z dzielenia przez $(z-z_0)$, a $q(z)$ to iloraz:
    $$q(z)=b_0+zb_1+z^2b_2+...+z^{n-1}b_{n-1}.$$
    Mamy wiec rownosc
    \begin{align*}
        a_0+za_1+z^2a_3+...+z^na_n&=(z-z_0)(b_0+zb_1+...+z^{n-1}b_{n-1})+p(z_0)
    \end{align*}
    i wspolczynniki przy zmiennych musza byc sobie rowne, wiec
    \begin{align*}
        a_n&=b_{n-1}\\
        a_{n-1}&=b_{n-2}-z_0b_{n-1}\\
        a_{k+1}&=b_k-z_0b_{k+1}\quad\text{dla }k=0,...,n-2\\
        a_0&=-z_0b_0+p(z_0)
    \end{align*}
    W takim razie
    $$p(z_0)=a_0+z_0b_0.$$

    Schemat Hornera do znalezienia $p(z_0)$ wyglada wtedy nastepujaco:
    \begin{align*}
        b_{n-1}&:=a_{n}\\
        b_{k}&:=a_{k+1}+z_0b_{k+1}\quad\text{dla }k=n-2,...,-1
    \end{align*}
    i szukana wartosc to $b_{-1}$.\podz{sep}
    \bigskip

    Analogicznie, dla pochodnej $p(z)$ mamy
    \begin{align*}
        a_1+2a_2z+...+na_nz^{n-1}=&(z-z_0)q'(z)+q(z)\\
        a_1+2a_2z+...+na_nz^{n-1}=&(z-z_0)(b_1+2zb_2+3z^2b_3+...+(n-1)z^{n-2}b_{n-1})+\\
        &+(b_0+zb_1+...+z^{n-1}b_{n-1})
    \end{align*}
    i aby wyrazy po obu stronach sie zgadzaly:
    \begin{align*}
        na_n&=b_{n-1}+(n-1)b_{n-1}\\
        (n-1)a_{n-1}&=(n-2)b_{n-2}+b_{n-2}-z_0b_{n-1}\\
        a_{k+1}&=b_k-z_0b_{k+1}\\
        a_1&=b_0-z_0b_1
    \end{align*}
    W trakcie rozniczkowania $p(z_0)$ sie uproscilo, ale mozemy zapisac, ze
    $$p'(z_0)=b_0=a_1+z_0b_1.$$
    Schemat Hornera dla znalezienia $p'(z_0)$ to w takim razie
    \begin{align*}
        b_{n-1}&:=a_n\\
        b_k&:=a_{k+1}+z_0b_{k+1}\quad\text{dla}k=n-2,...,0
    \end{align*}

    i wyraz $b_0$ to szukane $p'(z_0)$.\podz{sep}
    \medskip

    To teraz dla $p''(z_0)$
    \begin{align*}
        2a_2+6za_3+...+n(n-1)a_nz^{n-2}=&(z-z_0)q''(z)+2q'(z)\\
        2a_2+6za_3+...+n(n-1)a_nz^{n-2}=&(z-z_0)(2b_2+6zb_3+...+(n-1)(n-2)z^{n-3}b_{n-1})+\\
        &+2(b_1+2zb_2+3z^2b_3+...+(n-1)z^{n-2}b_{n-1})
    \end{align*}

    \begin{align*}
        n(n-1)a_n&=(n-1)(n-2)b_{n-1}+2(n-1)b_{n-1}\\
        a_{n-1}&=b_{n-2}-z_0b_{n-1}\\
        a_{k+1}&=b_{k}-z_0b_{k+1}\\
        2a_2&=2b_1-2z_0b_2
    \end{align*}

    Schemat Hornera dla $p''(z_0)$:
    \begin{align*}
        b_{n-1}&:=a_n\\
        b_k&:={k+1\over2}(a_{k+1}+z_0b_{k+1})\quad\text{dla }k=n-2,...,1
    \end{align*}


    \section*{ZAD 2.}

    Mamy dana funkcje
    $$f(z)=f(x+iy)=u(x,y)+iv(x,y).$$

    \begin{align*}
        z_1&=x_0+iy_0+{u(x_0,y_0)+iv(x_0,y_0)\over u'(x_0,y_0)+iv'(x_0,y_0)}\\
        &=x_0+iy_0+{(u(x_0,y_0)+iv(x_0,y_0))(u'(x_0,y_0)-iv'(x_0,y_0))\over u'(x_0,y_k)^2-v'(x_0,y_0)^2}
    \end{align*}
    
    Niech $u_k=u(x_k,y_k)$, $v_k=v(x_k,y_k)$.

    \begin{align*}
        x_{1}&=x_0+{u_0(u')_0-v_0(v')_0\over (u')_0^2-(v')_0^2}\\
        y_{1}&=y_0+{v_0(u')_0-u_0(v')_0\over (u')_0^2-(v')_0^2}\\
        x_{k+1}&=x_k+{u_k(u')_k-v_k(v')_k\over (u')_k^2-(v')_k^2}\\
        y_{k+1}&=y_k+{v_k(u')_k-u_k(v')_k\over (u')_k^2-(v')_k^2}
    \end{align*}

    \section*{ZAD 3.}

    Klasyczna definicja $\phi(x)$
    \begin{align*}
        \phi(\alpha)&=\alpha\\
        \phi(x_k)&=x_{k+1}\\
        \phi(x)&=x-{f(x)\over f'(x)}\\
        \phi'(x)&=1-{f'(x)^2-f(x)f''(x)\over f'(x)^2}={f(x)f''(x)\over f'(x)^2}
    \end{align*}
    Jesli jest to metoda liniowa, to:
    \begin{align*}
        x_{k+1}=\phi(x_k)&=\phi(\alpha)+(x_k-\alpha)\phi'(\xi_k)
    \end{align*}

    \begin{align*}
        \lim\limits_{x\to\alpha}\phi'(x)&=\lim\limits_{x\to\alpha}{f(x)f''(x)\over f'(x)^2}=\lim\limits_{x\to\alpha}{f(x)f'''(x)+f'(x)f''(x)\over 2f'(x)f''(x)}=\\
        &=\lim\limits_{x\to\alpha}{f'(x)f'''(x)+f(x)f^{(4)}(x)+f''(x)^2+f'(x)f'''(x)\over 2f''(x)^2+2f'(x)f'''(x)}=\\
        &={f''(\alpha)^2\over 2f''(\alpha)^2}=\frac12
    \end{align*}


    \section*{ZAD 4.}

    Metoda $\phi$ oblicza pierwiastek jakiejs funkcji $f$, ktory wypada w $\alpha$. Blad dla $n$-tego kroku wynosi
    $$E_n=x_n-\alpha$$
    a jezeli metoda jest rzedu $p+1$, to
    \begin{align*}
        \Big |{E_{n+1}\over (E_n)^{p+1}}\Big |< \infty
    \end{align*}

    \begin{align*}
        x_{k+1}&=\phi(x_k)=\\
        &=\phi(\alpha)+(x_k-\alpha)\phi'(\alpha)+(x_k-\alpha)^2{\phi''(\alpha)\over 2!}+...+(x_k-\alpha)^p{\phi^{(p)}(\alpha)\over p!}+(x_k-\alpha)^{p+1}{\phi^{(p+1)}(\xi_k)\over (p+1)!}
    \end{align*}
    dla pewnego $\xi_k\in[x_k,\alpha]$.
    \medskip

    Ale wszystko poza ostatnim wyrazem sie zeruje, wiec mamy
    $$x_{k+1}=\phi(x_k)=\alpha+(x_k-\alpha)^{p+1}{\phi^{(p+1)}(\xi_k)\over(p+1)!}$$
    i to do $\alpha$ bedzie zbiegac w potedze $p+1$.

    \begin{align*}
        {x_{k+1}-\alpha\over (x_k-\alpha)^{p+1}}&={\phi(x_k)-\alpha\over (x_k-\alpha)^{p+1}}=\\
        &={(x_k-\alpha)^{p+1}{\phi^{(p+1)}(\xi_k)\over(p+1)!}\over (x_k-\alpha)^{p+1}}=\\
        &={\phi^{(p+1)}(\xi_k)\over(p+1)!}\rarrow{}{k\to\infty}{\phi^{(p+1)}(\alpha)\over(p+1)!}
    \end{align*}

    
    \section*{ZAD 5.}

    Twierdzenie 3.5.15 z Kincaid Cheney
    \medskip

    Niech $z_1,z_2$ beda pierwiastkami pojedynczymi danego wielomianu.
    \medskip

    W kazdym kroku procesu wyliczamy
    $$p(z)=(z^2-uz-v)q(z)+b_1(z-u)+b_0.$$
    
    Pochodne czastkowe po $u$ i po $v$ to:
    \begin{align*}
        0&=-zq(z)+(z^2-uz-v){dq\over du}-b_1+{db_1\over du}(z-u)+{db_0\over du}\\
        0&=-q(z)+(z^2-uz-v){dq\over dv}+{b_1\over dv}(z-u)+{db_0\over dv}
    \end{align*}

    Teraz jesli podstawimy pod $z$ jeden z pierwiastkow, to dostaniemy
    \begin{align*}
        &\begin{cases}
            0=-z_kq(z_k)+{db_1\over du}(z_k-u)+{db_0\over du}\\
            0=-q(z_k)+{b_1\over dv}(z_k-u)+{db_0\over dv}
        \end{cases}\\
        &\begin{cases}
            z_kq(z_k)={db_1\over du}(z_k-u)+{db_0\over du}\\
            q(z_k)={b_1\over dv}(z_k-u)+{db_0\over dv}
        \end{cases}
    \end{align*}
    co zapisane w postaci macierzowej daje
    \begin{align*}
        \begin{bmatrix}
            {db_0\over du} & {db_1\over du}\\
            {db_0\over dv} & {db_1\over dv}
        \end{bmatrix}
        \begin{bmatrix}
            1 & 1\\
            z_1-u & z_2 - u
        \end{bmatrix}=
        \begin{bmatrix}
            z_1q(z_1) & z_2q(z_2)\\
            q(z_1) & q(z_2)
        \end{bmatrix}
    \end{align*}

    Z algebry wiemy, ze
    $$\det(AB)=\det(A)\det(B),$$
    wiec wystarczy pokazac, ze macierz z prawej strony rownania ma niezerowy wyznacznik.
    \begin{align*}
        \det\begin{bmatrix}
            z_1q(z_1) & z_2q(z_2)\\
            q(z_1) & q(z_2)
        \end{bmatrix}&=z_1q(z_1)q(z_2)-z_2q(z_1)q(z_2)=\\
        &q(z_1)q(z_2)(z_1-z_2)
    \end{align*}

    Poniewaz $z_1,z_2$ to pojedyncze pierwiastki $p(z)$, nie moga byc pierwiastkami wielomianu $q$, wiec $q(z_1)\neq0\neq q(z_2)$. Z kolei przez to, ze sa to rozne pierwiastki, to $z_1-z_2\neq0$.


    \section*{ZAD 6.}

    Rozwiazanie w pliku.


\end{document}