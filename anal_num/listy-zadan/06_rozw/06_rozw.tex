\documentclass{article}[16pt]

\usepackage{../../../uni-notes-eng}
\usepackage{multicol}
\usepackage{graphicx}

\begin{document}
\section*{ZAD 1.}

$$w(x)=\frac12 c_0T_0(x)+c_1T_1(x)+...+c_nT_n(x)$$

\begin{align*}
    \begin{matrix}
        B_{n+2}:=B_{n+1}:=0\\
        B_k:=2xB_{k+1}-B_{k+2}+c_k
    \end{matrix}
\end{align*}

wtedy $w(x)=\frac12(B_0-B_2)$.
\medskip

Wiemy, ze
$$\color{acc}T_k(x)=2xT_{k-1}(x)-T_{k-2}(x)$$

Indukcja po $n$? Dla $n=2$ mamy
\begin{align*}
    w(x)&=\frac12c_0T_0(x)+c_1T_1(x)+c_2T_2(x)=\frac12c_0+c_1x+c_2(2x^2-1)
\end{align*}

\begin{align*}
    B_4&=B_3=0\\
    B_2&=2xB_3-B_4+c_2=c_2\\
    B_1&=2xB_2-B_3+c_1=2xc_2+c_1\\
    B_0&=2xB_1-B_2+c_0=4x^2c_2+2xc_1-c_2+c_0\\
    w(x)&=\frac12(B_0-B_2)=\frac12(4x^2c_2+2xc_1-c_2+c_0-c_2)=2x^2c_2+xc_1-c    _2+\frac12c_0
\end{align*}
więc śmiga.

Załóżmu indukcyjnie, że algorytm działa dla dowolnego algorytmu zawierającego $T_0(x),...,T_n(x)$. Pokażemy, że działa wtedy też dla wielomianu z doklejonym $T_{n+1}(x)$.
\begin{align*}
    w(x)&=\frac12T_0(x)+...+c_nT_n(x)+c_{n+1}T_{n+1}(x)=\\
    &=\frac12T_0(x)+...+c_nT_n(x)+c_{n+1}(2xT_n(x)-T_{n-1}(x))=\\
    &=\frac12T_0(x)+...+T_{n-1}(x)(c_{n-1}-c_{n+1})+T_n(c_n+2xc_{n+1})
\end{align*}

Taki wielomian z założenia indukcyjnego można rozwiązać za pomocą algorytmu, więc mamy
\begin{align*}
    B_{n+2}&=B_{n+1}=0\\
    B_n&=2xB_{n+1}-B_{n+2}+c_n+2xc_n=c_n+2xc_{n+1}\\
    B_{n-1}&=2xB_n-B_{n+1}+c_{n-1}-c_{n+1}=4x^2c_{n+1}+2xc_n+c_{n-1}-c_{n+1}\\
    B_{n-2}&=2xB_{n-1}-B_nc_{n-2}...
\end{align*}
Rozważmy więc nowy ciąg, $C$, zdefiniowany rekurencyjnie:
\begin{align*}
    C_{n+3}&=C_{n+2}=0\\
    C_{n+1}&=c_{n+1}\\
    C_n&=2xc_{n+1}+c_n=B_n\\
    C_{n-1}&=2xC_n-C_{n+1}=4x^2c_{n+1}+2xc_n-c_{n+1}+c_{n-1}=B_{n-1}\\
    C_k&=2xC_{k+1}-C_{k+2}+c_k
\end{align*}
Ponieważ $C_n$ i $C_{n-1}$ odpowiadają $B_n$ i $B_{n-1}$ i oba ciągi mają tę samą definicję rekurencyjną, to są sobie równe od $n$ w dół. Skoro $C$ to algorytm dla $w(x)$ w całość, to
$$w(x)=\frac12(C_0-C_2)$$
i koniec.

\section*{ZAD. 3}

$$H_{2n+1}(x_i)=f(x_i)$$

Tutaj zauwazamy, ze drugi wyraz sumy zeruje sie dla kazdego $x_k$, bo
$$\sum\limits_{k=0}^nf'(x_k)\overline{h}_k(x_i)=\sum\limits_{k=0}^nf'(x_k)(x_i-x_k){(x_i-x_0)...(x_i-x_i)...(x_i-x_n)\over(x_i-x_k)p_{n+1}'(x_k)}=0$$

Teraz peirwszy wyraz, on chcialabym zeby sie uproscil do $f(x_i)$.

\begin{align*}
    h_k(x)=[1-2(x-x_k)\lambda_k'(x_k)]\lambda_k^2(x_k)
\end{align*}

Dla $k\neq i$ mamy
$$h_k(x_i)=[...]\lambda_l^2(x_i)=[...]{(x_i-x_0)...(x_i-x_{k-1})(x_i-x_{k+1})...(x_i-x_n)\over p_{n+1}'(x_i)}=[...]\cdot0=0$$
wiec wszystko poza $f(x_i)h_i(x_i)$ sie zeruje (winko rowniez).
\medskip

Chcemy teraz sprawdzic, czy $h_i(x_i)=1$
\begin{align*}
    h_i(x_i)&=[1-2(x_i-x_i)\lambda_i'(x_i)]\lambda_i^2(x_i)=\\
    &=[1-0]{(x_i-x_0)...(x_i-x_{i-1})(x_i-x_{i+1})...(x_i-x_n)\over \sum\limits_{l=0}^n\prod\limits_{j=0,j\neq i}^n(x_i-x_j)}=\\
    &={(x_i-x_0)...(x_i-x_{i-1})(x_i-x_{i+1})...(x_i-x_n)\over (x_i-x_0)...(x_i-x_i)...(x_i-x_n)+...+(x_i-x_0)...(x_i-x_{i-1})(x_i-x_{i+1})...(x_i-x_n)}=1.
\end{align*}

$$H_{2n+1}'(x_i)=f'(x_i)$$

\begin{align*}
    H_{2n+1}'(x)&=\sum\limits_{k=0}^nf(x_k)h_k'(x)+\sum\limits_{k=0}^nf'(x_k)\overline{h}'(x)
\end{align*}

Wpierw warto pokazac, ze

\begin{align*}
    h_k'(x_i)&=[1-2(x_i-x_k)\lambda_k'(x_k)]2\lambda_k(x_i)\lambda_k'(x_i)-2\lambda_k'(x_k)\lambda_k^2(x_i)
\end{align*}

Gdy $k=i$, to wtedy
\begin{align*}
    h_i'(x_i)&=[1-0]2\lambda_i(x_i)\lambda_i(x_i)-2\lambda_i'(x_i)\lambda_i^2(x_i)=\\
    &=2\lambda_i'(x_i)\lambda_i(x_i)[1-\lambda_i(x_i)]=\\
    &=(costam)[1-1]=0
\end{align*}

w przeciwnym wypadku
\begin{align*}
    h_i(x_i)&=[1-2(x_i-x_k)\lambda_k'(x_k)]2\lambda_k(x_i)\lambda_k'(x_i)-2\lambda_k'(x_k)\lambda_k^2(x_i)=\\
    &=2\lambda_k(x_i)[...]=2\cdot 0=0
\end{align*}

\begin{align*}
    \overline{h}_k'(x)&=\lambda_k^2(x)+2\lambda_k(x)\lambda_k'(x)(x-x_k)
\end{align*}

dla $k=i$ mamy
\begin{align*}
    \overline{h}_i'(x_i)&=\lambda_i^2(x_i)+2\lambda_i(x_i)\lambda_i'(x_i)(x_i-x_i)=\lambda_i^2(x_i)=1
\end{align*}

dla $k\neq i$ mamy
\begin{align*}
    \overline{h}_k'(x_i)&=\lambda_k^2(x_i)+2\lambda_k(x_i)\lambda_k'(x_i)(x_i-x_k)=\\
    &=0+2\cdot 0\cdot...=0.
\end{align*}


\section*{ZAD. 5}

Tabelka w ramach pomocy:

\begin{center}
    \begin{tabular}{ c c c c c c c c c c c }
        $x_0$ & & $x_0$ & & $x_1$ & & $x_1$ & & $x_2$ & & $x_2$\\
        7     & &  7    & &  6    & &  6    & &  22   & &  22\\
        &   -1    &  & -1   & &   0    & &  16   & &    56\\
        &    &     0    & &  1   & &    16   & &    40\\
        &    &     &    1   & &   ${15\over2}$ && 24\\
        & & & & ${13\over4}$ && ${33\over4}$\\
        & & & & & ${5\over2}$
    \end{tabular}
\end{center}

\begin{align*}
    p(x)&=7-(x+1)+x(x+1)^2+{13\over4}x^2(x+1)^2+{5\over2}x^2(x+1)^2(x-1)\\
    p'(x)&=\frac12x(25x^3+46x^2+30x+11)
\end{align*}


\section*{ZAD. 7}

Chcemy pokazac, ze
$$\int\limits_a^b[s''(x)]^2dx=\sum\limits_{k=1}^{n-1}s''(x_k)(f[x_k,x_{k+1}]-f[x_{k-1},x_k])$$

Zacznijmy od calki, calka na przedziale jest suma calek na podprzedzialach, czyli

\begin{align*}
    \int\limits_{x_0}^{x_n}[s''(x)]^2dx&=\sum\int\limits_{x_{k-1}}^{x_k}[s''(x)]^2dx=\\
    &=\sum\int\limits_{x_{k-1}}^{x_k}s''(x)s''(x)dx=\\
    &=\sum\begin{bmatrix}u=s''(x)& du=s''(x)\\v=s'(x)&dv=s''(x)\end{bmatrix}=\\
    &=\sum[s''(x_k)s'(x_k)-s''(x_{k-1})s'(x_{k-1})-\int\limits_{x_{k-1}}^{x_k}s'(x)(x_k-x_{k-1})^{-1}(s''(x_k)-s''(x_{k-1}))]=\\
    &=\sum[s''(x_k)s'(x_k)-s''(x_{k-1})s'(x_{k-1})]-\sum(x_k-x_{k-1})^{-1}\int\limits_{x_{k-1}}^{x_k}s'(x)(s''(x_k)-s''(x_{k-1}))dx=\\
    &=s''(x_1)s'(x)-s''(x_0)s'(x_0)+s''(x_2)s'(x_2)-s''(x_1)s'(x_1)+...+\\
    &-\sum(x_k-x_{k-1})^{-1}(s''(x_k)-s''(x_{k-1}))\int\limits_{x_{k-1}}^{x_k}s'(x)dx=\\
    &=-\sum (x_k-x_{k-1})^{-1}(s''(x_k)-s''(x_{k-1}))(s(x_k)-s(x_{k-1}))=\\
    &=-(x_1-x_0)^{-1}(s''(x_1)-s''(x_0))(s(x_1)-s(x_0))-(x_2-x_1)^{-1}(s''(x_2)-s''(x_1))(s(x_2)-s(x_1))-...+\\
    &-(x_n-x_{n-1})^{-1}(s''(x_n)-s''(x_{n-1}))(s(x_n)-s(x_{n-1}))=\\
    &=s''(x_1)\Big({s(x_2)-s(x_1)\over x_2-x_1}-{s(x_1)-s(x_0)\over x_1-x_0}\Big)+...+s''(x_{n-1})\Big({s(x_{n})-s(x_{n-1})\over x_{n}-x_{n-1}}-{s(x_{n-1})-s(x_{n-2})\over x_{n-1}-x_{n-2}}\Big)=\\
    &=\sum\limits_{k=1}^{n-1}s''(x_k)[{s(x_{k+1})-s(x_k)\over x_{k+1}-x_k}-{s(x_k)-s(x_{k-1})\over x_k-x_{k-1}}]=\sum\limits_{k=1}^{n-1}M_k(f[x_k, x_{k+1}]-f[x_k, x_{k-1}])
\end{align*}


\section*{ZAD. 8}

$$\int\limits_a^bs(x)dx=h\sum\limits_{k=0}^{n}{''}f(x_k)-{h^3\over12}\sum\limits_{k=0}^{n}{''}M_k$$

\begin{align*}
    \int M_{k-1}(x_k-x)^3dx&=M_{k-1}\int(x_k-x)^3dx=\\
    &M_{k-1}{-(x_k-x)^4\over4}=\\
    &=M_{k-1}{(x_k-x_{k-1})^4\over 4}=\\
    &=M_{k-1}{(a+kh-a-(k-1)h)^4\over4}=\\
    &=M_{k-1}{h^4\over4}
\end{align*}

\begin{align*}
    \int M_{k}(x-x_{k-1})^3dx&=M_k\int(x-x_{k-1})^3=\\
    &=M_k{(x-x_{k-1})^4\over4}=\\
    &=M_k{(x_k-x_{k-1})^4\over4}=\\
    &=M_k{(a+kh-a-(k-1)h)^4\over4}=\\
    &=M_k{h^4\over4}
\end{align*}

\begin{align*}
    \int(6f(x_{k-1})-M_{k-1}(x_k-x_{k-1})^2)(x_k-x)&=(6f(x_{k-1})-M_{k-1}h^2)\int(x_k-x)dx=\\
    &=(6f(x_{k-1})-M_{k-1}h^2){-(x_k-x)^2\over2}=\\
    &=(6f(x_{k-1})-M_{k-1}h^2){h^2\over2}
\end{align*}

\begin{align*}
    \int(6f(x_k)-M_kh^2)(x-x_{k-1})dx&=(6f(x_k)-M_kh^2){(x-x_{k-1})^2\over2}=\\
    &=(6f(x_k)-M_kh^2){h^2\over2}
\end{align*}

\begin{align*}
    \int\limits_a^bs(x)dx&=\sum\limits_{k=1}^n\int\limits_{x_{k-1}}^{x_k}s(x)dx=\\
    &=\sum\limits_{k=1}^nh^{-1}\frac16[M_{k-1}{h^4\over4}+M_k{h^4\over4}+6f(x_{k-1}-M_{k-1}h^2){h^2\over2}+(6f(x_k)-M_kh^2){h^2\over2}]=\\
    &=\sum\limits_{k=1}^nh^{-1}\frac16[-M_{k-1}{h^4\over4}-M_k{h^4\over4}+3(f(x_{k-1})+f(x_k))h^2]=\\
    &=\frac12f(x_0)h+\frac12f(x_1)h+\frac12f(x_1)h+\frac12f(x_2)h+...+\frac12f(x_{n-1})h+\frac12f(x_n)h+\\
    &-\frac1{24}M_0h^3-\frac1{24}M_1h^3-\frac1{24}M_1h^3-\frac1{24}M_2h^3-...-\frac1{24}M_{n-1}h^3-\frac1{24}M_nh^3=\\
    &=\frac12f(x_0)h+h\sum\limits_{k=1}^{n-1}f(x_k)+\frac12hf(x_n)-\frac1{24}M_0h^3-\frac1{12}h^3\sum\limits_{k=1}^{n-1}M_{k}-\frac1{24}M_nh^3
\end{align*}




\end{document}