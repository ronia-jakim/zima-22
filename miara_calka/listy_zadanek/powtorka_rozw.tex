\documentclass{article}

\usepackage{../../uni-notes-eng}

\begin{document}
\subsection*{ZAD. 1.}
\emph{Uzasadnić poniższe stwierdzenia, albo bezpośrednim argumentem, albo opierając się na poznanych faktach:}
\medskip

\emph{\color{def}Funkcja niemalejąca $h:\R\to\R$ jest borelowska.}
\smallskip

Niech $a\in \R$ oraz $y=f(a)$, wtedy zbiór
$$\{x\in\R\;:\;f(x)\leq f(a)\}=f^{-1}[(-\infty,y]$$
przy czym $(-\infty,y]\in Bor(\R)$, a wiemy, że to pociąga mierzalność funkcji.
\medskip

\emph{\color{def}Jeżeli zbiory $A_n,A\subseteq \R$ są borelowskie i $\lambda(A_n\Delta A)<\frac1n$ dla $n\in\N$, to istnieje ciąg $n_1<n_2<...$ taki, że funkcje charakterystyczne $\chi_{A_{n_k}}$ zbiegają do $\chi_A$ prawie wszędzie.}
\medskip

Zbieganie $\chi_{A_{n_k}}$ prawie wszędzie do $\chi_A$ oznacza, że zbiór gdzie się nie zgadzają jest miary zero. Nie zgadzają się na zbiorze $A\Delta A_{n_k}$, którego miara zbiega do zera. Koniec?
\medskip

\emph{\color{def}Jeżeli $A\subseteq\R$ jest zbiorem mierzalnym i $\lambda(A)=1$, to istnieje $r>0$ takie, że $\lambda(A\cap(-r,r))=\frac34$.}
\medskip

Może najpierw zróbmy funkcje $f:\R_+\to\R$ $f(x)=\lambda(A\cap(-x,x))$, gdzie dla $x=0$ przypisujemy $0$. Oczywiście taka funkcja jest zawsze nieujemna. Łatwo zobaczyć, że jest to funkcja ciągła oraz, że jej wartość nie może przekraczać $1$, bo $A\cap(-x,x)\subseteq A\implies \lambda(A\cap(-x,x))\leq\lambda(A)=1$. Dodatkowo, funkcja ta jest niemalejąca, bo dla $x<y$ mamy $A\cap(-x,x)\subseteq A\cap(-y,y)$. Czyli w pewnym miejscu musi przyjąć wartość $\frac34$.

\subsection*{ZAD. 2.}
\emph{Niech $f_n,f:(0,1)\to\R$ będą funkcjami mierzalnymi, takimi, że $|f_n(x)|\leq\frac1{\sqrt(x)}$ dla $x\in(0,1)$. Udowodnić, że jeżeli $f_n\xrightarrow[]{\lambda}f$, to $\lim_n\int_{[0,1]}|f_n-f|d\lambda$.}
\medskip

Po pierwsze, co to znaczy, że $f_n\xrightarrow[]{\lambda}f$:
$$\lim_n\lambda(\{x\;:\;|f_n(x)-f(x)|\geq\varepsilon\})$$
Ustalmy więc $\varepsilon>0$ i niech 
$$A=\{x\;:\;|f_n(x)-f(x)|\geq\varepsilon\}.$$


\subsection*{ZAD. 3.}
\emph{Niech $(X,\Sigma,\mu)$ będzie przestrzenią miarową, a $f_n,g_n:X\to\R$ będą funkcjami mierzalnymi.}
\medskip

\emph{\color{def}Udowodnić, że jeżeli $f_n\xrightarrow[]{\mu}f$ i $g_n\xrightarrow[]{\mu}g$, to $f_n - g_n\xrightarrow[]{\mu}f - g$}
\medskip

Ustalmy $\varepsilon>0$ i niech $N$ będzie takie, że dla każdego $n>N$ $|f_n-f| < \varepsilon$ oraz $|g_n-g|<\varepsilon$.

\begin{align*}
    |g_n-f_n-(g-f)|=|g_n-g+(f-f_n)|\leq|g_n-g|+|f-f_n|<2\varepsilon
\end{align*}

Poza zbiorem miary zero $\star$
\medskip

\emph{\color{def}Wyjaśnić, dlaczego warunki $f_n\xrightarrow[]{\mu}f$ i $f_n\xrightarrow[]{\mu}g$ implikują, że $f=g$ prawie wszędzie.}
\medskip

Wiemy, że poza pewnym zbiorem miary zero $A$ mamy $f_n(x)\to f(x)$ oraz poza $B$ miary zero $f_n(x)\to g(x)$. Co jeśli teraz weźmiemy $A\Delta B$? jest to nadal zbiór miary zero oraz
$$\lambda((A\Delta B)\cup A\cup B)\leq\lambda(A\Delta B)+\lambda(A)+\lambda(B)=0$$
ale $(A\Delta B)\cup A\cup B$ jest zbiorem gdzie $f_n(x)$ nie zbiega do $f(x)$ lub $f_n(x)$ nie zbiega do $g(x)$. Poza tym zbiorem mamy, że $f_n(x)\to f(x)$ oraz $f_n(x)\to g(x)$, więc $f(x)=g(x)$ poza tym brzydkim zbiorem.

\subsection*{ZAD. 4.}
\emph{Obliczyć i podać szczegółowe uzasadnienia rachunków:}
\medskip

Niech 
$$f_n={nx^2+1\over nx^4+n^2}={{x^2\over n}+\frac1{n^2}\over{x^4\over n}+1},$$ 
wtedy $f_n\to0$ prawie wszędzie, więc jak w zad. 2.

\begin{align*}
    \lim_{n\to\infty}\int_\R{nx^2+1\over nx^4+n^2}d\lambda(x)&=0
\end{align*}

\begin{align*}
    \int_0^1\sum_{n=1}^\infty n\cdot\chi_{(\frac1{n+1}, \frac 1n)}d\lambda
\end{align*}

Niech 
$$f_n=n{1\over n(n+1)}=\frac1{n+1}=\int_{[0,1]}n\cdot\chi_{(\frac1{n+1},\frac1n)}d\lambda.$$
Teraz popatrzmy na
$$g_N=\sum_{n=1}^N\frac1{n+1}=\sum_{n=1}^Nf_n=\sum_{n=1}^N\int_0^1n\cdot_{(\frac1{n+1},\frac1n)}d\lambda=\int_0^1\sum_{n=1}^Nn\cdot_{(\frac1{n+1},\frac1n)}d\lambda$$
oryginalnie bedzie $\infty$?


\subsection*{ZAD. 5.}
\emph{Niech funkcje mierzalne $f)n:[0,1]\to\R$ spiełniają warunek $\int_0^1|f_n|d\lambda\leq1$. Niech $B$ będzie zbiorem tych $x\in[0,1]$, dla których szereg $\sum_n {f_n(x)\over n^2}$ nie jest zbieżny. Udowodnić, że zbiór $B$ jest miary zero; wyjaśnić, dlaczego spełniona jest zależność $\int_0^1\sum_{n=1}^\infty$. Udowodnić, że zbiór $B$ jest miary zero; wyjaśnić dlaczego spełniona jest zależność}
$$\int_0^1\sum\limits_{n=1}^\infty{f_n(x)\over n^2}d\lambda=\sum\limits_{n=1}^\infty\int_0^1{f_n(x)\over n^2}d\lambda$$

\end{document}