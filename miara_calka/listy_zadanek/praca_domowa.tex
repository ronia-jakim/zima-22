\documentclass{article}

\usepackage{../../uni-notes-eng}

\title{Praca domowa ciąg dalszy\medskip

\large Problem \textbf{2.6.D}
}
\author{Weronika Jakimowicz}
\date{}

\color{black}
\pagecolor{white}

\begin{document}
\maketitle
\thispagestyle{empty}

\textbf{1. POKAZAĆ, ŻE DLA $x\in\Q$ mamy $f(x)=f(1)x$}
\medskip

Z addytywności funkcji mamy, że dla dowolnego $x\in\R$ i $k\in\N$ zachodzi:
$$f(kx)=kf(x).$$
Możemy pokazać to za pomocą indukcji. Dla $k=1$ jest to oczywista równość. Załóżmy, że dla $k\leq n$ jest to prawdą. Popatrzmy teraz na $k=n+1$
$$f((n+1)x)=f(xn+x)=f(nx)+f(x)\overset{*}{=}nf(x)+f(x)=(n+1)f(x)$$
gdzie równość z $*$ jest z założenia indukcyjnego.
\medskip

Specjalnym przypadkiem powyższej równości dla $k\in\N$ jest $x=\frac1k$. Wtedy mamy
$$f(kx)=kf(x)=kf(\frac1k)$$
a z drugiej strony
$$f(kx)=f(k\cdot\frac1k)=f(1),$$
czyli
$$f(\frac1k)=\frac1k f(1).$$

Liczby wymierne mają tę własność, że można je zapisać jako iloraz dwóch (względnie pierwszych) liczb naturalnych. Weźmy więc dowolny $q\in\Q$ taki, że $q={n\over k}$ dla $k,n\in\N$. Mamy wtedy
$$f(q)=f(\frac nk)=nf(\frac1k)=n\cdot\frac1k \cdot f(1)=f(1)\cdot{n\over k}=qf(1).$$
I to jest udowadniana zależność.
\bigskip

\textbf{2. JEŚLI $f$ MIERZALNA, TO $f(x)=f(1)x$ DLA $x\in\R$.}
\medskip

Dodając do przydatnych własności funkcji addytywnej, mamy
$$f(0)=f(0+0)=f(0)+f(0)=2f(0)\implies f(0)=0$$
$$0=f(0)=f(x-x)=f(x+(-x))=f(x)+f(-x)\implies f(-x)=-f(x)$$

Twierdzenie Steinhausa (1.11.H): jeżeli $A\subseteq\R$ jest mierzalny i $\lambda(A)>0$, to różnica kompleksowa $A$ z $A$ zawiera odcinek postaci $(-\delta,\delta)$ dla pewnego $\delta>0$.
\smallskip

$$|f(x)-f(y)|=|f(x)+f(-y)|=|f(x-y)|=f(|x-y|)$$

Aby pokazać ciągłość, chcemy, żeby dla każdego $\varepsilon>0$ istniało $\delta>0$ takie, że $\varepsilon>|f(x)-f(y)|=f(|x-y|)$ jeżeli $|x-y|<\delta$. Czyli tak naprawdę wystarczy nam pokazać ciągłość w okolicy zera: wtedy będziemy brać $z=x-y$ i sprawdzać, czy zachodzi $\delta>|z|\implies f(|z|)<\varepsilon$.
\medskip

Weźmy więc teraz dowolne $p\in\Q$, $p>0$ i niech $A=f^{-1}[(-p,p)]$. Ponieważ $f$ jest funkcją mierzalną, to również $A$ musi być mierzalne (Lemat 2.1.2). Jeśli $\lambda(A)>0$, to z twierdzenia Steinhausa wiemy, że dla istnieje $\delta>0$ takie, że $(-\delta,\delta)\subseteq (A-A)$. Popatrzmy teraz, jak wygląda obraz różnicy kompleksowej $A$
\begin{align*}
    f[(-\delta,\delta)]\subseteq f[A-A]&=f[\{x-y\;:\;x,y\in A\}]=\{f(x-y)\;:\;x,y\in A\}=\\
    &=\{f(x)-f(y)\;:\;x,y\in f^{-1}[(-p,p)]\}=\{x-y\;:\;x,y\in(-p,p)\}=(-2p,2p)
\end{align*}

Zauważmy, że dla $x,y\in(-p,p)$ mamy
$$|x-y|<|p-(-p)|=2p$$

Podsumowując, dla dowolnego $p>0$ istnieje $\delta>0$ taka, że dla $|x-y|<2\delta$, to znaczy $x,y\in(-\delta,\delta)$ takie, że 
$$|f(x-y)|=|f(x)-f(y)|<2p,$$
bo $f[(-\delta,\delta)]\subseteq (-2p,2p)$. To daje ciągłość w zerze, co z addytywności przekłada sie na ciągłość na całej prostej.


\end{document}