\documentclass{article}[13pt]

\usepackage{../uni-notes-eng}
\usepackage{multicol}
\usepackage{graphicx}

\setlength{\columnsep}{1cm}

\title{Miara i calka\medskip\\{\normalsize speedrun przed terminem 0}}
\author{by a MEEEE}
\date {21.03.2137}

\newcommand{\bor}{Bor(\R)}
\newcommand{\obor}{Bor(\R\times\R)}

\begin{document}
\maketitle
\thispagestyle{empty}

\subsection*{ROZDZIAŁ 1}

\begin{multicols}{2}
    
    $\color{def}\bor$ to najmniejsze $\sigma$-ciało zawierające wszystkie zbiory otwarte z $\R$
    \medskip

    Dla $\rodz R\subseteq\Po{X}$ funkcję $\mu:\rodz R\to[0,\infty]$ nazywamy {\color{def}addytywną funkcją zbioru} (\emph{miarą skończenie addytywną}), jeżeli
    \smallskip
    
    \point $\mu(\emptyset)=0$

    \point $A\cap B=\emptyset\implies \mu(A\cup B)=\mu(A)+\mu(B)$.
    \medskip

    {\color{def}Przeliczalnie addytywna funkcja zbioru} to $\mu$ takie, że dla parami rozłącznych $A_n$ zachodzi $\mu(\bigcup A_i)=\sum\mu(A_i)$. Jeżeli nie są rozłączne, to $\mu(\bigcup A_i)\leq\sum\mu(A_i)$.
    \medskip

    Następujące warunki są równoważne:
    \smallskip

    \point $\mu$ przeliczalnie addytywna

    \point $\mu$ jest ciągła z dołu, czyli jeśli $\color{acc}A_n\uparrow A$, to zachodzi $\color{acc}\lim_n\mu(A_n)=\mu(A)$

    \point $\mu$ jest ciągła z góry, jeśli $\color{acc}A_n\downarrow A$ to $\color{acc}\lim_n\mu(A_i)=\mu(A)$

    \point $\mu$ jest ciągła z góry na zbiorze $\emptyset$, czyli dla $\color{acc}A_n\downarrow\emptyset$ mamy $\color{acc}\lim_n\mu(A_n)=0$
    \medskip

    $[a_n,b_n)$ jest ciągiem parami rozłącznych przedziałów zawartych w $[a,b)$, to $\sum(b_n-a_n)\leq b-a$. Natomiast, jeżeli $[a,b)\subseteq\bigcup[a_n,b_n)$, to $b-a\leq\sum(b_n-a_n)$.
    \medskip

    \podz{sep}
    \medskip

    $\mu$ jest {\color{def}$\sigma$-skończona }na pierścieniu $\rodz{R}$, jeżeli istnieją $X_n$ takie, że $X=\bigcup X_n$ i $\mu(X_n)<\infty$.
    \medskip

    {\color{def}Twierdzenie o konstrukcji miary}: funckja $\sigma$-skończona oraz przeliczalnie addytywna rozszerza się do jednoznacznie do miary na $\sigma(\rodz{R})$.
    \medskip

    Dla każdego $A\in\sigma(\rodz{R})$ o $\mu(A)<\infty$ i $\varepsilon>0$ istnieje $R\in\rodz{R}$ takie, że $\mu(A\Delta R)<\varepsilon$.
    \medskip

    \podz{sep}
    \medskip

    Przestrzeń miarową $(X, \Sigma, \mu)$ nazywamy {\color{def}skończoną} jeśli $\mu(X)<\infty$, {\color{def}probabilistyczną}, jeżeli $\mu(X)=1$ lub {\color{def}$\sigma$-skończoną} jeżeli istnieją $X_k\in\Sigma$, $X=\bigcup X_k$ i $\mu(X_k)<\infty$.
    \medskip

    Przestrzeń miarowa jest {\color{def}zupełna}, jeśli dla każdego $A\in\Sigma$, $\mu(A)=0$ oznacza, że wszystkie podzbioru $A$ należą do $\Sigma$. Wtedy $\Sigma$ jest {\color{def}$\sigma$-ciałem zupełnym względem $\mu$}.
    \medskip

    Dla każdej przestrzeni miarowej $(X, \Sigma,\mu)$ istnieje przestrzeń miarowa $(X,\hat{\Sigma}, \hat{\mu})$ taka, że $\Sigma\subseteq \hat{\Sigma}$ oraz $\hat{\mu}$ jest rozszerzeniem $\mu$.
    \medskip

    \podz{sep}
    \medskip

    {\color{def}Zbiory mierzalne w sensie Lesbegue'a} [$\mathfrak{L}$] to zbiory które można zapisać jako $A=B\Delta N$ gdzie $B\in\bor$ i $N$ jest podzbiorem zbioru miary zero.
    \medskip

    Dla każdego zbioru mierzalnego $A\in\mathfrak{L}$ istnieje zbiór otwarty $V$ i zbiór domknięty $F$ takie, że $\color{acc}F\subseteq A\subseteq V$ i $\color{acc}\lambda(V\setminus F)<\varepsilon$. Dla $\lambda(A)<\infty$ możemy znaleźć {\color{dyg}skończoną sumę odcinków $J$ }taką, że $\lambda(A\Delta J)<\varepsilon$ oraz {\color{dyg}$K$ zwarty zawarty w $A$} taki, że $\lambda(A\setminus K)<\varepsilon$.
    \medskip

    Dla dowolnego $B\in\bor$ i $x\in\R$ mamy $x+B\in\bor$ oraz $\lambda(B)=\lambda(x+B)$.
    \medskip

    Rodzinę $\rodz{M}\subseteq\Po{X}$ nazywamy {\color{def}klasą monotoniczną}, jeśli dla dowolnego $A_n\in\rodz{M}$ mamy $A_n\uparrow A$ lub $A_n\downarrow A$, to $A\in\rodz{M}$.

    Klasa monotoniczna zawierająca pierścień to zawiera też $\sigma$-pierścień przez niego generowany.
    \medskip
    Niech $\mu$ będzie przeliczalnie addytywną funkcją zbioru na pierścieniu, załóżmy że $X=\bigcup S_k$ dla $S_k\in\rodz{R}$ takich, że $\mu(S_k)<\infty$, wtedy $\mu$ ma co {\color{acc}co najwyżej jedno przedłużenie do miary na $\sigma(\rodz{R})$}.
    \medskip

    \podz{sep}
    \medskip

    

\end{multicols}

\subsection*{ROZDZIAŁ 2}

\begin{multicols}{2}
    {\color{def}Funkcja $\Sigma$-mierzalna} $f:X\to\R$ to funkcja, która dla każdego $f^{-1}[B]\in\Sigma$ spełnia $B\in\bor$, równoważnie jeżeli $\rodz{G}\subseteq\bor$ takie, że $\sigma(\rodz{G})=\bor$, to wystarczy dla każdego $G\in\rodz{G}$ $f^{-1}[G]\in\Sigma$.
    \medskip

    Każdy z poniższych pociąga mierzalność:
    \smallskip

    $\{x\;:\;f(x)<t\}\in\Sigma$

    $\{x\;:\;f(x)\leq t\}\in\Sigma$

    $\{x\;:\;f(x)>t\}\in\Sigma$

    $\{x\;:\;f(x)\geq t\}\in \Sigma$
    \medskip

    Jeżeli funkcja $f:X\to\R$ jest $\Sigma$-mierzalna, a $g:\R\to\R$ jest ciągła, to $g\circ f:X\to\R$ jest $\Sigma$-mierzalna.
    \medskip

    {\color{acc}Granica punktowa} zbieżnego ciągu funkcji mierzalnych jest mierzalna.
    \medskip

    Każdą $\Sigma$-mierzalna funkcję $f:X\to\R$ można zapisać w postaci $f^+-f^-$, różnicy funkcji mierzalnych i nieujemnych.
    \medskip

    \podz{sep}
    \medskip

    {\color{def}Funkcja prosta} to funkcja o skończonym zbiorze wartości, czyli kombinacja liniowa skończenie wielu {\color{acc}funkcji charakterystycznych}
    \medskip

    Ciąg funkcji mierzalnych jest {\color{def}zbieżny prawie wszędzie}, jeżeli $\lim_nf_n(x)=f(x)$ poza zbiorem miary zero.
    \medskip

    Dla każdej $\lambda$-mierzalnej funkcji $f$ istnieje borelowska funkcja $g$ taka, że $f=g$ $\lambda$-prawie wszędzie.
    \medskip

    Jeżeli $f_n\to f$ prawie wszędzie, to dla każdego $\varepsilon>0$ istnieje $A\in\Sigma$ o $\mu(A)<\varepsilon$ i $f_n$ jest jednostajnie zbieżny do $j$ na zbiorze $A^c$.
    \medskip

    Ciąg funkcji mierzalnych jest {\color{def}niemal jednostajnie zbieżny}, jeżeli dla każdego $\varepsilon>0$ ciąg $f_n$ zbiega jednostajnie na dopełnieniu pewnego zbioru miary $<\varepsilon$.
    \medskip

    Mówimy, że ciąg jest {\color{def}zbieżny według miary}, jeżeli dla każdego $\varepsilon$ $\lim_n\mu(\{x\;:\;|f_n(x)-f(x)|\geq\varepsilon\})=0$.
    \medskip

    {\color{acc}\point} Zbieżny niemal jednostajnie $\implies$ zbieżny według miary.

    {\color{acc}\point} Zbieżny prawie wszędzie w mierze skończonej $\implies$ zbieżny według miary.
    \medskip

    {\color{def}Twierdzenie Riesza}: jeżeli ciąg funkcji spełnia {\color{acc}warunek Cauchy'ego według miary}, czyli dla dowolnego $\varepsilon>0$
    $$\lim_{n,m}\mu(\{x\;:\;|f_n(x)-f_m(x)|\geq\varepsilon\})=0$$
    to $f_n$ jest zbieżny według miary do pewnego $f$ oraz istnieje podciąg liczb naturalnych $n(k)$ taki, że $f_{n(k)}$ jest zbieżny prawie wszędzie oraz według miary.
\end{multicols}
\bigskip

\podz{sep2}
\bigskip

\subsection*{ROZDZIAŁ 3}

\begin{multicols}{2}
    Całkę po funkcji prostej $f=\sum a_i\chi_{A_i}$ definiujemy jako
    $$\int_Xfd\mu=\sum a_i\mu(A_i)$$

    Dla nieujemnej mierzalnej funkcji $f$ definiujemy
    $$\color{acc}\int_Xfd\mu=\sup\{\int_Xsd\mu\;:\;0\leq s\leq f\},$$
    czyli jeżeli $s_1\leq s_2\leq ...$ jest ciągiem funkcji prostych takich, że $\lim_ns_n=f$ prawie wszędzie, to
    $$\int_xfd\mu=\lim_x\int_Xs_nd\mu.$$

    Funkcja mierzalna jest {\color{def}całkowalna}, jeżeli $\color{acc}\int_X|f|d\mu<\infty$, wtedy definiujemy całkę wzorem $\int_Xfd\mu=\int_Xf^+d\mu-\int_Xf^-d\mu$ dla $f=f^x-f^-$ nieujemnych.
    \medskip

    Dla {\color{def}$f,g$ całkowalnych i $h$ mierzalnej}:
    \smallskip

    \point $\int_X(af+gb)d\mu=a\int_Xfd\mu+b\int_Xgd\mu$

    \point $h=0$ prawie wszędzie, to $\int_Xhd\mu=0$

    \point $f\leq g$ prawie wszędzie, to $\int_Xfd\mu\leq\int_Xgd\mu$

    \point $|\int_X(f+g)d\mu|\leq\int_X|f|d\mu+\int_X|g|d\mu$
    
    \point $a\leq f\leq b$ prawie wszędzie, to $a\mu(X)\leq\int_Xfd\mu\leq b\mu(X)$

    \point dla $A,B\in\Sigma$ jeżeli $A\cap B=\emptyset$, to $\int_{A\cup B}fd\mu=\int_Afd\mu+\int_Bfd\mu$
    \medskip

    \podz{sep}
    \medskip

    {\color{def}Twierdzenie o zbieżności monotonicznej:} niech $f_n$ będzie ciągiem nieujemnych funkcji mierzalnych takich, że $f_1\leq f_2\leq...$ zbieżnych prawie wszędzie do $f=\lim_nf_n$ 
    $$\int_Xfd\mu=\lim_n\int_Xf_nd\mu$$

    {\color{def}Lemat Fatou}: dla dowolnego ciągu funkcji nieujemnych $f_n$ zachodzi
    $$\int_x\lim\inf f_nd\mu=\lim\inf\int_X f_nd\mu$$

    {\color{def}Twierdzenie Lesbegue'a o zbieżności ograniczonej}: niech $f_n,g$ będą mierzalne, że dla każdego $n$ $|f_n|\leq g$ zachodzi prawie wszędzie przy czym $\int_Xgd\mu<\infty$. Jeżeli {\color{def}$f=\lim_nf_n$ prawie wszędzie}, to
    $$\lim_n\int_X|f_n-f|d\mu=0$$
    $$\lim_n\int_Xfd\mu=\lim\int_Xf_nd\mu$$

    Jeżeli teraz $\mu(X)<\infty$ oraz $f_n$ są wspólnie ograniczone i $\int_X\lim_nf_nd\mu=\lim_n\int_Xf_nd\mu$.
    \medskip

    Jeżeli $f$ jest mierzalna i nieujemna, to funkcja $\nu:\Sigma\to[0,\infty]$
    $$\nu(A)=\int_Afd\mu$$
    jest {\color{def}miarą na $\Sigma$}.
    \medskip

    \podz{sep}
    \medskip

    Jeżeli $f:[a,b]\to\R$ jest całkowalna w sensie Riemanna, to jest $\lambda$-mierzalna i obie całki są sobie równe: $\int_a^bf(x)dx=\int_{[a,b]}fd\lambda$.

\end{multicols}
\bigskip

\podz{sep2}
\bigskip

\subsection*{ROZDZIAŁ 4}

\begin{multicols}{2}

    Niech $(X,\Sigma)$ i $(Y,\Theta)$ będą przestrzeniami z $\Sigma,\Theta$ będącymi $\sigma$-ciałami. W $X\times Y$ możemy zdefiniować następujące $\sigma$-ciało:
    $$\Sigma\otimes\Theta=\sigma(\{A\times B\;:\;A\in\Sigma, B\in\Theta\}),$$
    wtedy $\Sigma\otimes\Theta$ jest {\color{def}produktem $\sigma$-ciał} $\Sigma$ i $\Theta$.
    \medskip

    Zbiór $F\subseteq X\times Y$ należy do ciała prostokątów na $\Sigma,\Theta$ wtw $F=\bigcup A_i\times B_i$ dla $A_i\in\Sigma$ i $B_i\in\Theta$.

    Jeżeli $E\in\Sigma\otimes\Theta$, to dla każdego $x\in X$ i $y\in Y$ definiujemy {\color{acc}cięcia pionowe i poziome} $E_x=\{z\in Y\;:\;(x,z)\in E\}$ i $E^y=\{z\in Z\;:\;(z,y)\in E\}$.
    \medskip

    Jeżeli $E\in\Sigma\otimes\Theta$, a funkcja $f:X\times Y\to\R$ jest $\Sigma\otimes\Theta$-mierzalna, to funkcja $f_x$ jest $\theta$-mierzalna, a $f^y$ jest $\Sigma$-mierzalna.
    \medskip

    $$\bor\otimes\bor=\obor$$

    {\color{def}Miarę} na produkcie przestrzeni $(X,\Sigma,\mu)$ i $(Y,\Theta,\nu)$ definiujemy:
    $$\color{acc}\mu\otimes\nu(A\times B)=\mu(A)\cdot\mu(B)$$

    \podz{sep}
    \medskip

    Na $\sigma$-ciele $\Sigma\otimes\Theta$ istnieje jedyna miara spełniająca dla każdego $A\in\Sigma$ i $B\in\Theta$
    $$\mu\otimes\nu(A\times B)=\mu(A)\cdot\nu(B)$$
    a dla dowolnego $E\in\Sigma\otimes\Theta$ funkcja $x\mapsto\nu(E_x)$ i $y\mapsto\mu(E^y)$ są mierzalne i zachodzą wzory:
    $$\mu\otimes\nu(E)=\int_X\nu(E_x)d\mu(x)=\int_Y\mu(E^y)d\nu(y)$$

    \podz{sep}
    \medskip

    {\color{def}Twierdzenie Fubiniego}: jeżeli funkcja $f:X\times Y\to\R$ jest $\Sigma\otimes\Theta$-mierzalna oraz $f$ jest nieujemna lub $f$ jest $\mu\otimes\nu$-całkowalna, to wtedy
    $$I:x\mapsto\int_Yf(x,y)d\nu(y)$$
    $$J:y\mapsto\int_Yf(x,y)d\mu(x)$$
    są mierzalne względem $\Sigma$ i odpowiednio $\Theta$, oraz
    \begin{align*}
        \int_{X\times Y}fd\mu\otimes\nu&=\int_X\left(\int_Yf(x,y)d\nu(y)\right)d\mu(x)=\\
        &=\int_Y\left(\int_Xf(x,y)d\mu(x)\right)d\nu(y)
    \end{align*}
    Czyli możemy {\color{acc}bezkarnie zamieniać granice całkowania}.
    \medskip

    \podz{sep}
    \medskip

    Jeżeli rozważamy przestrzenie metryczne $(X_i,\Sigma_i,\mu_i)$ dla $i=1,...,n$, to na $\sigma$-ciele $\color{acc}\bigotimes_{i\leq n}\Sigma_i$ podzbiorów $\prod_{i\leq n}X_i$ generowanym przez wszystkie kostki mierzalne $A_1\times...\times A_n$, to {\color{def}istnieje jedyna miara} $\mu=\bigotimes\mu_i$ spełniająca dla wszystkich $A_i\in\Sigma_i$
    $$\mu(A_1\times...\times A_n)=\mu_1(A_1)\cdot...\cdot\mu_n(A_n)$$

    \podz{sep}
    \medskip

    $K=\{0,1\}^\N$ - przestrzeń ciągów $0,1$, wtedy możemy określić metrykę $d(x,y)=\frac1n$ gdzie $n$ to pierwszy indeks gdzie te dwa ciągi się różnią.
    \medskip

    Funkcja $f:K\to[0,1]$ zadana $f(x)=\sum\limits_{n=1}^\infty{2x(n)\over3^n}$ jest homeomorfizmem między $K$ a zbiorem Cantora. Dla funkcji $\psi:\N\supseteq A\to\{0,1\}$ oznaczamy zbiór 
    $$[\psi]=\{x\in K\;:\;x(i)=\psi(i)\;i\in A\}.$$
    Cylindry ten postaci są otwarto-domknięte w $K$ i stanowią bazę topologii w $K$. Oznaczmy przez $\rodz C$ ciało podzbiorów $K$ generowanych przez wszystkie cylindry. Zbiór $C$ jest w $\rodz C$ wtw istnieje $n$ i $C'\subseteq\{0,1\}^n$ takie, że $C=C'\times\{0,1\}\times...$.
    \medskip

    Dla $C\in\rodz C$ definiujemy {\color{def}miarę na $K$}:
    $$\nu(C)={|C'|\over2^n}$$
    która spełnia własność: 
    $$\color{acc}(\forall\;B\in Bor(K))(\forall\;\varepsilon>0)(\exists\;C\in\rodz C)\;\nu(B\Delta C)<\varepsilon.$$
    
\end{multicols}
\bigskip

\podz{sep2}
\bigskip

\subsection*{ROZDZIAŁ 5}

\begin{multicols}{2}
    
    {\color{def}Miara znakowana} $\alpha:\Sigma\to[-\infty,\infty]$ to funkcja która przyjmuje co najwyżej jedną z wartości nieskończonych oraz
    $$\alpha(\emptyset)=0$$
    $$\alpha(\bigcup_nA_n)=\sum_n\alpha(A_n)$$
    dla każdego ciągu rozłącznych $A_i$.
    \medskip

    {\color{def}Rozkłada Hahna}: jeśli $\alpha$ jest miarą znakowaną na $\sigma$-ciele $\Sigma$ podzbiorów $X$, to istnieją rozłączne zbioru $X^+$ i $X^-$ takie, że $\color{acc}X=X^+\cup X^-$, $X=X^+\cup X^-$ oraz dla dowolnego $A\in\Sigma$, to
    
    \point jeżeli $A\subseteq X^+$, to $\alpha(A)\geq0$,

    \point a jeżeli $A\subseteq X^-$, to $\alpha(A)\leq0$.
    \medskip

    {\color{def}Rozkład Jordana}: jeżeli $\alpha$ jest miarą znakowaną na $\Sigma$, to istnieją miary $\alpha^+,\alpha^-$ takie, że $\alpha=\alpha^+-\alpha^-$.
    \medskip

    \podz{sep}
    \medskip

    Miara $\nu$ jest {\color{def}absolutnie ciągła względem miary $\mu$}, jeżeli $(\forall\;A\in\Sigma)\;\mu(A)=0\implies\nu(A)=0$. Oznaczamy $\color{acc}\nu\ll\mu$.
    \medskip

    Miara $\nu$ jest {\color{def}singularna względem miary} $\mu$, jeżeli istnieją $A,B\in\Sigma$ takie, że $X=A\cup B$, $A\cap B=\emptyset$, $\mu(A)=0$ i $\nu(B)=0$. Relację tę oznaczamy $\color{acc}\mu\perp\nu$.
    \medskip

    {\color{def}Absolutne wahanie} miary znakowanej $\alpha$ to $|\alpha|=\alpha^++\alpha^-$. Jeżeli $\alpha,\beta$ są określone na tym samym $\Sigma$, przyjmujemy $\alpha\ll\beta$ gdy $|\alpha|\ll|\beta|$, tak samo $\alpha\perp\beta$ jeżeli $|\alpha|\perp|\beta|$.
    \medskip

    $\nu$ jest miarą skończoną na $\Sigma$, to dla dowolnej $\mu$ na $\Sigma$
    \begin{align*}
        \nu\ll\mu\iff(\forall\;\varepsilon>0)(\exists\;\delta)(\forall\;A\in\Sigma)\;\mu(A)<\delta\implies\nu(A)<\varepsilon
    \end{align*}
    \medskip

    \podz{sep}
    \medskip

    $\mu,\nu$ - skończone miary na $\Sigma$ takie, że $\nu\neq0$ i $\nu\ll\mu$. Wtedy istnieje $P\in\Sigma$ takie, że $\mu(P)>0$ i $P$ jest pozytywny dla miary znakowanej $\nu-\varepsilon\mu$, to znaczy $\nu(B)\geq\varepsilon\mu(B)$ dla każdego mierzalnego $B\subseteq P$.
    \medskip

    {\color{def}Twierdzenie Radona-Nikodyma}: $(X,\Sigma,\mu)$ - $\sigma$-skończona przestrzeń miarowa i $\nu$ jest miarą znakowaną na $\Sigma$, że $|\nu|$ jest $\sigma$-skończone. Jeśli $\nu\ll\mu$, to istnieje $f:X\to\R$ takie, że dla wszystkich $A\in\Sigma$
    $$\color{acc}\nu(A)=\int_Afd\mu$$

    Funkcja $f$ spełniająca tezę twierdzenia Radona-Nikodyma bywa oznaczana $f={d\nu\over d\mu}$ i mówi się na nią {\color{dyg}pochodna Radona-Nikodyma} miary $\nu$ względem miary $\mu$.
    \medskip

    Dla miar $\mu$ i $\nu$ jak w twierdzeniu wyżej
    $$\color{acc}\int_Xgd\nu=\int_Xg\cdot{d\nu\over d\mu}d\mu$$
    dla każdej $\nu$-całkowalnej funkcji $g$.
    \medskip

    Dla $A\in\Sigma$ w $\sigma$-skończonej przestrzeni miarowej, dla $\sigma$-ciała $\Sigma_0\subseteq\Sigma$, istnieje $\Sigma_0$-mierzalna funkcja $f$ taka, że dla każdego $B\in\Sigma_0$
    $$\color{acc}\mu(A\cap B)=\int_Bfd\mu$$

    Dla $\mu,\nu$ $\sigma$-skończonych miar na tym samym ciele, wtedy istnieje $\color{acc}\nu=\nu_a+\nu_s$ taki, że $\nu_a\ll\mu$ i $\nu_s\perp\mu$.
    \medskip

    \podz{sep}
    \medskip

    Jeżeli $\mu,\nu$ są miarami na $\bor$ i dla każdego $a<b$ mamy $\mu[a,b)=\nu[a,b)<\infty$, to $\mu=\nu$.
    \medskip

    {\color{def}Miara Lebesgue'a-Stieltjesa} $\lambda_F$: dla każdej $F:\R\to\R$ lewostronnie ciągłej, niemalejącej istnieje jedyna miara $\lambda_F$ określona na $\bor$ taka, że:
    $$\lambda_F[a,b)=F(b)-F(a)$$

    Jeżeli $F$ niemalejąca ma ciągłą pochodną, to dla każdej $\lambda_F$-całkowalnej $g$:
    $$\int_\R gd\lambda_F=\int_\R g\cdot F'd\lambda$$

\end{multicols}
\bigskip

\podz{sep2}
\bigskip

\subsection*{ROZDZIAŁ 6}

\begin{multicols}{2}
    
    Dla dowolnych liczb dodatnich $a,b,p,q$ takich, że $\frac1p+\frac1q=1$ zachodzi
    $$ab\leq{a^p\over p}+{b^q\over q}$$

    {\color{def}$p$-norma całkowa funkcji} to $\|f\|_p=\left(\int_X|f|^pd\mu\right)^{\frac1p}$.
    \medskip

    {\color{def}Nierówność Cauchy'ego-H\"oldera}: dla dowolnych $f,g$ i $p,q>0$ jak wyżej zachodzi
    $$\color{acc}\int_X|fg|d\mu\leq\|f\|_p\|g\|_q$$

    {\color{def}Nierówność Minkowskiego}: $\|f+g\|_p\leq\|f\|_p+\|g\|_p$
    \medskip

    \podz{sep}
    \medskip

    Warunki, aby $\|\cdot\|$ było {\color{def}normą}:
    \smallskip

    \point $\|x\|=0\iff x=0$

    \point $\|cx\|=|c|\|x\|$

    \point $\|x+y\|\leq\|x\|+\|y\|$
    \smallskip

    Możemy określić na podstawie normy metrykę $p(x,y)=\|x-y\|$. Jeżeli przestrzeń unormowana z taką metryką jest zupełna, to jest nazywana {\color{def}przestrzenią Banacha}.
    \medskip

    Symbol $\color{acc}L_p(\mu)$ oznacza przestrzeń wszystkich funkcji mierzalnych $f:X\to\R$ dla których $\|f\|_p<\infty$. Elementy z $L_p(\mu)$ są utożsamiane jeżeli są równe prawie wszędzie. Jest to {\color{dyg}przestrzeń Banacha}.

\end{multicols}

\end{document}