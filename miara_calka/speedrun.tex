\documentclass{article}[13pt]

\usepackage{../uni-notes-eng}
\usepackage{multicol}
\usepackage{graphicx}

\setlength{\columnsep}{1cm}

\title{Miara i calka\medskip\\{\normalsize speedrun przed terminem 0}}
\author{by a MEEEE}
\date {21.03.2137}

\newcommand{\bor}{Bor(\R)}
\newcommand{\obor}{Bor(\R\times\R)}

\begin{document}
\maketitle
\thispagestyle{empty}

\subsection*{ROZDZIAŁ 2}

\begin{multicols}{2}
    {\color{def}Funkcja $\Sigma$-mierzalna} $f:X\to\R$ to funkcja, która dla każdego $f^{-1}[B]\in\Sigma$ spełnia $B\in\bor$, równoważnie jeżeli $\rodz{G}\subseteq\bor$ takie, że $\sigma(\rodz{G})=\bor$, to wystarczy dla każdego $G\in\rodz{G}$ $f^{-1}[G]\in\Sigma$.
    \medskip

    Każdy z poniższych pociąga mierzalność:
    \smallskip

    $\{x\;:\;f(x)<t\}\in\Sigma$

    $\{x\;:\;f(x)\leq t\}\in\Sigma$

    $\{x\;:\;f(x)>t\}\in\Sigma$

    $\{x\;:\;f(x)\geq t\}\in \Sigma$
    \medskip

    Jeżeli funkcja $f:X\to\R$ jest $\Sigma$-mierzalna, a $g:\R\to\R$ jest ciągła, to $g\circ f:X\to\R$ jest $\Sigma$-mierzalna.
    \medskip

    {\color{acc}Granica punktowa} zbieżnego ciągu funkcji mierzalnych jest mierzalna.
    \medskip

    Każdą $\Sigma$-mierzalna funkcję $f:X\to\R$ można zapisać w postaci $f^+-f^-$, różnicy funkcji mierzalnych i nieujemnych.
    \medskip

    \podz{sep}
    \medskip

    {\color{def}Funkcja prosta} to funkcja o skończonym zbiorze wartości, czyli kombinacja liniowa skończenie wielu {\color{acc}funkcji charakterystycznych}
    \medskip

    Ciąg funkcji mierzalnych jest {\color{def}zbieżny prawie wszędzie}, jeżeli $\lim_nf_n(x)=f(x)$ poza zbiorem miary zero.
    \medskip

    Dla każdej $\lambda$-mierzalnej funkcji $f$ istnieje borelowska funkcja $g$ taka, że $f=g$ $\lambda$-prawie wszędzie.
    \medskip

    Jeżeli $f_n\to f$ prawie wszędzie, to dla każdego $\varepsilon>0$ istnieje $A\in\Sigma$ o $\mu(A)<\varepsilon$ i $f_n$ jest jednostajnie zbieżny do $j$ na zbiorze $A^c$.
    \medskip

    Ciąg funkcji mierzalnych jest {\color{def}niemal jednostajnie zbieżny}, jeżeli dla każdego $\varepsilon>0$ ciąg $f_n$ zbiega jednostajnie na dopełnieniu pewnego zbioru miary $<\varepsilon$.
    \medskip

    Mówimy, że ciąg jest {\color{def}zbieżny według miary}, jeżeli dla każdego $\varepsilon$ $\lim_n\mu(\{x\;:\;|f_n(x)-f(x)|\geq\varepsilon\})=0$.
    \medskip

    {\color{acc}\point} Zbieżny niemal jednostajnie $\implies$ zbieżny według miary.

    {\color{acc}\point} Zbieżny prawie wszędzie w mierze skończonej $\implies$ zbieżny według miary.
    \medskip

    {\color{def}Twierdzenie Riesza}: jeżeli ciąg funkcji spełnia {\color{acc}warunek Cauchy'ego według miary}, czyli dla dowolnego $\varepsilon>0$
    $$\lim_{n,m}\mu(\{x\;:\;|f_n(x)-f_m(x)|\geq\varepsilon\})=0$$
    to $f_n$ jest zbieżny według miary do pewnego $f$ oraz istnieje podciąg liczb naturalnych $n(k)$ taki, że $f_{n(k)}$ jest zbieżny prawie wszędzie oraz według miary.
\end{multicols}
\bigskip

\podz{sep2}
\bigskip

\subsection*{ROZDZIAŁ 3}

\begin{multicols}{2}
    Całkę po funkcji prostej $f=\sum a_i\chi_{A_i}$ definiujemy jako
    $$\int_Xfd\mu=\sum a_i\mu(A_i)$$

    Dla nieujemnej mierzalnej funkcji $f$ definiujemy
    $$\color{acc}\int_Xfd\mu=\sup\{\int_Xsd\mu\;:\;0\leq s\leq f\},$$
    czyli jeżeli $s_1\leq s_2\leq ...$ jest ciągiem funkcji prostych takich, że $\lim_ns_n=f$ prawie wszędzie, to
    $$\int_xfd\mu=\lim_x\int_Xs_nd\mu.$$

    Funkcja mierzalna jest {\color{def}całkowalna}, jeżeli $\color{acc}\int_X|f|d\mu<\infty$, wtedy definiujemy całkę wzorem $\int_Xfd\mu=\int_Xf^+d\mu-\int_Xf^-d\mu$ dla $f=f^x-f^-$ nieujemnych.
    \medskip

    Dla {\color{def}$f,g$ całkowalnych i $h$ mierzalnej}:
    \smallskip

    \point $\int_X(af+gb)d\mu=a\int_Xfd\mu+b\int_Xgd\mu$

    \point $h=0$ prawie wszędzie, to $\int_Xhd\mu=0$

    \point $f\leq g$ prawie wszędzie, to $\int_Xfd\mu\leq\int_Xgd\mu$

    \point $|\int_X(f+g)d\mu|\leq\int_X|f|d\mu+\int_X|g|d\mu$
    
    \point $a\leq f\leq b$ prawie wszędzie, to $a\mu(X)\leq\int_Xfd\mu\leq b\mu(X)$

    \point dla $A,B\in\Sigma$ jeżeli $A\cap B=\emptyset$, to $\int_{A\cup B}fd\mu=\int_Afd\mu+\int_Bfd\mu$
    \medskip

    \podz{sep}
    \medskip

    {\color{def}Twierdzenie o zbieżności monotonicznej:} niech $f_n$ będzie ciągiem nieujemnych funkcji mierzalnych takich, że $f_1\leq f_2\leq...$ zbieżnych prawie wszędzie do $f=\lim_nf_n$ 
    $$\int_Xfd\mu=\lim_n\int_Xf_nd\mu$$

    {\color{def}Lemat Fatou}: dla dowolnego ciągu funkcji nieujemnych $f_n$ zachodzi
    $$\int_x\lim\inf f_nd\mu=\lim\inf\int_X f_nd\mu$$

    {\color{def}Twierdzenie Lesbegue'a o zbieżności ograniczonej}: niech $f_n,g$ będą mierzalne, że dla każdego $n$ $|f_n|\leq g$ zachodzi prawie wszędzie przy czym $\int_Xgd\mu<\infty$. Jeżeli {\color{def}$f=\lim_nf_n$ prawie wszędzie}, to
    $$\lim_n\int_X|f_n-f|d\mu=0$$
    $$\lim_n\int_Xfd\mu=\lim\int_Xf_nd\mu$$

    Jeżeli teraz $\mu(X)<\infty$ oraz $f_n$ są wspólnie ograniczone i $\int_X\lim_nf_nd\mu=\lim_n\int_Xf_nd\mu$.
    \medskip

    Jeżeli $f$ jest mierzalna i nieujemna, to funkcja $\nu:\Sigma\to[0,\infty]$
    $$\nu(A)=\int_Afd\mu$$
    jest {\color{def}miarą na $\Sigma$}.
    \medskip

    \podz{sep}
    \medskip

    Jeżeli $f:[a,b]\to\R$ jest całkowalna w sensie Riemanna, to jest $\lambda$-mierzalna i obie całki są sobie równe: $\int_a^bf(x)dx=\int_{[a,b]}fd\lambda$.

\end{multicols}
\bigskip

\podz{sep2}
\bigskip

\subsection*{ROZDZIAŁ 4}

\begin{multicols}{2}

    Niech $(X,\Sigma)$ i $(Y,\Theta)$ będą przestrzeniami z $\Sigma,\Theta$ będącymi $\sigma$-ciałami. W $X\times Y$ możemy zdefiniować następujące $\sigma$-ciało:
    $$\Sigma\otimes\Theta=\sigma(\{A\times B\;:\;A\in\Sigma, B\in\Theta\}),$$
    wtedy $\Sigma\otimes\Theta$ jest {\color{def}produktem $\sigma$-ciał} $\Sigma$ i $\Theta$.
    \medskip

    Zbiór $F\subseteq X\times Y$ należy do ciała prostokątów na $\Sigma,\Theta$ wtw $F=\bigcup A_i\times B_i$ dla $A_i\in\Sigma$ i $B_i\in\Theta$.

    Jeżeli $E\in\Sigma\otimes\Theta$, to dla każdego $x\in X$ i $y\in Y$ definiujemy {\color{acc}cięcia pionowe i poziome} $E_x=\{z\in Y\;:\;(x,z)\in E\}$ i $E^y=\{z\in Z\;:\;(z,y)\in E\}$.
    \medskip

    Jeżeli $E\in\Sigma\otimes\Theta$, a funkcja $f:X\times Y\to\R$ jest $\Sigma\otimes\Theta$-mierzalna, to funkcja $f_x$ jest $\theta$-mierzalna, a $f^y$ jest $\Sigma$-mierzalna.
    \medskip

    $$\bor\otimes\bor=\obor$$

    {\color{def}Miarę} na produkcie przestrzeni $(X,\Sigma,\mu)$ i $(Y,\Theta,\nu)$ definiujemy:
    $$\color{acc}\mu\otimes\nu(A\times B)=\mu(A)\cdot\mu(B)$$

    \podz{sep}
    \medskip

    Na $\sigma$-ciele $\Sigma\otimes\Theta$ istnieje jedyna miara spełniająca dla każdego $A\in\Sigma$ i $B\in\Theta$
    $$\mu\otimes\nu(A\times B)=\mu(A)\cdot\nu(B)$$
    a dla dowolnego $E\in\Sigma\otimes\Theta$ funkcja $x\mapsto\nu(E_x)$ i $y\mapsto\mu(E^y)$ są mierzalne i zachodzą wzory:
    $$\mu\otimes\nu(E)=\int_X\nu(E_x)d\mu(x)=\int_Y\mu(E^y)d\nu(y)$$

    \podz{sep}
    \medskip

    {\color{def}Twierdzenie Fubiniego}: jeżeli funkcja $f:X\times Y\to\R$ jest $\Sigma\otimes\Theta$-mierzalna oraz $f$ jest nieujemna lub $f$ jest $\mu\otimes\nu$-całkowalna, to wtedy
    $$I:x\mapsto\int_Yf(x,y)d\nu(y)$$
    $$J:y\mapsto\int_Yf(x,y)d\mu(x)$$
    są mierzalne względem $\Sigma$ i odpowiednio $\Theta$, oraz
    \begin{align*}
        \int_{X\times Y}fd\mu\otimes\nu&=\int_X\left(\int_Yf(x,y)d\nu(y)\right)d\mu(x)=\\
        &=\int_Y\left(\int_Xf(x,y)d\mu(x)\right)d\nu(y)
    \end{align*}
    Czyli możemy {\color{acc}bezkarnie zamieniać granice całkowania}.
    \medskip

    \podz{sep}
    \medskip

    Jeżeli rozważamy przestrzenie metryczne $(X_i,\Sigma_i,\mu_i)$ dla $i=1,...,n$, to na $\sigma$-ciele $\color{acc}\bigotimes_{i\leq n}\Sigma_i$ podzbiorów $\prod_{i\leq n}X_i$ generowanym przez wszystkie kostki mierzalne $A_1\times...\times A_n$, to {\color{def}istnieje jedyna miara} $\mu=\bigotimes\mu_i$ spełniająca dla wszystkich $A_i\in\Sigma_i$
    $$\mu(A_1\times...\times A_n)=\mu_1(A_1)\cdot...\cdot\mu_n(A_n)$$

    \podz{sep}
    \medskip

    $K=\{0,1\}^\N$ - przestrzeń ciągów $0,1$, wtedy możemy określić metrykę $d(x,y)=\frac1n$ gdzie $n$ to pierwszy indeks gdzie te dwa ciągi się różnią.
    \medskip

    Funkcja $f:K\to[0,1]$ zadana $f(x)=\sum\limits_{n=1}^\infty{2x(n)\over3^n}$ jest homeomorfizmem między $K$ a zbiorem Cantora. Dla funkcji $\psi:\N\supseteq A\to\{0,1\}$ oznaczamy zbiór 
    $$[\psi]=\{x\in K\;:\;x(i)=\psi(i)\;i\in A\}.$$
    Cylindry ten postaci są otwarto-domknięte w $K$ i stanowią bazę topologii w $K$. Oznaczmy przez $\rodz C$ ciało podzbiorów $K$ generowanych przez wszystkie cylindry. Zbiór $C$ jest w $\rodz C$ wtw istnieje $n$ i $C'\subseteq\{0,1\}^n$ takie, że $C=C'\times\{0,1\}\times...$.
    \medskip

    Dla $C\in\rodz C$ definiujemy {\color{def}miarę na $K$}:
    $$\nu(C)={|C'|\over2^n}$$
    która spełnia własność: 
    $$\color{acc}(\forall\;B\in Bor(K))(\forall\;\varepsilon>0)(\exists\;C\in\rodz C)\;\nu(B\Delta C)<\varepsilon.$$
    
\end{multicols}
\bigskip

\podz{sep2}
\bigskip

\subsection*{ROZDZIAŁ 5}

\begin{multicols}{2}
    
    

\end{multicols}

\end{document}