\section{Zbiory}

\subsection{Wstęp o zbiorkach}

\begin{multicols}{2}
    
$$A\Delta B=(A\setminus B)\cup (B\setminus A)$$

Mówimy, że ciąg zbiorów $A_n$ zbiega od dołu do $A$, [$A_n\uparrow A$] jeżeli
$$A=\bigcup\limits_{n=1}^\infty A_n$$
analogicznie zbieganie od góry [$A_n\downarrow A$]:
$$A=\bigcap\limits_{n=1}^\infty A_n.$$

{\color{def}Granice górna i dolna}:
$$\lim\sup_n A_n=\bigcap\limits_{n=1}^\infty\bigcup\limits_{k=n}^\infty A_k$$
$$\lim\inf A_n=\bigcup\limits_{n=1}^\infty\bigcap\limits_{k=n}^\infty A_k$$
i ogółem
$$A=\lim A_n\iff \lim\sup A_n=\lim\inf A_n=A.$$

\podz{sep}
\medskip

Każdy niepusty otwarty $U\subseteq \R$ można zapisać
$$U=\bigcup\limits_{n=1}^\infty (a_n,b_n),$$
gdzie $a_n,b_n\in\Q$.
\smallskip

Dla zbioru domkniętego $F\subseteq\R$ mamy z kolei
$$F\subseteq\bigcup\limits_{n=1}^\infty(a_n,b_n)$$
i istnieje wtedy takie $N$, że
$$F\subseteq\bigcup\limits_{n=1}^N(a_n,b_n).$$

\end{multicols}
\bigskip

\podz{sep}
\bigskip

\begin{multicols}{2}

Rodzina $\rodz{R}\subseteq\Po{X}$ jest {\color{def}pierścieniem} $[r(\rodz{R})]$, jeśli:\smallskip\\
\point $\emptyset\in\rodz{R}$\\
\point $(\forall\;A,B\in\rodz{R})\;A\setminus B\in\rodz{R}\;i\;A\cup B\in\rodz{R}$ (\emph{wnioskiem z tego jest, że $A\cap B\in\rodz{R}$}).
\smallskip

Pierścień, który jest dodatkowo zamknięty na przeliczalne sumy, tzn. 
$$(\forall\;A_n\in\rodz{R})\;\bigcup A_n\in\rodz{R}$$ 
nazywa się {\color{def}$\sigma$-pierścieniem} [$s(\rodz{R})$].
\medskip

Pierścień, który jest domknięty na dopełnienia, jest nazywany {\color{def}ciałem} [$a(\rodz{R})$], natomiast $\sigma$-pierścień domknięty na dopełnienia to {\color{def}$\sigma$-ciało} [$\sigma(\rodz{R})$]. W $\sigma$- mamy też domknięcie na {\color{acc}$\lim\sup$ i $\lim\inf$} ciągów zbiorów.
\medskip

Rodzina $\rodz{R}$ zbiorów $A\subseteq \R$ takich, że
$$A=\bigcup[a_n,b_n)$$
jest pierścieniem. Co więcej, każdy taki $A$ można zaprezentować za pomocą rozłącznych przedziałów.
\bigskip

{\color{def}$\sigma$-ciało zbiorów borelowskich} [$\sigma(\rodz{U})$] to najmniejsze $\sigma$-ciało zawierające rodzinę $\rodz U$ wszystkich podzbiorów otwartych $\R$.

Jeśli $\rodz{F}$ to zbiór przedziałów postaci $[p,q)$, $p,q\in\Q$ to mamy równość
$$\sigma(\rodz{F})=Bor(\R)$$

\end{multicols}
\bigskip

\podz{sep}
\bigskip

\subsection{Funkcje zbiorów}

\begin{multicols}{2}

Jeśli $\rodz{R}$ jest pierścieniem zbiorów i mamy funkcję 
$$\mu:\rodz{R}\to[0,\infty]$$
to $\mu$ jest {\color{def}addytywną funkcją zbiorów}, jeśli\smallskip\\
\point $\mu(\emptyset)=0$\\
\point $\mu(A\cup B)=\mu(A)+\mu(B)$ dla $A\cap B=\emptyset$.

Kilka fajnych {\color{acc}własności}:\smallskip\\
\point $A\subseteq B\implies \mu(A)\leq\mu(B)$\\
\point $A\subseteq B$ i $\mu(A)<\infty\implies \mu(B\setminus A)=\mu(B)=\mu(A)$\\
\point $A_1,..., A_n$ parami rozłączne, to $\mu(\bigcup A_i)=\sum\mu(A_i)$.
\medskip

{\color{def}Przeliczalnie addytywna funkcja zbioru} to $\mu$ jak wyżej takie, że dla dowolnego $A$ i $A_i$ rozłącznych takich, że
$$A=\bigcup A_i$$
zachodzi
$$\mu(A)=\mu(\bigcup A_i)=\sum\mu(A_i)$$

Jeśli $\mu$ jest przeliczalną funkcją zbioru i $A-\bigcup A_i$, to zachodzi
$$\mu(A)=\mu(\bigcup A_i)\leq \sum\mu(A_i)$$

Addytywna funkcja zbioru jest przeliczalna $\iff$ jest {\color{acc}ciągła z dołu} (alternatywnie z góry), tzn:
$$(\forall\;A)(\forall\;A_n)\;A_n\uparrow A\implies \lim\mu(A_n)=\mu(A).$$

Dla addytywnej $\mu$  {\color{acc}następujące warunki są równoważne}:\smallskip\\
\point $\mu$ - przeliczalnie addytywna\\
\point $\mu$ - ciągła z góry/dołu\\
\point $\mu$ - ciągła z góry na zbiór $\emptyset$.

\end{multicols}
\bigskip

\podz{sep}
\bigskip

\subsection{Miara Lebesgue'a I}

\begin{multicols*}{2}

Dla $A=\bigcup\limits_{i=1}^n[a_i, b_i)$, gdzie $[a_i, b_i)$ są rozłączne, definiujemy {\color{def}naturalną funkcję zbioru} $\lambda$:
$$\lambda(A)=\sum\limits_{i=1}^n[a_i,b_i).$$
Od razu warto zaznaczyć, że $[a_i,b_i)$ nie musi być ciągiem skończonym - ciągi nieskończone też śmigają, bo $\lambda$ jest przeliczalną addytywną funkcją zbioru.

\end{multicols*}