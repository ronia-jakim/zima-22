\subsection{Arbitrary subgraphs}

\begin{tabularx}{\textwidth}{ X!{\color{git90gray}\vrule} X }

    {\color{def}Forbidden subgraph problem}: given a graph $H$, how many edges can a $H$-free graph of order $n$ have?
    $$\color{def}ex(n; H)=\max\{e(G)\;:\;\text{G - H-free graph with }|G|=n\}.$$

    From previous theorems we know that:\smallskip

    \point $ex(n; K_{r+1})=t_r(n)$ and $ex(n;K_{r+1})\sim {n^2\over2}(1-\frac1r)$

    \point $ex(n; K_{t,t})=O(n^{2-\frac1t})$
    \medskip

    Let us write 
    $$\color{acc}e(H)=\lim_{n\to\infty}{ex(n;H)\over{n\choose 2}}$$

    and proof that for $n\geq 2$ the sequence
    $$({ex(n;H)\over{n\choose 2}})_{n=2}^\infty$$
    converges.

    &

    {\color{def}Problem zakazanego podgrafu}: mając dany graph $H$, ile krawędzi może mieć $H$-wolny graf rzędu $n$?
    $$\color{def}ex(n; H)=\max\{e(G)\;:\;\text{G - H-wolny graf z }|G|=n\}.$$

    Z poprzednich twierdzeń wiemy, że:\smallskip

    \point $ex(n; K_{r+1})=t_r(n)$ oraz $ex(n;K_{r+1})\sim {n^2\over2}(1-\frac1r)$

    \point $ex(n; K_{t,t})=O(n^{2-\frac1t})$
    \medskip

    Oznaczmy
    $$\color{acc}e(H)=\lim_{n\to\infty}{ex(n;H)\over{n\choose 2}}$$

    i udowodnijmy, że dla $n\geq 4$, ciąg
    $$({ex(n;H)\over{n\choose 2}})_{n=2}^\infty$$
    jest zbieżny.    

\end{tabularx}

\prooflink{forbidden-convergence}
\medskip

\prooflang{forbidden-convergence}{GB}
\medskip

DUNNO
\medskip

\prooflang{forbidden-convergence}{PL}
\medskip

Ciąg $(x_n)_{n=2}^\infty$ jest ograniczony od dołu przez $0$, więc wystarczy pokazać, że nie jest to ciąg rosnący. Niech $n\geq3$ i niech $G$ będzie $H$-wolnym grafem z $|G|=n$ oraz $e(G)=ex(n; H)$. Wtedy dla dowolnego $v\in G$ graf $G-\{v\}$ jest $H$-wolny i ma rząd $n-1$, implikując, że
$$e(G-\{v\})\leq ex(n-1, H).$$

Z drugiej strony, dowolna krawędź $uw\in E(G)$ jest używa t dokładnie $n-2$ grafach $G-\{v\}$ dla $v\in G$ (w tych, gdzie $v\notin\{u,w\}$). To z kolei implikuje, że
$$(n-2)e(G)=\sum\limits_{v\in G}e(G-\{v\})$$
i z tego mamy
$$x_n={ex(n;H)\over {n\choose2}}={2e(G)\over n(n-1)}=\sum\limits_{v\in G}{2e(G-\{v\})\over n(n-1)(n-2)}\leq {2ex(n-1;H)\over (n-1)(n-2)}$$
czyli ciąg nie jest rosnący, a więc musi do czegoś zbiegać.