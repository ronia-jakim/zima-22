\subsection{Arbitrary subgraphs}

\begin{tabularx}{\textwidth}{ X!{\color{git90gray}\vrule} X }

    {\color{def}Forbidden subgraph problem}: given a graph $H$, how many edges can a $H$-free graph of order $n$ have?
    $$\color{def}ex(n; H)=\max\{e(G)\;:\;\text{G - H-free graph with }|G|=n\}.$$

    From previous theorems we know that:\smallskip

    \point $ex(n; K_{r+1})=t_r(n)$ and $ex(n;K_{r+1})\sim {n^2\over2}(1-\frac1r)$

    \point $ex(n; K_{t,t})=O(n^{2-\frac1t})$
    \medskip

    Let us write 
    $$\color{acc}e(H)=\lim_{n\to\infty}{ex(n;H)\over{n\choose 2}}$$

    and proof that for $n\geq 2$ the sequence
    $$({ex(n;H)\over{n\choose 2}})_{n=2}^\infty$$
    converges.

    &

    {\color{def}Problem zakazanego podgrafu}: mając dany graph $H$, ile krawędzi może mieć $H$-wolny graf rzędu $n$?
    $$\color{def}ex(n; H)=\max\{e(G)\;:\;\text{G - H-wolny graf z }|G|=n\}.$$

    Z poprzednich twierdzeń wiemy, że:\smallskip

    \point $ex(n; K_{r+1})=t_r(n)$ oraz $ex(n;K_{r+1})\sim {n^2\over2}(1-\frac1r)$

    \point $ex(n; K_{t,t})=O(n^{2-\frac1t})$
    \medskip

    Oznaczmy
    $$\color{acc}e(H)=\lim_{n\to\infty}{ex(n;H)\over{n\choose 2}}$$

    i udowodnijmy, że dla $n\geq 4$, ciąg
    $$({ex(n;H)\over{n\choose 2}})_{n=2}^\infty$$
    jest zbieżny.    

\end{tabularx}

\prooflink{forbidden-convergence}
\medskip

\prooflang{forbidden-convergence}{GB}
\medskip

DUNNO
\medskip

\prooflang{forbidden-convergence}{PL}
\medskip

Ciąg $(x_n)_{n=2}^\infty$ jest ograniczony od dołu przez $0$, więc wystarczy pokazać, że nie jest to ciąg rosnący. Niech $n\geq3$ i niech $G$ będzie $H$-wolnym grafem z $|G|=n$ oraz $e(G)=ex(n; H)$. Wtedy dla dowolnego $v\in G$ graf $G-\{v\}$ jest $H$-wolny i ma rząd $n-1$, implikując, że
$$e(G-\{v\})\leq ex(n-1, H).$$

Z drugiej strony, dowolna krawędź $uw\in E(G)$ jest używa t dokładnie $n-2$ grafach $G-\{v\}$ dla $v\in G$ (w tych, gdzie $v\notin\{u,w\}$). To z kolei implikuje, że
$$(n-2)e(G)=\sum\limits_{v\in G}e(G-\{v\})$$
i z tego mamy
$$x_n={ex(n;H)\over {n\choose2}}={2e(G)\over n(n-1)}=\sum\limits_{v\in G}{2e(G-\{v\})\over n(n-1)(n-2)}\leq {2ex(n-1;H)\over (n-1)(n-2)}$$
czyli ciąg nie jest rosnący, a więc musi do czegoś zbiegać.
\medskip

\podz{sep}
\medskip

\begin{tabularx}{\textwidth}{ X!{\color{git90gray}\vrule} X }

    {\color{def}Chromatic number} of a graph $H$ [$\chi(H)$] is the smallest integer $r\geq1$ such that $H$ is $r-partite$.
    \medskip

    {\color{def}Erd\"os-Stone Theorem}

    Let $k,r$ be integers with $k-1\geq r\geq1$ and let $\varepsilon>0$. Then there exists an integer $N$ such that for all $n\geq N$, if $G$ is a graph with $|G|=n$ and $e(G)\geq(1-\frac1r+\varepsilon){n\choose 2}$, then $T_{r+1}(k)\leq G$.

    No proof for your stupid arse.
    \medskip

    Let $H$ be a graph with $e(H)\geq1$. Then $ex(H)=1-{1\over \chi(H)-1}$

    &
    
    {\color{def}Liczba chromatyczna} grafu $H$ [$\chi(H)$] to najmniejsza liczba całkowita $r\geq1$ taka, że $H$ jest $r$-dzielny.
    \medskip

    {\color{def}Twierdzenie Erd\"osa-Stone'a}

    Niech $k,R$ będą liczbami całkowitymi z $k-1\geq r\geq1$ i niech $\varepsilon>0$. Wtedy istnieje liczba całkowita $N$ taka, że dla każdego $N\geq N$, jeżeli $G$ jest grafem z $|G|=n$ i $e(G)\geq (1-\frac1r+\varepsilon){n\choose 2}$, wtedy $T_{r+1}(k)\leq G$.

    Nie ma dowodu dla twojej głupiej dupy.
    \medskip

    Niech $H$ będzie grafem z $e(H)\geq1$. Wtedy $ex(H)=1-{1\over\chi(H)-1}$.

\end{tabularx}

\prooflink{chi-before-erdos}
\medskip

\prooflang{chi-before-erdos}{GB}
\medskip

DUNNO
\medskip

\prooflang{chi-before-erdos}{PL}
\medskip

Niech $r\chi(H)-1$, wybierzmy $k$ takie, że $H\leq T_{r+1}(k)$ (na przykład możemy wziąć $k=(r+1)|H|$) i niech $\varepsilon>0$. Oznaczmy przez $N$ liczbę całkowitą z twierdzenia Erd\"osa-Stone'a. Wtedy dla dowolnego $n\geq N$ i dowolnego $H$-wolnego grafu $G$ z $|G|=n$ wiemy, że $G$ jest również $T_r(K)$-wolny i z tego powodu
$$e(G)<(1-\frac1r+\varepsilon){n\choose 2}.$$

Z tego wiemy, że $ex(n;H)<(1-\frac1r+\varepsilon){n\choose 2}$ dla wszystkich $N\geq N$, a więc
$$ex(H)\leq 1-\frac1r+\varepsilon.$$
Ale ponieważ $\varepsilon>0$ był z dowolnie mały, to mamy
$$ex(H)\leq 1-\frac1r.$$

Z drugiej strony dla dowolnego $n\geq r$ graf $T_r(n)$ jest $H$-wolny, bo $H$ nie jest $r$-dzielny i mamy $t_r(n)\sim (1-\frac1r){n\choose 2}$, implikując że $ex(H)\geq 1-\frac1r$.
\medskip

\podz{sep}
\medskip


\begin{tabularx}{\textwidth}{ X!{\color{git90gray}\vrule} X }

    {\color{def}Upper density} [$ud(G)$] of an infinite graph $G$ is defined as:
    $$ud(G)=\lim\sup\limits_{n\to\infty}1\max\Big\{{e(H)\over{n\choose2}}\;:\;H\leq G,\;|H|=n\Big\}$$

    In an infinite graph $G$ we have either $ud(G)=1$ or $ud(G)=1-\frac1r$ for some $r\geq1$.

    &

    {\color{def}Górna gęstość} [$ud(G)$] nieskończonego grafu $G$ jest zdefiniowana jako:
    $$ud(G)=\lim\sup\limits_{n\to\infty}1\max\Big\{{e(H)\over{n\choose2}}\;:\;H\leq G,\;|H|=n\Big\}$$

    Dla nieskończonego grafu $G$ następuje albo $ud(G)=1$ albo $ud(G)=1-\frac1r$ dla pewnego $r\geq1$.

\end{tabularx}