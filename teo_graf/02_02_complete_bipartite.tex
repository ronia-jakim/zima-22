\subsection{Complete bipartite subgraphs}

\begin{tabularx}{\textwidth}{ X!{\color{git90gray}\vrule} X }

    {\color{def}Jensen's Inequality}: $a<b\in\R$ and $f:[a,b]\to\R$ is convex. Then
    $$\frac1n\sum\limits_{i=1}^nf(x_1)\geq f(\frac1n\sum\limits_{i=1}^nx_i)$$
    for all $x_1,...,x_n\in[a,b]$.
    \smallskip

    A particular case of Jensen's Inequality is
    $$b_t(x)=\begin{cases}
        {x\choose t}=\frac1{t!}x(x-1)...(x-t+1)\quad x\geq t-1\\
        0
    \end{cases}$$

    &

    {\color{def}Nierówność Jensena}: $a<b\in\R$ i $f:[a,b]\to\R$ jest wypukła. Wtedy
    $$\frac1n\sum\limits_{i=1}^nf(x_1)\geq f(\frac1n\sum\limits_{i=1}^nx_i)$$
    dla wszystkich $x_1,...,x_n\in[a,b]$.
    \smallskip

    Specjalnym przypadkiem nierówności Jensena jest
    $$b_t(x)=\begin{cases}
        {x\choose t}=\frac1{t!}x(x-1)...(x-t+1)\quad x\geq t-1\\
        0
    \end{cases}$$
    \\

    & \\

    \hline

    & \\

    {\color{def}t-fan} is a graph $H$ such that $H\simeq K_{1,t}$.

    For any $t\geq2$, there exists a function $f=f_t:\N\to(0,\infty)$ with $f(n)=O(n^{2-\frac1t})$, such that if $G$ is a $K_{t,t}$-free graph with $|G|=n$ then $e(G)\leq f(n)$.

    &

    {\color{def}t-wachlarzem} jest graf $H$ taki, ze $H\simeq K_{1,t}$.

    Dla dowolnego $t\geq2$ istnieje funkcja $f=f_t:\N\to(0, \infty)$ z $f(n)=O(n^{2-\frac1t})$, taka, ze jesli $G$ jest $K_{t,t}$-wolnym grafem z $|G|=n$, to $e(G)\leq f(n)$.

\end{tabularx}

\prooflink{jensen-bullshit}
\medskip

\prooflang{jensen-bullshit}{GB}
\medskip

DUNNO
\medskip

\prooflang{jensen-bullshit}{PL}
\medskip

Niech $G$ bedzie $K_{t,t}$-wolnym grafem z $|G|=n\geq1$ i niech $e(G)=m$. Niech $k$ bedzie iloscia t-wachlarzy w $G$.
\medskip

Każdy wierzchołek $v\in G$ jest wierzchołkiem stopnia $t$ dla dokładnie ${d(v)\choose t}=b_t(d(v))$ t-grafów w $G$, implikując, że
$$k=\sum\limits_{v\in G}b_t(d(v))\geq n\cdot b_t(\frac1n\sum\limits_{v\in G}d(v))=n\cdot b_t\Big({2m\over n}\Big),$$
gdzie środkowa nierówność wynika z nierówności Jensena, a następująca po niej równość - z handshaking lemma. Z drugiej strony, ponieważ $G$ jest $K_{t,t}$-wolnym grafem, dowolny zbiór $t$ wierzchołków w $G$ jest wierzchołkiem stopnia $1$ w co najwyżej $(t-1)$ t-wachlarzach w $G$. To implikuje, że
$$k\leq {n\choose t}(t-1)\leq {n^t\over t!}t.$$

Ponieważ $tn=O(n^{2-\frac1t})$, to bez straty ogólności możemy założyć, że $m\geq tn$ i z tego dostajemy
$${2m\over n}\geq {m\over n}+y\geq t.$$
Z nierówności Jensena dostajemy
$$k\geq n{{2m\over n}\choose t}\geq {n\Big({2m\choose n}-t+1\Big)^t\over t!}>{n\over t!}\Big({m\over n}^t\Big)={m^t\over n^{t-1}t!}.$$
To w połączeniu z przykładem 2.4. ze skryptu, którego nie chce mi się przepisywać, daje $m^t\leq n^{2t-1}t$, czyli
$$m\leq \sqrt[t]{t}n^{2-\frac1t}.$$
Czyli funkcja 
$$f_t(n)=\max({tn, \sqrt[t]{t}n^{2-\frac1t}})$$ 
spełnia warunki twierdzenia.
\medskip

\podz{sep}
\medskip

\begin{tabularx}{\textwidth}{ X!{\color{git90gray}\vrule} X }

    Theorem above is similar to the {\color{acc}Zarankiewicz problem}, which, given $n\geq t\geq2$, asks about the smallest number $\color{def}z_t(n)$ such that any $K_{t,t}$-free graph $G$ with $n$ vertices in each class has $e(G)\leq z_t(n)$. We call $z_t(n)$ {\color{acc}Zarankiewicz numbers} and the theorem above implies that
    $$z_t(n)\leq f_t(2n)=O(n^{2-\frac1t})$$

    &
    
    Twierdzenie powyżej jest podobne do {\color{acc}problemu Zarankiewicza}, który, mając dane $n\geq t\geq 2$, pyta o najmniejszą liczbę $\color{def}z_t(n)$ taką, że dowolny $K_{t,t}$-wolny graf $G$ z $n$ wierzchołkami w każdej klasie ma $e(G)\leq z_t(n)$. Liczby $z_t(n)$ nazywamy liczbami {\color{acc}liczbami Zarankiewicza} i twierdzenie wyżej implikuje, że
    $$z_t(n)\leq f_t(2n)=O(n^{2-\frac1t})$$
\end{tabularx}