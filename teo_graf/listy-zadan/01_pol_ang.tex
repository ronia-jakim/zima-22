\documentclass{article}[13pt]

\usepackage{../../uni-notes-eng}
\usepackage{multicol}
\usepackage{graphicx}
\usepackage{tabularx}

\begin{document}
    \begin{tabularx}{\textwidth}{ X X }
        1. Find (by "drawing" pictures representing graphs) all pariwise non-isomorphic graphs of order 4. & 1. Znajdz (rysujac obrazki reprezentujace grafy) wszystkie parami nieizomorficzne grafy stopnia 4.\\

         & \\

        2. For a graph $G$, define a relation $\approx$ on $V(G)$ by saying $v\approx w$ if and only if there exists a path in $G$ with endpoints $v$ and $w$. Show that $\approx$ is an equivalence relation - that is, show that $(\forall\;u\in G)(u\approx ua)$, thath $(\forall\;u,v\in G)(u\approx v\implies v\approx u)$, and that $(\forall\;u,v,w\in G)([u\approx v\;\land\;v\approx w]\implies u\approx w)$. & 2. Dla grafu $G$ definiujemy relacje $\approx$ na $V(G)$ mowiac, ze $v\approx w$ wtedy i tylko wtedy, gdy istnieje sciezka w $G$ z koncami $v$ oraz $w$. Pokaz, ze $\approx$ jest relacja rownowaznosci - ze spelnia ...\\

         & \\

        3. Given a graph $G$, define its \emph{complement} $\overline G$ as a graph with vertices $V(\overline G)=V(G)$, such that given $v,w\in V(G)$ with $v\neq w$, we have $vw\in E(\overline G)$ if and only if $vw\notin E(G)$.

        a. Show thath if $G\simeq \overline G$, then $|G|\equiv0\;or\;1\;(\mod 4)$.

        b. Show thath for any graph $G$, either $G$ or $\overline G$ is connected. & 3. Majac dany graf $G$, definiujemy jego \emph{dopelnienie} $\overline G$ jako graf z wierzcholkami $V(\overline G)=V(G)$, taki, ze dla danych $v,w\in V(G)$, $w\neq v$, mamy $vw\in E(\overline G)$ wtw $vw\notin E(G)$

        a. Pokaz, ze jezeli $G\simeq \overline G$, to $|G|\equiv 0\;lub\;1\;(\mod 4)$.

        b. Pokaz, ze dla dowolnego grafu $G$ albo $G$ albo $\overline G$ jest spojny.\\

         & \\

        4. Show that any graph of order at least 2 has two vertices of the same degree. & 4. Pokaz, ze dowolny graf o stopniu co najmniej 2 ma dwa wierzcholki o tym samym stopniu.\\
        
         & \\

        5. a. Show that every connected graph $G$ contains a vertex $v\in G$ such that $G-\{v\}$ is connected.

        [\emph{Hing: pick $v$ so thath some connected component of $G-\{v\}$ is as big as possible}]

        b. A connected graph with at least one vertex is called a \emph{tree} if it has no cycles. Show that every tree with $\geq2$ vertices has a vertex of degree 1 (such a vertex is called a \emph{leaf})

        c. Deduce that if $T$ is a tree then $e(T)=|T|-1$

        d. Let $G$ be a graph with $|G|=n$. We say that a tuple $(d_G(v_1),...,d_G(v_n))$, where $\{v_1,..., v_n\}=V(G)$, is a \emph{degree sequence} of $G$. Show that a given tuple $(d_1,...,d_n)$ of integers, where $n\geq2$, is a degree sequence of a tree iff $d_i\geq1$ for all $i$ and $\sum\limits_{i=1}^nd_i=2n-2$. & 5. a. Pokaz, ze kazdy spojny graf $G$ posiada wierzcholek $v\in G$ taki, ze $G-\{v\}$ jest spojny.

        [\emph{cenzura <3}]

        b. Spojny graf z co najmniej jednym wierzcholkiem jest nazywany \emph{\color{def}drzewem} jezeli nie ma cykli. Pokaz, ze kazde {\color{def}drzewo} z $\geq2$ wierzcholkami ma wierzcholek stopnia 1 (taki wierzcholek nazywa sie \emph{lisciem}).

        c. Wydedukuj, ze jezeli $T$ jest drzewem, wtedy $e(T)=|T|-1$

        d. Niech $G$ bedzie grafem z $|G|=n$. Mowimy, ze krotka $(d_G(v_1),...,d_G(v_n))$, gdzie $\{v_1,..., v_n\}=V(G)$, jest \emph{ze jest to sekwencja wierzcholkow ze sie ich stopnie nie zwiekszaja, sry nie znam nazwy i nie chce mi sie szukac wiecej niz na wikipedii, buzi} grafu $G$. Pokaz, ze majac krotke $(d_1,...,d_n)$ liczb calkowitych, gdzie $n\geq2$, jest \emph{ta wlasnie seria} w {\color{def}drzewie} wtw $d_i\geq1$ dla wszystkich $i$ oraz $\sum\limits_{i=1}^nd_i=2n-2$.\\

         & \\

        6. Let $G=(V,E)$ be a graph. Show that there exists a partition $V=A\sqcup B$ such that all vertices of $G[A]$ and of $G[B]$ have even degree. & 6. Niech $G=(V,E)$ bedzie grafem. Pokaz, ze istnieje podzialy $V=A\sqcup B$ takie, ze wszystkie wierzcholki $G[A]$ i $G[B]$ maja parzyste stopnie\\

         & \\

        7. Suppose $G$ is a graph that has no induced cycles of odd legth - thath is, for any $A\subseteq V(G)$, the graph $G[A]$ is not a cycle of odd length. Show that $G$ is bipartite. & 7. Zaloz, ze $G$ jest grafem nie majacych \emph{induced cycle (takze chordless cycle) to taki cykl, ze nie ma takich brudaskow ktore lacza wierzcholki w cyklu, ale do cyklu nie naleza} nieparzystej dlugosci, tzn dla dowolnego $A\subseteq V(G)$, graf $G[A]$ nie ma cykli o nieparzystej dlugosci. Pokaz, ze $G$ jest dwudzielne.
    \end{tabularx}

    \begin{tabularx}{\textwidth}{ X X }
        8. Let $G$ be a regular bipartite graph with vertex classes $W$ and $M$. Show that $G$ contains a matching from $W$ to $M$. & 8. Niech $G$ bedzie regularnym dwudzielnym grafem z klasami wierzcholkow $W$ i $M$. Pokaz, ze $G$ zawiera laczenie z $W$ do $M$.\\

         & \\
        
        9. Let $n\geq m\geq1$. An $m\times n$ \emph{Latin rectangle} is an $m\times n$ matrix with entries in $[n]$ such that each $i\in[n]$ appears exactly once in each row and at most once in each column. Show that any $m\times n$ Latin rectangle forms the first $m$ rows of any $n\times n$ Latin rectangle. & 9. Niech $n\geq m\geq1$. $m\times n$ \emph{uposledzony trojkat magiczny} to macierz $m\times n$ z elementami w $[n]$ takimi, ze kazde $i\in [n]$ wystepuje dokladnie jeden raz w kazdym wierszu i dokladnie raz w kazdej kolumnie. Pokaz, ze dowolny $m\times n$ prostokacik tworzy pierwsze $m$ wierszy prostokacika $n\times n$.\\

        & \\

        10. Let $G$ be an infinite bipartite graph with (infinite) vertex classes $W$ and $M$, and suppose that $|N_G(A)|\geq|A|$ for every $A\subseteq W$.

        a. Show, by constructing an example, that such a graph $G$ does not need to contain a matching from $W$ to $M$

        b. Suppose that $W$ is countable and $d_G(w)<\infty$ for all $w\in W$. Show that in this case $G$ does contain a matching from $W$ to $M$. & 10. Niech $G$ bedzie nieskonczonym dwudzielnym grafem z (nieskonczonymi) klasami wierzcholkow $W$ i $M$ i zaloz, ze $|N_G(A)|\geq|A|$ dla kazdego $A\subseteq W$.

        a. Pokaz, poprzez konstrukcje przykladu, ze taki graf $G$ nie koniecznie musi zawierac polaczenia z $W$ do $M$

        b. Zaloz, ze $W$ jest przeliczalne i $d_G(w)<\infty$ dla kazdego $w\in W$. Pokaz, ze w takim wypadku $G$ nie zawiera polaczenia z $W$ do $M$.\\

        & \\

        11. For each $k\geq 2$, give an example of a $k$-edge-connected graph that is not $2$-connected. Is there a $k$-connected graph that is not $2$-edge-connected? & 11. Dla kazdego $k\geq 2$ podaj przyklad \emph{grafu, ktory jest spojny i jak wywalisz mniej niz $k$ krawedzi to nadal jest spojny}, ktory nie jest $2$-spojny. Czy jest $k$-spojny graf, ktory nie jest jednoczesnie $2$-to-z-krawedziami?\\

        & \\

        12. Let $G$ be a $k$-connected graph for some $k\geq2$.

        a. Show that for every $x\in G$ and every $U\subseteq V(G)\setminus\{x\}$ with $|U|\geq k$, there exists a collection of $(\{x\}, U)$-paths $P^{(1)},...,P^{(k)}$, where $P^{(i)}=xy_{i,1}...y_{i,m_i}$, such that $y_{i,j}\neq y_{i',j'}$ for $(i,j)\neq (i',j')$ and such thath $y_{i,j}\in U$ iff $j=m_i$.

        b. Show that if $|G|\geq2k$ then $G$ contains a cycle of length $\geq2k$.

        c. Show that every collection of $k$ vertices in $G$ is contained in a cycle. & 12. Niech $G$ bedzie $k$-spojnym grafem dla pewnego $k\geq2$.

        a. Pokaz, ze dla kazdego $x\in G$ i dla kazdego $U\subseteq V(G)\setminus\{x\}$, gdzie $|U|\geq k$, istnieje kolekcja $(\{x\}, U)$-sciezek $P^{(1)},...,P^{(k)}$, gdzie $P^{(i)}=xy_{i,1}...y_{i,m_i}$, taka, ze $y_{i,j}\neq y_{i',j'}$ for $(i,j)\neq (i',j')$ i $y_{i,j}\in U$ iff $j=m_i$.

        b. Pokaz, ze jezeli $|G|\geq2k$, to $G$ zawiera cyk ldlugosci $\geq2k$

        c. Pokaz, ze kazda kolekcja $k$ wierzcholkow w $G$ jest zawarta w cyklu.\\

        & \\

        13. Let $G$ be a $k$-edge-connected graph, and let $F\subseteq E(G)$ with $|F|=k$. Show that $G-F$ has at most two connected components. & 13. Niech $G$ bedzie $k$-to-z-usuwaniem-krawedzi-zad11 i niech $F\subseteq E(G)$, $|F|=k$. Pokaz, ze $G-F$ ma co najwyzej dwa spojne podgrafy (czesci?).\\

        & \\

        14. Let $G$ be an $r$-regular graph for some $r\geq1$, and let $H=L_G$ be the line graph of $G$ (appearing in the proof of the edge version of Menger's Theorem). 

        a. Show that $H$ is regular.

        b. Show thath $L_H\simeq G$ iff $r=2$. & 14. Niech $G$ bedzie $r$-regularnym grafem dla pewnego $r\geq1$ i niech $H=L_G$ bedzie grafem krawedziowym grafu $G$ (pojawiajacy sie w dowodzie krawedziowej wersji twierdzenie Megera).

    \end{tabularx}

    \kdowod
\end{document}