\documentclass{article}[13pt]

\usepackage{../../uni-notes-eng}
\usepackage{multicol}
\usepackage{graphicx}
\usepackage{tabularx}

\begin{document}

\subsection*{ZAD. 1.}

\emph{Find (by "drawing" pictures representing graphs) all pairwise non-simorfphic graphs of order 4}
\medskip

Trivial

\subsection*{ZAD. 2.}

\emph{For a graph $G$, define a relation $\approx$ on $V(G)$ by saying $v\approx w$ iff there exists a path in $G$ with endpoints $v$ and $w$. Show that $\approx$ is an equivalence relation.}
\medskip

{\color{acc}Reflexivity:} $v\approx v$ 
\smallskip

Any path of length $0$ begins and ends in one vertex without any in the middle.
\smallskip

{\color{acc}Symmetry:} $v\approx w\implies w\approx v$
\smallskip

Obviously if we have an undirected graph and path $v...w$, then we also have path going through the same vertices but in reversed order, that is $w...v$.
\smallskip

{\color{acc}Transitivity:} $(v\approx w\;\land\;w\approx z)\implies v\approx z$
\smallskip

We know that there exists a path $v...w$ and another path $w...z$. The end of the first one is the same as the beggining of the second, so we can connect those two paths. That gives us a path $v...ww...z=v...w...z$.

\subsection*{ZAD. 3.}

\emph{Given a graph $G$ define its complement $\overline{G}$ as a graph with vertices $V(\overline{G})=V(G)$ such that fiven $v,w\in V(G)$ with $v\neq w$ we have $vw\in E(\overline{G})$ iff $vw\notin E(G)$
}

(a) Show that if $G\simeq\overline{G}$, then $|G|\equiv 0\;or\;1\mod{4}$

(b) Show that for any graph $G$, either $G$ or $\overline{G}$ is connected.
\medskip

{\color{acc}(a)} Let us assume that $G$ has $n$ vertices. If we want $G$ to be isomorphic with $\overline G$, then they have to have the same number of edges. Between $n$ vertices we can have no more than
$${n\choose 2}={n!\over (n-2)!2!}={n(n-1)\over2}$$
We want to be able to divide this number into two equal parts, so $n(n-1)$ must be divisible by $2\cdot2=4$, so either $n$ is divisible by $4$ (then $n\equiv0\mod4$) or $(n-1)$ is divisible by $4$ ($n\equiv1\mod4$).
\medskip

{\color{acc}(b)} If $G$ is connected then śmiga. Otherwise, $G$ is not connected. Then there are at least two $U, W\subseteq V(G)$ such that for any $u\in U$ and $w\in W$ there is no $uw\in G$. Then we have $uw\in \overline G$ for every such $u, w$. So we can move between any two vertices just by jumping between $U$ and $W$.

\subsection*{ZAD. 4.}

\emph{Show that any graph of order at least $2$ has two vertices of the same degree.}
\medskip

Let us take a graph of order $n$. Any of its vertices can have degree between $0$ and $(n-1)$. 
\smallskip

If at least one of those vertices has order $0$, then no vertex can have order $(n-1)$, so we are left with $(n-2)$ possible degrees and $(n-1)$ vertices. By the pigeonhole principle, at least two vertices have the same degree.
\smallskip

If no vertex has order $0$, then we have $(n-1)$ possible degrees and $n$ vertices. Again, by the pigeonhole principle, at least two vertices have the same degree.

\subsection*{ZAD. 5.}

\emph{{\color{def}(a)} Show that every connected graph $G$ contains a vertex $v\in G$ such that $G-\{v\}$ is connected.}

\textbf{Hint:} pick $v$ so that some connected component of $G-\{v\}$ is as big as possible.
\medskip

We have a graph of order $n$ and we want to take away one of its edges. Let $v$ be a vertex such that $G-\{v\}$ contains a path as long as possible. Because $G$ was a connceted graph, then this path spanned across all $n$ of its vertices, so in $G-\{v\}$ the longest path would be $(n-1)$ vertices long. Which means, that the graph we obtained is connected.
\medskip

\emph{{\color{def}(b)} A connected graph with at least one vertex is called a \text{tree} if it has no cycles. Show that every tree with $\geq2$ vertices has a vertex of degree 1 (such a vertex is called a \text{leaf}).}
\medskip

Let us suppose that there is a tree with no vertices of degree $1$. We will label it as $G$. Now, let $P$ be the longest path in $G$. If we takie it out of $G$, we have a straight line with some vertex $u$ as its beggining and another vertex $v$ as its end. Now they are of degree $1$. But if we put them back inside $G$, they cannot have degree $1$, so they must be connected to some other vertex. But this is the longest path, so no new vertex can be added, therefore $u$ must be connected with $v$, which gives us a cycle and a contradiction.
\medskip

\emph{{\color{def}(c)} Deduce that if $T$ is a tree then $e(T)=|T|-1$}
\medskip

A quick throwback to the definition of $e(G)$ for my dumb self:
$$e(G)=\frac12\sum\limits_{v\in G}d(v).$$

Now we gonna do induction on the number of vertices. For $n=2$ we have a tree with $|T|=2$ and $e(T)=\frac12(1+1)=1=|T|-1$.
\smallskip

Let us suppose that for a tree with $n$ vertices this formula is true. We add one new vertex. Because in a tree we have at leas one (actually two but nevermind) vertex of degree one, by removing it we change $e(T)$ by only $1$ - one less degree from the removed vertex and one less in degree of the only vertex it was connected to. At the same time, we change $|T|$ by one, so we have:
$$e(T)=e(T')+1=(|T'|-1)+1=(|T|-2)+1=|T|-1.$$
The end <3

\emph{{\color{def}(d)} Let $G$ be a graph with $|G|=n$. We say that a tuple $(d_G(v_1),...,d_G(v_n))$, where $\{v_1,...,v_n\}=V(G)$ is a degree sequence of $G$. Show that a given tuple $(d_1,...,d_n)$ of integers, where $n\geq2$, is a degree sequence of a tree $\iff$ $d_i\geq1$ for all $i$ and $\sum\limits_{i=1}^nd_i=2n-2$.}

$\color{acc}\implies$

It is simply from the previous excercise. For a tree with $n$ degrees we have
$$\frac12\sum d_G(v_i)=n-1$$
$$\sum d_G(v_i)=2n-2.$$

$\color{acc}\impliedby$

Suppose by contradition, that there exists a graph $G$ with $n$ vertices such that $\sum d_i=2n-2$ which is not a tree.

First, let us assume that such a graph does not have any vertices of odd degree, as a tree must have at least one which has degree $1$. Then we must have at most $(2n-2)/2=n-1$ vertices of any degree. But we have $n$ vertices so one would be of degree $0$ and this is a contradiction.

Now, let us assume that at least one vertex has odd degree. But because the sum of all degrees is even, then there must be an even amount of vertices with odd degrees. If the graph is not connected, then we would have two classes of vertex $W, V$, with no edges between them and each would have to contain at least $2$ vertices.

\end{document}