\documentclass{article}[13pt]

\usepackage{../../uni-notes-eng}
\usepackage{multicol}
\usepackage{graphicx}
\usepackage{tabularx}

\begin{document}

\subsection*{ZAD. 1.}

\emph{Find (by "drawing" pictures representing graphs) all pairwise non-simorfphic graphs of order 4}
\medskip

Trivial

\subsection*{ZAD. 2.}

\emph{For a graph $G$, define a relation $\approx$ on $V(G)$ by saying $v\approx w$ iff there exists a path in $G$ with endpoints $v$ and $w$. Show that $\approx$ is an equivalence relation.}
\medskip

{\color{acc}Reflexivity:} $v\approx v$ 
\smallskip

Any path of length $0$ begins and ends in one vertex without any in the middle.
\smallskip

{\color{acc}Symmetry:} $v\approx w\implies w\approx v$
\smallskip

Obviously if we have an undirected graph and path $v...w$, then we also have path going through the same vertices but in reversed order, that is $w...v$.
\smallskip

{\color{acc}Transitivity:} $(v\approx w\;\land\;w\approx z)\implies v\approx z$
\smallskip

We know that there exists a path $v...w$ and another path $w...z$. The end of the first one is the same as the beggining of the second, so we can connect those two paths. That gives us a path $v...ww...z=v...w...z$.

\subsection*{ZAD. 3.}

\emph{Given a graph $G$ define its complement $\overline{G}$ as a graph with vertices $V(\overline{G})=V(G)$ such that fiven $v,w\in V(G)$ with $v\neq w$ we have $vw\in E(\overline{G})$ iff $vw\notin E(G)$
}

(a) Show that if $G\simeq\overline{G}$, then $|G|\equiv 0\;or\;1\mod{4}$

(b) Show that for any graph $G$, either $G$ or $\overline{G}$ is connected.
\medskip

{\color{acc}(a)} Let us assume that $G$ has $n$ vertices. If we want $G$ to be isomorphic with $\overline G$, then they have to have the same number of edges. Between $n$ vertices we can have no more than
$${n\choose 2}={n!\over (n-2)!2!}={n(n-1)\over2}$$
We want to be able to divide this number into two equal parts, so $n(n-1)$ must be divisible by $2\cdot2=4$, so either $n$ is divisible by $4$ (then $n\equiv0\mod4$) or $(n-1)$ is divisible by $4$ ($n\equiv1\mod4$).
\medskip

{\color{acc}(b)} If $G$ is connected then śmiga. Otherwise, $G$ is not connected. Then there are at least two $U, W\subseteq V(G)$ such that for any $u\in U$ and $w\in W$ there is no $uw\in G$. Then we have $uw\in \overline G$ for every such $u, w$. So we can move between any two vertices just by jumping between $U$ and $W$.

\subsection*{ZAD. 4.}

\emph{Show that any graph of order at least $2$ has two vertices of the same degree.}
\medskip

Let us take a graph of order $n$. Any of its vertices can have degree between $0$ and $(n-1)$. 
\smallskip

If at least one of those vertices has order $0$, then no vertex can have order $(n-1)$, so we are left with $(n-2)$ possible degrees and $(n-1)$ vertices. By the pigeonhole principle, at least two vertices have the same degree.
\smallskip

If no vertex has order $0$, then we have $(n-1)$ possible degrees and $n$ vertices. Again, by the pigeonhole principle, at least two vertices have the same degree.

\subsection*{ZAD. 5.}

\emph{{\color{def}(a)} Show that every connected graph $G$ contains a vertex $v\in G$ such that $G-\{v\}$ is connected.}

\textbf{Hint:} pick $v$ so that some connected component of $G-\{v\}$ is as big as possible.
\medskip

We have a graph of order $n$ and we want to take away one of its edges. Let $v$ be a vertex such that $G-\{v\}$ contains a path as long as possible. Because $G$ was a connceted graph, then this path spanned across all $n$ of its vertices, so in $G-\{v\}$ the longest path would be $(n-1)$ vertices long. Which means, that the graph we obtained is connected.
\medskip

\emph{{\color{def}(b)} A connected graph with at least one vertex is called a \text{tree} if it has no cycles. Show that every tree with $\geq2$ vertices has a vertex of degree 1 (such a vertex is called a \text{leaf}).}
\medskip

Let us suppose that there is a tree with no vertices of degree $1$. We will label it as $G$. Now, let $P$ be the longest path in $G$. If we takie it out of $G$, we have a straight line with some vertex $u$ as its beggining and another vertex $v$ as its end. Now they are of degree $1$. But if we put them back inside $G$, they cannot have degree $1$, so they must be connected to some other vertex. But this is the longest path, so no new vertex can be added, therefore $u$ must be connected with $v$, which gives us a cycle and a contradiction.
\medskip

\emph{{\color{def}(c)} Deduce that if $T$ is a tree then $e(T)=|T|-1$}
\medskip

A quick throwback to the definition of $e(G)$ for my dumb self:
$$e(G)=\frac12\sum\limits_{v\in G}d(v).$$

Now we gonna do induction on the number of vertices. For $n=2$ we have a tree with $|T|=2$ and $e(T)=\frac12(1+1)=1=|T|-1$.
\smallskip

Let us suppose that for a tree with $n$ vertices this formula is true. We add one new vertex. Because in a tree we have at leas one (actually two but nevermind) vertex of degree one, by removing it we change $e(T)$ by only $1$ - one less degree from the removed vertex and one less in degree of the only vertex it was connected to. At the same time, we change $|T|$ by one, so we have:
$$e(T)=e(T')+1=(|T'|-1)+1=(|T|-2)+1=|T|-1.$$
The end <3

\emph{{\color{def}(d)} Let $G$ be a graph with $|G|=n$. We say that a tuple $(d_G(v_1),...,d_G(v_n))$, where $\{v_1,...,v_n\}=V(G)$ is a degree sequence of $G$. Show that a given tuple $(d_1,...,d_n)$ of integers, where $n\geq2$, is a degree sequence of a tree $\iff$ $d_i\geq1$ for all $i$ and $\sum\limits_{i=1}^nd_i=2n-2$.}

$\color{acc}\implies$

It is simply from the previous excercise. For a tree with $n$ degrees we have
$$\frac12\sum d_G(v_i)=n-1$$
$$\sum d_G(v_i)=2n-2.$$

$\color{acc}\impliedby$

Induction on $n$. For $n=2$ we have that $\sum d_i=2$, so the edges are connected by one edge and this is a tree. 

Now, let us take any graph with $n+1$ vertices that has $\sum d_i=2n$. Then we cannot have all vertices of even degree because at least one would have degree 0. We have $n$ edges and so at least one vertex has degree $1$. We can cut it out to obtain a graph of $n$ vertices and $\sum d_i=2n-2$, which is a tree. Now, if we add a vertex of degree $1$ we still will have a tree which concludes my very insightful investigation.

\subsection*{ZAD. 6.}

\emph{Let $G=(V, E)$ be a graph. Show that there exists a partition $V=A\cup B$ such that all vertices of $G[A]$ and of $G[B]$ have even degree.}

\textbf{Hint:} consider what happens when we remove a vertex $x$ of odd degree and "invert" adjacency between the neighbors of $v$.
\medskip

Ok, so we doing induction once again. So for $n=1$ it is quite obvious. 

So let us assume that for all graphs that have $n$ vertices this works. Now, let us take a graph $G$ with $n+1$ vertices. If there are no vertices of odd degree then we all good. So let us assume that there exists at least one vertex of odd degree, we gonna name him $Krzys$. To simplify the upcoming struggles, we gonna say that $Krzys$' signature is $x$. And that he is a priest. Now, let us say that $Krzys$ is a pedophile and that he is currently serving his sentence in jail. However, we cannot let the sheep wander without a shepherd, so we want to rewire his believers so that they are connected with each other via a priest. That is, we make new graph $G'$ where $V(G')=V(G')\setminus x$ and for all $a,b\in N_G(x)$ if $ab\in G$ we remove this connection and if $ab\notin G$ we add this connection. This graph has $n$ vertices, so we can divide it into two vertex classes $A$ and $B$ such that all vertices in them have even degrees. Now, let 11 years pass and $Krzys$ is out of the prison. We need to let him back to the church. Therefore, we must choose if he joins the vertex class $A$ or $B$. We still want the degrees to be even and since $Krzys$ had odd degree, he must have an even degree of neighbors in one of $A$ or $B$. Therefore, we need to add him to that group. We need to severe the fragile, artificial connections that were made and revive those that were lost but without connecting $A$ and $B$. 

Such operation does not change the parity of degrees. Let us assume that in $A$ we have even number of neighbors of $Krzys$, let us name this number $k$. By adding $x$ without changing anything else we make them be of odd degree. We had $k$ neighbors of $x$ in $A$ and this is even. One vertex could have a total of $k-1$ edges with its neighbors, which would be an odd number so $k-1$ minus an even number would give us an odd number plus one from $Krzys$ would be even.

In $B$ let us say we had $m$ neighbors of $Krzys$, which is an odd number. Each could have been connected with a maximum of $m-1$ neighbors, which is an even number. So now after reversing it we would have $m-1$ minus an even number, yield once again an even number.

This way, we divided $G$ into two classes of vertices, both containing only vertices of even degree.

\subsection*{ZAD. 7.}

\emph{Suppose $G$ is a graph that has no induced cycles of odd length - that is, for any $A\subseteq V(G)$, the graph $G[A]$ is not a cycle of odd length. Show that $G$ is bipartite.}
\medskip

Suppose that $G$ is a graph such that it does not contain induced cycles of odd length. Without loss of generality, take $G$ that is connected, otherwise we would have several connected components each being bipartite. Let us choose one vertex, $v_0\in G$ and divide all the other vertices into two sets:
\begin{align*}
    A&=\{v\;:\;\text{the shortest }v...v_0\in G\text{ is of odd length}\}\\
    B&=\{v\;:\;\text{the shortest }v...v_0\in G\text{ is of even length}\}
\end{align*}
We will show that $A$ and $B$ are vertex classes of graph $G$.

Suppose that there exists an edge $wu\in A$. Then we have a cycle
$$w...v...uw\in G$$
consisting of a path $w...v$ of length $2k$ for some $k$ and $v...u$ of length $2n$. Additionally, we have the last path $wu$ of length $1$, which gives us a cycle of length $2(k+n)+1$, which certainly cannot be classified as even.

Now, if there exists an edge $wu\in B$, then we have
$$w...v...uw\in G$$
with $w...v$ having length $2k+1$ and $v...u$ of length $2n+1$, adding up to a cycle of length $2(k+n+1)+1$. 

Therefore, $G$ does not contain $wv$ for $w,v\in A$ or $w,v\in B$ and is bipartite.

\subsection*{ZAD. 8.}

\emph{Let $G$ be a regular bipartite graph with vertex classes $W$ and $M$. Show that $G$ contains a matching from $W$ to $M$.}
\medskip

A graph $G$ is regular if there exists an $r$ such that 
$$(\forall\;v\in G)\;d(v)=r.$$
A graph $G$ contains a matching if for any $A\subseteq V$ we have $|N(A)|\geq |A|$.
\medskip

Let us take a $r$-regular bipartite graph $G$ with vertex classes $W$ and $M$. Without the loss of generality, let us take any $A\subseteq W$. We know, that every $a\in A$ must have $r$ neighbors, each in $M$. We must have $r|A|$ edges leaving $A$ and we know that $|N(A)|\geq r$. If $|A|\leq r$, then we know that $|N(G)|\geq r\geq |A|$ and the Hall's condition is satisfied. Otherwise, we would have $|A|>r$ and if $n=|A|>|N(A)|=m$, then we $nr$ edges going to $m<n$ vertices, implying that some vertex from $N(A)$ has degree larger than $r$.

\end{document}