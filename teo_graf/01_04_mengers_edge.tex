\subsection{Menger's Theorem (so edgy)}

\begin{tabularx}{\textwidth}{ X!{\color{git90gray}\vrule} X}

    Graph $G$ is {\color{def}$k$-edge-connected} for $k\geq 0$ if for every $F\subseteq E(G)$, $|F|<k$, $G-F$ is connected.
    \medskip

    {\color{def}Line graph} of graph $G$ $[L_G]$ is a graph with $V(L_G)=E(G)$ and for $e,f\in L_G$ with $e\neq f$ we have 
    $$e\sim f\;in\;L_G\iff e,f\;common\;endpoint\;in\;G$$
    \bigskip

    {\color{def}Menger's Theorem \color{acc}edge version}

    Let $G$ be a graph and let $k\geq0$.

    Then $G$ is $k$-edge-connected iff for every $a,b\in G$ with $a\neq b$, there exists a collection of $k$ edge-disjoint $(a,b)$-paths in $G$.

    &

    Graf $G$ jest {\color{def}$k$-spojny krawedziowo} dla $k\geq0$ jesli dla kazdego $F\subseteq E(G)$, $|F|<k$, $G-F$ jest spojny.
    \medskip

    {\color{def}Graf krawedziowy} grafu $G$ $[L_G]$ jest grafem z $V(L_G)=E(G)$ i dla $e,f\in L_G$ z $e\neq f$ mamy
    $$e\sim f\;w\;L_G\iff e,f\;wspolny\;koniec\;w\;G$$
    \bigskip

    {\color{def}Twierdzenie Megera \color{acc}wersja krawedzie}

    Niech $G$ bedzie grafem i niech $k\geq0$.

    Wtedy $G$ jest $k$-spojny krawedziowo z $a\neq b$, wtedy istnieje zbior $k$ rozlacznych krawedziami $(a,b)$-krawedzi w $G$.

\end{tabularx}

\prooflink{mengers-edges}
\medskip

\prooflang{mengers-edges}{GB}
\medskip

{\color{cyan}SOMEBONY ONCE TOLD ME THE WORLD IS GONNA ROLL ME}
\bigskip

\prooflang{mengers-edges}{PL}
\medskip

$\color{acc}\implies$

Niech $L_G$ bedzie {\color{acc}grafem liniowym} grafu $G$. Wezmy $a,b\in G$ takie, ze $a\neq b$. Niech 
$$A=\{av\in E(G)\;:\;v\in N_G(a)\}$$
i niech
$$B=\{bv\in E(G)\;:\;v\in N_G(b)\}.$$
Oznaczmy przez $C$ $(A, B)$-ciecie w $L_G$, wiec 
$$C\subseteq E(G).$$
Wtedy nie istnieje $(a,b)$-sciezka w $G-C$, co implikuje, ze $|C|\geq k$. W takim razie, na mocy lematu z poprzedniego podrozdzialu, istnieje $k$ rozlaczna wzgledem wierzcholkow $(A,B)$-sciezka w $L_G$ i z tego powodu jest $k$ rozlaczna wzgledem krawedzi $(a,b)$-sciezka w $G$.
\smallskip

\emph{
    Mozemy wyprowadzic te implikacje z twierdzenia "max-flow min-cut" przez zamienianie kazdej krawedzi $vw$ przez pare skierowanych krawedzi $v\to w$ i $w\to v$. Ale my nie znamy tego twierdzenia, wiec nie chce mi sie pisac dalej :v
}
\medskip

$\color{acc}\impliedby$

Niech $f\subseteq E(G)$ i zalozmy, ze $G-F$ jest niespojny. Wybierzmy $a,b\in G-F$ nalezacy do roznych skladowych spojnosci $G-F$. Zgodnie z zalozeniem, $G$ zawiera $k$ rozlaczne wzgledem krawedzi $(a,b)$-sciezki i kazda z tych sciezek musi miec krawedzie w $F$. Z tego tez powodu $|F|\geq k$ tak jak chcielismy.