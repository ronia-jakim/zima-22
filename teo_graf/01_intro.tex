\section{Structural properties}

\subsection{Basic definitions}

\pdef

    {\color{def}Graph} - an ordered pair $G=(V,E)$:\smallskip\\
    \point {\color{acc}vertices} $:=V$ [singular: \emph{vertex}]\\
    \point {\color{acc}edges} $:=E$, $\{v,w\}:=vw$

\kdef 

\begin{tabularx}{\textwidth}{ X!{\color{git90gray}\vrule} X }

    For an edge $vw$, $v\neq w$ we say that $v,w$ are its {\color{acc}endpoints} and that it is {\color{acc}incident} to $v$ (or $w$). 
    \medskip

    &
    Dla krawedzi $vw$, $v\neq w$ mowimy, ze $v,w$ sa jej {\color{acc}koncami} i ze jest krawedzia {\color{acc}padajaca} na $v$ (lub $w$).
    \medskip

    \\

    \hline

    & \\

    Graphs $G$ and $H$ are {\color{def}isomorfic} ($G\simeq H$) if there exists $f:V(G)\xrightarrow[1-1]{on} V(H)$ such that
    
    $(\forall\;v,w\in V(G))\;vw\in E(G)\iff f(v)f(w)\in E(H)$
    \smallskip

    \emph{\color{dyg}Meaning that edges are like an operation on a group of vertices}
    \medskip

    $G$ is a {\color{def}subgraph} of $H$ [$G\leq H$] if $V(G)\subseteq V(H)$ and $E(G)\subseteq E(H)$.
    \medskip

    If $G$ is {\color{def}$H$-free} if it is has no subgraphs isomorfphic to $H$.
    \medskip

    &
    Grafy $G$ i $G$ sa {\color{def}izomorficzne}, jezeli istnieje $f:V(G)\rarrow[1-1]{na}V(H)$ takie, ze

    $(\forall\;v,w\in V(G))\;vw\in E(G)\iff f(v)f(w)\in E(H)$
    \medskip

    $G$ jest {\color{def}podgrafem} $H$ [$G\leq H$] jezeli $V(G)\subseteq V(H)$ oraz $E(G)\subseteq E(H)$.
    \medskip

    $G$ jest {\color{acc}H-free} (wolny od $H$?), jezeli nie ma podgrafow izomorficznych z $H$.
    \medskip

    \\

    \hline

    & \\

    A {\color{def}cycle} of length $n\geq3$ [$C_n$] is a graph with vertices
    $$V(C_n)=[n]$$ 
    and edges:
    $$E(C_n)=\{i(i+1)\;:\;i\leq i\leq n-1\}\cup\{1n\}.$$

    A {\color{def}path} of length $n-1$ [$P_{n-1}$] is a graph with vertices
    $$V(P_{n-1})=[n]$$
    and edges
    $$E(P_{n-1})=\{i(i+1)\;:\;1\leq i\leq n-1\}.$$

    &
    {\color{def}Cykl} dlugosci $n\geq3$ [$C_n$] to graf z wierzcholkami
    $$V(C_n)=[n]$$
    i krawiedziami:
    $$E(C_n)=\{i(i+1)\;:\;i\leq i\leq n-1\}\cup\{1n\}.$$

    {\color{def}Sciezka} dlugosci $n-1$ [$P_{n-1}$] to graf z wierzcholkami
    $$V(P_{n-1})=[n]$$
    i krawedziami
    $$E(P_{n-1})=\{i(i+1)\;:\;1\leq i\leq n-1\}.$$

    \\
    
    \hline

    & \\

    An {\color{def}induced} by $A\subseteq V(G)$ subgraph of $G$ is \\
    $G[A]=(A, E_A)$
    \medskip

    A {\color{acc}connected component} of $G$ is a subgraph $G[W]\leq G$ where $W\subseteq V$ is an equivalence class under $\approx$ given by
    $$v\approx w\iff \text{exists a path } v...w\text{ in }G$$

    A graph is {\color{def}connected} if $v\approx w$ for every $v,w\in V$ ($G$ has at most one connected component).

    & \\

    \hline

    & \\

    If $v$ is a vertex in graph $G$, we say that its {\color{def}neighbourhood} is $N_G(v)=\{w\in G\;:\;vw\in E(G)\}$. Furthermore, the {\color{def}degree of }$v$ is $|N_G(v)|$.
    \medskip

    If $A\subseteq V$, then $N(A):=\bigcup\limits_{v\in A}N(v)$.
    

    & \\


\end{tabularx}


\begin{tabularx}{\textwidth}{ X!{\color{git90gray}\vrule} X}
    We define:\smallskip
    
    \point {\color{acc}minimal degree} $\delta(G)=\min\limits_{v\in G}d(v)$
    \smallskip

    \point {\color{acc}maximal degree} $\Delta(G)=\max\limits_{v\in G}d(v)$
    \smallskip

    \point {\color{acc}average degree} $d(G)={\sum d(v)\over |G|}$.
    \medskip

    If there exists an $r\geq 0$ such that
    $$\delta(G)=\Delta(G)=d(G)=r$$
    then we say that the graph is {\color{def}r-regular} or, more generally, it is {\color{acc}regular} for some $r$.
    \medskip

    {\color{def}Handshaking Lemma}: for any graph $G$ we have $e(G)=\frac12\sum d(v)={|G|\over2}d(G)$

    & \\

    \hline

\end{tabularx}

\subsection{Hall's Marriage Theorem}

\begin{tabularx}{\textwidth}{ X!{\color{git90gray}\vrule} X}

    Graph $G$ is {\color{def}bipartite} with vertex classes $U$ and $W$ if $V=U\cup W$ so that every edge has form $uw$ for some $u\in U$ and $w\in W$.
    \smallskip

    $G$ is bipartite iff it has no cycles of odd length.

    &
    Graf $G$ jest {\color{def}dwudzielny} z klasami wierzcholkow $U$ i $W$, jesli $V=U\cup W$ takimi, ze kazda krawedz jest formy $uw$ dla pewnych $u\in U$ oraz $w\in W$.
    \smallskip

    $G$ jest dwudzielny wtw kiedy nie ma cykli o nieparzystej dlugosci.
\end{tabularx}

\hyperref[bipartite-even-cycle-GB]{[ \worldflag[length=10px, width=5px]{GB}]} \hyperref[bipartite-even-cycle-PL]{[ \worldflag[length=10px, width=5px]{PL}]}
\phantomsection
\label{bipartite-even-cycle-LAN}

\medskip

\hyperref[bipartite-even-cycle-LAN]{[ \worldflag[length=10px, width=5px]{GB}]}
\phantomsection
\label{bipartite-even-cycle-GB}
\medskip

$\color{acc}\implies$
\smallskip

Let $U,W$ be the vertex classes and $v_1,v_2,...,v_n,v_1$ be a cycle in $G$. WLG suppose that $v_1\in U$. Then $v_2\in W$ etc. Specifically we have $v_i\in U$ if $i$ is odd and $v_i\in W$ if $i$ is even. Then, we have $v_nv_i$, so $n$ must be even.
\medskip

$\color{acc}\impliedby$
\smallskip

Suppose $G$ has no cycles of odd length. WLOG, assume that $V(G)\neq\emptyset$ and that $G$ is connected, because $G$ will be bipartite if all its connected components are bipartite. Fix $v\in G$ and for all other $w\in G$ define distance $dist(v,w)$ as the smallest $n\geq 0$ such that there exists a path $v...w$ in $G$ of length $n$.
\smallskip

Now, let $V_n:=\{w\in G\;:\;dist(v,w)=n\}$ and set
\begin{align*}
    U&=V_0\cup V_2\cup V_4\cup...\\
    W&=V_1\cup V_3\cup V_5\cup...
\end{align*}
We want to show that there are no edges in $G$ of the form $v'v''$ where $v',v''\in U$ or $v',v''\in W$.
\smallskip

Suppose that $v'v''\in E(G)$ with $v'\in V_m,v''\in V_n$ and $m\leq n$. Then, we have a path
$$v...v'v''\in G$$
of length $m+1$, implying that 
$$n\in\{m,m+1\}.$$

Supose that $n=m$. Let $v_0'v_1'...v_m'$ and $v_0''v_1''...v_m''$ be paths in $G$ with $v=v_0'=v_0''$, $v'=v_m'$ and $v''=v_m''$. Note that $v_i',v_i''\in V_i$ for $0\leq i\leq m$. Let $k\geq 0$ be largest such that
$$v_k'=v_k''$$
and note that $k\leq m-1$ as $v'\neq v''$. Then 
$$v_k'v_{k+1}'...v_m'v_m''v_{m-1}''...v_k''$$
is a cycle of odd length, which is a contradiction.
\medskip

Therefore, we can only have $n=m+1$ and then exactly one of $n,m$ is even meaning that exactly one of $v'$ and  $v''$ is in $U$ as required for $G$ to be bipartite.
\bigskip

\hyperref[bipartite-even-cycle-LAN]{[ \worldflag[length=10px, width=5px]{PL}]}
\phantomsection
\label{bipartite-even-cycle-PL}
\medskip

$\color{acc}\implies$
\smallskip

Niech $U,W$ beda klasami wierzcholkow oraz niech $v_1,v_2,...,v_n,v_1$ niech bedzie cyklem w $G$. BSO zalozmy, ze $v_1\in U$. W takim razie, $v_2\in W$ etc. W szczegolnosci, mamy $v_i\in U$ jezeli $i$ jest nieparzyste oraz $v_i\in W$ jezeli $i$ jest parzyste. W takim razie, skoro $v_nv_1$, to $n$ musi byc parzyste.
\medskip

$\color{acc}\impliedby$
\smallskip

Zalozmy, ze $G$ nie ma cykli o nieparzystej dlugosci. BSO zalozmy, ze $V(G)\neq\emptyset$ i ze $G$ jest spojny, poniewaz $G$ bedzie dwudzielny, wtw gdy wszystkie jego skladowe spojne (????) beda dwudzielne. Ustalmy $v\in G$ i dla kazdego innego $w\in G$ zdefiniujmy dystans $dist(v,w)$ jako najmniejsze $n\geq0$ takie, ze istnieje sciezka $v...w$ w $G$ o dlugosci $n$.
\smallskip

Niech $V_n:=\{w\in G\;:\; dist(v,w)=n\}$ i zbiory
\begin{align*}
    U&=V_0\cup V_2\cup V_4\cup...\\
    V&=V_1\cup V_3\cup V_5\cup ...
\end{align*}
Chcemy pokazac, ze nie istnieja w $G$ krawedzie postaci $v'v''$, gdzie $v',v''\in U$ lub $v',v''\in W$.
\smallskip

Zalozmy, ze $v'v''\in E(G)$ z $v'\in V_m, v''\in V_n$ oraz $m\leq n$. Wtedy istnieje sciezka
$$v...v'v''\in G$$
dlugosci $m+1$, co implikuje, ze
$$n\in \{m,m+1\}.$$

Zalozmy, ze $n=m$. Niech $v_0'v_1'...v_m'$ oraz $v_0''v_1''...v_m''$ sa sciezkami w $G$ takimi, ze $v=v_0'v_0''$, $v'=v_m'$ oraz $v''=v_m''$. Zauwazmy, ze $v_i',v_i''\in V_i$ dla $0\leq i\leq m$. Niech $k\geq0$ bedzie najwiksze takie, ze
$$v_k'=v_k''$$
i zauwazmy, ze $k\leq m-1$ poniewaz $v'\neq v''$. Wtedy
$$v_k'v_{k+1}'...v_m'v_m''v_{m-1}''...v_{k}''$$
jest cyklem o nieparzystej dlugosci, co daje nam sprzecznosc.
\medskip

W takim raize, mozemy miec tylko $n=m+1$ i wtedy dokladnie jedno z $n,m$ moze byc parzystem, co daje nam dokladnie jedno z $v'$ i $v''$ w $U$ tak, jak jest wymagane zeby to byl graf dwudzielny.
\bigskip


\begin{tabularx}{\textwidth}{ X!{\color{git90gray}\vrule} X}

    \hline
    
    & \\

    If $G$ is a bipartite graph with $V=W\cup M$ and $W'\subseteq W$, a {\color{acc}partial matching} in $G$ from $W'$ to $M$ is
    $$\{wv_w\;:\;w\in W'\}\subseteq E(G)$$
    for some $v_w\in M$ such that $w\neq w'\implies v_w\neq v_{w'}$. A partial matching from $W$ to $M$ is called a {\color{def}matching}.
    \medskip

    Sufficient condition:
    $$|N(A)|\geq|A| \quad(\kawa)$$ 
    for every $A\subseteq W$
    
    &
    Jesli $G$ jest grafem dwudzielnym z $V=W\cup M$ oraz $W'\subseteq W$, wtedy {\color{acc}czesciowe skojarzenie} w $G$ z $W'$ do $M$ to
    $$\{wv_w\;:\;w\in W'\}\subseteq E(G)$$
    dla pewnych $v_w\in M$ takich, ze $w\neq w'\implies v_w\neq v_{w'}$. Czesciowe kojarzenie z $W$ do $M$ jest nazywane {\color{def}kojarzeniem}.
    \medskip

    Wystarczajacy warunek:
    $$|N(A)|\geq|A| \quad(\kawa)$$ 
    dla kazdego $A\subseteq W$

    \\

    \podz{sep}
    \medskip

    A bipartite graf $G$ contains a matching from $W$ to $M$ $iff$ $(G,W)$ satisfies Hall's condition $(\kawa)$.

    &

    \podz{sep}
    \medskip

    Dwudzielny graf $G$ zawiera kojarzeniem $iff$ gdy $(G,W)$ zadowala warunek Halla $(\kawa)$.

\end{tabularx}
\medskip

\hyperref[halls-condition-GB]{[ \worldflag[length=10px, width=5px]{GB}]} \hyperref[halls-condition-PL]{[ \worldflag[length=10px, width=5px]{PL}]}
\phantomsection
\label{halls-condition-LAN}

\hyperref[halls-condition-LAN]{[ \worldflag[length=10px, width=5px]{GB}]}
\phantomsection
\label{halls-condition-GB}
\medskip

$\color{acc}\implies$
\smallskip

Trivial.
\medskip

$\color{acc}\impliedby$
\smallskip

Using induction on $|W|$. For $|W|=0,1$ it is trivial.
\smallskip

We gonna break it into parts:  $|N(A)|>|A|$ and $|N(A)|=|A|$
\smallskip

Suppose that $|N(A)|>|A|$ for every non-empty subset $A\subsetneq W$. Take any $w\in W$ and $v\in N(w)$ and construct a new graph
$$G_0=G-\{w,v\}.$$
For any non-empty $B\subseteq W-\{w\}$ we have
$$N_{G_0}(B)=N_G(B)-\{v\}$$
and therefore
$$|N_{G_0}(B)|\geq |N_G(B)|-1\geq |B|$$
and so $(G_0,W-\{w\})$ satisfies Hall's condition. From induction we have a matching $P$ in $G_0$ from $W-\{w\}$ to $M-\{v\}$ and so $P\cup \{wv\}$ is a matching from $W$ to $M$.
\smallskip

Now, suppose that $|N(A)=|A|$ for some non-empty subset $A\subsetneq W$. Let 
$$G_1=G[A\cup N(A)]$$ 
and 
$$g_2=G[(W-A)\cup(M-N(A))].$$
We will show that both those graphs satisfy Hall's condition.

Let us take any $B\subseteq A$ in $G_1$. We have 
$$N_G(B)\subseteq N_G(A)\subseteq V(G_1)$$
$$|N_{G_1}(B)|=|N_G(B)|\geq|B|$$
and so graph $G_1$ satisfies Hall's condition.

Now, let us take any $B\subseteq W-A$ in $G_2$. We know that $N_{G_2}(B)\subseteq M-N(A)$ so 
$$N_{G_2}(B)= N_G(B)-N_G(A)=N_G(A\cup B)-N_G(A)$$
$$|N_{G_2}(B)|=|N_G(A\cup B)-N_G(A)|\geq |N_G(A\cup B)|-|N_G(A)|\geq |A\cup B|-|A|=|A|+|B|-|A|=|B|$$
Therefore, graph $G_2$ also satisfies Hall's condition.

Using inductive hypothesis, we have that there exists a matching $P_1$ in $G_1$ and a matching $P_2$ in $G_2$. The first one is from $A$ to $N_G(A)$ while the second is from $W-A$ to $M-N_G(A)$, so they are disjoint. Therefore, $P_1\cup P_2$ is a matching in $G$ from $W$ to $M$.
\bigskip

\hyperref[halls-condition-LAN]{[ \worldflag[length=10px, width=5px]{PL}]}
\phantomsection
\label{halls-condition-PL}
\medskip

$\color{acc}\implies$
\smallskip

Trywialne.
\medskip

$\color{acc}\impliedby$
\smallskip

Uzyjemy indukcji na $|W|$. Dla $|W|=0,1$ jest trywialne.
\smallskip

Podzielimy dowod na dwie czesci:  $|N(A)|>|A|$ oraz $|N(A)|=|A|$.
\smallskip

Zalozmy, że $|N(A)|>|A|$ dla kazdego niepustego podzbioru $A\subsetneq W$. Wezmy dowolne $w\in W$ oraz $v\in N(w)$ i skonstruujmy nowy graf
$$G_0=G-\{w,v\}.$$
Dla kazdego niepustego $B\subseteq W-\{w\}$ mamy
$$N_{G_0}(B)=N_{G}(B)-\{v\}$$
i w takim razie
$$|N_{G_0}(B)|\geq |N_G(B)|-1\geq |B|,$$
czyli $(G_0, W-\{w\})$ spelnia warunek Halla. Z zalozenia indukcyjnego istnieje kojarzenie $P$ w $G_0$ z $W-\{w\}$ do $M-\{v\}$, w takim razie $P\cup\{wv\}$ jest kojarzeniem z $W$ do $M$.
\smallskip

Zalozmy teraz, ze $|N(A)=A|$ dla pewnego niepustego podzbioru $A\subsetneq W$. Niech
$$G_1=G[A\cup N(A)]$$ 
oraz 
$$g_2=G[(W-A)\cup(M-N(A))].$$
Pokazemy, ze oba te grafy zaspokajaja warunek Halla.