\section{Structural properties}

\subsection{The basics}

\pdef

    {\color{def}Graph} - an ordered pair $G=(V,E)$:\smallskip\\
    \point {\color{acc}vertices} $:=V$ [singular \emph{vertex}]\\
    \point {\color{acc}edges} $:=E$, $\{v,w\}:=vw$

\kdef 

\begin{tabularx}{\textwidth}{ X X }

    For an edge $vw$, $v\neq w$ we say that $v,w$ are its {\color{acc}endpoints} and that it is {\color{acc}incident} to $v$ (or $w$). 
    \medskip
    
    \podz{sep}
    &
    Dla krawedzi $vw$, $v\neq w$ mowimy, ze $v,w$ sa jej {\color{acc}koncami} i ze jest krawedzia {\color{acc}padajaca} na $v$ (lub $w$).\\

    & \\

    Graphs $G$ and $H$ are {\color{def}isomorfic} ($G\simeq H$) if there exists $f:V(G)\xrightarrow[1-1]{on} V(H)$ such that\\
    $(\forall\;v,w\in V(G))\;vw\in E(G)\iff f(v)f(w)\in E(H)$\smallskip

    $G$ is a {\color{def}subgraph} of $H$ [$G\leq H$] if $V(G)\subseteq V(H)$ and $E(G)\subseteq E(H)$.\smallskip

    If $G$ is {\color{def}$H$-free} if it is has no subgraphs isomorfphic to $H$.

\end{tabularx}