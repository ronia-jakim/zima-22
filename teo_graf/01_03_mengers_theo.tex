\subsection{Menger's Theorem}

\begin{tabularx}{\textwidth}{ X!{\color{git90gray}\vrule} X}

    {\color{acc}Cut vertex} $v$ is a vertex in a connected graph $G$ such that $G-\{v\}$ is not connected.
    \smallskip

    Graph $G$ is a {\color{def}$k$-connected graph} if for any $A\subseteq V(G)$, $|A|<k$, $G-A$ is connected.
    \medskip

    {\color{def}Complete graph} has all vertices connected by an edge, that is for all $v,w\in G$ $v\neq w$ we have $vw\in G$.

    &

    {\color{acc}Tnacy wierzcholek} $v$ jest wierzcholkiem w spojnym grafie $G$ takim, ze $G-\{v\}$ jest niespojny.
    \smallskip

    Graf $G$ jest {\color{def}$k$-spojnym grafem}, jezeli dla kazdego $A\subseteq V(G)$, $|A|<k$, $G-A$ jest spojny.
    \medskip

    {\color{def}Graf pelny} ma wszystkie wierzcholki polaczone krawedzia, to znaczy dla kazdego $v,w\in G$, $v\neq w$ mamy $vw\in G$. 
    \\

    \hline

    {\color{def}$(A,B)$-path} is a path in $G$ for some $A,B\subseteq V$ of the form $a...b$ for some $a\in A$ and $b\in B$.

    {\color{def}$(A,B)$-cut} in $G$ is $C\subseteq V$ such that $G-C$ contains no $(A-C,B-C)$-paths.
    \smallskip

    If we take vertices $a,v\in V$ we call an $(\{a\},\{b\})$-path an $(a,b)$-path. Given a collection of $(a,b)$-paths
    $$P^{(1)},...,P^{(k)}$$
    we say such a collection is {\color{def}independent} if $P^{(i)}-\{a,b\}$ and $P^{(j)}-\{a,b\}$ have no common vertices for $i\neq j$.

    &

    {\color{def}$(A,B)$-sciezka} to sciezka w $G$ dla pewnych $A,B\subseteq V$ postaci $a...b$ dla jakis $a\in A$ i $b\in B$.

    {\color{def}$(A,B)$-ciecie} w $G$ to $C\subseteq V$ takie, ze $G-C$ nie zawiera zadnych $(A-C, B-C)$-sciezek.
    \smallskip

    Jesli wezmiemy wierzcholki $a,v\in V$, to $(\{a\},\{b\})$-sciezke nazywamy $(a,b)$-sciezka. Jesli dana jest kolekcja $(a,b)$-sciezek
    $$P^{(1)},...,P^{(k)}$$
    mowimy, ze ta kolekcja jest {\color{def}niezalezna}, jezeli $P^{(i)}-\{a,b\}$ i $P^{(j)}-\{a,b\}$ nie maja wspolnych wierzcholkow dla $i\neq j$.

\end{tabularx}

\begin{tabularx}{\textwidth}{ X!{\color{git90gray}\vrule} X}

    Given $A,B,C\subseteq V(G)$ and if $A\subseteq C$ or $B\subseteq C$, then $C$ is an $(A,B)$-cut and if $C$ is an $(A,B)$-cut then $A\cap B\subseteq C$.
    \bigskip

    Let $G$ be a graph, $A,B\subseteq V(G)$ and $k\geq0$. Suppose that for every $(A,B)$-cut $C$ in $G$ we have $|C|\geq k$. 

    Then $G$ contains a {\color{acc}collection of $k$ vertex-disjoint} $(A,B)$-paths.

    &

    Dla danych $A,B,C\subseteq V(G)$, jezeli $A\subseteq C$ albo $B\subseteq C$, to $C$ jest $(A,B)$-cieciem i jesli $C$ jest $(A,B)$-cieciem, to $A\cap B\subseteq C$.
    \bigskip

    Niech $G$ bedzie grafem, $A,B\subseteq V(G)$ i $k\geq0$. Zalozmy, ze dla kazdego $(A,B)$-ciecia $C$ w $G$ jest $|C|\geq k$.

    Wtedy $G$ zawiera {\color{acc}zbior $k$ rozlacznych wierzcholkami} $(A,B)$-sciezek.

\end{tabularx}

\prooflink{vertex-disjoined-lemma}
\medskip

\prooflang{vertex-disjoined-lemma}{GB}
\medskip

{\color{cyan}I WILL GET TO IT SOMEDAY}

\prooflang{vertex-disjoined-lemma}{PL}
\medskip


Uzyjemy indukcji na $e(G)$ \hyperref[handshaking-lemma]{[definicja dla debila]}.
\smallskip

Jako przypadek bazowy mamy $e(G)=0$, wtedy $A\cap B$ jest $(A,B)$-cieciem i w takim razie $k\leq |A\cap B|$, ale kazdy wierzcholek $A\cap B$ jest $(A,B)$-sciezka dlugosci 0 i wszystkie z nich sa rozlaczne, tak jak wymagamy.
\medskip

Zalozmy, ze $e(G)\geq 1$, wybiezmy krawedz $e\in E(G)$ i niech $H=G-\{e\}$. 
\smallskip

Jesli dla kazde $(A,B)$-ciecie w $H$ ma stopien co najmniej $k$, to przez hipoteze indukcyjna sa one $k$ wierzcholkowo rozlacznymi $(A,B)$-sciezkami w $H$ i w takim razie w $G$, wiec koniec.
\smallskip

Zalozmy teraz, bez starty ogolnosci, ze w $H$ istnieje co najmniej jedno $(A,B)$-ciecie $C$ takie, ze $|C|<k$. W takim razie $C$ nie jest $(A, B)$-cieciem w $G$, wiec $G-C$ zawiera co najmniej jedna $(A, B)$-sciezke postaci
$$a...vw...b$$
dla pewnych $a\in A$, $b\in B$, gdzie $v,w\in G$ sa koncami $e$. Co wiecej, kazda $(A, B)$-sciezka w $G-C$ zawiera wierzcholek $v$, co implikuje ze
$$C'=C\cup\{v\}$$
jest $(A, B)$-cieciem w $G$. Co wiecej, $|C'|=|C|+1\geq k$. {\color{dyg}Poniewaz $a...vw...b$ bylo jedyna sciezka ktora blokowala $C$ przed zostaniem $(A,B)$-cieciem w $G$, ale juz $|C'|$ nim jest}, to $|C|=k-1$ i mozemy przyjac, ze
$$C=\{c_1,...,c_{k-1}.$$
Teraz, poniewaz $v\in C'$, to kazde $(A, C')$-ciecie $D$ w $H$ jest takze $(A, C')$-cieciem w $G$. Poniewaz kazda $(A, B)$-sciezka w $G$ zawiera wierzcholek $C'$, to $D$ jest takze $(A, B)$-cieciem w $G$ i dlatego $|D|\geq k$. Korzystajac wiec z hipotezy indukcyjnej, wiemy, ze istnieja rozlaczne wierzcholkami $(A, C')$-sciezki 
$$P^{(1)},...,P^{(k-1)},P^{(k)}$$ 
w $H$ konczace sie odpowiednio w $c_1,..., c_{k-1}, v$. Niech $C''=C\cup\{w\}$. Wtedy analogicznie, mamy takie $(C'', B)$-sciezki 
$$Q^{(1)},...,Q^{{(k-1)}},Q^{(k)}$$
w $H$ zaczynajace sie od odpowiednio wierzcholkow $c_1,..., c_{k-1}, v$. Co wiecej, poniewaz $C'$ jest $(A, B)$-cieciem w $G$, to $P^{(i)}$ oraz $Q^{(j)}$ nie moga miec wspolnego wierzcholka $u$ poza przypadkiem $i=j\leq k-1$ i $u=c_i$. To sugeruje, ze 
$$P^{(1)}Q^{(1)},...,P^{(k-1)}Q^{(k-1)},P^{(k)}eQ^{(k)}$$
sa $k$ rozlacznymi wzgledem wierzcholkow $(A, B)$-sciezkami w $G$. Koniec.
\bigskip

\podz{sep}
\bigskip

\begin{tabularx}{\textwidth}{ X!{\color{git90gray}\vrule} X}

    Hall's Marriage Theorem may be deduced from this lemma:

    Let $G$ be a bipartite graph with vertex classes $W$ and $M$ and suppose that $(G, W)$ satisfies Hall's condition. Let $C$ be a $(W, M)$-cut in $G$. Then 
    $$N(W-C)\subseteq M\cap C$$

    &

    Twierdzenie Halla o malzenstwach moze byc wyprowadzone z tego lematu:

    Niech $G$ bedzie grafem dwudzielnym z klasami wierzcholkow $W$ i $M$ i zalozmy, ze $(G, W)$ zadowala warunek Halla. Niech $C$ bedzie $(W, M)$-cieciem w $G$. Wtedy
    $$N(W-C)\subseteq M\cap C$$

\end{tabularx}

\begin{tabularx}{\textwidth}{ X!{\color{git90gray}\vrule} X}
    
    and therefore
    $$
    \begin{aligned}
        |C|=|W\cap c|+|M\cap C|\geq \\
        |W\cap C|+|N(W-C)|\geq\\
        |W\cap C|+|W-C|=|W|
    \end{aligned}
    $$    
    thus $|W|$ contains vertex-disjoint $(W, M)$-paths, each of length $1$ implying that such a collection of paths is a matching.

    &
    
    i z tego
    $$
    \begin{aligned}
        |C|=|W\cap c|+|M\cap C|\geq \\
        |W\cap C|+|N(W-C)|\geq\\
        |W\cap C|+|W-C|=|W|
    \end{aligned}
    $$
    a wiec $|W|$ zawiera rozlaczne wzgledem wierzcholkow $(W, M)$-sciezki, kazda o dlugosci $1$, implikujac ze taki zbior sciezek jest kojarzeniem.\\

    & \\

    \hline

    & \\

    {\color{def}Menger's Theorem}
    \smallskip

    Let $G$ be an incomplete graph and let $k\geq0$. Then $G$ is $k$-connected iff for every $a,b\in G$ with $a\neq b$, there exists a collection of $k$ independent $(a,b)$-paths in $G$.

    &

    {\color{def}Twierdzenie Mengera}
    \smallskip

    Niech $G$ bedze niepelnym grafem i niech $k\geq0$. Wtedy $G$ jest $k$-spojne iff dla kazdego $a,b\in G$ z $a\neq b$ istnieje zbior $k$ niezaleznych $(a,b)$-sciezek w $G$.

\end{tabularx}

\prooflink{menger-theorem}
\medskip

\prooflang{menger-theorem}{GB}
\medskip

{\color{cyan}I WILL GET TO IT SOMEDAY}

\prooflang{menger-theorem}{PL}
\medskip

$\color{acc}\implies$

Niech $C\subseteq V(G)$ i zalozmy, ze $G-C$ jest niespojny. Wybierzmy dowolne $a,b\in G-C$ nalezace do roznych skladowych spojnosci $G-C$. Na mocy tego zalozenia, $G$ zawiera $k$ niezaleznych $(a,b)$-sciezek. Kazda z tych sciezek musi miec wierzcholek w $C$, ale zadne dwie sciezki nie maja wspolnego wierzcholka poza $a$ i $b$. Z tego wynika, ze $|C|\geq k$, tak jak wymagamy.
\smallskip

$\color{acc}\impliedby$

Bedziemy robic indukcje po $k$.\smallskip 

Przypadek bazowy dla $k=0$ jest trywialny. 

Niech wiec $k\geq 1$ i niech $a,b\in G$ beda rozne.

Zalozmy najpierw, ze $a\not\sim b$. Niech $A=N(a)$ oraz $B=N(b)$. Grafy $G-A$ i $G-B$ sa niespojne, bo nie maja ani jednej sciezki $a....b$. Daje to $|A|\geq k$ oraz $|B|\geq k$. Jezeli $C$ jest $(A, B)$-cieciem w $G$, to $G-C$ rowniez nie ma sciezem miedzy elementami $A-C$ oraz $B-C$. Dlatego, albo $A\subseteq C$ albo $B\subseteq C$, albo $G-C$ jest niespojne. W kazdym razie, mamy $|C|\geq k$ wiec z lematu wyzej, $G$ ma $k$ rozlacznych wzgledem wierzcholkow $(A, B)$-sciezek:
$$a_1...b_1,\;...,\;a_k...b_k.$$
Wtedy, 
$$aa_1...b_1b,\;...,\;aa_k...b_kb$$
sa $k$ niezaleznymi sciezkami $(a,b)$ tak jak wymagamy.

Zalozmy teraz, ze $a\sim b$ i niech $H=G-\{ab\}$. Pokazemy najpiew, ze $H$ jest $(k-1)$-spojne.

Zalozmy, ze tak nie jest. Niech $C\subseteq V(H)$ bedzie takim podzbiorem, ze $|C|<k-1$ i niech $H-C$ bedzie niespojne. Poniewaz $G$ jest $k$-spojne, to $G-C$ jest spojne i nie ma wierzcholkow tnacych ({\color{dyg}cut vertices}), co implikuje ze $H-C$ dokladnie dwie skladowe spojne, kazda zawierajaca jeden z wierzcholkow $a$ lub $b$. Ale wtedy $|G|=|H|=2+|C|\leq k$, wiec $G$ jest grafem $k$-spojnym z $|G|\leq k$, co daje sprzecznosc z tym, ze $G$ nie jest pelne.

W takim raize, $H$ musi byc $(k-1)$-spojne. Z hipotezy indukcyjnej zawiera wiec $k-1$ niezaleznych $(a,b)$-sciezem. Razem z krawedzie $ab$ te sciezki tworza zbior $k$ niezaleznych $(a,b)$ sciezek w $G$, co konczy dowod.