\documentclass{article}[13pt]

\usepackage{../uni-notes-eng}
\usepackage{multicol}
\usepackage{graphicx}

\title {Kombinatoryka $\&$ teoria grafów}
\author{by a fish}
\date {21.03.2137}

\begin{document}

\maketitle

\newpage

\begin{center}
    \includegraphics[height=\textheight]{potega.jpg}
\end{center}

\newpage

\section*{SYLABUS - teoria grafów:}
1. Basic concepts: graphs, paths and cycles, complete andbipartite graphs\\
2. Matchings: Hall's Marriage theorem and its variations\\
3. Forbidden subgraphs: complete bipartite and r-partite subgraphs, chromatic numbers, Tur"an's thorem, asymptotic behaviour og edge density, Erd"os-Stone theorem\\
4. Hamiltonian cycles (Dirac's Theorem), Eulerian circuits\\
5. Connectivity: connected and k-connected graphs, Menger's theorem\\
6. Ramsey theory: edge colourings of graphs, Ramsey's theorem and its variations, asymptotic bounds on Ramsey numbers\\
7. Planar graphs and colourings: statements of Kuratowski's and Four Colour theorems, proof of Five Colour theorem, graphs on other surfaces and Euler chracteristics, chromatic polynomial, edge colourings and Vizing's theorem\\
8. Random graphs: further asymptotic bounds on Ramsey numbers, Zarankiewicz numbers and their bounds, graphs of large firth and high chromatic number, cmplete subgraphs in random graphs.\\
9. Algebraic methods: adjavenvy matrix and its eigenvalues, strongly regular graphs, Moore graphs and their existence.

\newpage


\tableofcontents

\newpage

\begin{multicols*}{2}
    
\begin{mdframed}[style=definicja]
    {\color{def}Definicja} - to jest bardzo {\color{acc}wyedukowany} tekst w ramce 
\end{mdframed}

Troszke mniej wyedukowany tekst, bo jest poza fajna ramka ktora wyglada jak guwno i w sumie to nie wiem czemu ja ja w ten sposob ranie
\end{multicols*}

\section{Powtorka z poprzedniego roku}

\subsection{Grupy, pierscienie, ciala}
\begin{multicols*}{2}

    Dzialanie na zbiorze $X$:
    $$\Phi: X\times X\to X,$$
    zwykle zapisywane jako $xy$, $x\cdot y$, $x+y$.\smallskip

    {\color{def}Element neutralny} - takie $e$, ze dla kazdego $x\in X$ $ex=xe=x$. Dzialanie ma co najwyzej jeden element neutralny.\smallskip
    
    {\color{def}Element odwrotny} do $x$ to takie $y$, ze $xy=yx=e$. Jesli dzialanie jest laczne, to ma co najwyzej jeden element odwrotny do danego $x$.\medskip

    \podz{sep}\medskip

    {\color{def}Homomorfizm} algebry $\rodz X=(X,\{ \cdot\})$ na algebre $\rodz Y=(Y,\{\circ\})$ nazywamy przeksztalcenie $f:X\to Y$ spelniajace dla kazdego $a, b\in X$
    $$f(a\cdot b)=f(a)\circ f(b).$$
    \indent {\color{acc}$\bullet$ monomorfizm} - $f$ jest 1-1\smallskip\\
    \indent {\color{acc}$\bullet$ epimorfizm} - $f$ jest "na"\smallskip\\
    \indent {\color{acc}$\bullet$ izomorfizm} - $f$ jest 1-1 i "na"\smallskip\\
    \indent {\color{acc}$\bullet$ endomorfizm} - kiedy $\rodz Y=\rodz X$\smallskip\\
    \indent {\color{acc}$\bullet$ automorfizm} - enodmorfizm bedacy izomorfizmem\medskip

    \podz{sep}\medskip

    {\color{def}Polgrupa} to niepusty zbior z dzialaniem lacznym.\medskip
    
    {\color{def}GRUPA} to niepusty zbior z lacznym dzialaniem i elementem neutralnym (zwanym {\color{acc}jednoscia grupy}) oraz elementami odwrotnymi dla kazdego elementu.\smallskip\\
    \indent {\color{acc}$\hookrightarrow$ grupa abelowa} (przemienna) - grupa z dzialaniem przemiennym\smallskip\\
    Zbior $G$ z dzialaniem $\cdot$ jest grupa, jesli:\smallskip\\
    \indent 1. $(\forall\;a,b,c\in G)\;(ab)c=a(bc)$\\
    \indent 2. $(\exists\;e\in G)(\forall\;a\in G)\;ea=ae=e$\\
    \indent 3. $(\forall\;a\in G)(\exists\;b\in G)\;ab=ba=e$\\
    \indent *4. $(\forall\;a,b\in G)\;ab=ba$ w grupie \emph{abelowej}\bigskip

    {\color{def}PIERSCIEN} to niepusty zbior $X$ z dwoma dzialaniami ($\cdot,\;+$, mnozenie i dodawanie), ktory spelnia:\smallskip\\
    \indent 1. zbior $X$ z $+$ jest grupa abelowa\smallskip\\
    \indent 2. $\cdot$ jest laczne\smallskip\\
    \indent 3. $(\forall\;x,y,z\in X)\;x\cdot(y+z)=x\cdot y+x\cdot z\;\land\;(x+y)\cdot z=x\cdot z+y\cdot z$\smallskip\\
    Kolejne dzikie nazwy $\star$:\smallskip\\
    \indent {\color{acc}$\star$ pierscien przemienny} - jesli mnozenia jest przemienne\smallskip\\
    \indent {\color{acc}$\star$ pierscien z jednoscia} - dla mnozenia istnieje element neutralny\medskip

    {\color{def}CIALO} to pierscien przemienny, ktory dla kazdego elementu $\neq 0$ ma element odwrotny\medskip

    \podz{sep}\medskip

    Niech $G$ bedzie grupa, a $e$ jej elementem neutralnym. Wowczas:\smallskip\\
    \point $a,b\in G\implies (ab)^{-1}=b^{-1}a^{-1}$\smallskip\\
    \point $a\in G$ i $n=1, ..., n$ $a^{-n} = (a^n)^{-1} =^* (a^{-1})^n$\smallskip\\
    \point dla $m,n\in\Z$ oraz $a\in G$ mamy $a^{mn}=^* (a^m)^n$\smallskip\\
    \point dla $G$ grupy abelowej i $n\in\Z$ $(ab)^n=^*a^nb^n$\smallskip\\
    $^*$ trzeba udowodnic, ale mi sie nie chce\medskip

    $H\subseteq G$ jest {\color{def}podgrupa} $G$, jesli jest grupa ze wzgledu na te same dzialania, czyli wystarczy, ze 
    $$(\forall\;a,b\in H)\; ab^{-1}\in H.$$

    Jelsi $a\in G$ i istnieja $n\in\N$, $n\geq 1$, takie, ze $a^n=e$, to mowimy ze $n$ jest {\color{def}rzedem elementu} $a$ ($n=o(a)$). Jesli takie $n$ nie istnieja, to $a$ ma {\color{acc}rzad nieskonczony} ($o(a)=\infty$).\smallskip\\
    {\color{acc}\point} {\color{def}grupa torsyjna} - wszystkie elementy maja rzad skonczony\smallskip\\
    {\color{acc}\point} {\color{def}grupa beztorsyjna} - wszystkie elementy maja rzad nieskonczony\smallskip\\
    \emph{Jesli $n=o(a)$ oraz $a^N=e$ to $n|N$, fajny dowodzik, ale leniem jestem}

\end{multicols*}


\end{document}