\documentclass{article}[13pt]

\usepackage{../../uni-notes-eng}
\usepackage{multicol}

\begin{document}

    \section*{ZAD 1.}
    $$\lfloor an\rfloor+\lfloor(1-a)n\rfloor=n-1$$

    Niech $\Z\ni m=\lfloor an\rfloor$, wtedy
    \begin{align*}
        m\leq an &<m+1\\
        m-n\leq an-n&<m-n+1\\
        n-m\geq n-an&> n-m-1
    \end{align*}

    Poniewaz $n\notin \Q$, to $n-an\notin \Z$, wiec
    $$\lfloor n-an\rfloor=n-\lfloor an\rfloor -1$$
    a z tego
    $$\lfloor an\rfloor+\lfloor n(1-a)\rfloor = n-1$$

    
    $$\lceil an\rceil + \lceil n-an\rceil=n+1$$

    \section*{ZAD 2.}
    $$\lfloor {x\over m}\rfloor + \lfloor {x+1\over m}\rfloor+...+\lfloor {x+m-1\over m}\rfloor=\sum\limits_{i=0}^{m-1}\lfloor {x+i\over m}\rfloor\quad(\nietoperz)$$

    Po pierwsze pokazemy, ze dla dowolnych $n,m\in\Z$ oraz $x\in\R$ zachodzi
    $$\Big\lfloor {x+n\over m}\Big\rfloor=\Big\lfloor{\lfloor x\rfloor+n\over m}\Big\rfloor\quad (\kawa)$$
    
    Niech $p=\Big\lfloor{\lfloor x\rfloor+n\over m}\Big\rfloor$, wtedy
    \begin{align*}
        p&\leq {\lfloor x\rfloor+n\over m} < p+1\\
        p&\leq {\lfloor x\rfloor+n\over m}<p+1\\
        \Z\ni m\cdot p-n&\leq \lfloor x\rfloor<m\cdot(p+1)-n\in\Z\\
        m\cdot p-n&\leq x<m\cdot(p+1)-n\\
        p&\leq {x-n\over m}<p+1\\
        p&\leq \Big\lfloor {x+n\over m}\Big\rfloor<p+1\\
        p&=\Big\lfloor {x+n\over m}\Big\rfloor
    \end{align*}
    Czyli pokazalismy $(\kawa)$.
    \bigskip

    Po drugie, zauwazmy, ze dla dowolnego $n$ i dla kazdego $m\in\Z$, $n\geq m>1$ zachodzi\par
    $$n=\sum\limits_{i=0}^{m-1}\Big\lfloor{n+i\over m}\Big\rfloor$$
    
    Zauwazmy, ze jest to ilosc elementow w kazdej grupie przy podziale $n$ elementow na $m$ grup. We wszystkich kolumnach umiescimy co najmniej $\Big\lfloor \frac nm\Big\rfloor$ obiektow, ale w ostatnich $n\mod m$ kolumnach bedzie ich o 1 wiecej, co jest uzyskiwane przez zwiekszanie o 1 licznika po kazdej kolumnie.
    \medskip

    Wracajac do $(\nietoperz)$, mozemy powiedziec, ze
    $$\sum\limits_{i=0}^{m-1}\Big\lfloor{x+i\over m}\Big\rfloor=\sum\limits_{i=0}^{m-1}\Big\lfloor{\lfloor x\rfloor+i\over m}\Big\rfloor=\lfloor x\rfloor$$

    \section*{ZAD 3.}

    a) potrzebujemy $a_0$, $a_1$, natomiast $a_2$ mozemy juz obliczyc za pomoca $a_0$
    \medskip

    b) potrzebne jest $a_0$, $a_1$ oraz $a_2$, bo wyraz $a_3$ to juz suma wyrazow poprzednich
    \medskip

    c) potrzebny jest tylko wyraz $a_0$ - jest on potrzebny dla $a_1$, dla $a_2$ potrzebne jest $a_1$ i tak dalej - zawsze przy odpowiedniej ilosci podzielen na 2 otrzymujemy $a_0$


    \section*{ZAD 4.}

    a) $f_n=f_{n-1}+3^n$ dla $n>1$ i $f_1=3$.
    \medskip

    To jest suma geometric sequence:
    $$f_n=\sum\limits_{i=1}^n3^i=3\cdot{3^n-1\over 3-1}={3^{n+1}-3\over 2}$$

    b) $h_n=h_{n-1}+(-1)^{n+1}n$ dla $n>1$ i $h_1=1$

    $$h_n=-\lfloor\frac n2\rfloor+(n-2\lfloor\frac n2\rfloor)n$$

    c) $l_n=l_{n-1}l_{n-2}$ dla $n>2$ i $l_1=l_2=2$

    $$l_3=4=2^2$$
    $$l_4=8=2^3$$
    $$l_5=32=2^5$$
    $$l_6=256=2^8$$

    $l_n$ wyraz to $2^k$, gdzie $k$ to $n$-ty wyraz ciagu fibonacciego. Takze zostaje mi nic innego jak znalezc jawny wzor na ciag fibonacciego, $f_n$ :)
    \medskip

    Lecimy funkcja tworzaca, cuz why not. Niech
    $$F(x)=\sum\limits_{i=1}^nf_nx^n$$
    wtedy
    \begin{align*}
        F(x)&=\sum\limits_{i=0}f_ix^i\\
        F(x)&=x+\sum\limits_{i=2}(f_{i-1}+f_{i-2})x^i\\
        F(x)&=x+x\sum\limits_{i=2}f_{i-1}x^{i-1}+x^2\sum\limits_{i=2}f_{i-2}x^{i-2}\\
        F(x)&=x+x\sum\limits_{i=1}f_ix^i+x^2\sum\limits_{i=0}f_ix^i
    \end{align*}
    Zauwazmy, ze poniewaz $f_0=0$, to $\sum\limits_{i=0}f_ix^i=\sum\limits_{i=1}f_ix^i$

    \begin{align*}
        F(x)&=x+(x+x^2)\sum\limits_{i=0}f_ix^i\\
        F(x)&=x+(x+x^2)F(x)\\
        F(x)(1-x-x^2)&=x\\
        F(x)&={x\over 1-x-x^2}
    \end{align*}
    Zauwazamy, ze 
    $$1-x-x^2=0\iff x={-1\pm\sqrt{1+4}\over 2}$$
    Czyli mamy
    $$F(x)={x\over (x+{1+\sqrt5\over 2})(x+{1-\sqrt5\over 2})}$$

    Rozbicie tego na dwa dodawane ulamki zostawiam czytelnikowi oraz wikipedii. Tak samo jak dokonczenie tego rozwiazania.
    \medskip

    Zalozmy, ze czytelnik byl mniej leniwy niz autorka i wyliczyl jawny wzor na $n$-ty wyraz ciagu fibonaciego, ktory wg wikipedii wyglada mniej wiecej tak:
    $$f_n=\frac1{\sqrt5}{(1+\sqrt5)^n-(1-\sqrt5)^n\over 2^n}$$

    Otrzymujemy wiec:
    $$l_n=2^{f_n}=2^{f_{n-1}}\cdot 2^{f_{n-2}}$$


    \section*{ZAD 5.}

    a) $a_n=\frac 2{a_{n-1}}$ dla $a_0=1$

    $$a_n=\begin{cases}
        1\quad n|2\\
        2
    \end{cases}$$

    Dla $n=0, 1$ - dziala. Zalozmy, ze dziala tez dla wszystkich wyrazow mniejszych niz $n$. Rozwazamy dwa przypadki:\smallskip\\
    \indent $2|n$, wtedy $2\nmid n-1$ i mamy $a_{n-1}=2$
    $$a_n=\frac22=1$$
    czyli tak jak jest we wzorze.\smallskip\\
    \indent $2\nmid n$, wtedy $a_{n-1}=1$ i
    $$a_n=\frac21=2.$$

    \medskip

    b) $b_n={1\over 1+b_{n-1}}$ $b_0=0$
    \medskip

    Oznaczmy jako $f_n$ $n$-ty wyraz ciagu fibbonacciego, czyli $f_n=f_{n-1}+f_{n-2}$, gdzie $f_0=0,\;f_1=1$. Wtedy 
    $$b_n={f_n\over f_{n+1}}.$$

    Dla $n=0, 1$ mamy $b_0={0\over1}=0$ oraz $b_1={1\over1}=1$. Zalozmy, ze dla wszystkich wyrazow do $b_n$ wzor dziala. Wtedy
    \begin{align*}
        b_{n+1}&={1\over 1+b_{n}}={1\over 1+{f_{n}\over f_{n+1}}}=\\
        &={1\over {f_n+f_{n+1}\over f_{n+1}}}{f_{n+1}\over f_n+f_{n+1}}={f_{n+1}\over f_{n+2}}
    \end{align*}

    \medskip

    c) $c_n=\sum\limits_{i=0}^{n-1}c_i$ $c_0=1$

    Dla $n\neq 0$ zachodzi
    $$c_n=2^{n-1}$$
    natomiast dla $n=0$ mamy $c_0=1$.
    \medskip

    Dla $n=1,2$ jest $c_1=2^0=1,\;c_2=2^2=2$. Zalozmy, ze dla kazdego $n$ wzor jest prawdziwy, wtedy
    \begin{align*}
        c_{n+1}&=\sum\limits_{i=0}^nc_i=\sum\limits_{i=0}^{n-1}c_i+c_n=\\
        &=c_n+c_n=2\cdot 2^{n-1}=2^n
    \end{align*}

    \medskip

    d) $d_n={d_{n-1}^2\over d_{n-2}}$ $d_0=1$ $d_1=2$
    \medskip

    Dla $n=0,1$ mamy $d_0=2^0=1,\;d_1=2^1=2$. Zalozmy, ze dla wszystkich $n$ wzor jest prawdziwy, wowczas
    \begin{align*}
        d_{n+1}={d_n^2\over d_{n-1}}={(2^n)^2\over 2^{n-1}}=2^{2n-(n-1)}=2^{n+1}
    \end{align*}


    \section*{ZAD 6.}

    a) $y_0=y_1=1$, $y_n={y_{n-1}^2+y_{n-2}\over y_{n-1}+y_{n-2}}$

    $$y_n=1$$
    Dla $n=2$ $y_2=1$. Zalozmy, ze dla pierwszych $n$ wyrazow wzor jest prawdziwy, wowczas
    \begin{align*}
        y_{n+1}={y_{n-1}^2+y_{n-2}\over y_{n-1}+y_{n-2}}={1^2+1\over 1+1}=\frac 22=1
    \end{align*}

    \medskip

    b) $z_0=1$, $z_1=2$, $z_n={z_{n-1}^2-1\over z_{n-2}}$

    $$z_n=n+1$$
    Dla $z_2=3$, wiec zalozmy, ze dla wszystkich $n$ wzor zachodzi, wowczas
    \begin{align*}
        z_{n+1}={z_n^2-1\over z_{n-2}}={(n+1)^2-1\over n}={n^2+2n\over n}=n+2=(n+1)+1
    \end{align*}

    \medskip

    c) $t_0=0$, $t_1=1$, $t_n={(t_{n-1}-t_{n-2}+3)^2\over 4}$
    
    $$a_n=n^2$$
    Dla $n=2$ smiga, zalozmy, ze smiga tez dla wszystkich $n$. Wowczas
    \begin{align*}
        t_{n+1}&={(t_n-t_{n-1}+3)^2\over 4}={(n^2-(n-1)^2+3)^2\over 4}=\\
        &={(n^2-n^2+2n-1+3)^2\over 4}={(2n+2)^2\over 4}=\\
        &=\Big({2n+2\over 2}\Big)^2=(n+1)^2
    \end{align*}

    \section*{ZAD 7.}
    
    a) $a_{n+1}=(n+1)a_n+1$ dla $a_0=1$

    \begin{align*}
        a_n=n!+\sum\limits_{i=1}^n{n!\over i!}=\sum\limits_{i=0}^n{n!\over i!}
    \end{align*}
    
    Dla $n=1,2$ dziala. Zalozmy, ze dziala tez dla wszystkich $n$, wtedy
    \begin{align*}
        a_{n+1}=(n+1)a_n+1=(n+1)\sum\limits_{i=0}^n{n!\over i!}+{(n+1)!\over(n+1)!}=\sum\limits_{i=0}^n{(n+1)!\over i!}+{(n+1)!\over(n+1)!}=\sum\limits_{i=0}^{n+1}{(n+1)!\over i!}
    \end{align*}

    \medskip
    
    b) $b_0=\frac12$, $nb_n=(n-2)b_{n-1}+1$ czyli dla $n>0$ mamy $b_n={(n-2)b_{n-1}+1\over n}$

    $$b_n=\frac12$$

    Dla $n=1,2$ dziala, zalozmy ze dziala tez dla wszystkich $n$, wtedy
    \begin{align*}
        b_{n+1}={(n-1)b_{n}+1\over n+1}={\frac12(n-1)+1\over n+1}={n-1+2\over 2(n+1)}={n+1\over 2(n+1)}=\frac12
    \end{align*}

    \medskip

    c) $c_0=0$, $nc_n=(n+2)c_{n-1}+n+2$, dla $n>0$ $c_n={(n+2)c_{n-1}+n+2\over n}$

    $$c_n=\sum\limits_{i=1}^n(2i+1)=\sum\limits_{i=1}^n2i+n=2\sum\limits_{i=1}^ni+n=n(n+1)+n$$

    Dla $n=1,2$ dziala, zalozmy, ze jest zgodny dla wszystkich $n$, wtedy

    \begin{align*}
        c_{n+1}&={(n+3)c_n+n+3\over n+1}={(n+3)(n(n+1)+n)+n+3\over n+1}=\\=
        &={(n+3)(n(n+1)+n+1)\over n+1}={(n+3)(n+1)(n+1)\over n+1}=\\
        &=(n+3)(n+1)=(n+1)(n+2)+(n+1)
    \end{align*}

    \medskip

    d) $d_0=1$, $d_1=2$, $nd_n=(n-2)!d_{n-1}d_{n-2}$, czyli d;a $n>0$ $d_n={(n-2)!d_{n-1}d_{n-2}\over n}$

    \begin{align*}
        d_0=1\\
        d_1=2\\
        d_2=1\\
        d_3=\frac23\\
        d_4=\frac13\\
        d_5=\frac4{15}
    \end{align*}

\end{document}