\documentclass{article}[13pt]

\usepackage{../../uni-notes-eng}
\usepackage{multicol}

\begin{document}

    \section*{ZAD 1.}
    $$\lfloor an\rfloor+\lfloor(1-a)n\rfloor=n-1$$

    Niech $\Z\ni m=\lfloor an\rfloor$, wtedy
    \begin{align*}
        m\leq an &<m+1\\
        m-n\leq an-n&<m-n+1\\
        n-m\geq n-an&> n-m-1
    \end{align*}

    Poniewaz $n\notin \Q$, to $n-an\notin \Z$, wiec
    $$\lfloor n-an\rfloor=n-\lfloor an\rfloor -1$$
    a z tego
    $$\lfloor an\rfloor+\lfloor n(1-a)\rfloor = n-1$$

    
    $$\lceil an\rceil + \lceil n-an\rceil=n+1$$

    \section*{ZAD 2.}
    $$\lfloor {x\over m}\rfloor + \lfloor {x+1\over m}\rfloor+...+\lfloor {x+m-1\over m}\rfloor=\sum\limits_{i=0}^{m-1}\lfloor {x+i\over m}\rfloor\quad(\nietoperz)$$

    Po pierwsze pokazemy, ze dla dowolnych $n,m\in\Z$ oraz $x\in\R$ zachodzi
    $$\Big\lfloor {x+n\over m}\Big\rfloor=\Big\lfloor{\lfloor x\rfloor+n\over m}\Big\rfloor\quad (\kawa)$$
    
    Niech $p=\Big\lfloor{\lfloor x\rfloor+n\over m}\Big\rfloor$, wtedy
    \begin{align*}
        p&\leq {\lfloor x\rfloor+n\over m} < p+1\\
        p&\leq {\lfloor x\rfloor+n\over m}<p+1\\
        \Z\ni m\cdot p-n&\leq \lfloor x\rfloor<m\cdot(p+1)-n\in\Z\\
        m\cdot p-n&\leq x<m\cdot(p+1)-n\\
        p&\leq {x-n\over m}<p+1\\
        p&\leq \Big\lfloor {x+n\over m}\Big\rfloor<p+1\\
        p&=\Big\lfloor {x+n\over m}\Big\rfloor
    \end{align*}
    Czyli pokazalismy $(\kawa)$.
    \bigskip

    Po drugie, zauwazmy, ze dla dowolnego $n$ i dla kazdego $m\in\Z$, $n\geq m>1$ zachodzi\par
    $$n=\sum\limits_{i=0}^{m-1}\Big\lfloor{n+i\over m}\Big\rfloor$$
    
    Zauwazmy, ze jest to ilosc elementow w kazdej grupie przy podziale $n$ elementow na $m$ grup. We wszystkich kolumnach umiescimy co najmniej $\Big\lfloor \frac nm\Big\rfloor$ obiektow, ale w ostatnich $n\mod m$ kolumnach bedzie ich o 1 wiecej, co jest uzyskiwane przez zwiekszanie o 1 licznika po kazdej kolumnie.
    \medskip

    Wracajac do $(\nietoperz)$, mozemy powiedziec, ze
    $$\sum\limits_{i=0}^{m-1}\Big\lfloor{x+i\over m}\Big\rfloor=\sum\limits_{i=0}^{m-1}\Big\lfloor{\lfloor x\rfloor+i\over m}\Big\rfloor=\lfloor x\rfloor$$

    \section*{ZAD 3.}

    a) potrzebujemy $a_0$, $a_1$, natomiast $a_2$ mozemy juz obliczyc za pomoca $a_0$
    \medskip

    b) potrzebne jest $a_0$, $a_1$ oraz $a_2$, bo wyraz $a_3$ to juz suma wyrazow poprzednich
    \medskip

    c) potrzebny jest tylko wyraz $a_0$ - jest on potrzebny dla $a_1$, dla $a_2$ potrzebne jest $a_1$ i tak dalej - zawsze przy odpowiedniej ilosci podzielen na 2 otrzymujemy $a_0$


    \section*{ZAD 4.}

    a) $f_n=f_{n-1}+3^n$ dla $n>1$ i $f_1=3$.
    \medskip

    To jest suma geometric sequence:
    $$f_n=\sum\limits_{i=1}^n3^i=3\cdot{3^n-1\over 3-1}={3^{n+1}-3\over 2}$$

    b) $h_n=h_{n-1}+(-1)^{n+1}n$ dla $n>1$ i $h_1=1$

    $$h_n=-\lfloor\frac n2\rfloor+(n-2\lfloor\frac n2\rfloor)n$$

    c) $l_n=l_{n-1}l_{n-2}$ dla $n>2$ i $l_1=l_2=2$

    $$l_3=4=2^2$$
    $$l_4=8=2^3$$
    $$l_5=32=2^5$$
    $$l_6=256=2^8$$

    $l_n$ wyraz to $2^k$, gdzie $k$ to $n$-ty wyraz ciagu fibonacciego. Takze zostaje mi nic innego jak znalezc jawny wzor na ciag fibonacciego, $f_n$ :)
    \medskip

    Lecimy funkcja tworzaca, cuz why not. Niech
    $$F(x)=\sum\limits_{i=1}^nf_nx^n$$
    wtedy
    \begin{align*}
        F(x)&=\sum\limits_{i=0}f_ix^i\\
        F(x)&=x+\sum\limits_{i=2}(f_{i-1}+f_{i-2})x^i\\
        F(x)&=x+x\sum\limits_{i=2}f_{i-1}x^{i-1}+x^2\sum\limits_{i=2}f_{i-2}x^{i-2}\\
        F(x)&=x+x\sum\limits_{i=1}f_ix^i+x^2\sum\limits_{i=0}f_ix^i
    \end{align*}
    Zauwazmy, ze poniewaz $f_0=0$, to $\sum\limits_{i=0}f_ix^i=\sum\limits_{i=1}f_ix^i$

    \begin{align*}
        F(x)&=x+(x+x^2)\sum\limits_{i=0}f_ix^i\\
        F(x)&=x+(x+x^2)F(x)\\
        F(x)(1-x-x^2)&=x\\
        F(x)&={x\over 1-x-x^2}
    \end{align*}
    Zauwazamy, ze 
    $$1-x-x^2=0\iff x={-1\pm\sqrt{1+4}\over 2}$$
    Czyli mamy
    $$F(x)={x\over (x+{1+\sqrt5\over 2})(x+{1-\sqrt5\over 2})}$$

    Rozbicie tego na dwa dodawane ulamki zostawiam czytelnikowi oraz wikipedii. Tak samo jak dokonczenie tego rozwiazania.
    \medskip

    Zalozmy, ze czytelnik byl mniej leniwy niz autorka i wyliczyl jawny wzor na $n$-ty wyraz ciagu fibonaciego, ktory wg wikipedii wyglada mniej wiecej tak:
    $$f_n=\frac1{\sqrt5}{(1+\sqrt5)^n-(1-\sqrt5)^n\over 2^n}$$

    Otrzymujemy wiec:
    $$l_n=2^{f_n}=2^{f_{n-1}}\cdot 2^{f_{n-2}}$$


    \section*{ZAD 5.}

    a) $a_n=\frac 2{a_{n-1}}$ dla $a_0=1$

    \begin{align*}
        a_0=1\\
        a_1=2\\
        a_2=1\\
        a_3=2\\
        a_4=1\\
        a_5=2\\
        ...
    \end{align*}

    Kurwa no nie bede robic indukcji. Dla parzystych indeksow jest 1, dla nieparzystych - 2.

\end{document}