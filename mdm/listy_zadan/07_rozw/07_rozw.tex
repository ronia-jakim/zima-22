\documentclass{article}[13pt]

\usepackage{../../../uni-notes-eng}
\usepackage{multicol}
\definecolor{back2}{HTML}{ffffff}
\definecolor{text2}{HTML}{000000}
\usepackage{wrapfig}

\author{Weronika Jakimowicz}
\title{MDM Lista 7}
\date{}

%\pagecolor{back2}
%\color{text2}

\begin{document}
\maketitle

\textbf{\large\color{def}ZMIANY}:\smallskip

\point Zadanie 1. - suma

\point Zadanie 2. - krótkie uzasadnienie interpretacji $F_n$

\point Zadanie 14.b. - krótkie uzasadnienie

\section*{ZAD. 1.}

Niech $k$ będzie liczbą pionków do rozłożenia. Jeśli $k>n$, to wtedy co najmniej dwa muszą być w jednej kolumnie, a więc jeden nie będzie na lewo od drugiego. W takim razie musi być $k\leq n$.
\medskip

Ułóżmy najpierw $k$ pionków na planszy $k\times k$ tak, żeby w każdej parze jeden był na lewo i niżej niż drugi. Takie ułożenie jest jedno, to zaczy pionki muszą stać na przekątnej od lewego dolnego rogu do prawego górnego.

Jeśli ustawimy najpierw $k$ pionków na planszy $k\times k$. Utożsamimy kolumny zawierające pionki z liczbą $1$, natomiast kolumny puste z liczbą $0$. Wtedy sposobów żeby ustawić $n-k$ jedynek w ciąg $n$ elementowy mamy ${n\choose n-k}$. Analogiczna sytuacja zachodzi dla wierszy, a ogólna ilość rozwiązań to 
$${n\choose n-k}^2,$$ 
gdyż łączymy każde ustawienie kolumn z każdym ustawieniem wierszy.
\medskip

Sumując teraz po wszystkich możliwych wartościach $k$ dostajemy
\begin{align*}
    \sum\limits_{k=0}^n{n\choose n-k}^2&=\sum\limits_{k=0}^n{n\choose n-k}{n\choose k}=(1+1)^n=2^n
\end{align*}

\section*{ZAD. 2.}

Liczba Fibonaciego $F_n$ odpowiada na pytanie, ile jest ciągów składających się tylko z $1$ i $2$ sumujących się do $(n-1)$. \emph{Wynika to z tego, że ciąg sumujący się do $(n+1)$ możemy uzyskać dodając do dowolnego ciągu sumującego się do $n$ jedynkę na końcu. Możemy też dodać dwójkę do dowolnego ciągu sumującego się do $(n-2)$. Czyli ilość ciągów długości $(n+1)$ to $a_{n+1}=a_n+a_{n-1}$.} Popatrzmy teraz na sumę z zadania:
$$F_n=\sum\limits_{i=0}^{n-1}{n-1-i\choose i}.$$
Pierwszy wyraz, ${n-1\choose 0}$ to liczba ciągów składających się tylko z $1$ sumujących się do $(n-1)$. Drugi wyraz, ${n-2\choose 1}$ skraca ciąg $(n-1)$ jedynek o jeden i wybiera jedną z pozostałych $(n-2)$ jedynek która zostanie zamieniona na $2$. W ten sposób dostajemy ilość ciągów sumujących się do $(n-1)$ zawierających tylko jedną liczbę $2$. Tak więc dla $k$-tego wyrazu sumy usuwamy $k$ jedynek, a z pozostałych na $k$ sposobów wybieramy te, które zostaną zastąpione przez $2$ ${n-1-k\choose k}$.
\bigskip

\podz{sep}
\bigskip

Teza:
$$F_{m+2n}=\sum\limits{i=0}^n{n\choose i}F_{i+m}$$

Niech $x_n=\begin{pmatrix}
    F_{m+1+n}\\F_{n+m}
\end{pmatrix}$ oraz $A=\begin{pmatrix}
    1&1\\1&0
\end{pmatrix}$. Zauważmy, że
$$A^2=\begin{pmatrix}
    2&1\\1&1
\end{pmatrix}=A+I$$
czyli
$$A^{2n}=(A^2)^n=(A+I)^n=\sum\limits_{i=0}^n{n\choose i}A^i$$
Mnożąc obie strony przez $x_0=\begin{pmatrix}
    F_{m+1}\\F_m
\end{pmatrix}$ otrzymujemy
$$\begin{pmatrix}F_{m+2n+1}\\F_{m+2n}\end{pmatrix}=x_{2n}=A^{2n}x_0=\sum\limits_{i=0}^n{n\choose i}A^ix_0=\sum\limits_{i=0}^n{n\choose i}\begin{pmatrix}
    F_{m+i+1}\\F_{m+i}
\end{pmatrix}$$
i przyrównując drugie współrzędne otrzymujemy:
$$F_{m+2}=\sum\limits_{i=0}^n{n\choose i}F_{m+i}$$


\section*{ZAD. 3.}

Sposobów na ułożenie $2n$ skarpet (czyli $n$ par) jest $(2n)!$. Jednak zazwyczaj skarpety z jednej pary są nierozróżnialne, więc nie ma znaczenia które będzie pierwsza. W każdej parze mamy $2$ sposoby na wybranie która skarpeta jest pierwsza, mamy $n$ par więc ogółem tych sposobów jest $2^n$. Czyli ogółem sposobów na ułożenie $n$ par skarpet jest
$${(2n)!\over 2^n}.$$

Zastanówmy się teraz, ile jest sposobów na ułożenie $n$ par skarpet tak, żeby określone $k$ par było obok siebie. Zwijając $k$ par skarpet razem zmniejszamy liczbę elementów o $k$, czyli teraz mamy $(2n-k)$ rozróżnialnych skarpet. Rozłożyć niezwinięte $(2n-2k)$ skarpet tak, żeby skarpety z jednej pary nie były obok siebie można na ${(2n-2k)!\over 2^{n-2k}}$ sposobów. Mamy teraz ciąg $(2n-2k)$ ustawionych skarpet w który chcemy włożyć $k$ dodatkowych elementów. Całość będzie się sumować do $(2n-k)$, więc z $(2n-k)$ możemy wybrać które $k$ miejsc wybierzemy na ${2n-k\choose k}$ sposobów. Czyli $k$ par skarpet zmuszamy do bycia razem podczas gdy pozostałe są rozdzielone na 
$${(2n-2k)!\over 2^{n-2k}}{2n-k\choose k}$$
sposobów.

Teraz, z zasady włączeń i wyłączeń, dostajemy szukaną odpowiedź w postaci:
$$A_{n}={(2n)!\over 2^n}-\sum\limits_{i=1}^n(-1)^{k+1}{n\choose k}{(2n-2k)!\over 2^{n-2k}}{2n-k\choose k}=\sum\limits_{k=0}^n(-1)^k{n\choose k}{2n-k\choose k}{(2n-2k)!\over 2^{n-k}}$$


\subsection*{ZAD. 4.}

$$\begin{cases}
    a_0=1\\
    a_1=0\\
    a_n=\frac12(a_{n-1}+a_{n-2})
\end{cases}$$

Rozważmy ciąg geometryczny: $x_n=q^n$, wtedy
\begin{align*}
    q^n&=\frac12(q^{n-1}+q^{n-2})\\
    q^2&=\frac12(q+1)
\end{align*}
Czyli $x_n=q^n$ dla $q$ będących zerami wielomianu
$$w(x)=x^2-\frac12x-\frac12=(x-\frac14)^2-\frac9{16}$$
$$x-\frac14=\pm \frac34$$
$$x=1\;\lor\;x=-{1\over 2}.$$

Czyli ciąg $x_n$ rozwiązuje
$$x_n=c_11^n+c_2\Big(-{1\over 2}\Big)^n$$
Dla dwóch pierwszych wyrazów daje to
\begin{align*}
    \begin{cases}
        x_0=1=c_1+c_2\\
        x_1=0=c_1-{c_2\over2}
    \end{cases}\\
\end{align*}
czyli $c_2=\frac23$ oraz $c_1=\frac13$ a postać jawna ciągu to
$$x_n=\frac13+\frac23(-2)^{-n}$$


% \subsection*{ZAD. 5.}

% \href{https://math.stackexchange.com/questions/1258484/getting-rid-of-exponents-with-n-when-solving-with-annihilators-a-n-a-n-12a}{jakis losowy stackexchange}

% {\color{acc}(a)} $a_{n+2}=2a_{n+1}-a_n+3^n-1$
% \smallskip



% {\color{acc}(c)} $a_{n+2}=2^{n+1}-a_{n+1}-a_n$
% \smallskip

% Wprowadźmy nowy ciąg, $b_n$ taki, że
% $$b_n={a_n\over 2^{n-1}}$$
% wtedy
% \begin{align*}
%     b_{n+2}2^{n+1}&=2^{n+1}-b_{n+1}2^n-b_n2^{n-1}\\
%     b_{n+2}&=1-b_{n+1}2^{-1}-b_n2^{-2}
% \end{align*}
% Kolejny ciąg: $c_n=b_n-\frac47$, wtedy
% \begin{align*}
%     b_{n+2}-\frac47&=-\frac12(b_{n+1}-\frac47)-\frac14(b_n-\frac47)\\
%     c_{n+2}&=-\frac12c_{n+1}-\frac12c_n\\
%     w(x)&=x^2+\frac12x+\frac12\\
%     x&=-\frac14\pm{i\sqrt{14}\over8}
% \end{align*}

% $$c_n=c_1\Big(-\frac14+{i\sqrt{14}\over8}\Big)^n+c_2\Big(-\frac14-{i\sqrt{14}\over8}\Big)^n$$
% $$\begin{cases}
%     c_0=\frac{10}7=c_1+c_2\\
%     c_1=\frac37=c_1\Big(-\frac14+{i\sqrt{14}\over8}\Big)+c_2\Big(-\frac14-{i\sqrt{14}\over8}\Big)
% \end{cases}$$
% Co po żmudnych wyliczeniach daje $c_2=i\sqrt{14}-10$ oraz $c_1=i\sqrt{14}-9$. 

\subsection*{ZAD. 14.}

{\color{acc}(a)} $a_n=n^2$
\begin{align*}
    {1\over 1-x}&=\sum\limits_{n=0}^\infty x^n\\
    {1\over(1-x)^2}&=\sum\limits_{n=0}^\infty nx^{n-1}\\
    {x\over(1-x)^2}&=\sum\limits_{n=0}^\infty nx^n\\
    {x+1\over(1-x)^3}&=\sum\limits_{n=0}^\infty n^2x^{n-1}\\
    {x(x+1)\over(1-x)^3}&=\sum\limits_{n=0}^\infty n^2x^n
\end{align*}

{\color{acc}(b)} $a_n=n^3$
Z poprzedniego podpunktu mamy
$${x(x+1)\over(1-x)^3}=\sum\limits_{n=0}^\infty n^2x^n$$
jeśli więc obie strony zróżniczkujemy, to dostaniemy
\begin{align*}
    {x^2+4x+1\over(1-x)^4}&=\sum\limits_{n=0}^\infty n^3x^{n-1}\\
    {x^3+4x^2+x\over (1-x)^4}&=\sum\limits_{n=0}^\infty n^3x^n
\end{align*}

{\color{acc}(c)} $a_n={n+k\choose k}$, dla ułatwienia zmieniam ten zapis na $a(n,k)={n+k\choose k}$
\medskip

Dla $k=0$ mamy $a(n,0)=1$ oraz
$$f_0(x)=\sum\limits_{n=0}^\infty a_nx^n=\sum\limits_{n=0}^\infty x^n={1\over 1-x}$$

Teza: dla dowolnego $k$ mamy
$$f_k(x)={1\over(1-x)^{k+1}}$$

Załóżmy, że dla pierwszych $k$ działa. Teraz sprawdźmy jak to wygląda dla $k+1$. 
$$a(n, k+1)={n+k+1\choose k+1}={(n+k+1)!\over n!(k+1)!}={(n+k+1)(n+k)...(n+1)\over (k+1)!}$$

Dalej mamy
\begin{align*}
    f_{k+1}(x)&=\sum\limits_{n=0}^\infty a_nx^n=\sum\limits_{n=0}^\infty{(n+k+1)(n+k)...(n+1)\over (k+1)!}x^n={1\over(k+1)}\sum\limits_{n=0}^\infty(n+k+1)a(n, k)x^n=\\
    &=\frac1{k+1}\Big[\sum\limits_{n=0}^\infty(n+k)a(n, k)x^n+\sum\limits_{n=0}^\infty a(n, k)x^n\Big]=\\
    &={1\over k+1}\Big[{k(1-x)+(k+1)x\over(1-x)^{k+2}}+{1\over(1-x)^{k+1}}\Big]=\\
    &={1\over k+1}\Big[{k(1-x)+(k+1)x+(1-x)\over (1-x)^{k+2}}\Big]=\frac1{k+1}{(k+1)(1-x+x)\over(1-x)^{k+2}}={1\over(1-x)^{k+2}}
\end{align*}

\subsection*{ZAD 15.}

{\color{acc}(a)}

\begin{align*}
    f(x)&=\sum\limits_{n=0}^\infty a_nx^n=x+2x^2+\frac13x^3+4x^4+\frac15x^5+...=\\
    &=\sum\limits_{n=0}^\infty 2nx^{2n}+\sum\limits_{n=0}^\infty{x^{2n+1}\over 2n+1}
\end{align*}

Niech $g(x)=\sum\limits_{n=0}^\infty x^{2n}$

\begin{align*}
    g(x)&={1\over 1-x^2}\\
    g'(x)&=\sum\limits_{n=0}^\infty 2nx^{2n-1}={2x\over(1-x^2)^2}\\
    \sum\limits_{n=0}^\infty 2nx^{2n}&={2x^2\over(1-x^2)^2}
\end{align*}

Dalej, niech $h(x)=\sum\limits_{n=0}^\infty {x^{2n+1}\over 2n+1}$. Wtedy
\begin{align*}
    h'(x)&=\sum\limits_{n=0}^\infty x^{2n}={1\over 1-x^2}\\
    h(x)&=\int{dx\over 1-x^2}=\frac12\log\Big[{|x+1|\over|x-1|}\Big]
\end{align*}

Podstawiając te funkcje za odpowiednie składniki $f(x)$ dostajemy
$$f(x)={2x^2\over(1-x^2)^2}+\frac12\log\Big[{|x+1|\over|x-1|}\Big]$$

{\color{acc}(b)}
\smallskip

Po pierwsze, zauważmy, że
$$H_n=H_{n-1}+\frac1n.$$
\begin{align*}
    f(x)&=\sum\limits_{n=0}^\infty H_nx^n=\sum\limits_{n=1}^\infty H_nx^n=\sum\limits_{n=1}^\infty [H_{n-1}+\frac 1n]x^n=\\
    &=x\sum\limits_{n=1}^\infty H_{n-1}x^{n-1}+\sum\limits_{n=1}^\infty \frac1nx^n 
\end{align*}

Rozważmy funckję $h(x)=\sum\limits_{n=1}^\infty\frac1nx^n$. Wtedy
\begin{align*}
    h(x)&=\sum\limits_{n=1}^\infty \frac1nx^n\\
    h'(x)&=\sum\limits_{n=1}^\infty x^{n-1}=\sum\limits_{n=0}^\infty x^n={1\over 1-x}\\
    h(x)&=\int{dx\over 1-x}=-\log|1-x|
\end{align*}

co daje
\begin{align*}
    f(x)&=xf(x)-\log|1-x|\\
    f(x)&={\log|1-x|\over x-1}
\end{align*}

\end{document}