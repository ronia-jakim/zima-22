\documentclass{article}[13pt]

\usepackage{../../../uni-notes-eng}
\usepackage{multicol}
\definecolor{back2}{HTML}{ffffff}
\definecolor{text2}{HTML}{000000}
\usepackage{wrapfig}

\author{Weronika Jakimowicz}
\title{MDM Lista 7}
\date{}

%\pagecolor{back2}
%\color{text2}

\begin{document}
\maketitle

\section*{ZAD. 1.}

Niech $k$ będzie liczbą pionków do rozłożenia. Jeśli $k>n$, to wtedy co najmniej dwa muszą być w jednej kolumnie, a więc jeden nie będzie na lewo od drugiego. W takim razie musi być $k\leq n$.
\medskip

Ułóżmy najpierw $k$ pionków na planszy $k\times k$ tak, żeby w każdej parze jeden był na lewo i niżej niż drugi. Takie ułożenie jest jedno, to zaczy pionki muszą stać na przekątnej od lewego dolnego rogu do prawego górnego.

Jeśli ustawimy najpierw $k$ pionków na planszy $k\times k$. Utożsamimy kolumny zawierające pionki z liczbą $1$, natomiast kolumny puste z liczbą $0$. Wtedy sposobów żeby ustawić $n-k$ jedynek w ciąg $n$ elementowy mamy ${n\choose n-k}$. Analogiczna sytuacja zachodzi dla wierszy, a ogólna ilość rozwiązań to 
$${n\choose n-k}^2,$$ 
gdyż łączymy każde ustawienie kolumn z każdym ustawieniem wierszy.

\section*{ZAD. 2.}

Liczba Fibonaciego $F_n$ odpowiada na pytanie, ile jest ciągów składających się tylko z $1$ i $2$ sumujących się do $(n-1)$. Popatrzmy teraz na sumę z zadania:
$$F_n=\sum\limits_{i=0}^{n-1}{n-1-i\choose i}.$$
Pierwszy wyraz, ${n-1\choose 0}$ to liczba ciągów składających się tylko z $1$ sumujących się do $(n-1)$. Drugi wyraz, ${n-2\choose 1}$ skraca ciąg $(n-1)$ jedynek o jeden i wybiera jedną z pozostałych $(n-2)$ jedynek która zostanie zamieniona na $2$. W ten sposób dostajemy ilość ciągów sumujących się do $(n-1)$ zawierających tylko jedną liczbę $2$. Tak więc dla $k$-tego wyrazu sumy usuwamy $k$ jedynek, a z pozostałych na $k$ sposobów wybieramy te, które zostaną zastąpione przez $2$ ${n-1-k\choose k}$.
\bigskip

\podz{sep}
\bigskip

Teza:
$$F_{m+2n}=\sum\limits{i=0}^n{n\choose i}F_{i+m}$$

Niech $x_n=\begin{pmatrix}
    F_{m+1+n}\\F_{n+m}
\end{pmatrix}$ oraz $A=\begin{pmatrix}
    1&1\\1&0
\end{pmatrix}$. Zauważmy, że
$$A^2=\begin{pmatrix}
    2&1\\1&1
\end{pmatrix}=A+I$$
czyli
$$A^{2n}=(A^2)^n=(A+I)^n=\sum\limits_{i=0}^n{n\choose i}A^i$$
Mnożąc obie strony przez $x_0=\begin{pmatrix}
    F_{m+1}\\F_m
\end{pmatrix}$ otrzymujemy
$$\begin{pmatrix}F_{m+2n+1}\\F_{m+2n}\end{pmatrix}=x_{2n}=A^{2n}x_0=\sum\limits_{i=0}^n{n\choose i}A^ix_0=\sum\limits_{i=0}^n{n\choose i}\begin{pmatrix}
    F_{m+i+1}\\F_{m+i}
\end{pmatrix}$$
i przyrównując drugie współrzędne otrzymujemy:
$$F_{m+2}=\sum\limits_{i=0}^n{n\choose i}F_{m+i}$$


\end{document}