\documentclass{article}[13pt]

\usepackage{../../../uni-notes-eng}
\usepackage{multicol}
\definecolor{back2}{HTML}{ffffff}
\definecolor{text2}{HTML}{000000}
\usepackage{wrapfig}
\usepackage{svg}

\author{Weronika Jakimowicz}
\title{MDM Lista 9}
\date{}

\begin{document}
    \maketitle

    \subsection*{ZAD. 1.}

    Najpierw przewozimy kozę, a wilka z kapustą zostawiamy na starym brzegu. Potem wracamy z pustą łódką i zabieramy kapustę. Dobijamy do docelowego brzegu, wysadzamy kapustę i zabieramy kozę. Wracamy na startowy brzeg i zostawiamy kozę, po czym szybko zabieramy wilka zanim on złapie kozę. Odstawiamy wilka na brzeg z kapustą, po czym wracamy po kozę i koniec. 

    \pgraf
        \node (KOZA) at (2, 0) {
            \includesvg[width=20px]{goat}
        };

        \node (WOLF) at (0, 0) {
            \includesvg[width=20px]{wolf}
        };

        \node (CABBAGE) at (4, 0) {
            \includesvg[width=20px]{cabbage}
        };
    \kgraf

    \pgraf
        \node (000) at (0, 0) {(0, 0, 0)};
        \node (100) at (4, 0) {(1, 0, 0)};
        \node (010) at (0, 4) {(0, 1, 0)};
        \node (001) at (2.5, 1.5) {(0, 0, 1)};
        \node (110) at (4, 4) {(1, 1, 0)};
        \node (101) at (6.5, 1.5) {(1, 0, 1)};
        \node (111) at (6.5, 5.5) {(1, 1, 1)};
        \node (011) at (2.5, 5.5) {(0, 1, 1)};

        \draw (000)--(100)--(101)--(111)--(011)--(010)--(000);
        \draw (111)--(110)--(100);
        \draw (110)--(010);
        \draw[color=sep] (000)--(001)--(101);
        \draw[color=sep] (001)--(011);
        \draw[color=def] (000)--(010)--(110)--(100)--(101)--(111);
    \kgraf

    Przedstawiamy ten problem za pomocą grafu, którego wierzchołki pokrywają się z wierzchołkami sześcianu. Będziemy przemieszczać 
    
\end{document}