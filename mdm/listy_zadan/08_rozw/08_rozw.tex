\documentclass{article}[13pt]

\usepackage{../../../uni-notes-eng}
\usepackage{multicol}
\definecolor{back2}{HTML}{ffffff}
\definecolor{text2}{HTML}{000000}
\usepackage{wrapfig}

\author{Weronika Jakimowicz}
\title{MDM Lista 8}
\date{}

%\pagecolor{back2}
%\color{text2}

\begin{document}
\maketitle

\subsection*{ZAD. 2.}

Rozważmy równanie
$$z^k=1.$$
Jednym z jego pierwiastków jest liczba $z=e^{{2i\pi\over k}}$. Zauważmy też, że
\begin{align*}
    1+z+z^2+...+z^{k-1}=0,
\end{align*}
bo
\begin{align*}
    (z-1)(1+z+...+z^{k-1})&=z+z^2+...+z^k-1-z-...-z^{k-1}=z^k-1\\
    {z^k-1\over z-1}&=1+z+...+z^{k-1}
\end{align*}
ale $z^k-1=0$, więc
$$1+z+...+z^{k-1}=0.$$

Dla $k=2$ mamy $z=-1$ i:
\begin{align*}
    A(x)&=a_0+a_1x+a_2x^2+a_3x^3+...\\
    A(-x)&=a_0-a_1x+a_2x^2-a_3x^3+...
\end{align*}
co po wysumowaniu daje
\begin{align*}
    A(x)+A(-x)&=2a_0+a_1x(1-1)+2a_2x^2+a_3x^3(1-1)+...=\sum\limits_{n=0}^\infty 2a_{2n}x^{2n}
\end{align*}
i upraszczając dostajemy:
$$\color{acc}\sum\limits_{n=0}^\infty a_{2n}x^{n}=\frac12(A(\sqrt x)+A(-\sqrt x))$$

\begin{align*}
    A(x)&=a_0+a_1x+a_2x^2a_3x^3+a_4x^4+a_5x^5+a_6x^6+a_7x^7+...\\
    A(zx)&=a_0+a_1zx+a_2z^2x^2+a_3z^3x^3+a_4z^4x^4+a_5z^5x^5+a_6z^6x^6+a_7z^7x^7+...\\
    A(z^2x)&=a_0+a_1z^2x+a_2z^4x^2+a_3z^6x^3+a_4z^8x^4+a_5z^{10}x^5+a_6z^{12}x^6+a_7z^{14}x^7+...
\end{align*}
Zauważmy, że $z^{3k}=1^k=1$, więc przy $a_{3n}$ zawsze mamy tylko $x^{3n}$, natomiast przy pozostałych potęgach, czyli $z^{3k+r}$, gdzie $r=1$ lub $r=2$ mamy
$$z^{3k+r}=z^{3k}z^r=z^r.$$  
Dodając wszystkie te wartości, tak jakd dla przypadku $k=2$, dostajemy
\begin{align*}
    A(x)+A(zx)+A(z^2x)&=3a_0+a_1x(1+z+z^2)+a_2x^2(1+z+z^2)+3a_3x^3+a_4x^4(1+z^1+z^2)+...=\\
    &=3a_0+3a_{3}x^{3}+...=\sum\limits_{n=0}^\infty 3a_{3n}x^{3n}
\end{align*}
a więc aby dostać funkcję tworzącą ciągu $a_{3n}$ wystarczy wziąc 
$$\color{acc}\frac13(A(\sqrt[3]{x})+A(e^{{2i\pi\over3}}\sqrt[3]{x})+A(e^{{4i\pi\over3}}\sqrt[3]{x}))=\sum\limits_{n=0}^\infty a_{3n}x^n.$$

\subsection*{ZAD. 3.}
$$a_n=\sum\limits_{i=0}^\infty F_iF_{n-i}$$

\begin{align*}
    f(x)&=\sum\limits_{n=0}^\infty a_nx^n=\sum\limits_{n=0}^\infty x^n\sum\limits_{i=1}^n F_iF_{n-i}
\end{align*}



\end{document}