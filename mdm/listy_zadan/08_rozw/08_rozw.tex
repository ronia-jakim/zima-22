\documentclass{article}[13pt]

\usepackage{../../../uni-notes-eng}
\usepackage{multicol}
\definecolor{back2}{HTML}{ffffff}
\definecolor{text2}{HTML}{000000}
\usepackage{wrapfig}

\author{Weronika Jakimowicz}
\title{MDM Lista 8}
\date{}

%\pagecolor{back2}
%\color{text2}

\begin{document}
\maketitle

\begin{center}
\begin{tabular}{| c | c | c | c | c | c | c | c | c | c | c | c | c | c | c |}
    \hline

    1 & 2 & 3 & 4 & 5 & 6 & 7 & 8 & 9 & 10 & 11 & 12 & 13 & 14 & 15\\

    \hline


    - & + & - & + & - & + & + & - & - & +  & -  & -  & +  & +  & -\\

    \hline
\end{tabular}
\end{center}

\subsection*{ZAD. 2.}

Rozważmy równanie
$$z^k=1.$$
Jednym z jego pierwiastków jest liczba $z=e^{{2i\pi\over k}}$. Zauważmy też, że
\begin{align*}
    1+z+z^2+...+z^{k-1}=0,
\end{align*} 
bo
\begin{align*}
    (z-1)(1+z+...+z^{k-1})&=z+z^2+...+z^k-1-z-...-z^{k-1}=z^k-1\\
    {z^k-1\over z-1}&=1+z+...+z^{k-1}
\end{align*}
ale $z^k-1=0$, więc
$$1+z+...+z^{k-1}=0.$$

Dla $k=2$ mamy $z=-1$ i:
\begin{align*}
    A(x)&=a_0+a_1x+a_2x^2+a_3x^3+...\\
    A(-x)&=a_0-a_1x+a_2x^2-a_3x^3+...
\end{align*}
co po wysumowaniu daje
\begin{align*}
    A(x)+A(-x)&=2a_0+a_1x(1-1)+2a_2x^2+a_3x^3(1-1)+...=\sum\limits_{n=0}^\infty 2a_{2n}x^{2n}
\end{align*}
i upraszczając dostajemy:
$$\color{acc}\sum\limits_{n=0}^\infty a_{2n}x^{n}=\frac12(A(\sqrt x)+A(-\sqrt x))$$

\begin{align*}
    A(x)&=a_0+a_1x+a_2x^2a_3x^3+a_4x^4+a_5x^5+a_6x^6+a_7x^7+...\\
    A(zx)&=a_0+a_1zx+a_2z^2x^2+a_3z^3x^3+a_4z^4x^4+a_5z^5x^5+a_6z^6x^6+a_7z^7x^7+...\\
    A(z^2x)&=a_0+a_1z^2x+a_2z^4x^2+a_3z^6x^3+a_4z^8x^4+a_5z^{10}x^5+a_6z^{12}x^6+a_7z^{14}x^7+...
\end{align*}
Zauważmy, że $z^{3k}=1^k=1$, więc przy $a_{3n}$ zawsze mamy tylko $x^{3n}$, natomiast przy pozostałych potęgach, czyli $z^{3k+r}$, gdzie $r=1$ lub $r=2$ mamy
$$z^{3k+r}=z^{3k}z^r=z^r.$$  
Dodając wszystkie te wartości, tak jakd dla przypadku $k=2$, dostajemy
\begin{align*}
    A(x)+A(zx)+A(z^2x)&=3a_0+a_1x(1+z+z^2)+a_2x^2(1+z+z^2)+3a_3x^3+a_4x^4(1+z^1+z^2)+...=\\
    &=3a_0+3a_{3}x^{3}+...=\sum\limits_{n=0}^\infty 3a_{3n}x^{3n}
\end{align*}
a więc aby dostać funkcję tworzącą ciągu $a_{3n}$ wystarczy wziąc 
$$\color{acc}\frac13(A(\sqrt[3]{x})+A(e^{{2i\pi\over3}}\sqrt[3]{x})+A(e^{{4i\pi\over3}}\sqrt[3]{x}))=\sum\limits_{n=0}^\infty a_{3n}x^n.$$

\subsection*{ZAD. 3.}
$$a_n=\sum\limits_{i=0}^\infty F_iF_{n-i}$$

\begin{align*}
    f(x)&=\sum\limits_{n=0}^\infty a_nx^n=\sum\limits_{n=0}^\infty x^n\sum\limits_{i=1}^n F_iF_{n-i}
\end{align*}

\subsection*{ZAD. 4.}

$k$-ta pochodna funkcji $f(x)=(1+x)^\alpha$ to
$$f^{(k)}(x)=\alpha(\alpha-1)...(\alpha-k+1)(1+x)^{\alpha-k}$$
czyli szereg Maclaurina (tzn. szereg Taylora w zerze) dla funkcji $f(x)$ to:
\begin{align*}
    (1+x)^\alpha&=(1+0)^\alpha+{\alpha\over 1!}(1+0)^{\alpha-1}x+{\alpha(\alpha-1)\over 2!}(1+0)^{\alpha-2}x^2+...=\\
    &=\sum\limits_{n=0}^\infty {\alpha(\alpha-1)...(\alpha-n+1)\over n!}x^n=\sum\limits_{n=0}^\infty {a^{\underline{n}}\over n!}x^n
\end{align*}
Co zgadza się ze wzorem podanym w treści zadania.

\subsection*{ZAD. 6.}

$$d_{n+1}=n(d_n+d_{n-1})$$

Na podłodze mamy rozłożone $(n+1)$ par skarpet o różnych wzorach i kolorach. Chcemy być osobą nietuzinkową, więc nie możemy nosić dwóch skarpet z jednej pary, dlatego szukamy ilości sposobów na jakie możemy pomieszać naszą kolekcję. 

Podnosimy górną skarpetkę z pierwszej pary. Odkładamy ją na bok i na $n$ sposobów wybieramy skarpetkę z pozostałych $n$ par. Teraz mamy dwie możliwości:

\point jeśli po prostu zamienimy skarpetę numer $1$ z wybraną na $n$ sposobów skarpetą, to zostaje nam pomieszać pozostałe $(n-1)$ skarpet, czyli mamy $nd_{n-1}$

\point jeśli z kolei nie zadowolimy się zwykłą podmianą pierwszej skarpety i tej wybranej, to tymczasowo kładziemy pierwszą skarpetę na wybranym miejscu i uznajemy to parę, po czym dokonujemy przemieszania nowo utworzonych $n$ par skarpet, czyli $nd_n$.
\medskip

Po zsumowaniu tych dwóch możliwych scenariuszy dostajemy
$$d_{n+1}=nd_n+nd_{n-1}$$
potencjalnych sposobów ułożenia kolekcji skarpet w sposób ciekawy.
\bigskip

Do tej rekurencji potrzebujemy znać wartość $d_0$ oraz $d_1$, gdyż $d_2=1\cdot d_1+1\cdot d_0$. 

Przy $d_0$ mamy zbiór pusty, więc nie robimy nic. To można nie robić tylko na jeden sposób, czyli $d_0=1$. 

Przy $d_1$ mamy tylko jeden element, więc tak czy siak on musi wylądować na miejscu $1$, więc na singletonie nie jesteśmy w stanie utworzyć ani jednego nieporządku. Stąd $d_1=0$.
\smallskip

$$d_n=nd_{n-1}+(-1)^n$$

Dla $n=1$ mamy 
$$d_1=1d_0+(-1)^1=1-1=0$$
co jest prawdą jak tłumaczyłam wyżej. Teraz załóżmy, że wzór jest poprawny dla pierwszych $n$ wyrazów. Wtedy
\begin{align*}
    d_{n+1}&=n(d_n+d_{n-1})\overset{*}{=}nd_n+d_n-(-1)^n=(n+1)d_n+(-1)^{n+1}
\end{align*}
Równość z $*$ wynika z faktu, że
$$d_n=nd_{n-1}+(-1)^n$$
$$nd_{n-1}=d_n-(-1)^n$$

\begin{align*}
    d_n&=nd_{n-1}+(-1)^n=n((n-1)d_{n-2}+(-1)^{n-1})+(-1)^n=\\
    &=n(n-1)((n-2)d_{n-3}+(-1)^{n-2})+n(-1)^{n-1}+(-1)^n=\\
    &=n!d_0+\sum\limits_{k=1}^{n-1}{n!\over (n-k)!}(-1)^{n-k}=\sum\limits_{k=0}^{n}{n!\over(n-k)!}(-1)^{n+k}=\\
    &=\sum\limits_{k=0}^n{n\choose k}k!(-1)^{n+k}
\end{align*}
zamiana wykładnika przy $(-1)^{n-k}$ jest możliwa, bo $(-1)^{2k}=1$, czyli:
$$(-1)^{n-k}=(-1)^{n-k}(-1)^{2k}=(-1)^{n-k+2k}=(-1)^{n+k}$$
co jest wzorem ogólnym na ilość nieporządków.

\subsection*{ZAD. 7.}

Wykładnicza funkcja tworząca dla ciągu $a_n$ to funkcja postaci
$$f(x)=\sum\limits_{n=0}^\infty {a_n\over n!}x^n.$$

Z poprzedniego zadania wiemy, że
$$d_{n}=(n-1)(d_{n-2}+d_{n-1})=nd_{n-1}+(-1)^n$$

\begin{align*}
    f(x)&=\sum\limits_{n=0}^\infty {d_n\over n!}x^n=\sum\limits_{n=0}^\infty {nd_{n-1}+(-1)^n\over n!}x^n=x\sum\limits_{n=0}^\infty {d_{n-1}\over (n-1)!}x^{n-1}+\sum\limits_{n=0}^\infty {(-1)^n\over n!}x^n=\\
    &=x\sum\limits_{n=1}^\infty {d_n\over n!}x^n+e^{-x}=x\sum\limits_{n=0}^\infty {d_n\over n!}x^n-x+e^{-x}=xf(x)-x+e^{-x}\\
\end{align*}
$$f(x)(1-x)=e^{-x}-x$$
$$f(x)={e^{-x}-x\over 1-x}$$

\subsection*{ZAD. 10.}

Mamy zadanie $2n$ punktów, które chcemy połączyć za pomocą $n$ linii. Niech $a(n)$ będzie ilością takich podziałów. 

Biorąc dowolny punkt musimy dla niego wybrać partnera tak, żeby po obu stronach tak przedzielonego okręgu były parzyste ilości punktów. Na pewno możemy wziąć sąsiada po lewej stronie, a następne potencjalne punkty są co dwa, więc takich podziałów mamy $n$. 

Każdy taki wybór daje nam unikatowy podział pozostałych $2n-2$ punktów na dwie części. Jeśli w lewej części podziału mamy $2n-2k-2$ punktów, to w prawej zostaje ich $2k$, gdzie $k=0,...,n-1$. Daje to nam wzór rekurencyjny:
$$a(n)=\sum\limits_{k=0}^{n-1}a(k)a(n-k-1),$$
który jest identyczny ze wzorem na liczbę Catalana:
$$C_n=\sum\limits_{k=0}^{n-1}C_iC_{n-k-1}.$$
Warunki początkowe szukanego ciągu to $a(0)=1$, bo mamy $0$ punktów i chcemy je łączyć $0$ linii, więc nie robimy nic na jeden sposób. Zgadza się to z wartością $C_0=1$, czyli $a(n)={2n\choose n}-{2n\choose n+1}$, co odpowiada wzorowi na $n$-tą liczbę Catalana.

\subsection*{ZAD. 13.}

Pokażę najpierw, że $a_{n+1}=a_n+na_{n-1}$. Jeśli mamy dane $(n+1)$ elementów i chcemy zrobić z nich permutację rzędu $2$, to możemy ostatni zostawić tam gdzie był - jest on wtedy punktem stałym i dalej tworzymy inwolucję na $n$ elementach i jest ich $a_n$. Możemy też ostatni element wciągnąć w cykl dwuelementowy z jednym z pozostałych $n$ elementów na $n$ sposobów, a pozostałe elementy wchodzą w inwolucję na $(n-1)$ elementach. Czyli
$$a_{n+1}=a_n+na_{n-1},$$
gdzie $a_0=1$, bo mamy tylko jedną pustą permutację oraz $a_1=1$ bo mamy tylko identyczność.

Wykładnicza funkcja tworząca tego ciągu to
\begin{align*}
    f(x)&=\sum\limits_{n=0}^\infty {a_n\over n!}x^n=\sum\limits_{n=2}^\infty {a_{n-1}+(n-1)a_{n-2}\over n!}x^n+1+x=1+x+\sum\limits_{n=2}{a_{n-1}\over n!}x^n+\sum\limits_{n=2}{(n-1)a_{n-2}\over n!}x^n\\
    f'(x)&=1+\sum\limits_{n=2}{a_{n-1}\over(n-1)!}x^{n-1}+x\sum\limits_{n=2}{a_{n-2}\over (n-2)!}x^{n-2}=\\
    &=\sum\limits_{n=0}{a_n\over n!}x^n+x\sum\limits_{n=0}{a_n\over n!}x^n=f(x)+xf(x)
\end{align*}
\begin{align*}
    g(x)&=e^{x+{x^2\over2}}\\
    g'(x)&=(1+x)e^{x+{x^2\over2}}
\end{align*}
Czyli $f(x)=e^{x+{x^2\over2}}$.

\subsection*{ZAD. 14.}

Organizujemy przyjęcie na $n$ osób, ale mamy tylko $k$ stołów. Chcemy sprawdzić ile jest możliwych usadzeń $n$ osób przy $k$ stołach. Jeśli jeden gość jest delikwentem którego najlepiej posadzić w kącie przy osobnym stole, to pozostałe osoby rozsadzamy na $\begin{bmatrix}
    n-1\\k-1
\end{bmatrix}$ sposobów. Jeśli delikwent jest w stanie siedzieć przy ludziach, wybieramy mu na $(n-1)$ osobę po której prawej stronie (lub lewej, sytuacja lustrzana) usiądzie, a pozostałych rozsadzamy na $\begin{bmatrix}
    n-1\\k
\end{bmatrix}$ sposobów. Daje to zależność z treści zadania.

Dla $n=0$ mamy $x^{\overline0}=1=\sum\begin{bmatrix}0\\k\end{bmatrix}x^k=1+0=1$. Załóżmy więc, że dla pierwszych $n$ liczb to działa, wtedy
\begin{align*}
    x^{\overline{n+1}}&=(x+n)x^{\overline{n}}=(x+n)\sum\limits_{k=0}\begin{bmatrix}n\\k\end{bmatrix}x^k=\sum\limits_{k=0}\begin{bmatrix}n\\k\end{bmatrix}x^{k+1}+\sum\limits_{k=0}\begin{bmatrix}n\\k\end{bmatrix}nx^k=\\
    &=\sum\limits_{k=1}\begin{bmatrix}n\\k-1\end{bmatrix}x^k+\sum\limits_{k=1}\begin{bmatrix}n\\k\end{bmatrix}nx^k+0=\\
    &=\sum\limits_{k=1}\Big(\begin{bmatrix}n\\k-1\end{bmatrix}+n\begin{bmatrix}n\\k\end{bmatrix}\Big)x^k=\sum\limits_{k=1}\begin{bmatrix}
        n+1\\k
    \end{bmatrix}x^k=\sum\limits_{k=0}\begin{bmatrix}
        n+1\\k
    \end{bmatrix}x^k
\end{align*}
Ostatnia równość zachodzi, bo dla $n>0$ $\begin{bmatrix}
    n\\0
\end{bmatrix}=0$, więc jest równoznaczna z dodaniem zera.


\end{document}