\documentclass{article}

\usepackage{../../../uni-notes-eng}
\usepackage{multicol}
\definecolor{back2}{HTML}{ffffff}
\definecolor{text2}{HTML}{000000}
\usepackage{wrapfig}
\usepackage{svg}

\author{Weronika Jakimowicz}
\title{MDM Lista 10}
\date{}

\begin{document}
\maketitle
\thispagestyle{empty}

\subsection*{ZAD. 2}

{\color{acc}(a)} Taki {graf} nie może istnieć, bo mamy 3 wierzchołki stopnia nieparzystego, a z hand shaking lemma wiemy, że wierzchołków stopnia nieparzystego musi być parzyście wiele.
\medskip

{\color{acc}(b)} Mamy $5$ wierzchołków, w tym jeden stopnia $4$, czyli musi być połączony ze wszystkimi pozostałymi. Na to połączenie używamy wszystkich możliwych połączeń wierzchołków stopnia $1$ (bo mamy graf prosty, więc nie możemy go połączyć wielokrotnie z wierzchołkiem stopnia $3$) i zostaje nam wierzchołek który chce mieć stopień $3$. Szuka więc jeszcze $2$ sąsiadów, ale już każdy inny wierzchołek ma dość sąsiadów i dlatego nie możemy spełnić jego oczekiwań. Czyli taki graf nie może istnieć.
\medskip

{\color{acc}(c)} Mamy $5$ wierzchołków i graf dwudzielny. Możemy mieć więc albo jeden wierzchołek samotny i pozostałe $4$ w jednej klasie albo klasę o $3$ wierzchołkach i klasę o $2$ wierzchołkach. Pierwszy pomysł nie jest możliwy, gdyż z klasy o jednym wierzchołku wychodzą dwie krawędzie, ale z większej klasy chce do niej wejść $8$ krawędzi. Drugi graf obalamy w ten sam sposób: z mniejszej klasy wychodzi $4$ krawędzie, ale z drugiej chce wejść tych krawędzi aż $6$.

\subsection*{ZAD. 3.}

Niech $G$ będzie grafem takim, że $d(G)>3$. Weźmy $u,v\in G$ takie, że $d(u, v)=d(G)>3$. Chcemy pokazać, że odległość każdych dwóch wierzchołków w $\overline G$ jest mniejsza niż $3$. Niech więc $x,y\in \overline G$ będą dwoma dowolnymi wierzchołkami. Oczywiście pula wierzchołków $G$ i $\overline G$ jest taka sama.

Wierzchołki $u,v$ na pewno nie mogą być połączone w $G$, czyli są połączone w $\overline G$. W zbiorze $u,v,x,y$ mamy co najwyżej $4$ różne wierzchołki. Jeżeli $xy\in \overline G$ to koniec. W przeciwnym wypadku istnieje $xy\in G$. Nie możemy mieć $xu,xv\in G$ bo wtedy $uxv\in G$ i jest $d(u,v)\leq2$. Tak samo dla $uy,vy$. Co więcej, jeśli $ux\in G$, to mamy $vx\in\overline G$ i wtedy nie możemy mieć $vy\in G$, bo wtedy $uxyv\in G$ i jest $d(u,v)\leq 3$. Czyli jeśli $vx\in\overline G$, to mamy $vy\in\overline G$ i jest $xvy\in\overline G$ i wtedy w kontekście $\overline G$ jest $d(x,y)\leq 2$.

\subsection*{ZAD. 6.}

Zauważmy, że w drzewie jeśli mamy ścieżkę bez powtarzających się wierzchołków, to jest to unikalne takie połączenie. W naszym drzewie $ab$ i $cd$ są rozłączne, ale już $ac$ i $ad$ mają wspólne wierzchołki. Tak samo $bc$ i $bd$. Teraz zauważmy, że $ac$ i $bc$ mają również co najmniej jeden wspólny wierzchołek. Teraz rozważmy przypadki pod względem czy $c\in bd$. Symetryczne przypadki $d\in bd$ albo $a,b\in bd$ są analogiczne.

Jeżeli $c\in bd$, to wystarczy, że do $bc$ dokleimy ścieżkę $cd$ i mamy $bccd=bd\ni c$ ma wspólny co najmniej wierzchołek $c$ ze ścieżką $ac$. 

Jeżeli zaś $c\notin bd$, to wtedy również $c\notin ad$, bo $ad=abbd\not\ni c$ (lub $bd=baad$). Wtedy mamy $ac=addc$ i $d$ jest wspólnym wierzchołkiem $ac$ i $bd$.

\subsection*{ZAD. 13.}

Graf T\'urana $T_r(n)$ to zupełny graf $r$-dzielny o $n$ wierzchołkach, który w każdej klasie ma $\lfloor {n\over r}\rfloor$ lub $\lceil {n\over r}\rceil$ wierzchołków. Graf $T_r(n)$ jest $K_{r+1}$-wolny, bo wybierając $(r+1)$ wierzchołków co najmniej dwa muszą leżeć w tej samej klasie i być niepołączone. Grafy T\'urana pojawiają się w {\color{acc}twierdzeniu T\'urana}, które mówi, że jeśli $|G|=n$ jest $K_{r+1}$-wolny i $e(G)\geq e(T_r(n))$, to $G\cong T_r(n)$. Z tego wynika, że $T_r(n)$ jest grafem $K_{r+1}$-wolnym o największej liczbie krawędzi.
\medskip

W zadaniu rozważamy grafy $K_3$-wolne. Z uwagi wyżej wiemy, że $T_2(n)$ będzie grafem o $n$ wierzchołkach, który jest $K_3$ wolny i ma największą możliwą liczbę krawędzi. Teraz zauważmy, że $T_2(n)$ to graf dwudzielny o jak najbardziej równych klasach wierzchołków. Na poprzedniej liście w zadaniu 8 udowodniliśmy, że graf dwudzielny o $n$ wierzchołkach ma co najwyżej $\lfloor{n^2\over 4}\rfloor$ krawędzi i jest to nadal aktualne w tym zadaniu dla grafu $T_2(n)$.

\subsection*{ZAD. 14.}

{\color{def}$G$ jest $2$-spójny $\iff$ (a)}:
\medskip

$\color{acc}\implies$
\smallskip

Niech $G$ będzie $2$-spójnym grafem. Załóżmy nie wprost, że istnieją dwa wierzchołki $u,v\in G$ nie leżące na jednym cyklu. Ponieważ $G$ jest grafem spójnym, to istnieje ścieżka $u...v\in G$. Ponieważ nie istnieje cykl zawierający jednocześnie $u$ i $v$, to musimy mieć co najmniej jeden wierzchołek $x\in G$ przez który wszystkie takie ścieżki przechodzą. Inaczej moglibyśmy mieć $y\in G$ taki, że $u...y...v\in G$ i może się zdarzyć, że $u...x...v$ oraz $u...y...v$ mają wspólne tylko końce, czyli $u...x...v...y...u$ jest cyklem. Ale zauważmy, że jeśli jest co najmniej jeden taki wierzchołek $x$, to usunięcie go powoduje usunięcie wszystkich ścieżek $u...v$, a co za tym idzie - rozspójnienie grafu $G$. Czyli $G$ nie może być $2$-spójny.
\smallskip

$\color{acc}\impliedby$
\smallskip

Jeżeli każde dwa wierzchołki $uv$ leżą na cyklu, to mamy co najmniej dwie rozłączne ścieżki $P_1,P_2$ między $u$ a $v$. Czyli jeśli usuniemy dowolny wierzchołek z $P_1$ lub $P_2$, to zostaje nam zawsze druga, nietknięta i rozłączna z rozciętą, ścieżka. Czyli graf jest $2$-spójny.
\medskip

{\color{def}$G$ jest $2$-spójny $\iff$ (b)}:
\medskip

$\color{acc}\implies$
\smallskip

Mamy graf $2$-spójny oraz dwie krawędzie: $ab, cd\in G$. Przedzielmy $ab$ na pół nowym wierzchołkiem $u$, czyli dodamy $au,ub$ do grafu $G$. Taki graf jest nadal $2$-spójny, bo  stare wierzchołki nie zostały zmienione, więc jeśli usuniemy cokolwiek poza $a,b,u$ graf zachowuje się tak naprawdę po staremu, natomiast jeśli usuniemy $a$ lub $b$ to dostaniemy graf jakbyśmy usunęli z poprzedniego $a$ lub $b$, ale z dołączonym "liściem". Skoro graf bez liście był spójny, to z liściem nadal jest spójny. Na koniec, jeśli usuwamy $u$, to wpływamy tylko na wierzchołki połączone przez $ab$, ale przecież z punktu $(a)$ wiemy, że $a$ i $b$ są na jednym cyklu i rozłączenie cyklu w jednym punkcie zostawia nam ścieżkę $a...b$. Czyli po prostu zamiast krótkiej drogi $ab$ mamy $a...b$, ale nadal jest to spójny graf. To samo możemy zrobić z krawędzią $cd$: dzielimy ją nowym wierzchołkiem $v$. Po tych dwóch modyfikacjach możemy nazwać nowy graf $G'$.

Z poprzedniego podpunktu i fakty, że $G'$ jest $2$-spójny wiemy, że wierzchołki $u$ i $v$ leżą na jednym cyklu. Nazwijmy ten cykl $C'$. Zauważmy, że $C'$ zawiera $aub$ oraz $cvd$. Teraz jeśli przeniesiemy $C'$ na realia grafu $G$ i zaczniemy tworzyć cykl $C$, to wierzchołki różne od $a,b,c,d,u,v$ są bez zmian - prosta kalka z $C'$. Natomiast dla wierzchołków $a,b,c,d$ zamiast $aub$ i $cvd$ musimy do cyklu $C$ wrzucić tylko krawędzie $ab$ i $cd$, czyli wyjmujemy $u$ i $v$ i sklejamy z powrotem krawędzie które nimi rozrywaliśmy. W ten sposób powstał cykl $C$ zawierający nasze dwie oryginalne krawędzie.
\smallskip

$\color{acc}\impliedby$
\smallskip

Wiemy, że w grafie $G$ każde dwie krawędzie leżą na jednym cyklu. Czyli jeżeli będziemy chcieli usunąć jeden wierzchołek, to tylko rozrywamy taki cykl i dalej mamy przejście dłuższą drogą.

\end{document}