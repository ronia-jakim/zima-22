\documentclass{article}

\usepackage{../../../uni-notes-eng}
\usepackage{multicol}
\definecolor{back2}{HTML}{ffffff}
\definecolor{text2}{HTML}{000000}
\usepackage{wrapfig}
\usepackage{svg}

\author{Weronika Jakimowicz}
\title{MDM Lista 13}
\date{}

\begin{document}
\maketitle
\thispagestyle{empty}

\subsection*{ZAD. 1.}
\emph{Niech $G\bullet e$ oznacza graf $G$ po ściągnięciu krawędzi $e$. Pokaż, że jeśli $G$ jest planarny, to $G\bullet e$ też jest planarny. Czy graf Petersena jest planarny?}
\medskip

\podz{sep}
\medskip

Niech $G$ będzie grafem planarnym, a $uv=e\in G$ będzie dowolną jego krawędzią. Narysujmy jak wygląda sąsiedztwo $e$:
\begin{center}
\begin{tikzpicture}
    \node (u) at (0, 0) {$\bullet$};
    \node (v) at (2, 0) {$\bullet$};
    \node (1u) at (0, 0.5) {$u$};
    \node (1v) at (2, 0.5) {$v$};
    \node (1) at (4, 1.5) {$\bullet$};
    \node (2) at (4, 0.5) {$\bullet$};
    \node (3) at (4, -0.5) {$\bullet$};
    \node (4) at (4, -1.5) {$\bullet$};
    \node (5) at (-2, 1.5) {$\bullet$};
    \node (6) at (-2, 0.5) {$\bullet$};
    \node (7) at (-2, -0.5) {$\bullet$};
    \node (8) at (-2, -1.5) {$\bullet$};
    \draw [color = acc] (u)--(v);
    \draw (v)--(1);
    \draw (v)--(2);
    \draw (v)--(3);
    \draw (v)--(4);
    \draw (u)--(5);
    \draw (u)--(6);
    \draw (u)--(7);
    \draw (u)--(8);
\end{tikzpicture}
\end{center}
Co się stanie jeśli po prostu przysuniemy prawą stronę sąsiedztwa tak, żeby wierzchołki $u$ i $v$ były jeden na sobie?

\begin{center}
\begin{tikzpicture}[every loop/.style={}]
    \node (u) at (0, 0) {$\bullet$};
    \node (v) at (0, 0) {$\bullet$};
    \node (1u) at (0, 0.5) {$u$};
    \node (1v) at (0, -0.5) {$v$};
    \node (1) at (2, 1.5) {$\bullet$};
    \node (2) at (2, 0.5) {$\bullet$};
    \node (3) at (2, -0.5) {$\bullet$};
    \node (4) at (2, -1.5) {$\bullet$};
    \node (5) at (-2, 1.5) {$\bullet$};
    \node (6) at (-2, 0.5) {$\bullet$};
    \node (7) at (-2, -0.5) {$\bullet$};
    \node (8) at (-2, -1.5) {$\bullet$};
    \path (u) edge [loop, color=acc] (u);
    \draw (v)--(1);
    \draw (v)--(2);
    \draw (v)--(3);
    \draw (v)--(4);
    \draw (u)--(5);
    \draw (u)--(6);
    \draw (u)--(7);
    \draw (u)--(8);
\end{tikzpicture}
\end{center}
Niebieska pętla zostaje usunięta, gdyż pracujemy na grafach prostych. Przesunęliśmy część poprawnego rysunku w lewo, co nie psuje planarności $G\bullet e$.
\medskip

Graf Petersona to graf o $10$ wierzchołkach oraz $15$ krawędziach. W dodatku, najkrótsze cykle w grafie Petersona mają $5$ elementów, czyli jeśli $m$ to ilość krawędzi, a $f$ - ilość ścian, to mamy
$$2m\geq 5f$$
co w połączeniu z formułą Eulera (nad którą pochylamy się bardziej w zadaniu 4) dostajemy
$$n-m+f=2$$
$$f=2-n+m$$
$$2m\geq 5f=5[2-n+m]=10-5n+5m$$
$$2m=30\geq 10-5\cdot10+5\cdot 15=35=5[2-n+m]$$
i to daje nam sprzeczność.

\subsection*{ZAD. 4.}
\emph{Udowodnij, że jeśli $G$ jest grafem płaskim, to $n(G)+f(G)=m(G)+k(G)+1$, gdzie $f(G)$ jest liczbą obszarów, a $k(G)$ jest liczbą składowych spójności.}
\medskip

\podz{sep}
\medskip

Rozważmy najpierw graf spójny $G$. Wtedy $k(G)=1$, a więc $n+f-m=2$, co jest formułą Eulera, dość szeroko znaną. Zrobimy szybki dowód indukcyjny po ilości krawędzi. Jeżeli $m=0$, to mamy tylko samotny wierzchołek i jedną ścianę, więc $n+f=1+1=2$, co się zgadza. Dalej załóżmy, że formuła jest prawdziwa dla wszystkich grafów spójnych o co najwyżej $m$ krawędziach. Niech $G$ będzie grafem o $(m+1)$ krawędziach. Rozważmy dwie możliwości:
\smallskip

1. $G$ nie ma cykli, czyli jest drzewem, więc mamy tylko jedną ścianę (bo jest potrzebny cykl by oddzielić kolejną) i ilość wierzchołków jest o $1$ większa niż ilość krawędzi, czyli $n=m+2$ i mamy 
$$n-(m+1)+f=m+2-m-1+1=2$$
tak jak chcieliśmy.
\smallskip

2. $G$ ma cykle, wtedy niech $e\in G$ będzie krawędzią należącą do pewnego cyklu. Wtedy usunięcie takiej krawędzi połączy dwie ściany w jedną większą, czyli $f'=f-1$. Z założenia indukcyjnego wiemy, że dla grafu $G'=G-e$ mamy
$$2=n-m'+f'=n-(m-1)+(f-1)=n-m+1+f-1=n-m+f$$
tak jak chcieliśmy.
\bigskip

\podz{sep}
\bigskip

Niech teraz $G$ będzie grafem o $k$ składowych spójności $G_1,...,G_k$. Wtedy dla każdej z nich osobno możemy zastosować formułę Eulera, czyli
$$n_i-m_i+f_i=2$$
gdzie $n_i,m_i,f_i$ to odpowiednio ilość wierzchołków, krawędzi i ścian w składowej $G_i$. Rysunek każdej z tych składowych możemy ograniczyć do pewnego okręgu na płaszczyźnie, czyli tylko ta "zewnętrzna" ściana jest wspólna dla wszystkich składowych spójności. W takim razie jeśli dodamy wszystkie równania jak wyżej, dostaniemy
$$\sum\limits_{i=1}^k[n_i-m_i+f_i]=2\cdot k$$
ale ponieważ jedna ściana jest dublowana dla każdej z tych składowych, to musimy odjąć ją od każdego $f_i$ jeden i dodać to jako tę jedną wspólną ścianę, czyli
$$\sum\limits_{i=1}^k[n_i-m_i+(f_i-1)]+1=k+1$$
a z kolei lewa strona sumuje się do
$$\sum\limits_{i=1}^k[n_i-m_i+(f_i-1)]+1=n-m+f,$$
gdzie $n,m,f$ to wartości dla pełnego grafu. Ostatecznie, dostajemy
$$n-m+f=k+1.$$

\subsection*{ZAD. 5.}
\emph{Udowodnij, że jeśli $G$ jest spójnym grafem planarnym, w którym najkrótszy cykl ma długość $r$, to spełniona jest nierówność $(r-2)m\leq r(n-2)$. Kiedy nierówność ta staje się równościa?}
\medskip

\podz{sep}
\medskip

Niech $G$ będzie grafem planarnym o najkrótszym cyklu $r$, $n$ wierzchołkach i $m$ krawędziach. Niech $f$ będzie liczbą ścian w jego poprawnym rysunku. Wtedy z formuły Eulera mamy
$$n-m+f=2$$
$$f=2-n+m$$
natomiast ponieważ tylko cykle tworzą nowe ściany, każda krawędź wpada do dwóch ścian i na jedną ścianę potrzebujemy co najmniej $r$ krawędzi, to mamy nierówność
$$2m\geq r\cdot f$$
Podstawiając $f=2-n+m$ otrzymujemy
\begin{align*}
    2m&\geq r\cdot (2-n+m)\\
    2m-rm&\geq r\cdot(2-n)\\
    (2-r)m&\geq r\cdot(2-n)\\
    -(r-2)m&\geq -r(n-2)\\
    (r-2)m&\leq r(n-2)
\end{align*}

\subsection*{ZAD. 10.}
\emph{Na płaszczyźnie rozłożono pewną liczbę monet o jednakowej średnicy, z których żadne dwie nie nachodzą na siebie. Monety te kolorujemy tak, by te, które się stykają miały różne kolory. Nie krozystając z twierdzenia o czterech barwach pokaż, że cztery kolory zawsze wystarczą, a trzy nie zawsze.}
\medskip

\podz{sep}
\medskip

% \begin{center}
%     \begin{tikzpicture}
%         \draw [ultra thick, color=acc](0, 0) circle (1);
%         \draw [ultra thick, color=acc](2, 0) circle (1);
%         \draw [ultra thick, color=acc](-2, 0) circle (1);
%         \draw [ultra thick, color=def](0, 2) circle (1);
%         \draw [ultra thick, color=def](2, 2) circle (1);
%         \draw [ultra thick, color=def](-2, 2) circle (1);
%         \draw (1, -1.73) circle (1);
%         \draw(-1, -1.73) circle (1);
%         \draw (0, 0)--(2, 0)--(2, 2)--(0, 2)--(-2, 2)--(-2, 0)--(-1, -1.73)--(1, -1.73)--(2, 0);
%         \draw (1, -1.73)--(0, 0)--(-1, -1.73)--(-2, 0)--(0, 0);
%         \draw (0,0)--(0,2);
%         \node at (0.3, 0.3) {A};
%         \node at (2.3, 0) {B};
%         \node at (1.3, -1.7) {C};
%         \node at (-1.3, -1.7) {B};
%         \node at (-2.3, 0) {C};
%         \node at (-2.3, 2) {A};
%         \node at (2.3, 2) {A};
%         \node at (0, 2.3) {B};
%         \draw (-1, 3.73) circle (1);
%         \draw (1, 3.73) circle (1);
%         \draw (2, 2)--(1, 3.73)--(0, 2)--(-1, 3.73)--(-2, 2);
%         \draw(1, 3.73)--(-1, 3.73);
%         \node at (1, 4) {C};
%         \node at (-1, 4) {???};
%     \end{tikzpicture}
% \end{center}
% W przykładzie wyżej kolorowanie na dolnej części grafu indukuje kolorowanie na górnej części grafu. To znaczy w zależności od ustawienia kolorów na niebieskich kółkach dostajemy 
% \bigskip

{\color{def}NIE MAM KONTRPRZYKŁADU DLA 3}
\bigskip

Będziemy rozpatrywać kolorowanie monet w ramach kolorowania grafu w którym monety to wierzchołki, a krawędzi oznaczają stykanie się dwóch monet. Udowodnimy, że jeżeli $G_n$ to graf utworzony przez ułożenie $n$ monet na stole, to
$$\chi(G_n)\leq4.$$
Dla $n=1$ jest to trywialne. Załóżmy teraz, że dla $n$ monet zawsze działa i weźmy $n+1$ monet aby stworzyć $G_{n+1}$. Zauważmy, że w takim grafie zawsze mamy monetę "brzegową", czyli stopnia co najwyżej $3$. Jeżeli taką monetę wyjmiemy z grafu $G_{n+1}$, to dostajemy $G_n$, które możemy pokolorować $4$ kolorami. Dołączenie wyjętej monety nic nie zmieni, bo w jej sąsiedztwie były tylko $3$ inne, więc wybieramy kolor który się między nimi nie pojawił i tak też malujemy dokładaną monetę. Czyli $\chi(G_{n+1})\leq 4$.

\subsection*{ZAD. 11.}
\emph{Dla grafy $G$ oznaczamy przez $G\bullet e$ graf powstały w wyniku ściągnięcia krawędzi $e$, a przez $P_G(k)$ - liczbę pokolorowań grafu $G$ $k$ kolorami. Pokaż, że $P_G(k)=P_{G-e}(k)-P_{G\bullet e}(k)$.}
\medskip

\podz{sep}
\medskip

Kolorujemy graf $G$ na $x$ kolorów. Możemy to zrobić na $P_G(x)$ sposobów. Niech teraz $e=uv\in G$ będzie dowolną krawędzią. Graf $G-e$ ma dwa sposoby kolorowania: takie, w których $u,v$ mają ten sam kolor (niemożliwe w $G$) i takie, w których $u,v$ mają różne kolory (możliwe w $G$). Dalej zauważmy, że jeżeli ściągamy krawędź $e$ do jednego punktu, to tak jakbyśmy malowali $u,v$ na ten sam kolor. Czyli jeśli $P_{G-e}(x)-P_{G\bullet e}(x)$ usuwa te kolorowania $G-v$ na $x$ kolorów gdzie $u$ i $v$ mają ten sam kolor, bo tylko te kolorowania pokrywają nam się z kolorowaniami $G-e$.

\subsection*{ZAD. 12.}
\emph{Niech $T$ będzie drzewem $n$-wierzchołkowym, a $C_n$ grafem cyklicznym. Pokaż, że $P_T(k)=k(k-1)^{n-1}$ oraz $P_{C_n}(k)=(k-1)^n+(-1)^n(k-1)$.}
\medskip

\podz{sep}
\medskip

Najpierw rozważmy wielomian chromatyczny (chromian wg. jednego studenta IM) drzewa. Niech $T_n$ będzie drzewem i niech $e\in T_n$ będzie liściem, czyli wierzchołkiem stopnia $1$. Wtedy $T_n\bullet e$ jest drzewem $T_{n-1}'$, bo ściągnięcie krawędzi jest w tym przypadku analogiczne do usunięcia liścia. Graf $T_n-e$ jest natomiast niespójny, z jednym samotnym wierzchołkiem oraz drzewem $T_{n-1}$, które możemy kolorować niezależnie. Czyli korzystając z poprzedniego zadania (i ukrytej indukcji) mamy
\begin{align*}
    P_{T_n}(x)&=P_{T_n-e}(x)-P_{T_n\bullet e}(x)=P_{T_{n-1}}(x)\cdot x-P_{T_{n-1}'}(x)=\\
    &=x\cdot x(x-1)^{n-2}-x(x-1)^{n-2}=x(x-1)^{n-2}(x-1)=x(x-1)^{n-1}
\end{align*}
tak jak chcieliśmy.
\bigskip

\podz{sep}
\bigskip

Przypatrzmy się teraz na cykl długości $n$. Jeżeli wybierzemy dowolną krawędź $e\in C_n$, to graf $C_n-e$ jest ścieżką długości $n$, a więc możemy go pokolorować na $x\cdot(x-1)^{n-1}$ sposobów (pierwszy wierzchołek na $x$, każdy kolejny na $(x-1)$). Graf $C_n\bullet e$ to z kolei cykl $C_{n-1}$, czyli korzystając z ukrytej indukcji i poprzedniego zadania, dostajemy
\begin{align*}
    P_{C_n}(x)&=P_{C_n-e}(x)-P_{C_n\bullet e}(x)=x(x-1)^{n-1}-(x-1)^{n-1}-(-1)^{n-1}(x-1)=\\
    &=(x-1)^{n-1}(x-1)+(-1)^n(x-1)=(x-1)^n+(-1)^n(x-1)
\end{align*}

\subsection*{ZAD. 13.}
\emph{Wykaż, że liczba krawędzi dowolnego grafu wynosi co najmniej $\chi(G){\chi(G)-1\over2}$.}
\medskip

\podz{sep}
\medskip

$\chi(G)$ mówi nam, ile potrzebujemy co najmniej kolorów, żeby dostać kolorowanie grafu $G$. Czyli na ile najmniej klas wierzchołków możemy podzielić graf $G$. Ponieważ $\chi(G)$ jest minimalną taką liczbą, to między każdymi dwoma wierzchołkami które są w różnych klasach mamy co najmniej jedną krawędź. Klas jest $\chi(G)$, więc jeżeli miałyby one po jednym wierzchołku, to mielibyśmy co najmniej $(\chi(G)-1)$ krawędzi wychodzących z każdej klasy. Czyli sumarycznie z każdej klasy musi wychodzić $\chi(G)(\chi(G)-1)$ krawędzi, ale ponieważ sumujemy je dwukrotnie, to dostajemy dolne ograniczenie na krawędzie w $G$ postaci:
$$\frac12\chi(G)(\chi(G)-1)$$

\subsection*{ZAD. 14.}
\emph{Pokaż, że dla dowolnego grafu $G$ $\chi(G)\chi(\overline G)\geq n$.}
\medskip

\podz{sep}
\medskip

Weźmy dowolny graf $G$ o $n$ wierzchołkach i niech $c_1:V(G)\to [\chi(G)]$ będzie poprawnym kolorowaniem go za pomocą $\chi(G)$ kolorów. To samo powtórzmy dla $\overline G$, to znaczy $c_2:V(\overline G)\to[\chi(\overline G)]$ jest kolorowaniem za pomocą $\chi(\overline G)$ kolorów. Zauważmy teraz, że jeżeli połączymy $G$ i $\overline G$ razem, to dostaniemy graf $K_n$. 
\smallskip

Opiszmy kolorowanie na $K_n$ takie, że każdy wierzchołek $v\in K_n$ dostaje parę uporządkowaną $(c_1(v), c_2(v))$. Ponieważ kolorowanie na $G$ i na $\overline G$ było poprawne, a dana krawędź z $K_n$ musi istnieć w dokładnie jednym z nich, to dla dwóch stycznych wierzchołków nigdy nie będziemy mieli dokładnie tej samej pary. Co więcej, ponieważ możliwości na pierwszym miejscu jest $\chi(G)$, a na drugim miejscu każdej pary jest $\chi(\overline G)$, to ogółem takich par do poprawnego pokolorowania $K_n$ utworzyliśmy $\chi(G)\cdot\chi(\overline G)$, co musi być co najmniej tyle ile $\chi(K_n)=n$. Czyli dostajemy pożądany wynik.

\end{document}