\documentclass{article}

\usepackage{../../../uni-notes-eng}
\usepackage{multicol}
\definecolor{back2}{HTML}{ffffff}
\definecolor{text2}{HTML}{000000}
\usepackage{wrapfig}
\usepackage{svg}

\author{Weronika Jakimowicz}
\title{MDM Lista 11}
\date{}

\begin{document}
\maketitle
\thispagestyle{empty}

\subsection*{ZAD. 1.}
\emph{Dane jest drzewo $T$ oraz jego automorfizm $\phi$. Udowodnij, że istnieje wierzchołek $v$ taki, że $\phi(v)=v$ lub istnieje krawędź $\{u,v\}$ taka, że $\phi(\{u,v\})=\{u,v\}$}
\medskip

\podz{sep}
\medskip

Niech $n$ będzie liczbą wierzchołków w drzewie $T$. Dla $n=1$ mamy drzewo o jednym wierzchołku i tylko jeden automorfizm na nim - identyczność, która zachowuje nie tylko wierzchołki, ale i (nieistniejące) krawędzie. Dla $n=2$ mamy tylko jedną krawędź i dwa punkty, więc ta jedyna krawędź zawsze musi przejść na samą siebie.
\smallskip

Załóżmy teraz, że dla wszystkich drzew o co najwyżej $n$ wierzchołkach teza jest prawdziwa. Niech $|T|=n+1$. Zauważmy, że jeśli $\phi$ jest automorfizmem na $T$, a $v\in T$ jest jego dowolnym liściem, to $\phi(v)$ musi nadal być liściem - inaczej $v$ stopnia $1$ przeszłoby na wierzchołek będący węzłem, a więc mający co najmniej stopień $2$ i takie $\phi$ nie mogłoby być automorfizmem na $T$.

Wiemy też, że w drzewie jest na pewno jeden wierzchołek stopnia $1$, niech więc 
$$L=\{v\in T\;:\;d(v)=1\}$$
będzie wierzchołkiem wszystkich liści, który na pewno jest niepusty. Niech $T'=T\setminus L$. Wtedy jeśli $\phi'$ jest automorfizmem na $T'$, to na mocy założenia indukcyjnego $\phi'$ spełnia tezę. Z uwagi wyżej wiemy, że liście muszą przejść na siebie, więc jeśli będziemy rozszerzać $\phi'$ do całego $T$, to $\phi[L]=L$, czyli nie wpływa na poprawność tezy dla rozszerzenia $\phi'$ do całego $T$.

\subsection*{ZAD. 2.}
\emph{Graf prosty $G$ jest samodopełniający wtedy i tylko wtedy, gdy jest izomorficzny ze swym dopełnieniem. Pokaż, że samodopełniający graf $n$ wierzchołkowy istnieje dokładnie wtedy, gdy $n\equiv0$ lub $n\equiv1\mod 4$}
\medskip

\podz{sep}
\medskip

Graf pełny o $n$ wierzchołkach ma ${n(n-1)\over 2}$ krawędzi. My chcemy je rozdzielić po równo między dopełnienie i graf sam w sobie, czyli musimy być w stanie liczbę krawędzi $K_n$ podzielić dodatkowo na $2$, a więc $n(n-1)$ musi być podzielne przez $4$. Jest to wtedy, gdy 
$$n\equiv0\mod4$$
lub 
$$(n-1)\equiv0\mod4$$
$$n\equiv1\mod4.$$

\subsection*{ZAD. 3.}
\emph{Niech $C_1,C_2,...,C_{m-n-1}$ będą zbiorami krawędzi wszystkich $(m-n+1)$ cykli otrzymanych poprzez dodanie do drzewa spinającego $T$ grafu prostego $G$ jednej krawędzi $G$ która nie należy do $T$. Pokaż, że zbiór krawędzi dowolnego cyklu w $G$ jest różnica symetryczną pewnej liczby zbiorów wybranych spośród $C_1,C_2,...,C_{m-n-1}$.}
\medskip

\podz{sep}
\medskip

No ale to widać

\subsection*{ZAD. 15.}
\emph{Pokaż, że jeśli $G$ jest grafem prostym i dla każdej pary niesąsiednich wierzchołków $u,v$}
$$deg(u)+deg(v)\geq n(G)-1,$$
\emph{to w $G$ istnieje droga Hamiltona.}
\medskip

\podz{sep}
\medskip

Niech $G'$ będzie grafem $G$ z dodanym wierzchołkiem $w$ tak, że $(\forall\;v\in G)\;vw\in G'$. Teraz dla dowolnych niesąsiednich wierzchołków mamy
$$deg(u)+deg(v)\geq n.$$
Z twierdzenia Ore'a wiemy, że wtedy w $G'$ istnieje cykl Hamiltona. Niech teraz $C$ będzie tym cyklem i niech dla pewnych $v, u\in G$ $vw, uw\in C$. Wtedy jeśli usuniemy z $C$ te dwie krawędzie oraz wierzchołek $w$, to wrócimy do ścieżki zawartej w $G$, która przechodzi wszystkie wierzchołki.

\subsection*{ZAD. 17.}
\emph{Niech $G$ będzie grafem prostym. Pokaż, że $G$ zawiera drogę o długości równej co najmniej $2\cdot{e(G)\over |G|}$.}
\medskip

\podz{sep}
\medskip

Indukcja po ilości wierzchołków.
\smallskip

Dla $|G|=1$ jest to dość proste. Mamy $e(g)=0$, a więc $2\cdot{e(G)\over |G|}=2\cdot{0\over1}=0$.
\smallskip

Niech $|G|=n+1$ będzie grafem prostym, a $v\in G$ będzie wierzchołkiem o minimalnym stopniu $d=d(v)$. Wiemy, że $d\leq n-1$, bo $G$ jest prosty. Jeżeli teraz usuniemy wierzchołek $v$, to dostajemy $G'$ o $n$ wierzchołkach oraz $e(G)-d\geq e(G)-n+1$ krawędziach. W $G'$ możemy znaleźć ścieżkę o
$$2\cdot {e(G')\over n}=2\cdot{e(G)-d\over n}\geq 2\cdot {e(G)-n+1\over n}\geq2\cdot{e(G)\over n+1}-2\cdot{n-1\over n+1}\geq 2\cdot{e(G)\over n+1}$$

\end{document}