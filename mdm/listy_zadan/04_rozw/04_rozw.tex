\documentclass{article}[13pt]

\usepackage{../../../uni-notes-eng}
\usepackage{multicol}
%\definecolor{back2}{HTML}{ffffff}
%\definecolor{text2}{HTML}{000000}

\author{Weronika Jakimowicz}
\title{MDM Lista 4}
\date{}

%\pagecolor{back2}
%\color{text2}

\begin{document}
\maketitle

%\begin{center}
%    \begin{tabular}{| c | c | c | c | c | c | c | c | c | c | c | c | c | c | c |}
%        \hline

%        1 & 2 & 3 & 4 & 5 & 6 & 7 & 8 & 9 & 10 & 11 & 12 & 13 & 14 & 15\\

%        \hline

%        X & X & - & X & - & - & X & X & - & -  & -  & -  & -  & -  & - \\
        
%        \hline
%    \end{tabular}
%\end{center}

\section*{ZAD 1.}

Na początku warto zauważyć, że
$$lcm(n,m)={nm\over gcd(n,m)}.$$
Jeśli liczby są wystarczająco duże, może okazać się, że iloczyn $nm$ przekracza górny zakres liczb całkowitych języka z jakiego korzystamy. Żeby temu zapobiedz, możemy podzielić większą z nich przez $gcd(n,m)$ i dopiero wynik pomnożyć przez mniejszą z liczb.

Algorytm napisany w języku Python.

\begin{lstlisting}[language=Python]
# funkcja obliczajaca gcd na podstawie algorytmu Euklidesa
def gcd(n, m):
    if m == 0: return n
    else: return gcd(m, n % m)


def lcm(n, m):
    div = gcd(n, m)

    # wybranie wiekszej z liczb n,m
    mx = m
    mn = n

    if n > m:
        mn = m
        mx = n

    # dziele wieksza liczbe, zeby na pewno po pomnozeniu nie wyjsc poza zakres
    mx = mx / div

    # tak naprawde zwracam (n*m)/gcd(n,m)
    return mn * mx
\end{lstlisting}

\section*{ZAD 2.}

Zauważmy, że
$$gcd(a, b, c, d)=gcd(gcd(a, b), c, d)=gcd(gcd(a, b), gcd(c, d))$$
czyli listę liczb całkowitych $m_i$ możemy za każdym razem dzielić na pół aż dojdziemy do momentu kiedy mamy listy 2 lub 1 elementowe. Zakładamy, że $gcd(a)=a$. 

Poniższy algorytm, napisany w Pythonie, jest analogiczny do merge sort, gdzie dzielimy listę na podlisty o podobnym rozmiarze i wykonujemy na nich operacje, po czym łączymy je z powrotem w całość.

\begin{lstlisting}[language=Python]
# implementacja algorytmu Euklidesa jak w Zad 1.
def euclid(n, m):
    if m == 0: return n
    else: return euclid(m, n % m)

def gcd(k, M):
    if k == 1: return M[0] # gcd(a) = a
    if k == 2: return euclid(M[0], M[1]) # mamy liste dwuelementowa

    else: 
        # na poczatku rozbijamy liste na dwie podlisty: L i R
        L = []
        R = []

        i = 0
        while i < k//2:
            L.append(M[i])
            i += 1
        while i < k:
            R.append(M[i])
            i += 1

        # gcd(M) to gcd(L, R) - analogicznie jak w merge sort
        return gcd(2, [gcd(k//2, L), gcd(k-k//2, R)])
\end{lstlisting}

\section*{ZAD 3.}

Zacznijmy od pomysłu nie w pełni wydajnego. Chcemy znaleźć wyznaczniki takie, że
$$x_1m_1+x_2m_2+...+x_km_k=gcd(m_1,m_2,...,m_k).$$
Możemy zająć się najpierw problemem pośrednim i znaleźć
$$x_1m_1+y_1(m_2+m_3+...+m_k)=gcd(m_1, gcd(m_2,...,m_k))=gcd(m_1,...,m_2).$$
Po odpowiedniej liczbie takich modyfikacji powinniśmy otrzymać
$$x_1m_1+y_1(m_2+y_2(m_3+...+y_{k-2}(m_{k-1}+y_{k-1}m_k)))=gcd(m_1,gcd(m_2,gcd(...gcd(m_k)))).$$
Rozwiązanie tego problemu od najbardziej zagnieżdżonych do najmniej zagnieżdżonych nawiasów korzysta z rozszerzonego algorytmu Euklidesa. W jego oryginalnej wersji szukamy $a,b$ takich, że
$$ax+by=gcd(x,y).$$
Czyli za każdym razem chcemy doprowadzić do sytuacji, gdzie mamy dwie liczby $x,y$. Pierwszy pomysł po zaimplementowaniu będzie przechodził przez $k$ liczb, co nie jest bardzo efektowne. Zastanówmy się, czy możemy to przyśpieszyć.

Rozważmy przypadek dla $k=5$.
$$x_1m_1+x_2m_2+x_3m_3+x_4m_4+x_5m_5=gcd(m_1,...,m_5)$$
$$z_1(m_1+m_2+m_3)+y_1(m_4+m_5)=gcd(gcd(m_1,m_2,m_3),gcd(m_4,m_5))$$
$$z_1(z_2(m_1+m_2)+m_3)+y_1(m_4+m_5)=gcd(gcd(gcd(m_1,m_2),m_3), gcd(m_4, m_5))$$

Zauważmy, że możemy nadal dzielić listę na połowe i wywoływać rozszerzony algorytm Euklidesa kiedy już dojdziemy do $2$ elementów. W przypadku jednoelementowego odcinka możemy zwracać ten jedyny element.


\begin{lstlisting}[language=Python]
def ext_gcd(a, b):
    R = [a, b]
    S = [1, 0]
    T = [0, 1]

    while R[1] != 0:
        q = R[0] // R[1]
        R = [R[1], R[0] - q * R[1]]
        S = [S[1], S[0] - q * S[1]]
        T = [T[1], T[0] - q * T[1]]

    ret = [S[0], T[0]]

    return ret

def zad(k, M):
    if k == 1: return [M[0]]
    if k == 2:
        return ext_gcd(M[0], M[1])

    else:
        L = []
        R = []

        i = 0
        
        # dziele liste na dwie podlisty
        while i < k//2:
            L.append(M[i])
            i += 1
        while i < k:
            R.append(M[i])
            i += 1

        # wykonuje na nich rekurencyjnie to samo dzialanie
        coef_L = zad(k//2, L)
        coef_R = zad(k-k//2, R)

        new_L = []
        new_R = []

        # mnoze wczesniej otrzymane wspolczynniki przez aktualnie otrzymane, a potem sumuje
        i = 0
        while i < len(L): 
            new_L.append(coef_L[i] * L[i])
            i += 1

        i = 0
        while i < len(R): 
            new_R.append(coef_R[i] * R[i])
            i += 1

        sum_L = sum(new_L)
        sum_R = sum(new_R)

        # wspolczynniki przy sumie podlist
        vec = ext_gcd(sum_L, sum_R)

        i = 0
        while i < len(L):
            coef_L[i] *= vec[0]
            i += 1
        i = 0
        while i < len(R):
            coef_R[i] *= vec[1]
            i += 1

        # zwroc wspolczynniki lewej podlisty zlaczone ze wspolczynnikami prawej podlisty
        ret = coef_L + coef_R

        return ret
\end{lstlisting}


\section*{ZAD 4.}

Jeśli $a,b$ są parzyste, to obie są podzielne przez $2$, więc $2$ wystąpi co najmniej raz w rozkładzie $gcd(a,b)$ na czynniki pierwsze. Możemy więc zapisać, że istnieją $a',b'\in\N$ takie, że $a=2a',b=2b'$ oraz
$$gcd(a,b)=2gcd(a',b').$$

\begin{lstlisting}[language=Python]
def binary_gcd(a, b):
    if a == 0:
        return b

    # XOR sprawdza, czy tylko jedna liczba jest parzysta
    bol = (a % 2 == 0) ^ (b % 2 == 0)

    if bol:
        if a % 2 == 1: return binary_gcd(b, a)
        return binary_gcd(a//2, b)

    # jezeli obie sa nieparzyste
    elif (a+b) % 2 == 0:
        if a > b: return binary_gcd(a-b, b)
        return binary_gcd(b-a, a)
    
    # jezeli obie sa parzyste, to na pewno obie dzieli 2, wiec mozna zwrocic 2 * gcd(a/2, b/2)
    else:
        return 2 * binary_gcd(a//2, b//2)
\end{lstlisting}

Zauważmy, że jeśli chociaż jedna z liczb $a,b$ jest parzysta, to zmniejszamy je 2 razy. Jeżeli obie są nieparzyste, to $a-b$ jest parzyste i wtedy co drugi krok zmniejszamy liczby o polowe. Czyli co najwyżej co dwa obroty zmniejszamy liczby dwukrotnie, więc wykona się $2\log_2\max(a,b)$ operacji, co daje nam złożoność $O(\log_2\max(a,b))$.


\section*{ZAD 5.}

\section*{ZAD 6.}

Zapis liczby $m$ w układzie kolejnych liczb pierwszych to zapis postaci
$$m=\prod\limits_{i=1}^\infty q_i^{m_i}$$
gdzie $q_i$ to kolejne liczby pierwsze. Oczywiście, dla skończonych liczb naturalnych $m$ od pewnego momentu $k$ dla każdego $i>k$ będzie $m_i=0$.
\medskip

{\color{def}a) $k=gcd(m,n)\iff k_i=\min(m_i,n_i)$ dla każdego $i=1,2...$}
\smallskip

$\color{acc}\implies$

Skoro $k=gcd(m,n)$, to $k$ jest największe takim, że $k|m$ oraz $k|n$.Podzielność obu liczb jest oczywista. Weźmy dowolną liczbę pierwszą $q_i$. Wtedy, ponieważ $q_i$ jest pierwsze, to musi zachodzić
$$q_i^{k_i}|m\;\land\;q_i^{k_i}|n.$$
Znowu, dla $m$ i $n$ sprowadza się to do podzielnośći $q^{m_i}$ oraz $q^{n_i}$ przez $q^{k_i}$. W takim razie $k_i$ jest największym takim, że 
$$k_i\leq m_i\;\land\;k_i\leq n_i,$$
więc $k_i=\min(m_i,n_i)$.
\smallskip

$\color{acc}\impliedby$

Wiemy, że dla każdego $i\in\N$ zachodzi $k_i=\min(m_i,n_i)$. Chcemy pokazać, że wtedy $k$ dzieli jednocześnie $n$ i $m$ oraz, że jest to największy taki dzielnik. 

To, że dzieli jest proste:
$${q^{m_i}\over q^{k_i}}=q^{m_i-k_i}\geq 1$$
$${q^{n_i}\over q^{k_i}}=q^{n_i-k_i}\geq 1$$
i obie te liczby są naturalne, bo jeśli $m_i\geq n_i$, to $q^{n_i-k_i}=1$, natomaist 
$$(\exists\;a\in\N)\;m_i-k_i=a$$ 
co daje $q^{m_i-k_i}=q^a\in\N$.

Załóżmy teraz, że istnieje liczba $p>k$ taka, że $p|m$ i $p|n$. Wtedy istnieje chociaż jedno $i$ takie, że $p_i>k_i$. Niech więc $p_i=k_i+d$, gdzie $d\in\N$, $d\geq 1$. Znowu załóżmy, że $m_i\geq n_i$.
\begin{align*}
    {q_i^{n_i}\over q_i^{p_i}}&={q^{n_i}\over q_i^{k_i+d}}={q^{n_i}\over q_i^{n_i+d}}=q^{n_i-n_i-d}=q^d
\end{align*}
ale wtedy $q^{p_i}\nmid q^{n_i}$, czyli $p\nmid n$ i mamy sprzeczność. W takim razie $k$ jest największą liczbą dzielącą jednocześnie $n$ i $m$.
\medskip

{\color{def}b) $k=lcm(m,n)\iff k_i=\max(m_i,n_i)$ dla każdego $i=1,2...$}
\medskip

$\color{acc}\implies$

Skoro $k=lcm(m,n)$ to $k$ jest najmniejsze takie, że $m|k$ i $n|k$. Weźmy dowolną liczbę pierwszą $q_i$. Tak samo jak w poprzednim podpunkcie, wystarczy, żeby potęgi liczby pierwszej były przez siebie podzielne:
$$q_i^{m_i}|q_i^{k_i}\;\land\;q_i^{n_i}|q_i^{k_i},$$
czyli $k_i\geq m_i$ oraz $k_i\geq n_i$. Poniewawż $k$ ma być najmniejszą wspólną wielokrotnością, to szukamy najmniejszego takiego $k_i$, czyli
$$k_i=\max(m_i,n_i).$$

$\color{acc}\impliedby$

Po pierwsze, chcemy pokazać że $m|k$ oraz $n|k$. Wystarczy, że pokażemy dla dowolnej liczby pierwszej $q_i$, że $q_i^{m_i}|q_i^{k_i}$ i $q_i^{n_i}|q_i^{k_i}$.
\begin{align*}
    {q^{k_i}\over q^{m_i}}&=q^{k_i-m_i}\geq 1\\
    {q^{k_i}\over q^{n_i}}&=q^{k_i-n_i}\geq 1
\end{align*}
Teraz, jeżeli BSO $m_i\geq n_i$, to $m_i=k_i$ i
$$q^{k_i-m_i}=1\in\N$$
$$q^{k_i-n_i}=q^{m_i-n_i}\in\N$$
a więc $k|m$ i $k|n$.

Załóżmy, nie wprost, że istnieje liczba $p$ mniejsza niż $k$, która jest wielokrotnościa $m$ i $n$. Wtedy na chociaż jednym miejscu, niech będzie to $i$, mamy $p_i<k_i$, a więc istnieje $d\in\N$ takie, że $p_i=k_i-d$. Mamy wtedy
$${q^{p_i}\over q^{m_i}}={q^{k_i-d}\over q^{m_i}}={q^{m_i-d}\over q^{m_i}}=q^{m_i-d-m_i}=q_i^{-d}$$
co nie jest liczbą naturalną, a więc $m\nmid p$ i mamy sprzeczność. W takim razie $k=lcm(m,n)$.


\section*{ZAD 7.}

{\color{def}a) $xz\equiv yz\mod mz\iff x\equiv y\mod m$ dla $z\neq 0$}
\medskip

$\color{acc}\impliedby$

Skoro $x\equiv y\mod m$, to znaczy, że istnieje $k\in\Z$ takie, że
$$x=km+y.$$
Jeżeli teraz pomnożymy obie strony przez $z$, to dostaniemy
$$xz=kmz+yz.$$
To oznacza, że $xz$ jest podzielne przez $mz$ reszta $yz$, czyli
$$xz\equiv yz\mod mz.$$

$\color{acc}\implies$

Jeżeli $xz\equiv yz\mod mz$, to dla pewnego $k\in\Z$ zachodzi
$$xz=kmz+yz$$
skoro $z\neq0$, to możemy obustronnie podzielić przez $z$
$$x=km+y.$$
Z tego z kolei wynika, że
$$x\equiv y\mod m$$

{\color{def}b) $xz\equiv yz\mod m\iff x\equiv y\mod {{m\over gcd(z, m)}}$ dla $x,y,z,m\in\Z$}
\medskip

$\color{acc}\impliedby$

Z założenia, że $x\equiv y\mod {{m\over gcd(z,m)}}$, to istnieje $k\in\Z$ takie, że
$$x=k{m\over gcd(z,m)}+y.$$
Jeżeli pomnożymy obie strony przez $z$, to dostaniemy
$$xz=km{z\over gcd(z,m)}+yz$$
I zauważmy, że ${z\over gcd(z,m)}\in\Z$, czyli istnieje $p\in\Z$ takie, że
$$xz=pm+yz$$
a więc
$$xz\equiv yz\mod m$$

$\color{acc}\implies$

Zakładamy, że $xz\equiv yz\mod m$, czyli istnieje $k\in\Z$ takie, że
$$xz=km+yz$$
Podzielmy teraz obustronnie przez $gcd(z,m)$, dostajemy:
$$x{z\over gcd(z,m)}={k\over gcd(z,m)}m+y{z\over gcd(z,m)}.$$
Zauważmy, że istnieje pewno $p\in\Z$ takie, że
$$z=p\cdot gcd(z,m)$$
oraz $gcd(p, m)=1$. Dalej mamy
\begin{align*}
    xp&=k{m\over gcd(z,m)}+yp\quad(\kawa)\\
    p(x-y)&=k{m\over gcd(z,m)}
\end{align*}
Ponieważ prawa strona równania jest liczbą całkowitą podzielną przez $p$, to również lewa strona musi być podzielna przez $p$. Zauważmy, że $p\nmid m$, a więc również $p\nmid {m\over gcd(z,m)}$. W takim razie $p\mid k$ i wtedy $\frac kp\in\Z$. Czyli jeśli podzielimy obie strony równania $(\kawa)$ przez $p$, dostajemy
$$x=\frac kp {m\over gcd(z,m)}+y$$
i z tego wynika, że
$$x\equiv y\mod {{m\over gcd(z,m)}}$$

{\color{def}c) $x\equiv y\mod mz\implies x\equiv y\mod m$}
\medskip

Ponieważ zwykle operacja modulo $n$ jest zdefiniowana dla $n$ całkowitych, założę, że jednocześnie $mz$ jak i $m$ są liczbami całkowitymi. Dodatkowo, $z\in\Z$, bo jeśli $m=18$, $z=\frac13$ i $x=10$:
$$10\equiv 4\mod 6$$
$$10\equiv 10\mod 18$$
to mamy sprzeczność.
\smallskip

Istnieje $k\in\Z$ takie, że $gcd(a,b)=1$ oraz
$$x=kmz+y$$
i wtedy również $kz\in\Z$, czyli $x\equiv y\mod m$.
\smallskip

% W przeciwnym wypadku $z\in\Q$, a więc istnieje $a\in\Z$ oraz $b\in\N$ takie, że
% $$z={a\over b}.$$
% I ponieważ $m$ jest liczbą pierwszą, to $gcd(m,b)=m$. Istnieje $p\in\Z$ takie, że $pm=b$ i jeżeli $p\neq 1$, to mamy
% $$mz=m{a\over b}=m{a\over pm}={a\over p}\in\Z$$
% co daje sprzeczność z $gcd(a,b)=1$, więc $p=1$ i $b=m$.
% \begin{align*}
%     x&=kmz+y\\
%     x&=km{a\over m}+y\\
%     xm&=kam+ym
% \end{align*}
% więc $xm\equiv ym\mod m$. Z poprzedniego podpunktu mamy
% $$x\equiv y\mod {{m\over gcd(m,m)}}=x\equiv y\mod 1$$
% czyli $y$ = 0, więc $mz|x$ i $m|x$ i mamy $x\equiv 0\mod m$.

\section*{ZAD 8.}

{\color{def}a) $2^n-1$ - liczba pierwsza $\implies$ $n$ jest liczbą pierwszą}
\smallskip

Załóżmy nie wprost, że istnieje $n$ takie, że $n$ nie jest liczbą pierwszą, ale $2^n-1$ jest liczbą pierwszą. Ponieważ $n$ nie jest pierwsze, to istnieją $k,m\in\Z$ większe od 1 takie, że $n=km$. Wtedy
\begin{align*}
    2^n-1&=2^km-1=(2^k)^m-1=(2^k-1)(2^{k(m-1)}+2^{k(m-2)}+...+2^1+2^0)
\end{align*}
Ponieważ $2^k>2^1=2$, to mnożymy liczbę całkowitą ($2^{k(m-1)}+2^{k(m-2)}+...+2^1+2^0$) przez liczbę całkowitą różną od $1$ ($2^k-1$). W takim razie $2^n-1$ nie jest liczbą pierwszą i mamy sprzeczność.
\medskip

{\color{def}b) $a^n-1$ jest liczbą pierwszą, to $a=2$}
\smallskip

Wiemy, że $a^n-1$ jest liczbą pierwszą, czyli nie ma dzielników całkowitych. Podobnie jak w poprzednim podpunkcie, rozpiszmy wyrażenie jako iloczyn sum:
\begin{align*}
    a^n-1=(a-1)(a^{n-1}+a^{n-2}+...+a^1+a^0).
\end{align*}
W takim razie $a-1$ musi być równe $1$, wtedy $a=2$, lub
$$a^{n-1}+a^{n-2}+...+a+1=1,$$
a więc
$$a^{n-1}+a^{n-2}+...+a=0$$
co dawałoby $a=0$, co nie jest poprawne, bo $0^n-1=-1$ nie jest liczbą pierwszą.
\medskip

{\color{def}c) $2^n+1$ - liczba pierwsza $\implies$ $(\exists\;k)\;2^k=n$}
\medskip

Załóżmy nie wprost, że $n$ nie jest potęgą liczby $2$. W takim razie $n=km$ gdzie $k,m\in\N$ i $m$ jest liczbą nieparzystą.
$$2^n+1=2^{km}+1=(2^k)^m+1$$
niech $a=2^k$, wtedy
$$a^x+1=a^x-(-1)^x=(a+1)(a^{x-1}+a^{x-2}+...+a+1)$$
Czyli $2^n+1$ jest iloczynem $(a+1)=2^y+1$ gdzie $y>1$, więc $2^n+1$ jest iloczynem dwóch liczb całkowitych i nie może być liczbą pierwszą.


\end{document}