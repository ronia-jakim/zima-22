\documentclass{article}[13pt]

\usepackage{../../../uni-notes-eng}
\usepackage{multicol}
%\definecolor{back2}{HTML}{ffffff}
%\definecolor{text2}{HTML}{000000}

\author{Weronika Jakimowicz}
\title{MDM Lista 3}
\date{}

%\pagecolor{back2}
%\color{text2}

\begin{document}
    \maketitle

    \section*{ZAD 1.}

    {\color{acc}Poprawność wzoru}
    $$f(n)=n-1+f(\lceil{n\over 2}\rceil)+f(\lfloor{n\over2}\rfloor)$$
    pokażę przez indukcję.
    \medskip

    Dla $n=2$
    $$f(2)=\sum\limits_{k=1}^2\lceil\log_2k\rceil=1$$
    $$2-1+f(1)+f(1)=1+0+0=1=f(2)$$
    czyli się zgadza.
    \medskip

    Załóżmy teraz, że wzór zachodzi dla pierwszych $n$ wyrazów. Pokażemy, że wówczas zachodzi również dla wyrazu $n+1$. 
    \medskip
    
    Rozważmy dwa przypadki:
    \smallskip

    I. $2|n+1$, wtedy możemy zapisać $n+1=2k+2$ oraz $n=2k+1$ dla pewnego $k\in\N$.
    \begin{align*}
        f(n+1)&=\sum\limits_{k=1}^{n+1}\lceil\log_2k\rceil=f(n)+\lceil\log_2n+1\rceil\overset{ind}{=}\\
        &\overset{ind}{=}n-1+f(\lceil{n\over2}\rceil)+f(\lfloor{n\over2}\rfloor)+\lceil\log_2n+1\rceil=\\
        &=n-1+f(k+1)+f(k)+\lceil\log_22(k+1)\rceil=\\
        &=n-1+\sum\limits_{i=1}^{k+1}\lceil\log_2i\rceil+\sum\limits_{i=1}^k\lceil\log_2i\rceil+\lceil1+\log_2k+1\rceil=\\
        &=n-1+\sum\limits_{i=1}^{k+1}\lceil\log_2i\rceil+\sum\limits_{i=1}^k\lceil\log_2i\rceil+1+\lceil\log_2k+1\rceil=\\
        &=(n+1)-1+\sum\limits_{i=1}^{k+1}\lceil\log_2i\rceil+\sum\limits_{i=1}^{k+1}\lceil\log_2i\rceil=\\
        &=(n+1)-1+f(\lceil{n+1\over2}\rceil)+f(\lfloor{n+1\over2}\rfloor)
    \end{align*}

    II. $2\nmid n+1$, czyli, dla pewnego $k\in\N$, mamy $n+1=2k+1$ i $n=2k$. Zauważmy, że wtedy $\lceil\log_2n+1\rceil=\lceil\log_2n+2\rceil$, bo logarytm dwójkowy liczby nieparzystej nigdy nie będzie liczbą całkowitą.

    \begin{align*}
        f(n+1)&=\sum\limits_{k=1}^{n+1}\lceil\log_2k\rceil=f(n)+\lceil\log_2n+1\rceil\overset{ind}{=}\\
        &\overset{ind}{=}n-1+f(\lceil{n\over2}\rceil)+f(\lfloor{n\over2}\rfloor)+\lceil\log_2n+1\rceil=\\
        &=n-1+f(k)+f(k)+\lceil\log_22k+1\rceil=\\
        &=n-1+\sum\limits_{i=1}^{k}\lceil\log_2i\rceil+\sum\limits_{i=1}^k\lceil\log_2i\rceil+\lceil\log_22(k+1)\rceil=\\
        &=n-1+\sum\limits_{i=1}^{k}\lceil\log_2i\rceil+\sum\limits_{i=1}^k\lceil\log_2i\rceil+\lceil1+\log_2k+1\rceil=\\
        &=n-1+\sum\limits_{i=1}^{k}\lceil\log_2i\rceil+\sum\limits_{i=1}^k\lceil\log_2i\rceil+1+\lceil\log_2k+1\rceil=
    \end{align*}
    \begin{align*}
        &=(n+1)-1+\sum\limits_{i=1}^{k}\lceil\log_2i\rceil+\sum\limits_{i=1}^{k+1}\lceil\log_2i\rceil=\\
        &=(n+1)-1+f(\lfloor{n+1\over2}\rfloor)+f(\lceil{n+1\over2}\rceil)
    \end{align*}

    {\color{acc}Jedynośc $f(n)$ jako rozwiązania podanej relacji}
    \medskip

    Jeżeli istnieją dwie funkcje $f,g$ spełniające tę zależność i
    $$f(1)=0=g(1)$$
    oraz niech $m$ oraz $d\in[0, 2^m)$ będą najmniejsze takie, że
    $$f(2^m+d)\neq g(2^m+d),$$
    czyli są pierwszym punktem w którym te funkcje są różne.

    \begin{align*}
        f(2^m+d)&=2^m+d-1+f(\lfloor 2^{m-1}+\frac d2\rfloor)+f(\lceil2^{m-1}+\frac d2\rceil)=\\
        &=(2^m+d)-1+2f(2^{m-1})+f(\lfloor \frac d2\rfloor)+f(\lceil\frac d2\rceil)
    \end{align*}
    \begin{align*}
        g(2^m+d)&=2^m+d-1+g(\lfloor 2^{m-1}+\frac d2\rfloor)+g(\lceil2^{m-1}+\frac d2\rceil)=\\
        &=(2^m+d)-1+2g(2^{m-1})+g(\lfloor \frac d2\rfloor)+g(\lceil\frac d2\rceil)
    \end{align*}
    Skoro te funkcje są po raz pierwszy różne, to $(\forall\;x<2^m+d)f(x)=g(x)$, a różnica $f(2^m+d)-g(2^m+d)\neq0$
    \begin{align*}
        f(2^m+d)-g(2^m+d)&=(2^m+d)-1+2f(2^{m-1})+f(\lfloor \frac d2\rfloor)+f(\lceil\frac d2\rceil)-((2^m+d)-1+2g(2^{m-1})+g(\lfloor \frac d2\rfloor)+g(\lceil\frac d2\rceil))=\\
        &=2f(2^{m-1})+f(\lfloor \frac d2\rfloor)+f(\lceil\frac d2\rceil)-2g(2^{m-1})-g(\lfloor \frac d2\rfloor)-g(\lceil\frac d2\rceil))
    \end{align*}
    ale ponieważ punkt który wybrałam to pierwszy punkt w którym te funkcje się różnią, wówczas $f(2^{m-1})=g(2^{m-1})$ oraz $f(\lfloor \frac d2\rfloor)=g(\lfloor \frac d2\rfloor)$ i $f(\lceil\frac d2\rceil))=g(\lceil\frac d2\rceil))$, czyli ta suma jest równa zero. W takim razie te funkcje są sobie równe.

    \section*{ZAD 2.}

    Zauważmy, że dla każdego $k$ i pewnego $m\in\N$ zachodzi:
    $$2^{m-1}< k \leq 2^m$$
    i wtedy $\lceil\log_2 k\rceil = m$. W jednym takim przedziale mamy $2^{m-1}(2-1)$ liczb całkowitych, których powały z logarytmów sumują się do $m2^{m-1}(2-1)=m2^{m-1}$. Niech dla pewnego $N$ zachodzi $N=\lceil\log_2n\rceil$, a więc
    $$2^{N-1}<n\leq 2^N$$
    i dalej
    \begin{align*}
        \sum\limits_{k=1}^n\lceil\log_2k\rceil&=\lceil\log_2n\rceil+...+\lceil\log_22^{N-1}+1\rceil+\lceil\log_22^{N-1}\rceil+...=\\
        &=N(n-2^{N-1})+\sum\limits_{k=1}^{N-1}k2^{k-1}=\\
        N(n-2^{N-1})+\frac12\sum\limits_{k=1}^{N-1}k2^k
    \end{align*}

    Oznaczmy
    $$S_n=\sum\limits_{k=1}^nk2^k,$$
    wtedy suma w drugiej części sumy wynosi:
    \begin{align*}
        S_n&=1\cdot2+2\cdot 2^2+...+ n\cdot2^n=\\
            &=2+2^2+...+2^n+(2^2+...+(n-1)2^n)=\\
            &=\sum\limits_{k=1}^n2^k+2S_{n-1}
    \end{align*}

    Wzrór na pierwszą część sumy jest łatwy do osiągnięcia:
    $$\sum\limits_{k=1}^n2^n=2^{n+1}-2$$

    Zauważamy też, że z definicji $S_n$ można wyprowadzić:
    $$S_n=\sum\limits_{k=1}^{n-1}k2^k+2^n=n2^n+S_{n-1}$$

    czyli mamy równość:
    \begin{align*}
        2^{n+1}-2+2S_{n-1}&=n2^n+S_{n-1}\\
        S_{n-1}&=2^n(n-2)+2\\
        S_n&=2^{n+1}(n-1)+2
    \end{align*}

    Wracając do sumy z zadania:
    \begin{align*}
        \sum\limits_{k=1}^n\lceil\log_2k\rceil&=N(n-2^{N-1})+\frac12S_{N-1}=\\
        &=N(n-2^{N-1})+2^{N-1}(N-2)+1=\\
        &=nN-N2^{N-1}+N2^{N-1}-2^N+1\\
        &=nN-2^N+1
    \end{align*}

    Czyli wzór jawny na szukaną funkcję to:
    $$f(n)=n\lceil\log_2n\rceil-2^{\lceil\log_2n\rceil}+1$$

    \section*{ZAD 3.}

    I. istnienie takiego zapisu: 
    \medskip

    Dla $n=1$ mamy
    $$1=1\cdot 1=1\cdot F_2.$$
    Załóżmy, że jest to prawdą również dla wszystkich liczb naturalnych do $n$ włącznie. Niech wtedy $k$ będzie największą liczbą naturalną taką, że
    $$F_k\leq n.$$
    Jeżeli $n=F_k$, to zapis jest oczywisty. W przeciwnym wypadku, liczba $m=n-F_k$ jest liczbą naturalną mniejszą niż $n$, a więc z założenia indukcyjnego możemy ją zapisać tak jak w poleceniu. Dalej zauważmy, że dla takiego $n$ mamy:
    $$F_k< n<F_{k+1}$$
    $$0< n-F_k< F_{k+1}$$
    czyli dla $m$ zauważamy, że zachodzi:
    $$m=n-F_k<F_{k+1}-F_k=F_{k-1}$$
    a więc zapis
    $$n=m+F_k$$
    nie zawiera $F_{k-1}$ czyli jest zgodny z treścią zadania.
    \bigskip

    II. jedyność:
    \medskip

    Po pierwsze, zauważmy że jeśli dany jest nam zbiór $S_j$ różnych, nienastępujących po sobie liczb Fibonacciego, to jeśli $F_j$, dla $j\geq 2$, jest największą spośród nich, ich suma jest ostro mniejsza niż $F_{j+1}$. Łatwo to udowodnić przez indukcję.
    \smallskip
    
    Dla $j=2$ mamy zbiór jednoelementowy: $S_2=\{F_2\}$ i jego suma wynosi $1<F_3=2$. Zakładamy, że dla wszystkich $j\leq n$ jest to prawdą. Wtedy dla $j=n+1$ Możemy rozdzielić taki zbiór $S_{n+1}$ na dwie części:
    $$S_{n+1}=(S_{n+1}\cap\{F_k\;:\;2\leq k\leq n-1\})\cup\{F_{n+1}\}$$
    Zauważmy, że pierwsza część tej sumy pozwala nam użyć założenia indukcyjnego, gdyż zawiera różne, nienastępujące po sobie liczby Fibonacciego nie większe niż $F_{n-1}$ (nie może być $F_n$ bo dalej mamy $F_{n+1}$ a wykluczamy występowanie dwóch kolejnych liczb Fibonacciego). Czyli ich suma jest ostro mniejsza niż $F_{n}$. Czyli mamy:
    \begin{align*}
        \sum\limits_{f\in S_{n+1}}f<F_n+F_{n+1}=F_{n+2}.
    \end{align*}

    Załóżmy, że dla pewnej liczby $n$ mamy dwa zbiory liczb Fibonacciego $U$ i $W$, spełniające założenia, takie, że
    $$\sum\limits_{f\in U}f=\sum\limits_{f\in W}f.$$
    Usuńmy teraz części wspólne tych zapisów, czyli niech $U'=U-W$ oraz $W'=W-U$. Ponieważ $U\neq U$ to te zbiory nie mogą być puste i 
    $$\sum\limits_{f\in U'}f=\sum\limits_{f\in W'}f.$$
    Weźmy teraz $u$ największe takie, że $F_u\in U$ oraz $w$ największe takie, że $F_w\in W$. Ponieważ usunęliśmy część wspólną, mamy $F_w\neq F_u$ i bez straty ogólności możemy założyć, że $F_u<F_w$. Ale wtedy mamy, zgodnie ze spostrzeżeniem na początku, że
    $$\sum\limits_{f\in U}f<F_{u+1}\leq F_w$$
    co daje nam sprzeczność z faktem, że sumy zbiorów $U$ i $W'$ są równe. Czyli któryś z nich musi być pusty. Ale wtedy jego suma jest równa 0 i musi być równa sumie drugiego zbioru, czyli oba są puste. Czyli zostaje nam, że $U=W$, bo niezerową sumę dają tylko liczby wspólne, które usunęliśmy w pierwszym kroku.

    
    \section*{ZAD 4.}

    Zauważmy, że jeżeli
    \begin{align*}
        A\mod n&=a\\
        B\mod n&=b
    \end{align*}
    to wtedy
    $$AB\mod n=(ab\mod n)$$

\begin{lstlisting}[language=juleczka]
function modulo (x, k, n)
    if k == 1
        return x % n
    else if k % 2 == 1
        return ((x % n) * modulo(x, (k-1)/2, n) * modulo(x, (k-1)/2, n)) % n
    else
        return (modulo(x, k/2, n) * modulo(x, k/2, n)) % n
\end{lstlisting}

    Algorytm wykonuje $O(\log_2k)$ mnożeń.


    \section*{ZAD 5.}
    $$
    \begin{pmatrix}
        1 & 1\\
        1 & 0
    \end{pmatrix}^n=
    \begin{pmatrix}
        F_{n+1} & F_n\\
        F_n & F_{n-1}
    \end{pmatrix}
    $$
    Dla $n=1$ mamy
    $$\begin{pmatrix}
        1&1\\1&0
    \end{pmatrix}^1=\begin{pmatrix}
        F_2&F_1\\F_1&F_0
    \end{pmatrix}$$
    Załóżmy, że zależność jest prawdziwa dla wszystkich liczb naturalnych $\leq n$. Wtedy dla $n+1$:
    \begin{align*}
        \begin{pmatrix}
            1 & 1\\
            1 & 0
        \end{pmatrix}^{n+1}&=
        \begin{pmatrix}
            1 & 1\\
            1 & 0
        \end{pmatrix}^n
        \begin{pmatrix}
            1 & 1\\
            1 & 0
        \end{pmatrix}=\\
        &=\begin{pmatrix}
            F_{n+1} & F_n\\
            F_n & F_{n-1}
        \end{pmatrix}
        \begin{pmatrix}
            1 & 1\\
            1 & 0
        \end{pmatrix}=\\
        &=\begin{pmatrix}
            F_{n+1}+F_n & F_{n+1}\\
            F_n+F_{n-1} & F_n
        \end{pmatrix}=\\
        &=\begin{pmatrix}
            F_{n+2} & F_{n+1}\\
            F_{n+1} & F_n
        \end{pmatrix}
    \end{align*}

\begin{lstlisting}[language=juleczka]
function tm(R, M)
    R = [
            R[0]*M[0]+R[1]*M[2], 
            R[0]*M[1]+R[1]*M[3], 
            R[2]*M[0]+R[3]*M[2], 
            R[2]*M[1]+R[3]*M[3]
        ]
    return R


function fibb_matrix(n)
    if n == 0 || n == 1
        return n
    
    M = [1, 1, 1, 0]
    R = [1, 0, 0, 1]

    while n > 0
        if n % 2 == 0
            n /= 2
            M = tm(M, M)
        else
            R = tm(R, M)
            n -= 1
            n /= 2
            M = tm(M, M)
    
    return R[1]
\end{lstlisting}

Zauważmy, że przy liczeniu złożoności obchodzi nas tylko największa liczba, którą mnożymy. W przypadku liczenia $n$-tej liczby Fibonacciego będzie to $F_{n+1}$. Liczby Fibonacciego można ograniczyć od góry i od dołu przez dwie funkcje wykładnicze, czyli możemy powiedzieć, że:
$$F_{n+1}=O(2^{n+1}).$$
Zauważmy, że istnieje $C$ takie, że
$$C2^{n+1}\geq 2^{n+1}+2^{n+2}=3\cdot2^{n+1}>2^{n+1}$$

W takim razie w jednym mnożeniu badanej macierzy wykonuje się $O(2^{n+1})$ mnożeń, a ponieważ złożoność została określona, to mamy $O(M(2^{n+1}))$ operacji przy mnożeniu macierzy. Dalej zauważmy, że takie mnożenie wykona się $\log_2 n+1$ razy, czyli mamy złożoność \\
$O(\log_2M(2^{n+1}))=O(M(n+1))$.

\section*{ZAD 6.}

\begin{align*}
    \begin{pmatrix}
        1 & 1\\
        1 & 0
    \end{pmatrix}^n
    \begin{pmatrix}
        F_2\\
        F_1
    \end{pmatrix}&=
    \begin{pmatrix}
        1 & 1\\
        1 & 0
    \end{pmatrix}^{n-1}
    \begin{pmatrix}
        1 & 1\\
        1 & 0
    \end{pmatrix}
    \begin{pmatrix}
        F_2\\
        F_1
    \end{pmatrix}=\\
    &=
    \begin{pmatrix}
        1 & 1\\
        1 & 0
    \end{pmatrix}^{n-1}
    \begin{pmatrix}
        F_2+F_1\\
        F_2
    \end{pmatrix}=\\
    &=\begin{pmatrix}
        1 & 1\\
        1 & 0
    \end{pmatrix}^{n-1}
    \begin{pmatrix}
        F_3\\
        F_2
    \end{pmatrix}=\\
    &=\begin{pmatrix}
        1 & 1\\
        1 & 0
    \end{pmatrix}^{n-k}
    \begin{pmatrix}
        F_{2+k}\\
        F_{1+k}
    \end{pmatrix}=\\
    &=\begin{pmatrix}
        1 & 1\\
        1 & 0
    \end{pmatrix}^{n-n}
    \begin{pmatrix}
        F_{2+n}\\
        F_{1+n}
    \end{pmatrix}=\begin{pmatrix}
        F_{2+n}\\
        F_{1+n}
    \end{pmatrix}
\end{align*}

Zauważmy, że algorytm wyliczający
$$a_n=\sum\limits_{i=1}^kp_ia_{n-i}$$
można zapisać jako
$$
\begin{pmatrix}
    0 & 1 & 0 & 0 & ... & 0\\
    0 & 0 & 1 & 0 & ... & 0\\
    0 & 0 & 0 & 1 & ... & 0\\
    & & ... \\
    p_1 & p_2 & p_3 & p_4 & ... & p_k
\end{pmatrix}^{n-k}
\begin{pmatrix}
    a_1\\
    a_2\\
    a_3\\
    ...\\
    a_k
\end{pmatrix}=
\begin{pmatrix}
    a_{n-(k-1)}\\
    a_{n-(k-2)}\\
    a_{n-(k-3)}\\
    ...\\
    a_{n}
\end{pmatrix}
$$

Czyli wystarczy, że skonstruujemy algorytm szybkiego potęgowania macierzy rozmiaru $k\times k$ i przemnożymy ostatni wiersz wyniku przez wektor $[a_1,a_2,...,a_k]$
\bigskip

Kod napisany w Pythonie:

\begin{lstlisting}[language=Python]
def tm(k, M, N):
    RET = []
    for i in range (0, k):
        r = []
        for j in range (0, k):
            s = 0
            for n in range(0, k):
                s += M[i][n] * N[n][j]
            r.append(s)
        RET.append(r)

    return RET

def calc(n, k, P, A):
    n = n-k
    M = []
    R = []
    for i in range(0, k-1):
        r = []
        p = []
        for j in range(0, k):
            if i == j:
                p.append(1)
                r.append(0)
            elif i + 1 == j:
                r.append(1)
                p.append(0)
            else:
                r.append(0)
                p.append(0)
        M.append(r)
        R.append(p)

    p = []

    for i in range(0, k-1):
        p.append(0)
    p.append(1)

    R.append(p)
    M.append(P)

    while n > 0:
        if n % 2 == 0:
            n /= 2
            M = tm(k, M, M)
        else:
            R = tm(k, M, R)
            n -= 1
            n /= 2
            M = tm(k, M, M)
    
    ret = 0
    for i in range(0, k):
        ret += A[i] * R[k-1][i]

    return ret
\end{lstlisting}


\end{document}