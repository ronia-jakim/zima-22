\documentclass{article}[13pt]

\usepackage{../../../uni-notes-eng}
\usepackage{multicol}
\definecolor{back2}{HTML}{ffffff}
\definecolor{text2}{HTML}{000000}
\usepackage{wrapfig}

\author{Weronika Jakimowicz}
\title{MDM Lista 6}
\date{}

%\pagecolor{back2}
%\color{text2}

\begin{document}
\maketitle

\textbf{\large\color{def}ZMIANY}:\smallskip

\point Zadanie 2.

\point Zadanie 5.

\point Zadanie 6.


\section*{ZAD. 1.}

Rozważmy najpierw prawą stronę równania. Spośród $n$ osób wybieramy najpierw lidera delegacji. Możemy to zrobić na $n$ sposobów. Chcemy mu dobrać pewną delegację osób. Ponieważ lider został już wybrany, to zostaje nam $(n-1)$ osób. Dla każdej z nich mamy dwie możliwości: albo osoba zostanie wybrana albo nie. Czyli dla każdej z $(n-1)$ możemy zadecydować jej los na $2$ sposoby, co daje nam
$$n\cdot 2^{n-1}$$
sposobów na wybranie delegacji z co najmniej $1$ osobą.
\medskip

Teraz zajmujemy się lewą stroną równania. Suma przechodzi przez wszystkie możliwe rozmiary delegacji: możemy wybrać delegację która ma tylko jedną osobę (wtedy $k=1$),a możemy też przecież wybrać delegację o $k=n-1$ lub $k=n$ członkach. W każdej z takich $k$-osobowych delegacji lidera możemy wybrać na $k$ sposobów. Całość daje to samo rozwiązanie co przechodzenie przez każdą potencjalną osobę po kolei i decydowanie czy ona trafia do delegacji czy też nie.

\section*{ZAD. 2.}

{\color{def}Zamieniłam pomysł z łączeniem w pary na o wiele prostsze wyjaśnienie.}
\medskip

Jeśli jedynek jest o co najmniej 2 więcej niż zer, to całość nam nie zadziała. Na przykład w 1101 nie możemy rozdzielić dwóch pierwszych 1. Załóżmy więc, że
$$k<l+2$$

Rozstawmy nasze $l$ zer w jednorodny ciąg. Między nimi jest $(l+1)$ pustych miejsc w które chcemy wstawiać nasze $k$ jedynek, ale każde miejsce możemy wybrać tylko raz. Czyli wybieramy $k$ spośród $(l+1)$ miejsc, co daje
$${l+1\choose k}.$$

\section*{ZAD. 3.}

Określmy zbiory:
$$A=\{k\;:\;1\leq k\leq n,\; 2 | k\}$$
$$B=\{k\;:\;1\leq k\leq n,\; 3 | k\}$$

Zacznijmy od znalezienia $|A\cup B|$. Z zasady włączeń i wyłączeń jest to
$$|A\cup B|=|A|+|B|-|A\cap B|=\lfloor{n\over 2}\rfloor+\lfloor{n\over 3}\rfloor-\lfloor{n\over 6}\rfloor,$$
bo $A\cap B$ to liczby podzielne jednocześnie przez $3$ i $2$, czyli podzielne przez $6$.
\medskip

Teraz niech
$$C=\{k\;:\;1\leq k\leq n,\; 14 | k\;lub\;21 | k\}$$
$$D=\{k\;:\;1\leq k\leq n,\; 10 | k\;lub\;15 | k\}$$
$$|C|=\lfloor{n\over 14}\rfloor+\lfloor{n\over 21}\rfloor -\lfloor{n\over 42}\rfloor$$
$$|D|=\lfloor{n\over 10}\rfloor+\lfloor{n\over15}\rfloor-\lfloor{n\over 30}\rfloor$$
$$|C\cup D|=|C|+|D|-\lfloor{n\over 210}\rfloor$$

Czyli teraz od sumy $|A\cup B|$ chcemy odjąć $|C\cup D|$ i to jest nasz wynik.


\section*{ZAD. 4.}

Liczba wszystkich permutacji to $n!$. Teraz wystarczy od wszystkich permutacji odjąć te, które nam nie pasują, czyli mają co najmniej jedną liczbę $i\leq k$ na pozycji $i$.

Rozważmy ciąg rekurencyjny $a(n, k)$ taki, że $a(n, k)$ to liczba permutacji zbioru $n$-elementowego, że pierwsze $k$ elementów nie jest na swoim wyjściowym miejscu. Dla $a(n, 0)$ mamy oczywiście wartość $n!$ a dla $a(n, 1)=(n-1)(n-1)!$.

Prześledźmy wędrówkę pierwszego elementu. Jeżeli zamienimy go z jednym z pierwszych $k$ elementów, to możęmy to zrobić na $(k-1)$ sposobów i zostaje nam wtedy $(k-2)$ elementy w $(n-2)$ elementach. Jeżeli zamienimy pierwszy element z jednym z ostatnich $(n-k)$ elementów, możemy to zrobić na $(n-k)$ sposobów i zostaje nam wtedy do ustawienia $(k-1)$ elementów spośród zbioru $(n-2)$ elementowego. Jeśli natomiast na miejscu pierwszego elementu postawimy jeden z pozostałych $(n-1)$ elementów ale nie zamienimy go z jedynką, to pozostaje nam ustawienie $(k-1)$ elementów spośród zbioru $(n-1)$ elementowego. Daje to poniższy wzór rekurencyjny:
$$a(n, k)=(k-1)a(n-2, k-2)+(n-k)a(n-2, k-1)+(n-1)a(n-1, k-1)$$

\section*{ZAD. 5.}

Szukamy liczby permutacji zbioru $\{1,2,...,n\}$ takich, że dla każdego $i$ nie stoi ono na pozycji $i$.
\medskip

Popatrzmy na liczbę permutacji, w których dokładnie $k$ elementów pozostaje na swoim miejscu. Z $n$ elementów te $k$ które zostanie pozostawione w swoim miejscu może zostać wybranie na ${n\choose k}$ sposobów. Z zasady włączeń i wyłączeń, mamy, że ilość permutacji, w których przynajmniej jeden element jest na swoim miejscu to
\begin{align*}
    {n\choose 1}(n-1)!-{n\choose 2}(n-2)!+{n\choose 3}(n-3)!-...=\sum\limits_{k=1}^n(-1)^{k-1}{n\choose k}(n-k)!=\sum\limits_{k=1}^n(-1)^{k-1}{n!\over k!}.
\end{align*}
Nieporządek to permutacja, w której żaden element nie jest na swoim miejscu to liczba wszystkich permutacji pomniejszona o liczbę permutacji w których przynajmniej jeden element jest na swoim miejscu:
\begin{align*}
    d_n=n!-\sum\limits_k=1^n(-1)^{k-1}{n!\over k!}=n!+\sum\limits_{k=1}^n(-1)^k{n!\over k!}=\sum\limits_{k=0}^n{n!\over k!}(-1)^k=n!\sum\limits_{k=0}^n{(-1)^k\over k!}
\end{align*}

Wzór Taylora na $e^x$ to
$$e^x=\sum\limits_{k=0}^\infty{x^k\over k!}$$
i dla $x=-1$ przyjmuje to wartość
$$e=\sum\limits_{k=0}^\infty{(-1)^k\over k!}$$
czyli liczba nieporządków to w przybliżeniu
$$d_n\approx n!e^{-1}={n!\over e}$$

{\color{def}Uzasadnienie, że różnica między $d_n$ a ${n!\over e}$ jest bardzo mała:}\smallskip

W zadaniu proszeni jesteśmy o uzasadnienie, że 
$$d_n=\lfloor{n!\over e}+\frac12\rfloor$$

\begin{align*}
    {n!\over e}&=n!\sum\limits_{k=0}^\infty {(-1)^k\over k!}=n!\sum\limits_{k=0}^{n}{(-1)^k\over k!}+n!\sum\limits_{k=n+1}^\infty {(-1)^k\over k!}=\\
    &=d_n+\sum\limits_{k=n+1}^\infty{(-1)^k\over k\cdot (k-1)...(n+1)}\leq d_n+\sum\limits_{k=n+1}^\infty {(-1)^k\over 2^{k-(n+1)+1}}=\\
    &=d_n+\sum\limits_{k=0}^\infty {(-1)^{k+n+1}\over 2^{k+1}}
\end{align*}
zauważmy, że suma $\sum\limits_{k=0}^\infty {(-1)^{k+n+1}\over 2^{k+1}}$ to szereg geometryczny, czyli jego suma to
$$\sum\limits_{k=0}^\infty {(-1)^{k+n+1}\over 2^{k+1}}=\pm\frac 12\cdot{1\over 1+\frac12}=\pm\frac12\cdot{2\over 3}=\pm\frac13$$
Czyli różnica między $d_n$ a ${n!\over e}$ jest mniejsza niż $\frac12$, więc możemy otrzymać $d_n$ poprzez przybliżanie ${n!\over e}$ do najbliższej liczby całkowitej:
$$d_n=\lfloor{n!\over e}+\frac12\rfloor.$$

\section*{ZAD. 6.}

{\color{def}Zamiana na wersje z czarnymi skarpetami}
\medskip

Najpierw zajmijmy się przyporządkowaniem jednolicie czarnych skarpet do komody. Każda z nich ma być pełna, więc jeśli $n<5$, to oczywiście nie możemy zrobić żeby żadna nie była pusta. Dajmy każdej komodzie po jednej skarpecie, w każdej z nich wybieramy szufladę na $4$ sposobów, więc rozłożyć po jednej jednolicie czarnej skarpecie do $5$ komód można na $4^5$ sposobów. Pozostaje nam rozmieszczenie $n-5$ skarpet do $20$ szuflad.
\medskip

Weźmy $19$ jednolicie białych kapci, które będą nam wskazywać które skarpety idą do której szuflady i ułóżmy skarpety w ciąg. Koniec zawartości ostatniej szuflady jest wyznaczany przez koniec ciągu, stąd też o jeden kapeć mniej niż ilość szuflad.

Mamy $(n-5)+19=n+14$ miejsc w które możemy postawić kapcia. Wybieramy więc $19$ takich miejsc i wkładamy w nie kapcie na
$${n+14\choose 19}$$
sposobów. Ostatecznie dostajemy
$$4^5\cdot{n+14\over 19}$$
sposobów rozłożenia jednolicie czarnych skarpet do $5$ komód po $4$ szuflady tak, żeby żadna komoda nie była pusta.

% Pozostałe $n-5$ skarpet możemy ułożyć w ciąg na $(n-5)!$ sposobów. Dla dowolnego takiego ciągu chcemy włożyć między skarpety kapcia, który rozgranicza które skarpety trafią do której szuflady. Takich kapciów potrzeba nam tylko $19$, bo zawartość ostatniej szuflady jest ograniczana końcem ciągu skarpet. Teraz wybieramy miejsca dla tych $19$ kapciów spośród $n-5+19=n+14$ miejsc w ciągu. Możemy to zrobić na ${n+14\choose 19}$ sposobów.
% \medskip

% Łącząc wszystko w jedną odpowiedź, dostajemy
% $$5!\cdot(n-5)!\cdot{n+14\choose 19}$$
% sposobów na rozmieszczenie $n$ skarpet w $5$ komód po $4$ szuflady tak, żeby żadna komoda nie była pusta.


\section*{ZAD. 12.}

{\color{acc}(a)} $\{id, (12345), (13524), (14253), (15432)\}$
\smallskip

TAK, jest to grupa, ponieważ wszystkie elementy to potęgi $(12345)$ (czyli mamy podgrupę cykliczną).
$$(12345)(12345)=(13524)$$
$$(12345)(13524)=(14253)$$
$$(12345)(14253)=(15432)$$
$$(12345)(15432)=(1)(2)(3)(4)(5)$$

{\color{acc}(b)} $\{id, (12)(34), (13)(24), (14)(23)\}$
\smallskip

TAK:

$$(12)(34)(13)(24)=(14)(23)$$
$$(12)(34)(14)(23)=(13)(24)$$
$$(12)(34)(12)(34)=id$$

$$(13)(24)(14)(23)=(12)(34)$$
$$(13)(24)(13)(24)=id$$
$$(13)(24)(12)(34)=(14)(32)$$

$$(14)(23)(13)(24)=(12)(34)$$
$$(14)(23)(12)(34)=(13)(24)$$
$$(14)(23)(14)(23)=id$$

Zbiór ten jest zamknięty na składanie permutacji, branie odwrotności i jego działanie jest łączne ze względu na definicję $S_5$.
\medskip

{\color{acc}(c)} $\{id, (12)(345), (135)(24), (15324), (12)(45), (134)(25), (143)(25)\}$
\smallskip

NIE:
$$(12)(345)(135)(24)=(14)(25)$$
nie jest zamknięta na składanie permutacji, bo wynik powyższego działania nie należy do danego zbioru.

\subsection*{ZAD. 14.}

Obracając wierzchołki dwunastościanu dostajemy $60$ symetrii:
\smallskip

Weźmy dowolny wierzchołek. Sąsiaduje on z $3$ innymi, więc jeśli przerzucimy ten wybrany wierzchołek na dowolny inny, co można zrobić na $20$ sposobów, bo mamy $20$ wierzchołków, to dla jednego z jego sąsiadów miejsce możemy wybrać na trzy sposoby. Pozostali sąsiedzi muszą sią dostosować do tego układu tak, żeby sądiedztwo wierzchołków nie zostało naruszone. Daje to $20\cdot 3=60$ symetrii obrotowych.
\medskip

Teraz zauważmy, że dla każdego takiego obrotu możemy go złożyć z obrotem o $\pi$ względem osi poprowadzanej przez wybrany wierzchołek i wierzchołek naprzeciwko niego. Takie złożenie jest nadal symetrią i mamy $60\cdot 2=120$ symetrii na dwunastościanie foremnym.


\end{document}