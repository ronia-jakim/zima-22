\documentclass{article}[13pt]

\usepackage{../../../uni-notes-eng}
\usepackage{multicol}
\definecolor{back2}{HTML}{ffffff}
\definecolor{text2}{HTML}{000000}
\usepackage{wrapfig}

\author{Weronika Jakimowicz}
\title{MDM Lista 6}
\date{}

%\pagecolor{back2}
%\color{text2}

\begin{document}
\maketitle

\section*{ZAD. 1.}

Rozważmy najpierw prawą stronę równania. Spośród $n$ osób wybieramy najpierw lidera delegacji. Możemy to zrobić na $n$ sposobów. Chcemy mu dobrać pewną delegację osób. Ponieważ lider został już wybrany, to zostaje nam $(n-1)$ osób. Dla każdej z nich mamy dwie możliwości: albo osoba zostanie wybrana albo nie. Czyli dla każdej z $(n-1)$ możemy zadecydować jej los na $2$ sposoby, co daje nam
$$n\cdot 2^{n-1}$$
sposobów na wybranie delegacji z co najmniej $1$ osobą.
\medskip

Teraz zajmujemy się lewą stroną równania. Suma przechodzi przez wszystkie możliwe rozmiary delegacji: możemy wybrać delegację która ma tylko jedną osobę (wtedy $k=1$),a możemy też przecież wybrać delegację o $k=n-1$ lub $k=n$ członkach. W każdej z takich $k$-osobowych delegacji lidera możemy wybrać na $k$ sposobów. Całość daje to samo rozwiązanie co przechodzenie przez każdą potencjalną osobę po kolei i decydowanie czy ona trafia do delegacji czy też nie.

\section*{ZAD. 2.}

Jeśli jedynek jest o co najmniej 2 więcej niż zer, to całość nam nie zadziała. Na przykład w 1101 nie możemy rozdzielić dwóch pierwszych 1. Załóżmy więc, że
$$k<l+2$$

Aby ułatwić sobie zadanie, sklejmy $k-1$ jedynek z zerami. Na razie niech zera będą zawsze przez jedynką, to znaczy tworzymy pary 01. Ustawiamy je jedna koło drugiej i zrobić to możemy na jeden sposób. Zostaje nam $k-(k-1)=1$ jedynka, którą musimy wstawić na sam przód ciągu i $l-(k-1)$ zer. Nie mamy ograniczeń na położenie zer, więc możemy je wstawić na dowolne miejsce między dotychczasowymi $2k-1$ elementami lub na jednym z końców, co daje nam
$${2k\choose l-k+1}$$
miejsc na wstawienie 0. Dostajemy więc ${2k\choose l-k+1}$ ciągów kiedy zera stoją przed jedynkami.
\smallskip

Teraz zauważmy, że jeśli odbijemy początkowe pary jedynek i zer, tzn. postawimy jedynki przed zerami, dostaniemy sytuację lustrzaną. Czyli kolejne ${2k\choose l-k+1}$ sposobów na ustawienie ciągu. Daje to ostateczną liczbę ciągów, gdzie jedynki nigdy nie są koło siebie, czyli
$$2\cdot {2k\choose l-k+1}$$

\section*{ZAD. 3.}

Zasada włączeń i wyłączeń, ale chwilowo mi się nie chce.

\section*{ZAD. 4.}

Liczba wszystkich permutacji to $n!$. Teraz wystarczy od wszystkich permutacji odjąć te, które nam nie pasują, czyli mają co najmniej jedną liczbę $i\leq k$ na pozycji $i$.

Rozważmy ciąg rekurencyjny $a_k$ taki, że $a_k$ to liczba permutacji zbioru $n$-elementowego, że pierwsze $k$ elementów nie jest na swoim wyjściowym miejscu. Dla $a_0$ mamy oczywiście wartość $n!$.

Dla $k+1$-elementu ciągu możemy skorzystać z poprzednich wartości.

\section*{ZAD. 5.}

Szukamy liczby permutacji zbioru $\{1,2,...,n\}$ takich, że dla każdego $i$ nie stoi ono na pozycji $i$. Niech $d_n$ oznacza liczbę nieporządków na zbiorze $n$ elementowym.


\section*{ZAD. 12.}

{\color{acc}(a)} $\{id, (12345), (13524), (14253), (15432)\}$
\smallskip

Tak, jest to grupa, ponieważ wszystkie elementy to potęgi $(12345)$ (czyli mamy podgrupę cykliczną).
$$(12345)(12345)=(13524)$$
$$(12345)(13524)=(14253)$$
$$(12345)(14253)=(15432)$$
$$(12345)(15432)=(1)(2)(3)(4)(5)$$

{\color{acc}(b)} $\{id, (12)(34), (13)(24), (14)(23)\}$
\smallskip

$$(12)(34)(13)(24)=(14)(23)$$
$$(12)(34)(14)(23)=(13)(24)$$
$$(12)(34)(12)(34)=(1)(2)(3)(4)$$

$$(13)(24)(14)(23)=(12)(34)$$
$$(13)(24)(13)(24)=id$$
$$(13)(24)(12)(34)=(14)(32)$$

$$(14)(23)(13)(24)=(12)(34)$$
$$(14)(23)(12)(34)=(13)(24)$$

\end{document}