\section{Wstep}
\subsection{Przestrzenie normalne}

\begin{multicols}{2}
    
    {\color{def}Norma} na $X$ to funkcja $x\mapsto \|x\|\in[0, \infty)$ taka, ze\smallskip\\
        \point $\|x\|=0\iff x=0$\smallskip\\
        \point $(\forall\;\lambda\in\C)(\forall\;x\in X) \|\lambda x\|=|\lambda|\|x\|$ - \emph{jednorodnosc}\smallskip\\
        \point $(\forall\;x,y\in X)\| x+y\| \leq \|x\|+\|y\|$
    \bigskip

    Przestrzen metryczna jest {\color{def}zupelna}, jesli kazdy ciag Cauchy'ego jest zbiezny.
    \medskip

    {\color{def}Przestrzen Banacha} - unormowana przestrzen zupelna w metryce $d(x, y)=\|x-y\|$.
    \bigskip

    \pdef

    Szereg $\sum\limits_{n=1}^\infty x_n$ jest {\color{acc}zbiezny}, jesli szereg sum czesciowych jest zbiezny.\\

    \kdef

    Szereg jest {\color{acc}bezwzglednie zbiezny}, jesli zbiezny jest $\sum\limits_{n=1}^\infty\|x_n\|$
    \medskip

    \pdef
    
    Przestrzen jest unormowana $\iff$ kazdy szereg bezwzglednie zbiezny jest zbiezny.


    \kdef
    \medskip

    Normy $\|\;\|_1$ i $\|\;\|_2$ sa {\color{def}rownowazne}, jesli istnieja $c_1, c_2>0$ takie, ze
    $$(\forall\;x)\;c_2\|x\|_2\leq \|x\|_1\leq c_1\|x\|_2.$$

    {\color{acc}\point} Jesli zbieznosc ciagow w dwoch normach jest rownowazna, to sa one rownowazne.

    {\color{acc}\point} Przestrzenie $\C^n$ oraz $\R^n$ sa zupelne w dowolnej normie.

    {\color{acc}\point} Przestrzen unormowana skonczona jest zawsze zupelna.
    \medskip

    {\color{def}Twierdzenie o najlepszej aproksymacji} - dla skonczonej podprzestrzeni liniowej $E$ przestrzeni unormowanej $X$ zachodzi:
    $$(\forall\;x\in X)(\exists\;x_0\in E)\;\|x-x_o\|=\inf\limits_{y\in E}\|x-y\|.$$

    \podz{sep}\bigskip

    \pdef

    Podzbior $A\subseteq X$ jest zbiorem {\color{def}gestym}, jezeli\\
    $(\forall\;x\in X)(\forall\varepsilon>0)(\exists\;a\in A)\;d(x,a)<\varepsilon$
    \medskip

    Przestrzen jest {\color{def}osrodkowa}, gdy posiada przeliczalny zbior gesty.


    \kdef

    Kazda przestrzen unormowana mozna uzupelnic do przestrzeni Banacha.
    \medskip

    Niech $Y\subseteq X$ bedzie domkniety, wtedy
    $$(\forall\;0<\theta<1)(\exists\;x\in X)\;\inf\{\|x-y\|\;:\;y\in Y\}\geq\theta$$

    Niech $X$ - unormowana, skonczona przestrzen liniowa, wtedy
    $$(\exists\;(x_n)\subseteq X)(\forall\;n\neq m)\;\|x_n\|=1\;\land\;\|x_n-x_m\|\geq \frac12$$

    Baza nieskonczonej przestrzeni Banacha jest nieprzeliczalna.

\end{multicols}