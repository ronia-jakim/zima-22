\section{Przestrzenie normalne}

\begin{multicols*}{2}
    
    {\color{def}Norma} na $X$ to funkcja $x\mapsto \|x\|\in[0, \infty)$ taka, ze\smallskip\\
        \point $\|x\|=0\iff x=0$\smallskip\\
        \point $(\forall\;\lambda\in\C)(\forall\;x\in X) \|\lambda x\|=|\lambda|\|x\|$ - \emph{jednorodnosc}\smallskip\\
        \point $(\forall\;x,y\in X)\| x+y\|\leq \|x\|+\|y\|$
    \bigskip

    Przestrzen metryczna jest {\color{def}zupelna}, jesli kazdy ciag Cauchy'ego jest zbiezny.
    \medskip

    {\color{def}Przestrzen Banacha} - unormowana przestrzen zupelna w metryce $d(x, y)=\|x-y\|$.
    \bigskip

    \pdef

    Szereg $\sum\limits_{n=1}^\infty x_n$ jest {\color{acc}zbiezny}, jesli szereg sum czesciowych jest zbiezny.\\

    \kdef

    Szereg jest {\color{acc}bezwzglednie zbiezny}, jesli zbiezny jest $\sum\limits_{n=1}^\infty\|x_n\|$
    \medskip

    \pdef
    
    Przestrzen jest unormowana $\iff$ kazdy szereg bezwzglednie zbiezny jest zbiezny.


    \kdef

\end{multicols*}